21. Чётное число, не кратное 6, может иметь вид $6k+2$ или $6k+4.$ Тогда $(6k+2)^2=36k^2+24k+4=12(3k^2+2k)+4,\ (6k+4)^2=36k^2+48k+16=12(3k^2+4k+1)+4.$ В обоих случаях квадрат этого числа при делении на 12 даёт остаток 4.\\
