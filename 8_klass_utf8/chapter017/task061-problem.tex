61. а) Пусть куплено $x$ красных карандашей и $y$ синих. Тогда выполнены три условия\\ $\begin{cases}17x+13y\leqslant495,\\ x+y=32,\\ |x-y|\leqslant5.\end{cases}
\Leftrightarrow \begin{cases}17(32-y)+13y\leqslant495,\\ x=32-y,\\ |32-y-y|\leqslant5.\end{cases}
\Leftrightarrow \begin{cases}49\leqslant4y,\\ x=32-y,\\ |32-2y|\leqslant5.\end{cases}$
Этим условиям удовлетворяют, например, $x=y=16.$\\
б) Аналогично пункту а) запишем $\begin{cases}17x+13y\leqslant495,\\ x+y=35,\\ |x-y|\leqslant5.\end{cases}
\Leftrightarrow \begin{cases}17(35-y)+13y\leqslant495,\\ x=35-y,\\ |35-y-y|\leqslant5.\end{cases}
\Leftrightarrow$\\$ \begin{cases}100\leqslant4y,\\ x=35-y,\\ |35-2y|\leqslant5.\end{cases}
\Leftrightarrow \begin{cases}25\leqslant y,\\ x=35-y,\\ |35-2y|\leqslant5.\end{cases}$
В этом случае $|35-2y|\geqslant|35-50|=15,$ так что 35 карандашей купить нельзя.\\
в) Пусть куплено $n$ карандашей, тогда $\begin{cases}17x+13y\leqslant495,\\ x+y=n,\\ |x-y|\leqslant5.\end{cases}
\Leftrightarrow \begin{cases}17(n-y)+13y\leqslant495,\\ x=n-y,\\ |n-y-y|\leqslant5.\end{cases}
\Leftrightarrow$\\$ \begin{cases}17n-495\leqslant4y,\\ x=n-y,\\ |n-2y|\leqslant5.\end{cases}$
Так как $17n-495\leqslant4y,$ верно неравенство $y\geqslant \cfrac{17n-495}{4}.$ Тогда $|n-2y|=2y-n\geqslant \cfrac{17n-495}{2}-n=\cfrac{15n-495}{2}.$ Для выполнения третьего условия должно выполняться неравенство $\cfrac{15n-495}{2} \leqslant 5 \Leftrightarrow3n-99\leqslant 2\Leftrightarrow 3n\leqslant 101
\Leftrightarrow n\leqslant 33.$ При $n=33$ можно взять $x=16,\ y=17.$\\
