178. {{PIC:g8-178.png}}\\
а) Посчитаем углы: $\angle B=90^\circ-60^\circ=30^\circ,\ \angle CAP=\angle BAP=60^\circ:2=30^\circ.$ В прямоугольном треугольнике $ACP$ катет $PC$ лежит напротив угла в $30^\circ,$ значит $AP=2PC=2,$ а треугольник $APB$ является равнобедренным и $AP=PB=2.$ По теореме Пифагора $AC=\sqrt{4-1}=\sqrt{3},$ поэтому $S_{\Delta ABC}=\cfrac{\sqrt{3}(1+2)}{2}=\cfrac{3\sqrt{3}}{2}.$\\
б) $S_{\Delta ABP}=S_{\Delta ABC}-S_{\Delta ACP}=\cfrac{3\sqrt{3}}{2}-\cfrac{\sqrt{3}\cdot1}{2}=\sqrt{3}.$\\
в) Угол $ABP$ опирается на дугу $AP,$ значит $\smile AP=2\angle ABP=60^\circ.$ Угол $AOP$ является центральным, значит $\angle AOP=60^\circ.$ В треугольнике $AOP$ стороны $AO$ и $OP$ равны радиусу окружности, значит он равнобедренный, а раз угол при его вершине равен $60^\circ,$ то и равносторонний. Поэтому $R=AP=2.$\\
