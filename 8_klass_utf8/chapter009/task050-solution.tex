50. {{PIC:g8-49.png}}\\
Проведём перпендикуляры из точки пересечения диагоналей ко всем сторонам параллелограмма. Треугольники $AOH_2$ и $COH_1$ равны по второму признаку: $AO=OC,$ Углы $AOH_2$ и $COH_1$ вертикальные, а $H_2AO$ и $H_1CO$ --- накрест лежащие. Значит, $H_1O=H_2O,$ поэтому $H_1H_2=2OH_1.$ Поэтому высоты параллелограмма в два раза больше, чем расстояния от точки пересечения диагоналей до сторон. Тогда одна сторона равна $48:6=8$см, а другая $48:8=6$см, таким образом периметр параллелограмма равен $(8+6)\cdot2=28$см.\\
