88. Пусть скорость второго пешехода равна $x$ км/ч, тогда скорость первого равна $x+2$ км/ч. Так как места встречи они достигли одновременно, верно соотношение $\cfrac{15}{x+2}+\cfrac{1}{2}=\cfrac{12}{x},\
30x+x^2+2x=24x+48,\ x^2+8x-48=0,\ (x-4)(x+12)=0,\ x=4$км/ч. Тогда скорость пешехода, шедшего из $A,$ равна $4+2=6$км/ч.
\newpage
