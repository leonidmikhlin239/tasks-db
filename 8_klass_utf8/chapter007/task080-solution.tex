80. Пусть в растворе было $m$г соли тогда имеем систему уравнений $\begin{cases} (m+60)\cdot\cfrac{x}{100}=m,\\ (m+80)\cdot\cfrac{x-5}{100}=m.\end{cases}\Leftrightarrow \begin{cases} mx+60x=100m,\\ mx-5m+80x-400=100m.\end{cases}\Rightarrow
5m-20x+400=0\Rightarrow m=4x-80.$ Значит, $4x^2-80x+60x=400x-8000,\ x^2-420x+8000=0,\ (x-25)(x-80)=0,\ x=25$ или $x=80.$ Значит, раствор мог содержать $4\cdot25-80=20$г соли или $4\cdot80-80=240$г соли.\\
