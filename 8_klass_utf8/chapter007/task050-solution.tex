50. Пусть скорость велосипедиста равна $x\text{ км/ч},$ а длина пути равна $S$ км. Тогда имеем систему уравнений
$\begin{cases}\cfrac{S}{x}=\cfrac{S}{x+5}+\cfrac{1}{5},\\ \cfrac{S}{x}=\cfrac{S}{x-8}-\cfrac{2}{3}.\end{cases}
\Leftrightarrow
\begin{cases}\cfrac{Sx+5S-Sx}{x(x+5)}=\cfrac{1}{5},\\ \cfrac{Sx-Sx+8S}{x(x-8)}=\cfrac{2}{3}.\end{cases}
\Leftrightarrow
\begin{cases}\cfrac{5S}{x(x+5)}=\cfrac{1}{5},\\ \cfrac{8S}{x(x-8)}=\cfrac{2}{3}.\end{cases}
\Leftrightarrow$\\$
\begin{cases}x^2+5x=25S,\\ 2x^2-16x=24S.\end{cases}
\Leftrightarrow
\begin{cases}x^2+5x=25S,\\ 26x=26S.\end{cases}
\Leftrightarrow
\begin{cases}x(x-20)=0,\\ x=S.\end{cases}
\Leftrightarrow
\begin{cases}x=20\text{ км/ч},\\ S=20\text{ км}.\end{cases}$\\
