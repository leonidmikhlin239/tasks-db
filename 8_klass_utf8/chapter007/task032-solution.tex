32. Пусть скорость байдарки в стоячей воде равна $x\text{ км/ч},$ тогда $\cfrac{15}{x}+\cfrac{9}{x-2}=\cfrac{42}{x+2}\Leftrightarrow$\\$
\cfrac{15x-30+9x}{x^2-2x}=\cfrac{42}{x+2}\Leftrightarrow 24x^2-30x+48x-60=42x^2-84x\Leftrightarrow
18x^2-102x+60=0\Leftrightarrow 9x^2-51x+30=0 \Leftrightarrow x=5\text{ км/ч}$(второй корень равен $\cfrac{2}{3}\text{ км/ч},$ но с такой скоростью нельзя было бы плыть против течения).\\