81. Пусть скорость первого трактора была равна $x$ км/ч, тогда скорость второго трактора была равна $x+5$ км/ч. Переведя минуты в часы (второй трактор потратил на дорогу на $5+3=8$ минут меньше), составим уравнение: $\cfrac{20}{x}-\cfrac{20}{x+5}=\cfrac{2}{15},\
\cfrac{20x+100-20x}{x(x+5)}=\cfrac{2}{15},\ x^2+5x=750,\ x^2+5x-750=0,\ x=25$км/ч, тогда скорость второго трактора равна $25+5=30$ км/ч.\\
