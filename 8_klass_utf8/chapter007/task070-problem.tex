70. При температуре $0^\circ C$ рельс имеет длину $l_0=10$м. При возрастании температуры происходит тепловое расширение рельса, и его длина, выраженная в метрах, меняется по закону:
$$l(t^\circ)=l_0(1+\alpha\cdot t^\circ),$$
где $\alpha=1,2\cdot10^{-5}(^\circ C)^{-1}$ --- коэффициент теплового расширения, $t^\circ$ --- температура (в градусах Цельсия). При какой температуре рельс удлинится на 3 мм? Ответ выразите в градусах Цельсия.\\