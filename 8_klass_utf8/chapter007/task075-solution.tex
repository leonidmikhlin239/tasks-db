75. Пусть один тратит на забег внутри озера $x$ минут, а другой --- $x+9.$ Тогда их скорости равны $\cfrac{1}{x}$ и $\cfrac{1}{x+9}$ (кругов в минуту). В первый раз они встретятся, когда преодолеют вместе один круг, поэтому $\cfrac{1}{\cfrac{1}{x}+\cfrac{1}{x+9}}=20,\ \cfrac{x(x+9)}{2x+9}=20,\
x^2-31x-180=0,\ (x+5)(x-36)=0,\ x=36$мин. Значит, скорость одного равна $\cfrac{1}{36},$ а другого --- $\cfrac{1}{45}.$ Если они побегут в одном направлении, то впервые встретятся, когда один обгонит второго на один круг, на это они потратят $\cfrac{1}{\cfrac{1}{36}-\cfrac{1}{45}}=180$ минут или 3 ч.\\