73. Пусть первого раствора изначально было взято $x$ литров, а второго --- $y.$ Тогда имеем систему уравнений $\begin{cases} 0,3x+0,65y=0,55(x+y),\\
0,3(x+20)+0,65(y+20)=0,51(x+y+40).\end{cases}\Leftrightarrow$\\$\begin{cases} 0,25x=0,1y,\\
0,3x+0,65y+19=0,51x+0,51y+20,4.\end{cases}\Leftrightarrow\begin{cases} y=\cfrac{5}{2}x,\\
0,21x+1,4=0,14y.\end{cases}\Leftrightarrow$\\$\begin{cases} y=\cfrac{5}{2}x,\\
0,21x+1,4=0,35x.\end{cases}\Leftrightarrow\begin{cases} y=25,\\
x=10.\end{cases}$ Значит, первого раствора было 10 литров, а второго --- 25.\\
