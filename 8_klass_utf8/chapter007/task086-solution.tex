86. В конце концов в первом сосуде оказалось $(30+2):2=16$л спирта, а во втором --- $30-16=14$ литров. Пусть изначально в первом сосуде было $x$л спирта, тогда во втором было $30-x$ литров. Доля спирта в первом сосуде после дополнения стала $\cfrac{x}{30},$ тогда во второй сосуд после переливания попало $x\cdot\cfrac{x}{30}=\cfrac{x^2}{30}$ литров спирта, и всего в нём стало $30-x+\cfrac{x^2}{30}$ спирта. Когда из него вылили 12 литров из 30,
спирта осталось $\cfrac{18}{30}\cdot\left(30-x+\cfrac{x^2}{30}\right)$ литров спирта, значит $\cfrac{18}{30}\cdot\left(30-x+\cfrac{x^2}{30}\right)=14,\
90-3x+\cfrac{x^2}{10}=70,\ x^2-30x+200=0,\ (x-20)(x-10)=0,\ x=10$л или $x=20$л. Значение $x=10$л не подходит, так как тогда из первого сосуда сначала выливают 10 литров, а потом доливают 12, отчего он должен переполниться. Таким образом, в первом сосуде было 20 литров спирта, а во втором --- 10 литров.\\
