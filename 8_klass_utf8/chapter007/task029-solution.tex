29. Пусть скорость третьего мотоциклиста равна $x\text{ км/ч}.$ Первый мотоциклист за 20 минут проедет $75:3=25$ километров, а второй --- $60:3=20$ километров. Тогда $\cfrac{25}{x-75}-\cfrac{20}{x-60}=1\Leftrightarrow\cfrac{25x-1500-20x+1500}{x^2-135x+4500}=1\Leftrightarrow 5x=x^2-135x+4500\Leftrightarrow
x^2-140x+4500=0\Leftrightarrow x=90\text{ км/ч}$ (второй корень равен $50\text{ км/ч},$ но с такой скоростью третий мотоциклист никого бы не догнал).\\
