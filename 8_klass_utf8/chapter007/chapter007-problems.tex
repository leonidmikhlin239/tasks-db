\section{Стандартные задачи}
1. При одновременной работе двух насосов пруд был очищен за 2 ч 55 мин. За сколько времени мог бы очистить пруд каждый насос, работая отдельно, если один из них может эту работу выполнить на 2 ч быстрее другого?\\
2. За 4 дня совместной работы двух тракторов различной мощности было вспахано $\cfrac{2}{3}$ поля. За сколько дней можно было бы вспахать всё поле каждым трактором в отдельности, если первым трактором это можно сделать на 5 дней быстрее, чем вторым?\\
3. Положительное число $a$ составляет $200\%$ от своего квадрата. Найдите число $a.$\\
4. Положительное число $a$ составляет $400\%$ от своего квадрата. Найдите число $a.$\\
5. Товар первоначально стоил $a$ рублей. Затем он подорожал на $10\%,$ а потом ещё на $15\%.$ На сколько процентов от первоначальной стоимости подорожал товар?\\
6. Товар первоначально стоил $a$ рублей. Затем он подорожал на $15\%,$ а потом ещё на $10\%.$ На сколько процентов от первоначальной стоимости подорожал товар?\\
7. Путь от города до посёлка автомобиль проезжает за 2,5 часа. Если он увеличит скорость на 20 км/ч, то за 2 часа он проедет путь на 15 км больше, чем расстояние от города до посёлка. Найти расстояние от города до посёлка.\\
8. Из Москвы в Санкт-Петербург выехал автобус. Спустя час вслед за ним вышла
легковая машина, скорость которой на 20 км/ч больше скорости автобуса. Машина обогнала автобус и через 5 часов после своего выхода находилась впереди него на 70 км. Найти скорость автобуса.\\
9. Саша и Стас вскапывают грядку за 10 минут, а один Стас за 15 минут. За сколько минут Саша один вскопает грядку?\\
10. Таня и Лена пропалывают грядку за 12 минут, а одна Лена за 20 минут. За сколько минут прополет грядку одна Таня?\\
11. На двух копировальных машинах, работающих одновременно, можно сделать копию пакета документов за 10 минут. За какое время можно выполнить эту работу на каждой машине в отдельности, если известно, что на первой машине её можно сделать на 15 минут быстрее, чем на второй?\\
12. На двух копировальных машинах, работающих одновременно, можно сделать копию пакета документов за 20 минут. За какое время можно выполнить эту работу на каждой машине в отдельности, если известно, что на первой машине её можно сделать на 30 минут быстрее, чем на второй?\\
13. Комната с размерами 4 метра и 6 метров составляет $75\%$ от всей квартиры. Найдите площадь квартиры.\\
14. Комната с размерами 3 метра и 4 метра составляет $60\%$ от всей квартиры. Найдите площадь квартиры.\\
15. Из А и В одновременно выехали 2 автомобиля. Первый проехал весь путь с постоянной скоростью. Второй проехал первую половину пути со скоростью 24 км/ч, а вторую половину пути --- со скоростью на 16 км/ч больше скорости первого, в результате чего прибыл в В одновременно с первым автомобилем. Найдите скорость первого автомобиля.\\
16. Из А и В одновременно выехали 2 автомобиля. Первый проехал весь путь с постоянной скоростью. Второй проехал первую половину пути со скоростью 33 км/ч, а вторую половину пути --- со скоростью на 22 км/ч больше скорости первого, в результате чего прибыл в В одновременно с первым автомобилем. Найдите скорость первого автомобиля.\\
17. Товар стоил $a$ рублей. Потом он подорожал на $10\%.$ После чего подешевел на $10\%$ от новой цены. Сколько стал стоить товар после удешевления?\\
18. Товар стоил $a$ рублей. Потом он подешевел на $10\%.$ После чего подорожал на $10\%$ от новой цены. Сколько стал стоить товар после подорожания?\\
19. Пароход прошёл 9 км по озеру и 20 км по течению реки за 1 час. Найти скорость парохода при движении по озеру, если скорость течения реки равна 3 км/ч.\\
20. Пароход прошёл 9 км по озеру и 16 км против течения реки за 1 час. Найти скорость парохода при движении по озеру, если скорость течения реки равна 3 км/ч.\\
21. Велосипедист проехал за 2,5 часа 58 км, а за следующий час ещё 19 км. Найдите среднюю скорость велосипедиста.\\
22. Велосипедист проехал за 1,5 часа 36 км, а за следующие 2 часа ещё 34 км. Найдите среднюю скорость велосипедиста.\\
23. Рубашка на $20\%$ дешевле пиджака. На сколько процентов пиджак дороже рубашки?\\
24. Из Санкт-Петербурга в Псков выехал автомобиль Москвич со скоростью 60 км/ч. В то же время навстречу ему из Пскова выехал автомобиль Жигули со скоростью 80 км/ч. Какое расстояние было между ними за час до встречи?\\
25. Сколько граммов воды надо добавить к 180 граммам сиропа, содержащего $25\%$ сахара, чтобы получить сироп, процентное содержание сахара в котором равно $20\%?$\\
26. Сколько граммов воды надо добавить к 220 граммам сиропа, содержащего $25\%$ сахара, чтобы получить сироп, процентное содержание сахара в котором равно $20\%?$\\
27. Цена билета на стадион была 150 рублей. После снижения цены билета количество посетителей увеличилось на $50\%,$ а сбор увеличился на $25\%.$ Найти новую цену билета.\\
28. Цена билета на стадион была 120 рублей. После снижения цены билета количество посетителей увеличилось на $50\%,$ а сбор увеличился на $25\%.$ Найти новую цену билета.\\
29. Одновременно из пункта А в одном направлении выехали два мотоциклиста: скорость одного из них 75 км/ч, а скорость второго 60 км/ч. Через 20 минут вслед за ними из пункта А выехал третий мотоциклист. Найдите скорость третьего мотоциклиста, если известно, что он догнал первого мотоциклиста на 1 час позже, чем второго. Ответ дайте в километрах в час.\\
30. Турист проплыл на байдарке 25 км по озеру и 9 км против течения реки за столько же времени, за сколько он проплыл бы по течению той же реки 56 км. Найдите скорость байдарки в стоячей воде, если скорость течения реки равна 2 км/ч. Ответ дайте в километрах в час.\\
31. Одновременно из пункта А в одном направлении выехали два велосипедиста: скорость одного из них 15 км/ч, а скорость второго 12 км/ч. Через 20 минут вслед за ними из пункта А выехал третий велосипедист. Найдите скорость третьего велосипедиста, если известно, что он догнал первого велосипедиста на 1 час позже, чем второго. Ответ дайте в километрах в час.\\
32. Турист проплыл на байдарке 15 км по озеру и 9 км против течения реки за столько же времени, за сколько он проплыл бы по течению той же реки 42 км. Найдите скорость байдарки в стоячей воде, если скорость течения реки равна 2 км/ч. Ответ дайте в километрах в час.\\
33. Торговая база закупила партию альбомов и поставила её магазину по оптовой цене на $130\%$ выше закупочной. Магазин установил розничную цену на альбом на $5\%$ выше оптовой. При распродаже в конце сезона магазин снизил розничную цену на альбом на $40\%.$ На сколько рублей больше заплатил покупатель по сравнению с закупочной ценой, если на распродаже он приобрёл альбом за 57,96 рубля?\\
34. Торговая база закупила партию альбомов и поставила её магазину по оптовой цене на $20\%$ выше закупочной. Магазин установил розничную цену на альбом на $110\%$ выше оптовой. При распродаже в конце сезона магазин снизил розничную цену на альбом на $30\%.$ На сколько рублей больше заплатил покупатель по сравнению с закупочной ценой, если на распродаже он приобрёл альбом за 123,48 рубля?\\
35. Торговая база закупила партию альбомов и поставила её магазину по оптовой цене на $5\%$ выше закупочной. Магазин установил розничную цену на альбом на $120\%$ выше оптовой. При распродаже в конце сезона магазин снизил розничную цену на альбом на $40\%.$ На сколько рублей больше заплатил покупатель по сравнению с закупочной ценой, если на распродаже он приобрёл альбом за 69,3 рубля?\\
36. Торговая база закупила партию альбомов и поставила её магазину по оптовой цене на $110\%$ выше закупочной. Магазин установил розничную цену на альбом на $5\%$ выше оптовой. При распродаже в конце сезона магазин снизил розничную цену на альбом на $30\%.$ На сколько рублей больше заплатил покупатель по сравнению с закупочной ценой, если на распродаже он приобрёл альбом за 92,61 рубля?\\
37. В банк 01.01.2014 г. положили 50000 р. 31 декабря каждого года банк увеличивает вклад на одно и то же число процентов. На какое число процентов ежегодно увеличивается вклад, если 01.01.2016 г. вклад составил 55125 р.?\\
38. В банк 01.01.2014 г. положили 2000 р. 31 декабря каждого года банк увеличивает вклад на одно и то же число процентов. На какое число процентов ежегодно увеличивается вклад, если 01.01.2016 г. вклад составил 2420 р.?\\
39. Один км от дома до остановки автобуса Петя проходит за 15 мин. Следующие 7 км на автобусе он проезжает за 12 мин. Затем 2 км от остановки до школы мальчик пробегает за 13 мин. Какова средняя скорость Пети в школу?\\
40. Спортсмен-триатлонист сначала проплыл 1 км за 15 мин, потом пробежал 10 км за 50 мин, затем проехал на велосипеде 29 км за 55 мин. Какова средняя скорость спортсмена?\\
41. На соревнованиях по кольцевой трассе один лыжник проходил круг на 3 минуты быстрее другого и через час обогнал его ровно на круг. За сколько минут каждый лыжник проходит круг?\\
42. На соревнованиях по кольцевой трассе один лыжник проходил круг на 5 минут быстрее другого и через час обогнал его ровно на круг. За сколько минут каждый лыжник проходит круг?\\
43. Свежий виноград содержит $80\%$ влаги, а сушёный виноград (изюм) --- $5\%.$ Сколько требуется свежего винограда для приготовления 1 кг изюма?\\
44. Винни Пух и Пятачок могут изготовить подарок ослику Иа-Иа за 8 дней. Определите, за сколько дней Пятачок изготовит этот подарок, работая отдельно, если известно, что он сделает это на 12 дней быстрее, чем Винни.\\
45. Малыш и Карлсон вместе съедают банку варенья за 12 минут. Определите, за сколько минут справится с банкой варенья Карлсон, если известно, что он сделает это на 10 минут быстрее, чем Малыш.\\
46. Из Петербурга в Москву одновременно отправились курьер и гонец. Курьер проехал с постоянной скоростью весь путь. Гонец проехал первую половину пути со скоростью 102 км/ч, а вторую половину пути --- со скоростью, на 17 км/ч меньшей скорости курьера, в результате чего прибыл в Москву одновременно с курьером. Известно, что скорость курьера меньше 60 км/ч. Найдите скорость курьера.\\
47. Из Москвы в Петербург одновременно выехали генерал и чиновник. Генерал проехал с постоянной скоростью весь путь. Чиновник проехал первую половину пути со скоростью, меньшей скорости генерала на 13 км/ч, а вторую половину пути --- со скоростью 78 км/ч, в результате чего прибыл в Петербург одновременно с генералом. Найдите скорость генерала, если известно, что она больше 48 км/ч.\\
48. После того, как рабочий увеличил производительность своего труда на $16\%,$ он сократил на 1 час время выполнения задания. Сколько времени теперь он стал тратить на выполнение этого задания?\\
49. Смешали 30 г $20\%$-го раствора соли с 10 г другого раствора и получили раствор с концентрацией соли $25\%.$ Определить концентрацию соли во втором растворе.\\
50. Если велосипедист увеличит скорость на 5 км/ч, то получит выигрыш во времени 12 минут при прохождении некоторого пути. Если же он уменьшит скорость на 8 км/ч, то потеряет 40 минут на том же пути. Найдите скорость велосипедиста и длину пути.\\
51. Численность волков в двух заповедниках составляла 210 особей. Через год обнаружили, что в первом заповеднике численность волков выросла на $10\%,$ а во втором --- на $30\%.$ В результате общая численность волков в этих двух заповедниках составила 251 особь. Сколько волков было в каждом из двух заповедников первоначально?\\
52. Сколько граммов четырёхпроцентного и сколько граммов девятипроцентного растворов соли необходимо взять, чтобы получить 250 граммов её шестипроцентного раствора?\\
53. Некоторый товар стоил 1000 рублей, но не пользовался спросом. Поэтому его цена дважды снижалась на $40\%$ и один раз на $20\%.$ Сколько стал стоить этот товар после последнего снижения цены?\\
54. Некоторый товар стоил 1000 рублей, но не пользовался спросом. Поэтому его цена дважды снижалась на $20\%$ и один раз на $60\%.$ Сколько стал стоить этот товар после последнего снижения цены?\\
55. Мотоциклист задержался с выездом на 9 минут. Чтобы наверстать потерянное время, он увеличил намеченную скорость на 10 км/ч. С какой скоростью ехал мотоциклист, если весь путь равен 30 км?\\
56. Петя вышел из школы и пошёл домой со скоростью 4,5 км/ч. Через 20 минут по той же дороге из школы выехал Вася на велосипеде со скоростью 12 км/ч. На каком расстоянии от школы Вася догонит Петю?\\
57. Нина поехала на велосипеде на рынок со скоростью 15 км/ч. Через 6 минут по той же дороге поехал на мопеде её брат со скоростью 40 км/ч. На каком расстоянии от дома брат догонит Нину?\\
58. Сколько граммов $15\%$-ного раствора соли надо добавить к 50 г $60\%$-ного раствора соли, чтобы получить $40\%$-ный раствор соли?\\
59. Сколько граммов $75\%$-ного раствора кислоты надо добавить к 30 г $15\%$-ного раствора кислоты, чтобы получить $50\%$-ный раствор кислоты?\\
60. По двум параллельным железнодорожным путям в одном направлении следуют пассажирский и товарный поезда, скорости которых равны 70 км/ч и 30 км/ч. Длина товарного поезда равна 1400 метрам. Найдите длину пассажирского поезда, если время, за которое он прошел мимо товарного поезда, равно 3 минутам.\\
61. По двум параллельным железнодорожным путям в одном направлении следуют пассажирский и товарный поезда, скорости которых равны 75 км/ч и 45 км/ч. Длина товарного поезда равна 800 метрам. Найдите длину пассажирского поезда, если время, за которое он прошел мимо товарного поезда, равно 2 минутам.\\
62. В сосуд, содержащий 7 литров 14-процентного раствора некоторого вещества, добавили 21 литр воды. Сколько процентов составляет концентрация полученного раствора?\\
63. В сосуд, содержащий 5 литров 12-процентного раствора некоторого вещества, добавили 7 литров воды. Сколько процентов составляет концентрация полученного раствора?\\
64. Повысив скорость на 10 км/ч, поезду удалось сократить на 1 час время, затрачиваемое на прохождение 720 км. Найти первоначальную скорость поезда.\\
65. Повысив скорость на 10 км/ч, поезду удалось сократить на 1 час время, затрачиваемое на прохождение 560 км. Найти первоначальную скорость поезда.\\
66. Смешали 8 литров $25\%$ водного раствора некоторого вещества с 12 литрами $20\%$ водного раствора этого же вещества. Сколько процентов составляет концентрация получившегося раствора?\\
67. Смешали 4 литра $15\%$ водного раствора некоторого вещества с 6 литрами $25\%$ водного раствора этого же вещества. Сколько процентов составляет концентрация получившегося раствора?\\
68. Два велосипедиста одновременно отправились в 88-километровый пробег. Первый ехал со скоростью на 3 км/ч большей, чем скорость второго, и прибыл к финишу на 3 часа раньше второго. Найти скорость велосипедиста, пришедшего к финишу вторым. Ответ дайте в км/ч.\\
69. Два велосипедиста одновременно отправились в 96-километровый пробег. Первый ехал со скоростью на 4 км/ч большей, чем скорость второго, и прибыл к финишу на 4 часа раньше второго. Найти скорость велосипедиста, пришедшего к финишу вторым. Ответ дайте в км/ч.\\
70. При температуре $0^\circ C$ рельс имеет длину $l_0=10$м. При возрастании температуры происходит тепловое расширение рельса, и его длина, выраженная в метрах, меняется по закону:
$$l(t^\circ)=l_0(1+\alpha\cdot t^\circ),$$
где $\alpha=1,2\cdot10^{-5}(^\circ C)^{-1}$ --- коэффициент теплового расширения, $t^\circ$ --- температура (в градусах Цельсия). При какой температуре рельс удлинится на 3 мм? Ответ выразите в градусах Цельсия.\\
71. Морская вода содержит $5\%$ соли (по весу). Сколько кг пресной воды нужно прибавить к 40 кг морской воды, чтобы содержание соли составило $2\%?$\\
72. Смешали какое-то количество 40-процентного и 75-процентного раствора кислоты, в результате чего получился 61-процентный раствор кислоты. Если бы каждого раствора взяли на 40 литров меньше, то получился бы 65-процентный раствор. Сколько литров каждого раствора было взято первоначально?\\
73. Смешали какое-то количество 30-процентного и 65-процентного раствора кислоты, в результате чего получился 55-процентный раствор кислоты. Если бы каждого раствора взяли на 20 литров больше, то получился бы 51-процентный раствор. Сколько литров каждого раствора было взято первоначально?\\
74. Двое бегают вокруг озера. Скорость каждого постоянна и на один круг один из них тратит на 7 минут меньше другого. Если они начинают бежать с общего старта одновременно и в одном направлении, то впервые встретятся через 1 ч 24 мин. Через какое время они впервые встретятся, если побегут одновременно с общего старта в противоположных направлениях?\\
75. Двое бегают вокруг озера. Скорость каждого постоянна и на один круг один из них тратит на 9 минут меньше другого. Если они начинают бежать с общего старта одновременно в противоположных направлениях, то впервые встретятся через 20 мин. Через какое время они впервые встретятся, если побегут одновременно с общего старта в одном направлении?\\
76. Первые два часа автомобиль ехал со скоростью 50 км/ч, следующий час --- со скоростью 100 км/ч, а затем два часа --- со скоростью 75 км/ч. Найдите среднюю скорость автомобиля на протяжении всего пути. Ответ дайте в км/ч.\\
77. Первые 190 км автомобиль ехал со скоростью 50 км/ч, следующие 180 км --- со скоростью 90 км/ч, а затем 170 км --- со скоростью 100 км/ч. Найдите среднюю скорость автомобиля на протяжении всего пути. Ответ дайте в км/ч.\\
78. Имеются два раствора серной кислоты в воде: первый --- $40\%,$ а второй --- $60\%.$ Эти растворы смешали, после чего добавили 5 кг чистой воды и получили $20\%$-й раствор. Если бы вместо 5 кг чистой воды добавили 5 кг $80\%$-го раствора, то получился бы $70\%$-й раствор. Сколько было $40\%$-го и $60\%$-го растворов?\\
79. Смешав $30\%$ и $60\%$ растворы кислоты и добавив 10 кг чистой воды, получили $36\%$ раствор кислоты. Если бы вместо 10 кг воды добавили 10 кг $50\%$ раствора той же кислоты, то получили бы $41\%$ раствор кислоты. Сколько килограммов $30\%$ и $60\%$ растворов использовали для получения смеси?\\
80. Водный раствор соли содержал 60 г чистой воды и $x\%$ соли. После того как в раствор добавили 20 г чистой воды, масса соли стала составлять $(x-5)\%$ массы раствора. Сколько граммов соли содержит раствор?\\
81. Из города $A$ в город $B,$ находящийся на расстоянии 20 километров от него, выехал трактор,
а через 5 минут ему навстречу выехал второй трактор, скорость которого была на 5 км/ч больше. Второй трактор приехал в $A$
на 3 минуты раньше, чем первый в $B.$ Найдите скорость второго трактора.\\
82. Из города $A$ в город $B,$ находящийся на расстоянии 15 километров от него, выехал трактор,
а через 5 минут ему навстречу выехал второй трактор, скорость которого была на 5 км/ч больше. Второй трактор приехал в $A$
на 4 минуты раньше, чем первый в $B.$ Найдите скорость второго трактора.\\
83. Цену на товар подняли на $30\%,$ а зарплату на $4\%.$ На сколько процентов меньше товара теперь удастся купить на зарплату?\\
84. Цену на товар подняли на $70\%,$ а зарплату на $2\%.$ На сколько процентов меньше товара теперь удастся купить на зарплату?\\
85. Бак наполнен спиртом. Из бака вылили часть спирта и
дополнили его водой. Потом из бака вылили столько же
литров смеси. В баке осталось 49 литров спирта. Сколько
литров спирта вылили в первый раз и сколько во второй
раз, если вместимоcть бака 64 литра?\\
86. Имеется два тридцатилитровых сосуда, в которых
содержится всего 30 л спирта. Первый сосуд доливают
доверху водой и полученной смесью дополняют второй
сосуд, из которого затем переливают 12 л новой смеси в
первый. Сколько литров спирта было сначала в каждом
сосуде, если во втором оказалось на 2 литра спирта
меньше, чем в первом.\\
87. Из пунктов $A$ и $B,$ расстояние между которыми равно 19
км, вышли одновременно навстречу друг другу два
пешехода и встретились в 9 км от $A.$ Найдите скорость
пешехода, шедшего из $A,$ если он шёл со скоростью на
1 км/ч большей, чем пешеход, шедший из $B,$ и сделал в
пути получасовую остановку.\\
88. Из пунктов $A$ и $B,$ расстояние между которыми равно 27
км, вышли одновременно навстречу друг другу два
пешехода и встретились в 15 км от $A.$ Найдите скорость
пешехода, шедшего из $A,$ если он шел со скоростью на
2 км/ч большей, чем пешеход, шедший из $B,$ и сделал в
пути получасовую остановку.
\newpage
