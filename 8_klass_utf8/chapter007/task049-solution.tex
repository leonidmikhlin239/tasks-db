49. Пусть концентрация соли во втором растворе выражается десятичной дробью $x.$ Тогда $30\cdot0,2+10\cdot x=(30+10)\cdot0,25,\ 6+10x=10,\ x=0,4.$ Значит, содержание соли во втором растворе составляет $40\%.$\\