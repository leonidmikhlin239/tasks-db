135. По теореме Виета имеем соотношения $x_1+x_2=-\cfrac{2}{3},\ x_1x_2=-3.$ Пусть $y_1=\cfrac{2-x_1}{x_2},\ y_2=\cfrac{2-x_2}{x_1},$ тогда выполняются равенства
$y_1+y_2=\cfrac{2-x_1}{x_2}+\cfrac{2-x_2}{x_1}=\cfrac{2x_1-x_1^2+2x_2-x_2^2}{x_1x_2}=$\\$
\cfrac{2(x_1+x_2)-((x_2+x_2)^2-2x_1x_2)}{x_1x_2}=\cfrac{-\cfrac{4}{3}-\left(\cfrac{4}{9}+6\right)}{-3}=\cfrac{70}{27},\
y_1y_2=\cfrac{(2-x_1)(2-x_2)}{x_1x_2}=$\\$\cfrac{4-2(x_1+x_2)+x_1x_2}{x_1x_2}=\cfrac{4+\cfrac{4}{3}-3}{-3}=-\cfrac{7}{9}=-\cfrac{21}{27}.$ По обратной теореме Виета числа $y_1$ и $y_2$ являются корнями квадратного уравнения $27x^2-70x-21=0.$\\
