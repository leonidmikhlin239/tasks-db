9. $y=x^2-7x+6=\left(x-\cfrac{7}{2}\right)^2-\cfrac{49}{4}+6=\left(x-\cfrac{7}{2}\right)^2-\cfrac{25}{4}.$ Наименьшее значение функции достигается, когда квадрат равен 0, а наибольшее --- когда квадрат имеет наибольший модуль (в том из концов отрезка, который дальше от вершины). Значит, наименьшее значение равно $-\cfrac{25}{4},$ а наибольшее ---
$\left(8-\cfrac{7}{2}\right)^2-\cfrac{25}{4}$ или $\left(-1-\cfrac{7}{2}\right)^2-\cfrac{25}{4}.$ Оба потенциально наибольших значения равны 14.\newpage\noindent
