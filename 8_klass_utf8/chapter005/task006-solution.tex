6. Пусть $f(x)=ax^2+bx+c.$ Тогда $c\cdot(a-b+c)=f(0)\cdot f(-1)<0.$ Значит, на концах отрезка $[-1;0]$ функция $f(x)$ принимает значения разных знаков, поэтому где-то на интервале $(-1;0)$ её значение равно 0. Иметь только один корень эта функция не может, так как тогда все остальные значения были бы одного знака, что невозможно. Значит, квадратный трёхчлен $ax^2+bx+c$ имеет два корня.\\
