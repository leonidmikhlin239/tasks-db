23. Пусть $x_1=2x_2,$ тогда $x_1+x_2=3x_2=k+4$ по теореме Виета. Значит, $x_2=\cfrac{k+4}{3}$ является корнем уравнения, подставим его:
$\left(\cfrac{k+4}{3}\right)^2-\cfrac{(k+4)^2}{3}+2k+4=0,\ \cfrac{k^2+8k+16}{9}-\cfrac{(k+4)^2}{3}+2k+4=0,\ k^2+8k+16-3k^2-24k-48+18k+36=0,\
2k^2-2k+4=0,\ k^2-k+2=0,\ (k+1)(k-2)=0,\ k=-1$ или $k=2.$ Рассмотренное нами условие являлось необходимым, но не достаточным, так что найденные значения $k$ необходимо проверить. При $k=-1:\ x^2-3x+2=0,$ корни $x=2$ и $x=1,$ значит $k=-1$ подходит. При $k=2:\ x^2-6x+8=0,$ корни $x=2$ и $x=4,$ значит $k=2$ также подходит.\\
