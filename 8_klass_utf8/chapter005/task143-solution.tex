144. $\cfrac{ax^2-(3a+2)x+8}{x^2-x-2}=0\Leftrightarrow \begin{cases}
ax^2-(3a+2)x+8=0,\\ (x-2)(x+1)\neq0.\end{cases}\Leftrightarrow \begin{cases}
ax^2-(3a+2)x+8=0,\\ x\notin\{-1;2\}.\end{cases}$ У этого уравнения может быть один корень либо если у квадратного уравнения он один и не совпадает с числами $-1$ и 2, либо если у квадратного уравнения два корня, но один из них совпадает с числом $-1$ или 2. В первом случае равен нулю может быть либо коэффициент при $x^2,$ либо дискриминант. Если $a=0,$ то $-2x+8=0,\ x=4,$ этот случай подходит. Если $(3a+2)^2-4\cdot a\cdot8=0,\ 9a^2+12a+4-32a=0,\
9a^2-20a+4=0,\ a=2$ или $a=\cfrac{2}{9}.$ Если $a=2,$ то $x=\cfrac{3\cdot2+2}{2\cdot2}=2,$ а значит у этого уравнения нет корней вообще. Если $a=\cfrac{2}{9},$ то $x=\cfrac{3\cdot\cfrac{2}{9}+2}{2\cdot\cfrac{2}{9}}=6,$ значит этот случай подходит. Случай, когда один из корней совпал с числом 2, мы уже разобрали. Если один из корней совпал с числом $-1,$ то $a+3a+2+8=0,\ 4a=-10,\ a=-\cfrac{5}{2},$ этот случай также подходит.\\
