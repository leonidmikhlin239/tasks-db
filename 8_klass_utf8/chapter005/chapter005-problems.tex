\section{Исследование функций и уравнений задачи}
1. В уравнении $x^2-4x+a=0$ сумма квадратов корней уравнения равна 16. Найти $a.$\\
2. В уравнении $x^2-2x+a=0$ квадрат разности корней уравнения равен 16. Найти $a.$\\
3. Найти область определения функции $f(x)=\sqrt{\cfrac{|x-3|(x+4)(x^2+9x+20)}{x^2-x-6}}.$\\
4. Найти область определения функции $f(x)=\sqrt{\cfrac{|x-1|(x+3)(x^2+8x+15)}{x^2+x-2}}.$\\
5. Доказать, что если $c\cdot(a+b+c)<0,$ то квадратный трёхчлен $ax^2+bx+c$ имеет корни.\\
6. Доказать, что если $c\cdot(a-b+c)<0,$ то квадратный трёхчлен $ax^2+bx+c$ имеет корни.\\
7. При каких $k$ уравнение $(k-2)x^2+2(k-1)x+k=0$ имеет единственный корень?\\
8. При каких $k$ уравнение $(k+2)x^2+2(k+1)x+k=0$ имеет единственный корень?\\
9. Найдите наименьшее и наибольшее значение функции на данном промежутке: $y=x^2-7x+6,$ \\
$-1\le x\le 8.$\\
10. Найти область определения функции $y=\sqrt{\cfrac{(x-1)^2(2x-x^2+2)}{(x^2+x-6)|x+2|}}.$\\
11. Найдите наименьшее значение выражения $x^2-2xy+8y^2,$
если $x+2y=4.$\\
12. Найдите наименьшее значение выражения $x^2+2xy+8y^2,$
если $x-2y=4.$\\
13. При каких значениях $a$ уравнение $(x-3)(x-a)(x-2a)=0$ имеет ровно два различных корня?\\
14. При каких значениях $a$ уравнение $(x-5)(x-a)(x-2a)=0$ имеет ровно два различных корня?\\
15. При каких значениях $q$ сумма квадратов корней уравнения $x^2+10x+q=0$ равна 2?\\
16. При каких значениях $q$ сумма квадратов корней уравнения $x^2-10x+q=0$ равна 2?\\
17. Решить относительно $x$ уравнение $(k-1)x^2-(2k-1)x+2=0.$\\
18. Решить относительно $x$ уравнение $(k-1)x^2-(2k-3)x-2=0.$\\
19. Не решая уравнения $2x^2-3x-11=0$ найти $\cfrac{x_1}{x_2}+\cfrac{x_2}{x_1},$ где $x_1,\ x_2$ --- корни этого уравнения.\\
20. Не решая уравнения $2x^2+3x-11=0$ найти $\cfrac{x_1}{x_2}+\cfrac{x_2}{x_1},$ где $x_1,\ x_2$ --- корни этого уравнения.\\
21. Найдите наибольшее значение, которое может принимать $xy,$ если $2x+3y=6.$\\
22. Найдите наибольшее значение, которое может принимать $xy,$ если $3x+2y=6.$\\
23. При каких $k$ один из корней уравнения $x^2-(k+4)x+2k+4=0$ в два раза больше другого?\\
24. При каких $k$ один из корней уравнения $x^2-(k+5)x+2k+6=0$ в два раза больше другого?\\
25. При каких $a$ уравнение $\cfrac{x^2-4ax+3a^2}{x-3}=0$ имеет ровно один корень?\\
26. При каких $a$ уравнение $\cfrac{x^2-5ax+4a^2}{x-4}=0$ имеет ровно один корень?\\
27. Найти те значения $a,$ при которых сумма квадратов корней уравнения $x^2-ax+20=0$ равна 24.\\
28. Найти те значения $b,$ при которых сумма квадратов корней уравнения $x^2-bx+10=0$ равна 16.\\
29. Определить, при каких $x$ и $y$ выражение $5x^2-4x+y^2+2xy+1$ принимает наименьшее значение.\\
30. Определить, при каких $x$ и $y$ выражение $5y^2+4y+x^2-2xy+1$ принимает наименьшее значение.\\
31. При каких значениях $q$ сумма квадратов корней уравнения $x^2-qx+4=0$ равна 17?\\
32. При каких значениях $q$ сумма квадратов корней уравнения $x^2-qx+3=0$ равна 10?\\
33. При каких значениях $t$ уравнение $\cfrac{(x-t)(x-2)}{x-2t}=0$ имеет ровно два разных корня?\\
34. При каких значениях $t$ уравнение $\cfrac{(x-t)(x-4)}{x-4t}=0$ имеет ровно два разных корня?\\
35. При каких значениях $a$ уравнение $(a+1)x^2+2x-a+1=0$ имеет ровно один корень?\\
36. При каких значениях $a$ уравнение $(1-a)x^2-2x+a+1=0$ имеет ровно один корень?\\
37. Вычислить $3x^2-2x-1$ при $x=\cfrac{1-\sqrt{2}}{3}.$\\
38. Вычислить $3x^2+2x-1$ при $x=\cfrac{\sqrt{2}-1}{3}.$\\
39. При каких значениях $x$ выполняется равенство $\sqrt{x^2-4x+3}=\sqrt{1-x}\cdot\sqrt{3-x}?$\\
40. При каких значениях $x$ выполняется равенство $\sqrt{x^2-6x+5}=\sqrt{1-x}\cdot\sqrt{5-x}?$\\
41. Найдите наименьшее значение выражения $2x^2+4xy+4y^2+3.$\\
42. Найдите наименьшее значение выражения $4x^2-4xy+2y^2+5.$\\
43. При каких значениях $t$ неравенство $\cfrac{\sqrt{x-t}}{(x-2)(x-3)}<0$ не имеет корней?\\
44. При каких значениях $t$ неравенство $\cfrac{\sqrt{x-t}}{(x-3)(x-4)}<0$ не имеет корней?\\
45. При каких значениях $a$ уравнение $(a-1)x^2+2ax+4=0$ имеет два различных корня?\\
46. При каких значениях $a$ уравнение $(a+1)x^2-2ax-4=0$ имеет два различных корня?\\
47. Определите, при каких значениях параметра $a$ уравнение $ax^2-(2a+6)x+3a+3=0$ имеет единственное решение.\\
48. Определите, при каких значениях параметра $k$ уравнение $kx^2-2(k+1)x+k+3=0$ имеет единственное решение.\\
49. Пусть $x_1$ и $x_2$ --- корни уравнения $3x^2-5x-11=0.$ Составить квадратное уравнение с корнями $\cfrac{1}{x_1}$ и $\cfrac{1}{x_2}.$\\
50. Пусть $x_1$ и $x_2$ --- корни уравнения $5x^2-3x-11=0.$ Составить квадратное уравнение с корнями $\cfrac{1}{x_1}$ и $\cfrac{1}{x_2}.$\\
51. Найдите произведение вещественных корней уравнения $x^4-3x^2-4=0.$\\
52. Найдите произведение вещественных корней уравнения $x^4+3x^2-4=0.$\\
53. При каких значениях $a$ уравнение $ax^2-4x+a=0$ имеет два различных корня?\\
54. При каких значениях $a$ уравнение $ax^2+4x+a=0$ имеет два различных корня?\\
55. При каких значениях $t$ уравнение $\cfrac{(x-t)(x-2t)}{x-1}=0$ имеет два различных корня?\\
56. При каких значениях $t$ уравнение $\cfrac{(x-t)(x-2t)}{x-2}=0$ имеет два различных корня?\\
57. Пусть $x_1$ --- меньший корень уравнения $2x^2+4x-5=0.$ Вычислите $4x_1^2+8x_1-7.$\\
58. Пусть $x_1$ --- меньший корень уравнения $3x^2+2x-7=0.$ Вычислите $6x_1^2+4x_1-11.$\\
59. Если $x^2-12x+15=(x+a)^2+b,$ то чему равно значение $b?$\\
60. Если $x^2-8x-3=(x+a)^2+b,$ то чему равно значение $b?$\\
61. Пусть $x_1$ и $x_2$ --- корни уравнения $x^2+px+6=0.$ Решить уравнение, если $x_1^2+x_2^2=37.$\\
62. Пусть $x_1$ и $x_2$ --- корни уравнения $x^2-px+10=0.$ Решить уравнение, если $x_1^2+x_2^2=29.$\\
63. При каких значениях $a$ уравнение $x^3+6x^2+ax=0$ имеет два различных корня?\\
64. При каких значениях $a$ уравнение $4x^3+4x^2+ax=0$ имеет два различных корня?\\
65. При каких значениях $a$ система $\begin{cases} y=x^2-4x+3,\\ y=a.\end{cases}$ имеет единственное решение?\\
66. При каких значениях $a$ система $\begin{cases} y=x^2+4x+3,\\ y=a.\end{cases}$ имеет единственное решение?\\
67. Найти сумму квадратов корней квадратного уравнения $x^2+9x+20=0.$\\
68. Найдите наименьшее значение выражения $x^2-2xy+8y^2,$ если $x-2y=4.$\\
69. Найдите область определения функции $f(x)=\sqrt{\cfrac{x^2+8x+15}{x-2}}.$\\
70. Найти область определения функции $\cfrac{\sqrt{x^2-6x+8}}{x-5}.$\\
71. Найти область определения функции $\cfrac{\sqrt{x^2-7x+12}}{x-6}.$\\
72. При каких значениях $k$ уравнение $7x^2-2x+4k=0$ имеет только положительные корни?\\
73. При каких значениях $k$ уравнение $3x^2-2x+9k=0$ имеет только положительные корни?\\
74. Какие значения может принимать $y,$ если $3x+2y=6$ и $|x|<8?$\\
75. Какие значения может принимать $x,$ если $4x+3y=8$ и $|y|<12?$\\
76. При каких значениях параметра $a$ число 1 расположено между корнями уравнения \\$x^2+(a+1)x-a^2=0?$\\
77. При каких значениях параметра $a$ число 1 расположено между корнями уравнения \\$x^2+(1-a)x-a^2=0?$\\
78. Решить уравнение $x^2-15x+q=0,$ если известно, что его корни $x_1,\ x_2$ связаны соотношением $\cfrac{1}{x_1}+\cfrac{1}{x_2}=\cfrac{5}{12}.$\\
79. Решить уравнение $x^2+px+36=0,$ если известно, что его корни $x_1,\ x_2$ связаны соотношением $\cfrac{1}{x_1}+\cfrac{1}{x_2}=\cfrac{5}{12}.$\\
80. При каких значениях параметра $a$ уравнение $(a^2-1)x^2-2(a^2-2a+1)x+4a-4=0$ имеет более 2 корней?\\
81. При каких значениях параметра $a$ уравнение $(a^2-a)x^2-2(a^2-3a+2)x-3a+3=0$ имеет более 2 корней?\\
82. Не решая уравнение $x^2-4x-1=0,$ найдите сумму кубов его корней.\\
83. Не решая уравнение $x^2-3x-2=0,$ найдите сумму кубов его корней.\\
84. Решите уравнение $x^3-x^2+bx+24=0,$ если известно, что один из его корней равен $3.$\\
85. Решите уравнение $x^3+x^2+bx-24=0,$ если известно, что один из его корней равен $-2.$\\
86. При каких значениях $x$ и $y$ выражение $6-2x^2-2xy-6x-y^2$ принимает наибольшее значение?\\
87. При каких значениях $x$ и $y$ выражение $20-2x^2+2xy-4x-y^2$ принимает наибольшее значение?\\
88. При каких $a$ уравнение $ax^2-2(a-2)x+a+1=0$ имеет ровно один корень?\\
89. При каких $a$ уравнение $ax^2+4(a-1)x+4a-3=0$ имеет ровно один корень?\\
90. Найдите все значения параметра $a,$ при которых сумма корней уравнения $x^2-(a^2-5a)x+4a^2=0$ будет отрицательной.\\
91. Найдите все значения параметра $a,$ при которых сумма корней уравнения $x^2+(a^2-5a)x+4a^2=0$ будет положительной.\\
92. При каких значениях параметра $a$ уравнение $(x-5)(2x-a)=x-5$ имеет ровно один корень?\\
93. При каких значениях параметра $a$ уравнение $(x-3)(2x-a)=x-3$ имеет ровно один корень?\\
94. При каких положительных значениях параметра $a$ уравнение $ax=|x-2|$ имеет единственное решение?\\
95. Найти область определения функции $f(x)=\sqrt{24-x^2}+\cfrac{1}{\sqrt{x^2+x-20}}.$\\
96. При каких значениях параметра $a$ уравнение $ax=|x+2|$ имеет единственное решение?\\
97. Найдите все значения параметра $k,$ при которых следующая система имеет бесконечно много решений:
$$\begin{cases}
(k+2)x+3y=9+kx,\\
x+(k+4)y=2.
\end{cases}$$\\
98. Найдите все значения параметра $a,$ при которых следующее уравнение имеет ровно одно решение: $(ax^2+3x+1)(x-3)=(x-3).$\\
99. Найдите все значения параметра $a,$ при которых сумма корней следующего уравнения отрицательна: $x^2-(a^2-5a)x+4=0.$\\
100. Пусть $x_1,\ x_2$ --- корни уравнения $2x^2-7x+4=0.$ Не решая это уравнение, найдите $\cfrac{x_1^2}{x_2}+\cfrac{x_2^2}{x_1}.$\\
101. Найдите все значения параметра $a,$ при которых уравнение $7x^2-2x+4a=0$\\
а) имеет корень, равный 3.\\
б) имеет два различных вещественных корня.\\
в) имеет только положительные корни.\\
г) не имеет отрицательных корней.\\
102. а) При каких значениях параметра $a$ уравнение $x^2-ax+a-1=0$ имеет два различных корня?\\
б) При каких значениях параметра $a$ уравнение $\cfrac{x^2-ax+a-1}{x+5}=0$ имеет единственное решение?\\
103. а) При каких значениях параметра $a$ уравнение $x^2-ax+3a-9=0$ имеет два различных корня?\\
б) При каких значениях параметра $a$ уравнение $\cfrac{x^2-ax+3a-9}{x-4}=0$ имеет единственное решение?\\
104. Найдите все значения параметра $a,$ при каждом из которых уравнение
$(a - 2)x^2 - 2(a - 2)x + 3 = 0$ имеет единственный корень.\\
105. Найдите все значения параметра $a,$ при каждом из которых уравнение
$(a + 3)x^2 - 2(a + 3)x - 5 = 0$ имеет единственный корень.\\
106. $f(x)=ax+b,\ ab
eq 0.$ График $f(x)$ проходит через I, II и IV четверти. Определите знаки $a,b.$\\
107. $f(x)=ax+b,\ ab
eq 0.$ График $f(x)$ проходит через I, III и IV четверти. Определите знаки $a,b.$\\
108. Решением неравенства $ax^2 + bx + c > 0$ является промежуток $(x_1; x_2),$ причём $x_1\cdot x_2 <  0.$ Определите знаки $a$ и $c.$\\
109. Решением неравенства $ax^2 + bx + c < 0$ является промежуток $(x_1; x_2),$ причём $x_1\cdot x_2 <  0.$ Определите знаки $a$ и $c.$\\
110. Найдите наибольшее значение выражения и определите, при каких значениях $x$ и $y$ оно достигается: $\cfrac{10}{x^2+y^2+4x-6y+14}.$\\
111. Найдите наибольшее значение выражения и определите, при каких значениях $x$ и $y$ оно достигается: $\cfrac{8}{x^2+y^2-2x-10y+30}.$\\
112. Найдите область определения функции: $f(x)=\sqrt{\cfrac{1-2x}{3x+5}+2}.$\\
113. Найдите область определения функции: $f(x)=\sqrt{\cfrac{5+6x}{3x+4}-1}.$\\
114. При каком $a$ квадрат разности корней квадратного уравнения квадратного уравнения\\ $x^2-3x+a=0$ равен 25?\\
115. Найдите наименьшее значение выражения $\sqrt{2x+2y+10}+\sqrt{x+3y-3}$ и укажите пары значений $x$ и $y,$ при которых оно достигается.\\
116. Найдите наименьшее значение выражения $\sqrt{2x-2y+10}+\sqrt{x+3y-3}$ и укажите пары значений $x$ и $y,$ при которых оно достигается.\\
117. Дано уравнение $x^2-2x-1=0.$ Не решая его найдите сумму квадратов его корней.\\
118. Решить уравнение $x^2-15x+q=0,$ если известно, что сумма квадратов его корней равна 125.\\
119. Решить уравнение $x^2-5x+q=0,$ если известно, что сумма квадратов его корней равна 125.\\
120. $x > 0,\ y > 0,\ x+y=6.$ Найдите наибольшее значение произведения $xy.$\\
121. $x > 0,\ y > 0,\ x+y=8.$ Найдите наибольшее значение произведения $xy.$\\
122. При каких значениях $a$ уравнение $ax^2+2(1-a)x-4=0$ имеет единственное решение?\\
123. При каких значениях $a$ уравнение $ax^2-2(1+a)x+4=0$ имеет единственное решение?\\
124. При каких значениях $a$ уравнение $\cfrac{(x-a)(x-2a)}{x-3a}=0$ не имеет решений?\\
125. При каких значениях $a$ уравнение $\cfrac{(x-a)(x-4a)}{x-2a}=0$ не имеет решений?\\
126. При каких значениях параметра $a$ неравенство $2|x-1|\geqslant a-x$ выполнено при всех значениях $x?$\\
127. При каких значениях параметра $a$ неравенство $2|x+1|\geqslant x-a$ выполнено при всех значениях $x?$\\
128. Найдите наименьшее и наибольшее значение выражения $|2a-b|,$ если $-2\leqslant a \leqslant 1$ и $1\leqslant b \leqslant 5.$\\
129. Найдите наименьшее и наибольшее значение выражения $|b-2a|,$ если $-2\leqslant a \leqslant 1$ и $1\leqslant b \leqslant 5.$\\
130. Найдите наименьшее значение выражения $x^2+y^2,$ если $x-y=2.$\\
131. Найдите наименьшее значение выражения $x^2+y^2,$ если $y-x=2.$\\
132. При каких значениях $k$ прямая $y=kx-k$ и парабола $y=-x^2$ не имеют общих точек?\\
133. Найдите все значения параметра $p,$ при каждом из которых уравнение\\ $\cfrac{(2p-3)x^2-(3p+2)x+p-1}{x-2}=0$ имеет ровно один корень.\\
134. Найдите все значения параметра $q,$ при каждом из которых уравнение\\ $\cfrac{(2q+1)x^2+(3q-2)x+q+2}{x+3}=0$ имеет ровно один корень.\\
135. Составьте квадратное уравнение с целыми коэффициентами, имеющее корни $\cfrac{2-x_1}{x_2}$ и $\cfrac{2-x_2}{x_1},$ где $x_1$ и $x_2$ --- корни уравнения $3x^2+2x-9=0.$\\
136. Составьте квадратное уравнение с целыми коэффициентами, имеющее корни $\cfrac{x_1-1}{x_2}$ и $\cfrac{x_2-1}{x_1},$ где $x_1$ и $x_2$ --- корни уравнения $2x^2-3x-8=0.$\\
137. Найдите значение выражения $\cfrac{3x+2y+z}{2x-3y+z},$ если $x:y:z=2:1:3.$\\
138. Найдите значение выражения $\cfrac{2y-4x+z}{5x-3y+8z},$ если $x:y:z=3:1:2.$\\
139. Не вычисляя корней уравнения $3x^2+8x-1=0,$ найдите: $x_1x_2^3+x_2x_1^3.$\\
140. Не вычисляя корней уравнения $2x^2-7x-3=0,$ найдите: $x_1x_2^3+x_2x_1^3.$\\
141. При каких значениях параметра $a$ уравнение $ax^2-(a+3)x-1=0$ имеет единственный корень?\\
142. Числа $x_1$ и $x_2$ --- корни уравнения $5x^2-3x-11=0.$ Вычислите $x_1^3x_2+x_2^3x_1.$\\
143. Числа $x_1$ и $x_2$ --- корни уравнения $5x^2+3x-13=0.$ Вычислите $x_1^3x_2+x_2^3x_1.$\\
144. При каких значениях параметра $a$ уравнение $\cfrac{ax^2-(3a+2)x+8}{x^2-x-2}=0$ имеет единственный корень?\\
145. При каких значениях параметра $a$ уравнение $\cfrac{ax^2-(5a+2)x+18}{x^2-x-6}=0$ имеет единственный корень?\\
146. Выяснить, существует ли такое $a,$ при котором произведение корней уравнения $x^2+2ax+a+3=0$ равно минус двум.\\
147. Выяснить, существует ли такое $a,$ при котором произведение корней уравнения $x^2+2ax+a+3=0$ равно двум.

ewpage
