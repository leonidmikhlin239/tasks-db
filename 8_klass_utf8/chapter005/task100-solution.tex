101. а) Подставим $x=3:\ 7\cdot9-2\cdot3+4a=0,\ a=-\cfrac{57}{4}.$\\
б) Для этого должно выполняться неравенство $D=4-4\cdot7\cdot4a>0,\ 28a<1,\ a\in\left(-\infty;\cfrac{1}{28}\right).$\\
в) Сумма корней по теореме Виета равна $\cfrac{2}{7}>0,$ также положительно должно быть их произведение $\cfrac{4a}{7}>0,\ a>0.$ При этом корень может быть и один, поэтому в отличие от пункта б) $a=\cfrac{1}{28}$ также подходит. Таким образом, ответ $a\in\left(0;\cfrac{1}{28}\right].$\\
г) При $a<0$ согласно пункту а) корни есть, при этом их произведение отрицательно, значит среди них точно есть отрицательный. При $a=0$ корни $x=0$ и $x=\cfrac{2}{7},$ отрицательных нет. При $a\in\left(0;\cfrac{1}{28}\right]$ согласно пункту в) оба корня положительны, а при $a>\cfrac{1}{28}$ корней нет вообще, а значит нет и отрицательных. Таким образом, ответ $a\in[0;+\infty).$\\
