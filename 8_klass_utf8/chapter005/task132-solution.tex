133. У этого уравнения может быть только один корень в двух случаях. Во-первых, у числителя может быть один корень, не равный 2. Для этого уравнение $(2p-3)x^2-(3p+2)x+p-1=0$ должно быть линейным или его дискриминант должен быть равен нулю. Уравнение линейно при $2p-3=0,\ p=\cfrac{3}{2},$ тогда $\cfrac{13}{2}x+\cfrac{1}{2}=0,\ x=-13,$ это значение $p$ подходит. Дискриминант равен нулю при $(3p+2)^2-4(2p-3)(p-1)=0,\ p^2+32p-8=0,\ p=-16\pm2\sqrt{66}.$ При этих значениях $p$ единственный корень также не равен 2. Во втором случае, подставив 2 в числитель и приравняв его к нулю, получим $8p-12-6p-4+p-1=0,\ 3p-17=0,\ p=\cfrac{17}{3}.$ Все найденные значения $p$ подходят, таким образом $p\in\left\{-16\pm2\sqrt{66};\cfrac{3}{2};\cfrac{17}{3}\right\}.$\\
