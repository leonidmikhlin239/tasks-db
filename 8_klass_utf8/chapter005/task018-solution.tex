18. Сначала разберём случай, когда уравнение не является квадратным: $k-1=0,\ k=1,\ x-2=0,\ x=2.$ Теперь разберём случай, когда у уравнения один корень:
$(2k-3)^2+4\cdot2\cdot(k-1)=0,\ 4k^2-12k+9+8k-8=0,\ (2k-1)^2=0,\ k=\cfrac{1}{2}.$ В этом случае $x=\cfrac{2k-3}{2(k-1)}=\cfrac{-2}{-1}=2.$ Во всех остальных случаях $x=\cfrac{2k-3\pm|2k-1|}{2(k-1)}=\left[\begin{array}{l}\cfrac{2k-3+2k-1}{2(k-1)}=2,\\ \cfrac{2k-3-2k+1}{2(k-1)}=\cfrac{1}{1-k}.\end{array}\right.$\\
