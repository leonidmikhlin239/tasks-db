96. $ax=|x+2|\Leftrightarrow \begin{cases}\left[\begin{array}{l}x+2=ax,\\ x+2=-ax.\end{array}\right.\\ ax\geqslant0.\end{cases}\Leftrightarrow
\begin{cases}\left[\begin{array}{l}x(a-1)=2,\\ x(a+1)=-2.\end{array}\right.\\ ax\geqslant0.\end{cases}$
Во-первых, один корень у этого уравнения может быть из-за того, что при вычислении второго получается 0 в знаменателе, тогда $a=1$ или $a=-1.$ При $a=1$ имеем корень $x=-1,$ который не удовлетворяет условию $ax\geqslant0,$ а при $a=-1$ --- $x=-1$ этому условию удовлетворяет. Поэтому при $a=1$ корней нет вообще, а при $a=-1$ он как раз один. При положительных значениях $a$ корень $x=-\cfrac{2}{a+1}<0,$ а значит не подходит. Поэтому должен подходить второй корень, значит $a-1>0,\ a>1.$ При отрицательных значениях $a$ корень $x=\cfrac{2}{a-1}<0$ точно подходит, значит второй корень должен не подходить. Поэтому $a+1<0,\ a<-1.$ Также корни могут совпадать, $\cfrac{2}{a-1}=-\cfrac{2}{a+1},\ a-1=-a-1,\ a=0.$ Таким образом, один корень данное уравнение имеет при $a\in(-\infty;-1]\cup\{0\}\cup(1;+\infty).$\\