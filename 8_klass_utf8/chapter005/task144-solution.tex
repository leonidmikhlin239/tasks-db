145. $\cfrac{ax^2-(5a+2)x+18}{x^2-x-6}=0\Leftrightarrow \begin{cases}
ax^2-(5a+2)x+18=0,\\ (x-3)(x+2)\neq0.\end{cases}\Leftrightarrow \begin{cases}
ax^2-(5a+2)x+18=0,\\ x\notin\{-2;3\}.\end{cases}$ У этого уравнения может быть один корень либо если у квадратного уравнения он один и не совпадает с числами $-2$ и 3, либо если у квадратного уравнения два корня, но один из них совпадает с числом $-2$ или 3. В первом случае равен нулю может быть либо коэффициент при $x^2,$ либо дискриминант. Если $a=0,$ то $-2x+18=0,\ x=9,$ этот случай подходит. Если $(5a+2)^2-4\cdot a\cdot18=0,\ 25a^2+20a+4-72a=0,\
25a^2-52a+4=0,\ a=2$ или $a=\cfrac{2}{25}.$ Если $a=2,$ то $x=\cfrac{5\cdot2+2}{2\cdot2}=3,$ а значит у этого уравнения нет корней вообще. Если $a=\cfrac{2}{25},$ то $x=\cfrac{5\cdot\cfrac{2}{25}+2}{2\cdot\cfrac{2}{25}}=15,$ значит этот случай подходит. Случай, когда один из корней совпал с числом 3, мы уже разобрали. Если один из корней совпал с числом $-2,$ то $4a+10a+4+18=0,\ 14a=-22,\ a=-\cfrac{11}{7},$ этот случай также подходит.\\
