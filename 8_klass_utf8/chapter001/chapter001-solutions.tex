\section{Числовые выражения решения}
1. $\left(\cfrac{4}{3}\sqrt{3}+\sqrt{2}+\sqrt{3\frac{1}{3}}\right): \left(-\sqrt{\frac{1}{3}}\right)+\sqrt{2}(\sqrt{3}+\sqrt{5})+\cfrac{9-2\sqrt{3}}{3\sqrt{6}-2\sqrt{2}}=
\left(\cfrac{4}{3}\sqrt{3}+\sqrt{2}+\cfrac{\sqrt{10}}{\sqrt{3}}\right)\cdot(-\sqrt{3})+\sqrt{6}+\sqrt{10}+\cfrac{(9-2\sqrt{3})(3\sqrt{6}+2\sqrt{2})}{54-8}=
-4-\sqrt{6}-\sqrt{10}+\sqrt{6}+\sqrt{10}+\cfrac{27\sqrt{6}+18\sqrt{2}-18\sqrt{2}-4\sqrt{6}}{46}=\cfrac{\sqrt{6}}{2}-4.$\\
2. $\left(\cfrac{1}{4}+\sqrt{\frac{1}{5}}-2\sqrt{2}\right): \left(\cfrac{1}{2}\sqrt{\frac{1}{5}}\right)-\cfrac{1}{2}\sqrt{5}+4\sqrt{10}+\cfrac{2\sqrt{6}+3\sqrt{3}}{4+3\sqrt{2}}=
\left(\cfrac{1}{4}+\cfrac{1}{\sqrt{5}}-2\sqrt{2}\right)\cdot2\sqrt{5}-\cfrac{1}{2}\sqrt{5}+4\sqrt{10}+\cfrac{(2\sqrt{6}+3\sqrt{3})(3\sqrt{2}-4)}{18-16}=
\cfrac{1}{2}\sqrt{5}+2-4\sqrt{10}-\cfrac{1}{2}\sqrt{5}+4\sqrt{10}+\cfrac{12\sqrt{3}-8\sqrt{6}+9\sqrt{6}-12\sqrt{3}}{2}=
\cfrac{\sqrt{6}}{2}+2.$\\
3. $\sqrt{9+4\sqrt{5}}-\sqrt{6-2\sqrt{5}}=\sqrt{2^2+2\cdot2\cdot\sqrt{5}+(\sqrt{5})^2}-\sqrt{1^2-2\cdot1\cdot\sqrt{5}+(\sqrt{5})^2}=$\\$=
\sqrt{(2+\sqrt{5})^2}-\sqrt{(1-\sqrt{5})^2}=|2+\sqrt{5}|-|1-\sqrt{5}|=(2+\sqrt{5})-(\sqrt{5}-1)=3.$\\
4. $\sqrt{12+6\sqrt{3}}-\sqrt{4-2\sqrt{3}}=\sqrt{3^2+2\cdot3\cdot\sqrt{5}+(\sqrt{3})^2}-\sqrt{1^2-2\cdot1\cdot\sqrt{3}+(\sqrt{3})^2}=$\\$=
\sqrt{(3+\sqrt{3})^2}-\sqrt{(1-\sqrt{3})^2}=|3+\sqrt{3}|-|1-\sqrt{3}|=(3+\sqrt{3})-(\sqrt{3}-1)=4.$\\
5. $2\sqrt{8\frac{1}{2}}-\sqrt{306}+5\sqrt{1\frac{9}{25}}=2\sqrt{\frac{17}{2}}-3\sqrt{34}+5\sqrt{\frac{34}{25}}=
\sqrt{34}-3\sqrt{34}+\sqrt{34}=-\sqrt{34}.$\\
6. $4\sqrt{10\frac{1}{2}}+\sqrt{168}-15\sqrt{1\frac{17}{25}}=4\sqrt{\frac{21}{2}}+2\sqrt{42}-15\sqrt{\frac{42}{25}}=
2\sqrt{42}+2\sqrt{42}-3\sqrt{42}=\sqrt{42}.$\\
7. $\sqrt{7+4\sqrt{3}}+\sqrt{7-4\sqrt{3}}=\sqrt{2^2+2\cdot2\cdot\sqrt{3}+(\sqrt{3})^2}+\sqrt{2^2-2\cdot2\cdot\sqrt{3}+(\sqrt{3})^2}=
|2+\sqrt{3}|+|2-\sqrt{3}|=2+\sqrt{3}+2-\sqrt{3}=4.$\\
8. $\sqrt{8+2\sqrt{7}}+\sqrt{8-2\sqrt{7}}==\sqrt{1^2+2\cdot1\cdot\sqrt{7}+(\sqrt{7})^2}+\sqrt{1^2-2\cdot1\cdot\sqrt{7}+(\sqrt{7})^2}=
|1+\sqrt{7}|+|1-\sqrt{7}|=1+\sqrt{7}+\sqrt{7}-1=2\sqrt{7}.$\\
9. $\cfrac{1}{\sqrt{6-\sqrt{20}}+1}-\cfrac{1}{\sqrt{6+\sqrt{20}}-1}=\cfrac{1}{\sqrt{1^2-2\cdot1\cdot\sqrt{5}+(\sqrt{5})^2}+1}-\cfrac{1}{\sqrt{1^2+2\cdot1\cdot\sqrt{5}+(\sqrt{5})^2}-1}=
\cfrac{1}{\sqrt{(1-\sqrt{5})^2}+1}-\cfrac{1}{\sqrt{(1+\sqrt{5})^2}-1}=\cfrac{1}{\sqrt{5}-1+1}-\cfrac{1}{\sqrt{5}+1-1}=
\cfrac{1}{\sqrt{5}}-\cfrac{1}{\sqrt{5}}=0.$\\
10. $\cfrac{1}{\sqrt{4-\sqrt{12}}+1}-\cfrac{1}{\sqrt{4+\sqrt{12}}-1}=\cfrac{1}{\sqrt{1^2-2\cdot1\cdot\sqrt{3}+(\sqrt{3})^2}+1}-\cfrac{1}{\sqrt{1^2+2\cdot1\cdot\sqrt{3}+(\sqrt{3})^2}-1}=
\cfrac{1}{\sqrt{(1-\sqrt{3})^2}+1}-\cfrac{1}{\sqrt{(1+\sqrt{3})^2}-1}=\cfrac{1}{\sqrt{3}-1+1}-\cfrac{1}{\sqrt{3}+1-1}=
\cfrac{1}{\sqrt{3}}-\cfrac{1}{\sqrt{3}}=0.$\\
11. $\cfrac{\left(\cfrac{1}{4}-\cfrac{5}{24}\right)\cdot 8-\cfrac{1}{3}}{1,85-1,62:0,9}=
\cfrac{\cfrac{1}{24}\cdot 8-\cfrac{1}{3}}{1,85-1,62:0,9}=\cfrac{0}{1,85-1,62:0,9}=0.$\\
12. $\cfrac{\left(1\cfrac{2}{9}:7\cfrac{1}{3}-\cfrac{1}{6}\right)\cdot0,23}{2\cfrac{1}{8}+\cfrac{1}{2}}=
\cfrac{\left(\cfrac{11}{9}\cdot\cfrac{3}{22}-\cfrac{1}{6}\right)\cdot0,23}{2\cfrac{1}{8}+\cfrac{1}{2}}=
\cfrac{0\cdot0,23}{2\cfrac{1}{8}+\cfrac{1}{2}}=0.$\\
13. $\sqrt{28-10\sqrt{3}}+\sqrt{28+10\sqrt{3}}=\sqrt{5^2-2\cdot5\cdot\sqrt{3}+(\sqrt{3})^2}+\sqrt{5^2+2\cdot5\cdot\sqrt{3}+(\sqrt{3})^2}=\sqrt{(5-\sqrt{3})^2}+
\sqrt{(5+\sqrt{3})^2}=|5-\sqrt{3}|+|5+\sqrt{3}|=5-\sqrt{3}+5+\sqrt{3}=10.$\\
14. $\sqrt{14+6\sqrt{5}}+\sqrt{14-6\sqrt{5}}=\sqrt{3^2+2\cdot3\cdot\sqrt{5}+(\sqrt{5})^2}+\sqrt{3^2-2\cdot3\cdot\sqrt{5}+(\sqrt{5})^2}=\sqrt{(3+\sqrt{5})^2}+
\sqrt{(3-\sqrt{5})^2}=|3+\sqrt{5}|+|3-\sqrt{5}|=3+\sqrt{5}+3-\sqrt{5}=6.$\\
15. Пусть $3458\cfrac{239}{9876}=x,$ тогда искомое выражение равно $x(x-1)-(x+1)(x-2)=x^2-x-x^2+2x-x+2=2.$\\
16. Пусть $4357\cfrac{239}{9876}=x,$ тогда искомое выражение равно $(x+2)(x-1)-(x+1)x=x^2-x+2x-2-x^2-x=-2.$\\
17. $\sqrt{6+2\sqrt{5}}-\sqrt{6-2\sqrt{5}}=\sqrt{1^2+2\cdot1\cdot\sqrt{5}+(\sqrt{5})^2}-\sqrt{1^2-2\cdot1\cdot\sqrt{5}+(\sqrt{5})^2}=\sqrt{(1+\sqrt{5})^2}-
\sqrt{(1-\sqrt{5})^2}=|1+\sqrt{5}|-|1-\sqrt{5}|=1+\sqrt{5}-\sqrt{5}+1=2.$\\
18. $\sqrt{7+2\sqrt{6}}-\sqrt{7-2\sqrt{6}}=\sqrt{1^2+2\cdot1\cdot\sqrt{6}+(\sqrt{6})^2}-\sqrt{1^2-2\cdot1\cdot\sqrt{6}+(\sqrt{6})^2}=\sqrt{(1+\sqrt{6})^2}-
\sqrt{(1-\sqrt{6})^2}=|1+\sqrt{6}|-|1-\sqrt{6}|=1+\sqrt{6}-\sqrt{6}+1=2.$\\
19. $b=\cfrac{2}{\sqrt{7+3\sqrt{5}}}=\cfrac{2\sqrt{7-3\sqrt{5}}}{\sqrt{49-45}}=\sqrt{7-3\sqrt{5}}=a.$\\
20. $b=\cfrac{2}{\sqrt{8+2\sqrt{15}}}=\cfrac{2\sqrt{8-2\sqrt{15}}}{64-60}=\sqrt{8-2\sqrt{15}}=a.$\\
21. $\cfrac{\sqrt{2+\sqrt{3}}-\sqrt{2-\sqrt{3}}}{\sqrt{2}}=\cfrac{\sqrt{4+2\sqrt{3}}-\sqrt{4-2\sqrt{3}}}{2}=$\\$=
\cfrac{\sqrt{1^2+2\cdot1\cdot\sqrt{3}+(\sqrt{3})^2}-\sqrt{1^2-2\cdot1\cdot\sqrt{3}+(\sqrt{3})^2}}{2}=
\cfrac{\sqrt{(1+\sqrt{3})^2}-\sqrt{(1-\sqrt{3})^2}}{2}=$\\$=\cfrac{1+\sqrt{3}-\sqrt{3}+1}{2}=1.$\\
22. $\cfrac{\sqrt{3+\sqrt{5}}-\sqrt{3-\sqrt{5}}}{\sqrt{2}}=\cfrac{\sqrt{6+2\sqrt{5}}-\sqrt{6-2\sqrt{5}}}{2}=$\\$=
\cfrac{\sqrt{1^2+2\cdot1\cdot\sqrt{5}+(\sqrt{5})^2}-\sqrt{1^2-2\cdot1\cdot\sqrt{5}+(\sqrt{5})^2}}{2}=
\cfrac{\sqrt{(1+\sqrt{5})^2}-\sqrt{(1-\sqrt{5})^2}}{2}=$\\$=\cfrac{1+\sqrt{5}-\sqrt{5}+1}{2}=1.$\\
23. Пусть $135=x,$ тогда искомое выражение равно $(x-4)(x+4)-(x-2)(x+2)=x^2-16-x^2+4=-12.$\\
24. Пусть $145=x,$ тогда искомое выражение равно $(x-4)(x+4)-(x-2)(x+2)=x^2-16-x^2+4=-12.$\\
25. $\sqrt{4+2\sqrt{3}}-\sqrt{4-2\sqrt{3}}=\sqrt{1^2+2\cdot1\cdot\sqrt{3}+(\sqrt{3})^2}-\sqrt{1^2-2\cdot1\cdot\sqrt{3}+(\sqrt{3})^2}=
\sqrt{(1+\sqrt{3})^2}-\sqrt{(1-\sqrt{3})^2}=1+\sqrt{3}-\sqrt{3}+1=2.$\\
26. $\sqrt{16+2\sqrt{15}}-\sqrt{16-2\sqrt{15}}=\sqrt{1^2+2\cdot1\cdot\sqrt{15}+(\sqrt{15})^2}-\sqrt{1^2-2\cdot1\cdot\sqrt{15}+(\sqrt{15})^2}=$\\$
\sqrt{(1+\sqrt{15})^2}-\sqrt{(1-\sqrt{15})^2}=1+\sqrt{15}-\sqrt{15}+1=2.$\\
27. Пусть $5379=x,$ тогда искомое выражение равно $x^2-(x-1)(x+1)=x^2-x^2+1=1.$\\
28. Пусть $9551=x,$ тогда искомое выражение равно $(x+1)(x-1)-x^2=x^2-1-x^2=-1.$\\
29. $\sqrt{27+10\sqrt{2}}-\sqrt{2}=\sqrt{5^2+2\cdot5\cdot\sqrt{2}+(\sqrt{2})^2}-\sqrt{2}=\sqrt{(5+\sqrt{2})^2}-\sqrt{2}=
5+\sqrt{2}-\sqrt{2}=5.$\\
30. $\sqrt{21-8\sqrt{5}}+\sqrt{5}=\sqrt{4^2-2\cdot4\cdot\sqrt{2}+(\sqrt{5})^2}+\sqrt{5}=\sqrt{(4-\sqrt{5})^2}+\sqrt{5}=
4-\sqrt{5}+\sqrt{5}=4.$\\
31. $\cfrac{5,1}{0,017}+\cfrac{0,09}{0,003}+\cfrac{1}{0,1}=300+30+10=340.$\\
32. $\cfrac{2,4}{0,08}+\cfrac{0,21}{0,07}+\cfrac{4}{0,4}=30+3+10=43.$\\
33. $4+2\sqrt{2}\ ??\ \sqrt{11}+\sqrt{13},\ 16+16\sqrt{2}+8\ ??\ 11+2\sqrt{11}\cdot\sqrt{13}+13,\ 8\sqrt{2}\ ??\ \sqrt{143},\ 128<143.$\\
34. $3+2\sqrt{5}\ ??\ \sqrt{14}+\sqrt{15},\ 9+12\sqrt{5}+20\ ??\ 14+2\sqrt{14}\cdot\sqrt{15}+15,\ 6\sqrt{5}\ ??\ \sqrt{210},\ 180<210.$\\
35. $\cfrac{\sqrt{52}}{\sqrt{\sqrt{17}+2}\sqrt{\sqrt{17}-2}}=\cfrac{\sqrt{52}}{\sqrt{17-4}}=\sqrt{4}=2.$\\
36. Пусть $566=x,$ тогда искомое выражение равно $(x-4)(x+4)-(x+2)(x-2)=x^2-16-x^2+4=-12.$\\
37. $\cfrac{15^{n+4}}{3^{n+2}\cdot5^{n+3}}=\cfrac{3^{n+4}\cdot5^{n+4}}{3^{n+2}\cdot5^{n+3}}=3^2\cdot5=45.$\\
38. $\sqrt{3}\cdot\sqrt{24}\cdot\cfrac{\sqrt{63\cdot16-72\cdot6}}{\sqrt{19^2-17^2}}=
\sqrt{72}\cdot\cfrac{\sqrt{9\cdot16\cdot(7-3)}}{\sqrt{(19-17)(19+17)}}=\sqrt{72}\cdot\cfrac{3\cdot4\cdot2}{\sqrt{2\cdot36}}=24.$\\
39. $\sqrt{2}\cdot\sqrt{18}\cdot\cfrac{\sqrt{36\cdot92-48\cdot9}}{\sqrt{21^2-19^2}}=
\sqrt{36}\cdot\cfrac{\sqrt{36\cdot(92-12)}}{\sqrt{(21-19)(21+19)}}=6\cdot\cfrac{6\sqrt{80}}{\sqrt{80}}=36.$\\
40. $\sqrt{7}\cdot\sqrt{28}\cdot\cfrac{\sqrt{225\cdot32-72\cdot16}}{\sqrt{23^2-19^2}}=
\sqrt{196}\cdot\cfrac{\sqrt{16\cdot9\cdot(50-8)}}{\sqrt{(23-19)(23+19)}}=14\cdot\cfrac{4\cdot3\sqrt{42}}{2\sqrt{42}}=84.$\\
41. $\sqrt{5}\cdot\sqrt{45}\cdot\cfrac{\sqrt{126\cdot32+48\cdot18}}{\sqrt{19^2-15^2}}=
\sqrt{225}\cdot\cfrac{\sqrt{16\cdot9\cdot(28+6)}}{\sqrt{(19-15)(19+15)}}=
15\cdot\cfrac{4\cdot3\sqrt{34}}{2\sqrt{34}}=90.$\\
42. $\sqrt{1-2\sqrt{3}+3}+\sqrt{3}=\sqrt{1^2-2\cdot1\cdot\sqrt{3}+(\sqrt{3})^2}+\sqrt{3}=|1-\sqrt{3}|+\sqrt{3}=\sqrt{3}-1+\sqrt{3}=2\sqrt{3}-1.$\\
43. $-\sqrt{5}-\sqrt{1-2\sqrt{5}+5}=-\sqrt{5}-\sqrt{1^2-2\cdot1\cdot\sqrt{5}+(\sqrt{5})^2}=-\sqrt{5}-|1-\sqrt{5}|=-\sqrt{5}-\sqrt{5}+1=1-2\sqrt{5}.$\\
44. $-\sqrt{3}-\sqrt{1-2\sqrt{3}+3}=-\sqrt{3}-\sqrt{1^2-2\cdot1\cdot\sqrt{3}+(\sqrt{3})^2}=-\sqrt{3}-|1-\sqrt{3}|=-\sqrt{3}-\sqrt{3}+1=1-2\sqrt{3}.$\\
45. $\sqrt{5}+\sqrt{1-2\sqrt{5}+5}=\sqrt{5}+\sqrt{1^2-2\cdot1\cdot\sqrt{5}+(\sqrt{5})^2}=\sqrt{5}+|1-\sqrt{5}|=\sqrt{5}+\sqrt{5}-1=2\sqrt{5}-1.$\\
46. $\sqrt{83-18\sqrt{2}}-\sqrt{2}=\sqrt{9^2-2\cdot9\cdot\sqrt{2}+(\sqrt{2})^2}-\sqrt{2}=\sqrt{(9-\sqrt{2})^2}-\sqrt{2}=|9-\sqrt{2}|-\sqrt{2}=9-2\sqrt{2}.$\\
47. $\sqrt{54-14\sqrt{5}}+\sqrt{5}=\sqrt{7^2-2\cdot7\cdot\sqrt{5}+(\sqrt{5})^2}+\sqrt{5}=\sqrt{(7-\sqrt{5})^2}+\sqrt{5}=|7-\sqrt{5}|+\sqrt{5}=7.$\\
48. $(2-\sqrt{5})(\sqrt{9+4\sqrt{5}})=(2-\sqrt{5})\left(\sqrt{2^2+2\cdot2\cdot\sqrt{5}+(\sqrt{5})^2}\right)=(2-\sqrt{5})\left(\sqrt{(2+\sqrt{5})^2}\right)=$\\$=
(2-\sqrt{5})(2+\sqrt{5})=4-5=-1.$\\
49. $(2-\sqrt{7})(\sqrt{11+4\sqrt{7}}=(2-\sqrt{7})\left(\sqrt{2^2+2\cdot2\cdot\sqrt{7}+(\sqrt{7})^2}\right)=(2-\sqrt{7})\left(\sqrt{(2+\sqrt{7})^2}\right)=$\\$=
(2-\sqrt{7})(2+\sqrt{7})=4-7=-3.$\\
50. $\cfrac{180\cdot3,91-168+859\cdot1,8-768}{239\cfrac{5}{6}-237\cfrac{2}{3}}=
\cfrac{1,8\cdot391+859\cdot1,8-936}{2\cfrac{1}{6}}=\cfrac{1,8\cdot(391+859)-936}{\cfrac{13}{6}}=$\\$=\cfrac{2250-936}{\cfrac{13}{6}}=
(1313+1)\cdot\cfrac{6}{13}=606\cfrac{6}{13}.$\\
51. $\cfrac{1,7\cdot229-1155+7,91\cdot170+937}{366\cfrac{5}{6}-364\cfrac{29}{42}}=
\cfrac{1,7\cdot229+791\cdot1,7-218}{2\cfrac{1}{7}}=\cfrac{1,7\cdot(229+791)-218}{\cfrac{15}{7}}=$\\$=\cfrac{1734-218}{\cfrac{15}{7}}=
(1515+1)\cdot\cfrac{7}{15}=707\cfrac{7}{15}.$\\
52. $\cfrac{(8^{2020}+8^{2019})^2}{(4^{2019}-4^{2018})^3}=\cfrac{(8^{2019}\cdot(8+1))^2}{(4^{2018}\cdot(4-1))^3}=
\cfrac{8^{4038}\cdot81}{4^{6054}\cdot27}=\cfrac{2^{12114}\cdot3}{2^{12108}}=2^6\cdot3=192.$\\
53. $\cfrac{(4^{3021}-4^{3020})^3}{(8^{3020}+8^{3019})^2}=\cfrac{(4^{3020}\cdot(4-1))^3}{(8^{3019}\cdot(8+1))^2}=
\cfrac{4^{9060}\cdot27}{8^{6038}\cdot81}=\cfrac{2^{18120}}{2^{18114}\cdot3}=\cfrac{2^6}{3}=\cfrac{64}{3}.$\\
54. Пусть $30\cfrac{1}{239}=x,$ тогда искомое выражение равно $x^2-(x+1)(x-1)=x^2-x^2+1=1.$\\
55. $(2-\sqrt{5})\sqrt{18+8\sqrt{5}}=(2-\sqrt{5})\sqrt{2(2^2+2\cdot2\cdot\sqrt{5}+(\sqrt{5})^2)}=(2-\sqrt{5})\sqrt{2(2+\sqrt{5})^2}=$\\$=
(2-\sqrt{5})\cdot\sqrt{2}(2+\sqrt{5})=(4-5)\sqrt{2}=-\sqrt{2}.$\\
56. $\cfrac{1}{\sqrt{7+4\sqrt{3}}}+\cfrac{1}{\sqrt{7-4\sqrt{3}}}=\cfrac{1}{\sqrt{2^2+2\cdot2\cdot\sqrt{3}+(\sqrt{3})^2}}+\cfrac{1}{\sqrt{2^2-2\cdot2\cdot\sqrt{3}+(\sqrt{3})^2}}=$\\
$=\cfrac{1}{\sqrt{(2+\sqrt{3})^2}}+\cfrac{1}{\sqrt{(2-\sqrt{3})^2}}=\cfrac{1}{2+\sqrt{3}}+\cfrac{1}{2-\sqrt{3}}=
\cfrac{2-\sqrt{3}+2+\sqrt{3}}{4-3}=4.$\\
57.
$\cfrac{1}{\sqrt{9+4\sqrt{5}}}+\cfrac{1}{\sqrt{9-4\sqrt{5}}}=\cfrac{1}{\sqrt{2^2+2\cdot2\cdot\sqrt{5}+(\sqrt{5})^2}}+\cfrac{1}{\sqrt{2^2-2\cdot2\cdot\sqrt{5}+(\sqrt{5})^2}}=$\\
$=\cfrac{1}{\sqrt{(2+\sqrt{5})^2}}+\cfrac{1}{\sqrt{(2-\sqrt{5})^2}}=\cfrac{1}{2+\sqrt{5}}+\cfrac{1}{\sqrt{5}-2}=
\cfrac{\sqrt{5}-2+\sqrt{5}+2}{5-4}=2\sqrt{5}.$\\
58. $\cfrac{1}{\sqrt{3}-\sqrt{2}}=\cfrac{\sqrt{3}+\sqrt{2}}{3-2}=\sqrt{3}+\sqrt{2}>\sqrt{2}+\sqrt{2}=2\sqrt{2}.$\\
59. $\sqrt{10}+\sqrt{13}\ ??\ \sqrt{12}+\sqrt{11},\ 10+2\sqrt{10}\sqrt{13}+13\ ??\ 12+2\sqrt{12}\sqrt{11}+11,\ 130<132.$\\
60. $\cfrac{1}{2+\sqrt{17-12\sqrt{2}}}=\cfrac{1}{2+\sqrt{3^2-2\cdot3\cdot2\sqrt{2}+(2\sqrt{2})^2}}=
\cfrac{1}{2+\sqrt{(3-2\sqrt{2})^2}}=\cfrac{1}{2+3-2\sqrt{2}}=$\\$=\cfrac{1}{5-2\sqrt{2}}=
\cfrac{5+2\sqrt{2}}{25-8}=\cfrac{5+2\sqrt{2}}{17}<\cfrac{5+3}{17}=\cfrac{8}{17}.$ Заметим, что $3>2\sqrt{2}$ так как $9>8.$\\
61. $(4+\sqrt{15})(\sqrt{10}-\sqrt{6})\sqrt{4-\sqrt{15}}=(4+\sqrt{15})\sqrt{2}(\sqrt{5}-\sqrt{3})\sqrt{4-\sqrt{15}}=$\\$=
(4+\sqrt{15})(\sqrt{5}-\sqrt{3})\sqrt{8-2\sqrt{15}}=(4+\sqrt{15})(\sqrt{5}-\sqrt{3})\sqrt{(\sqrt{5})^2-2\sqrt{5}\sqrt{3}+(\sqrt{3})^2}=
(4+\sqrt{15})(\sqrt{5}-\sqrt{3})\sqrt{(\sqrt{5}-\sqrt{3})^2}=(4+\sqrt{15})(\sqrt{5}-\sqrt{3})^2=
(4+\sqrt{15})(5-2\sqrt{15}+3)=(4+\sqrt{15})2(4-\sqrt{15})=2(16-15)=$\\$=2.$\\
62. $\cfrac{3^2-0,363^2}{3,363}=\cfrac{(3-0,363)(3+0,363)}{3,363}=2,637.$\\
63. $\cfrac{\left(0,875-\cfrac{1}{8}\right):0,75-1}{0,125+\cfrac{7}{8}}=\cfrac{0,75:0,75-1}{0,125+\cfrac{7}{8}}=\cfrac{0}{0,125+\cfrac{7}{8}}=0.$\\
64. $\left(\sqrt{3+\sqrt{5}}+\sqrt{3-\sqrt{5}}\right)^2=3+\sqrt{5}+2\sqrt{(3+\sqrt{5})(3-\sqrt{5})}+3-\sqrt{5}=
6+2\sqrt{9-5}=6+4=10.$\\
65. Пусть $2019=x,$ тогда искомое выражение равно $\sqrt{\sqrt{(x-3)(x-1)(x+1)(x+3)+16}+5}=\sqrt{\sqrt{(x^2-9)(x^2-1)+16}+5}=
\sqrt{\sqrt{x^4-9x^2-x^2+9+16}+5}=\sqrt{\sqrt{x^4-10x^2+25}+5}=$\\$=\sqrt{\sqrt{(x^2-5)^2}+5}=\sqrt{x^2-5+5}=x=2019.$\\
66. Пусть $2017=x,$ тогда искомое выражение равно $\sqrt{\sqrt{(x-4)(x-2)(x+2)(x+4)+36}+10}=\sqrt{\sqrt{(x^2-16)(x^2-4)+36}+10}=
\sqrt{\sqrt{x^4-16x^2-4x^2+64+36}+10}=\sqrt{\sqrt{x^4-20x^2+100}+10}=$\\$=\sqrt{\sqrt{(x^2-10)^2}+10}=\sqrt{x^2-10+10}=x=2017.$\\
67. $\left(\cfrac{1}{2+\sqrt{3}}-\cfrac{1}{2-\sqrt{3}}\right)(\sqrt{12}-\sqrt{75})=\cfrac{2-\sqrt{3}-2-\sqrt{3}}{4-3}\cdot(2\sqrt{3}-5\sqrt{3})=
-2\sqrt{3}\cdot(-3\sqrt{3})=18.$\\
68. $\left(\cfrac{1}{3+\sqrt{5}}-\cfrac{1}{3-\sqrt{5}}\right)(\sqrt{5}+\sqrt{45})=\cfrac{3-\sqrt{5}-3-\sqrt{5}}{9-5}\cdot(\sqrt{5}+3\sqrt{5})=
\cfrac{-2\sqrt{5}}{4}\cdot4\sqrt{5}=-10.$\\
69. $\sqrt{(\sqrt{2}-3)^2}+\sqrt{27+10\sqrt{2}}=|\sqrt{2}-3|+\sqrt{5^2+2\cdot5\cdot\sqrt{2}+(\sqrt{2})^2}=
3-\sqrt{2}+\sqrt{(5+\sqrt{2})^2}=3-\sqrt{2}+5+\sqrt{2}=8.$\\
70. $\cfrac{\left(0,5:1,25+\cfrac{7}{5}:1\cfrac{4}{7}-\cfrac{3}{11}\right)\cdot3}{\left(1,5+\cfrac{1}{4}\right):18\cfrac{1}{3}}=
\cfrac{\left(\cfrac{2}{5}+\cfrac{49}{55}-\cfrac{3}{11}\right)\cdot3}{\cfrac{7}{4}:\cfrac{55}{3}}=
\cfrac{\cfrac{56}{55}\cdot3}{\cfrac{21}{220}}=\cfrac{56\cdot3\cdot220}{55\cdot21}=32.$\\
71. $\left(\cfrac{(2,7-0,8)\cdot2\cfrac{1}{3}}{(5,2-1,4):\cfrac{3}{70}}+0,125\right):2\cfrac{1}{2}+0,43=
\left(\cfrac{\cfrac{19}{10}\cdot\cfrac{7}{3}}{\cfrac{38}{10}\cdot\cfrac{70}{3}}+\cfrac{1}{8}\right):\cfrac{5}{2}+0,43=
\left(\cfrac{\cfrac{19\cdot7}{30}}{\cfrac{38\cdot70}{30}}+\cfrac{1}{8}\right)\cdot\cfrac{2}{5}+0,43=
\left(\cfrac{1}{20}+\cfrac{1}{8}\right)\cdot\cfrac{2}{5}+0,43=\cfrac{7}{40}\cdot\cfrac{2}{5}+0,43=0,07+0,43=\cfrac{1}{2}.$\\
72. $\left(\cfrac{1}{2}\sqrt{6}-3\sqrt{3}+5\sqrt{2}-\sqrt{8}\right)\cdot\sqrt{24}+18\sqrt{2}-12\sqrt{3}=
\cfrac{1}{2}\sqrt{144}-3\sqrt{72}+5\sqrt{48}-\sqrt{192}+18\sqrt{2}-12\sqrt{3}=
\cfrac{1}{2}\cdot12-18\sqrt{2}+20\sqrt{3}-8\sqrt{3}+18\sqrt{2}-12\sqrt{3}=6.$\\
73. $\left(\cfrac{1}{2}\sqrt{32}-\cfrac{1}{3}\sqrt{3}+4\sqrt{15}-\sqrt{8}\right)\cdot\sqrt{12}-4\sqrt{6}-24\sqrt{5}=
\cfrac{1}{2}\sqrt{384}-\cfrac{1}{3}\sqrt{36}+4\sqrt{180}-\sqrt{96}-4\sqrt{6}-24\sqrt{5}=
\cfrac{1}{2}\cdot8\sqrt{6}-2+24\sqrt{5}-4\sqrt{6}-4\sqrt{6}-24\sqrt{5}=-2-4\sqrt{6}.$\\
74. $\sqrt{7+4\sqrt{3}}+\sqrt{7-4\sqrt{3}}=\sqrt{2^2+2\cdot2\sqrt{3}+(\sqrt{3})^2}+\sqrt{2^2-2\cdot2\sqrt{3}+(\sqrt{3})^2}=|2+\sqrt{3}|+|2-\sqrt{3}|=
2+\sqrt{3}+2-\sqrt{3}=4.$\\
75. $\sqrt{8+2\sqrt{7}}-\sqrt{8-2\sqrt{7}}=\sqrt{1^2+2\cdot1\sqrt{7}+(\sqrt{7})^2}+\sqrt{1^2-2\cdot1\sqrt{7}+(\sqrt{7})^2}=|1+\sqrt{7}|+|1-\sqrt{7}|=
1+\sqrt{7}+1-\sqrt{7}=2.$\\
76. $2\sqrt{9\cfrac{1}{2}}-\sqrt{342}+5\sqrt{1\cfrac{13}{25}}=2\sqrt{\cfrac{19}{2}}-3\sqrt{38}+5\sqrt{\cfrac{38}{25}}=\sqrt{38}-3\sqrt{38}+\sqrt{38}=-\sqrt{38}.$\\
77. $\cfrac{2^{19}\cdot27^3+15\cdot4^9\cdot9^4}{6^9\cdot2^{10}+12^{10}}=\cfrac{2^{19}\cdot3^9+3\cdot5\cdot2^{18}\cdot3^8}{2^9\cdot3^9\cdot2^{10}+2^{20}\cdot3^{10}}=
\cfrac{2^{18}\cdot3^9\cdot(2+5)}{2^{19}\cdot3^9\cdot(1+2\cdot3)}=\cfrac{1}{2}.$\\
78. $\cfrac{4^{15}\cdot9^9-4\cdot3^{20}\cdot8^9}{2^9\cdot6^{19}-5\cdot2^{29}\cdot27^6}=
\cfrac{2^{30}\cdot3^{18}-2^2\cdot3^{20}\cdot2^{27}}{2^9\cdot2^{19}\cdot3^{19}-5\cdot2^{29}\cdot3^{18}}=
\cfrac{2^{29}\cdot3^{18}\cdot(2-3^2)}{2^{28}\cdot3^{18}\cdot(3-5\cdot2)}=2.$\\
79. $\left(\cfrac{1}{2}\sqrt{6}-\sqrt{12}+0,5\sqrt{24}+\cfrac{3\sqrt{48}}{4}\right)\cdot2\sqrt{2}=
\left(\cfrac{1}{2}\sqrt{6}-2\sqrt{3}+\sqrt{6}+\cfrac{12\sqrt{3}}{4}\right)\cdot2\sqrt{2}=
\left(\cfrac{3}{2}\sqrt{6}+\sqrt{3}\right)\cdot2\sqrt{2}=6\sqrt{3}+2\sqrt{6}.$\\
80. $(8\sqrt{24}-12\sqrt{54}+6\sqrt{96}-4\sqrt{150}):2\sqrt{3}=
(16\sqrt{6}-36\sqrt{6}+24\sqrt{6}-20\sqrt{6}):2\sqrt{3}=(-16\sqrt{6}):2\sqrt{3}=-8\sqrt{2}.$\\
81. $\sqrt{5-\sqrt{24}}+\sqrt{5+\sqrt{24}}=\sqrt{3-2\sqrt{6}+2}+\sqrt{3+2\sqrt{6}+2}=\sqrt{(\sqrt{3}-\sqrt{2})^2}+\sqrt{(\sqrt{3}+\sqrt{2})^2}=
\sqrt{3}-\sqrt{2}+\sqrt{3}+\sqrt{2}=2\sqrt{3}.$ Так как $3^2=9<12=(2\sqrt{3})^2<16=4^2,$ данное число лежит в промежутке $(3;4).$\\
82. $\cfrac{17\sqrt{3}}{3\sqrt{5}-7\sqrt{3}}-\cfrac{3\sqrt{5}-\sqrt{3}}{5\sqrt{3}-4\sqrt{5}}=
\cfrac{17\sqrt{3}(3\sqrt{5}+7\sqrt{3})}{45-147}-\cfrac{(3\sqrt{5}-\sqrt{3})(5\sqrt{3}+4\sqrt{5})}{75-80}=$\\$
\cfrac{51\sqrt{15}+357}{-102}-\cfrac{15\sqrt{15}+60-15-4\sqrt{15}}{-5}=
\cfrac{-7-\sqrt{15}}{2}+\cfrac{11\sqrt{15}+45}{5}=\cfrac{11}{2}+\cfrac{17\sqrt{15}}{10}.$\\
83. $\cfrac{2\sqrt{5}-\sqrt{3}}{5\sqrt{3}-4\sqrt{5}}-\cfrac{21\sqrt{5}}{4\sqrt{5}-6\sqrt{3}}=
\cfrac{(2\sqrt{5}-\sqrt{3})(5\sqrt{3}+4\sqrt{5})}{5\sqrt{3}-4\sqrt{5}}-\cfrac{21\sqrt{5}(4\sqrt{5}+6\sqrt{3})}{4\sqrt{5}-6\sqrt{3}}=$\\$
\cfrac{10\sqrt{15}+40-15-4\sqrt{15}}{75-80}-\cfrac{420+126\sqrt{15}}{80-108}=
\cfrac{6\sqrt{15}+25}{-5}-\cfrac{420+126\sqrt{15}}{-28}=
-\cfrac{6\sqrt{15}+25}{5}+\cfrac{30+9\sqrt{15}}{2}=10+\cfrac{33\sqrt{15}}{10}.$\\
84. $(\sqrt{10}-\sqrt{5})\cdot\sqrt{\cfrac{20}{(1-\sqrt{2})^2}}=\sqrt{5}(\sqrt{2}-1)\cdot\cfrac{2\sqrt{5}}{\sqrt{2}-1}=10>9,99.$\\
85. $(\sqrt{8}+\sqrt{10})\cdot\sqrt{32\cdot(2-\sqrt{5})^2}=\sqrt{2}(2+\sqrt{5})\cdot4\sqrt{2}(\sqrt{5}-2)=8\cdot(5-4)=8>7,99.$\\
86. $\cfrac{\left(5\cfrac{7}{30}-3\cfrac{5}{18}\right):2\cfrac{2}{3}}{(1,2)^2-1,26:0,9}=
\cfrac{\cfrac{88}{45}\cdot\cfrac{3}{8}}{1,44-1,4}=\cfrac{\cfrac{11}{15}}{0,04}=\cfrac{55}{3}.$ Тогда искомое число равно
$\cfrac{55}{3}:36\cfrac{2}{3}\cdot100=\cfrac{55}{3}:\cfrac{110}{3}\cdot100=50.$\\
87. $\cfrac{18^{n+3}}{3^{2n+5}\cdot2^{n+2}}=\cfrac{2^{n+3}\cdot3^{2n+6}}{3^{2n+5}\cdot2^{n+2}}=2\cdot3=6.$\\
88. $\left(\cfrac{76^2+44^2+2\cdot44\cdot76}{120}\right):(37^2-13^2)=\cfrac{(76+44)^2}{120}:((37-13)(37+13))=\cfrac{120^2}{120}:(24\cdot50)=\cfrac{120}{1200}=\cfrac{1}{10}.$\\
89. $4322^2-4321\cdot4323=4322^2-(4322-1)(4322+1)=4322^2-(4322^2-1)=1.$\\
90. $\sqrt{9-4\sqrt{5}}-\cfrac{4}{\sqrt{5}+1}=\sqrt{(\sqrt{5}-2)^2}-\cfrac{4(\sqrt{5}-1)}{5-1}=
\sqrt{5}-2-(\sqrt{5}-1)=-1.$\\
91. $\sqrt{48}+\sqrt{11}\ ??\ \sqrt{27}+5,\ 4\sqrt{3}+\sqrt{11}\ ??\ 3\sqrt{3}+5,\ \sqrt{3}+\sqrt{11}\ ??\ 5,\ 3+2\sqrt{33}+11\ ??\ 25,\
2\sqrt{33}\ ??\ 11,\ 132>121.$\\
92. $\sqrt{8}+5\ ??\ \sqrt{18}+\sqrt{13},\ 8+10\sqrt{8}+25\ ??\ 18+2\sqrt{234}+13,\ 20\sqrt{2}+2\ ??\ 2\sqrt{234},\ 10\sqrt{2}+1\ ??\ \sqrt{234},\ 200+20\sqrt{2}+1\ ??\ 234,\ 20\sqrt{2}\ ??\ 33,\ 800<1089.$\\
93. $\sqrt{(2\sqrt{2}-3)^2}-\cfrac{8}{4+2\sqrt{2}}=[(2\sqrt{2})^2=8<9=3^2]=3-2\sqrt{2}-\cfrac{8(4-2\sqrt{2})}{16-8}=
3-2\sqrt{2}-4+2\sqrt{2}=-1.$\\
94. $\sqrt{(\sqrt{17}-3\sqrt{2})^2}+\cfrac{1}{\sqrt{17}-4}-3\sqrt{2}=[(\sqrt{17})^2=17<18=(3\sqrt{2})^2]=3\sqrt{2}-\sqrt{17}+\cfrac{\sqrt{17}+4}{17-16}-3\sqrt{2}=
3\sqrt{2}-\sqrt{17}+\sqrt{17}+4-3\sqrt{2}=4.$
\newpage
