85. Пусть парабола задана уравнением $y=ax^2+bx+c.$ Так как в точке $A(0;-3)$ находится вершина, $-\cfrac{b}{2a}=0,\ b=0,\ 0+0+c=-3,\ c=-3.$ Так как парабола проходит через точку $B(6;15),$ имеем уравнение $36a-3=15, a=\cfrac{1}{2}.$ Найдём точки пересечения с осью $Ox:\ \cfrac{1}{2}x^2-3=0,\ x^2=6,\ x=\pm\sqrt{6}.$ Значит, это точки $(-\sqrt{6};0)$ и $(\sqrt{6};0)$\\
