74. $(2-\sqrt{5})x>2+\sqrt{5}\Leftrightarrow x<\cfrac{2+\sqrt{5}}{2-\sqrt{5}}=\cfrac{4+4\sqrt{5}+5}{4-5}=-9-4\sqrt{5}=-9-\sqrt{80}.$ Так как $9+\sqrt{80}<9+9=18,$
наибольшим целым решением неравенства является $x=-18.$\\