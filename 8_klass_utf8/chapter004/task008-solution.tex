8. Сделаем замену: $x=\sqrt{a},\ y=\sqrt{b},\ z=\sqrt{c}.$ Тогда исходное неравенство примет вид $x^2+y^2+z^2\geqslant xy+yz+xz\Leftrightarrow
2x^2+2y^2+2z^2\geqslant 2xy+2yz+2xz\Leftrightarrow x^2-2xy+y^2+y^2-2yz+z^2+x^2-2xz+z^2\geqslant 0\Leftrightarrow (x-y)^2+(y-z)^2+(x-z)^2\geqslant 0.$
Полученное неравенство верно для любых значений переменных, так как значения квадратов всегда неотрицательны.\\
