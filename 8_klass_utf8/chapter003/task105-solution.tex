105. а) Так как $x^2-(\sqrt{2}+\sqrt{7})x+\sqrt{14}=(x-\sqrt{2})(x-\sqrt{7})=0,$ имеем $x=\sqrt{2}$ или $x=\sqrt{7}.$\\
б) Понятно, что $\sqrt{2}<2,$ а $\sqrt{7}>2.$ Тогда произведём сравнение $2-\sqrt{2}\ ??\ \sqrt{7}-2,\ 4\ ??\ \sqrt{2}+\sqrt{7},\ 16\ ??\ 2+2\sqrt{14}+7,\ 7\ ??\ 2\sqrt{14},\ 49<56.$ Значит, $\sqrt{2}$ находится ближе к числу 2.\\
