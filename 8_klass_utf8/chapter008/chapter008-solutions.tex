\section{Нестандартные задачи решения}
1. Если $a\ \vdots\ d$ и $a+b\ \vdots\ d,$ то $a+b-a=b\ \vdots\ d.$ Так как НОД$(a, b)=1,$ $d=1.$\\
2. Если $a\ \vdots\ d$ и $a-b\ \vdots\ d,$ то $a-(a-b)=b\ \vdots\ d.$ Так как НОД$(a, b)=1,$ $d=1.$\\
3. $\cfrac{3n-1}{n+1}=\cfrac{3n+3-4}{n+1}=3-\cfrac{4}{n+1}.$ Если это число целое, 4 делится на $n+1,$ тогда $n+1=2$ или $n+1=4,$
тогда $n\in\{1; 3\}.$\\
4. $\cfrac{3n+1}{n-1}=\cfrac{3n-3+4}{n-1}=3+\cfrac{4}{n-1}.$ Если это число целое, 4 делится на $n-1,$ тогда $n-1=1,\ n-1=2$ или $n-1=4,$
тогда $n\in\{2; 3; 5\}.$\\
5. Пусть маленьких раков было $x,$ а больших --- $y.$ Тогда  $5x+7y=101,\ 5x=101-7y.$ Чтобы $101-7y$ делилось на 5, число $7y$ должно кончаться на 1 или 6, поэтому $y$ может быть равно только $3,\ 8$ или 13, тогда $x$ равно $16,\ 9$ или 2. То есть могли купить 2 маленьких и 13 больших, 9 маленьких и 8 больших или 16 маленьких и 3 больших.\\
6. Пусть маленьких раков было $x,$ а больших --- $y.$ Тогда  $5x+8y=116,\ 5x=116-8y.$ Чтобы $116-8y$ делилось на 5, число $8y$ должно кончаться на 1 или 6, поэтому $y$ может быть равно только $2,\ 7$ или 12, тогда $x$ равно $20,\ 12$ или 4. То есть могли купить 4 маленьких и 12 больших, 12 маленьких и 7 больших или 20 маленьких и 2 больших.\\
7. $\cfrac{4n-5}{2n-1}=\cfrac{4n-2-3}{2n-1}=2-\cfrac{3}{2n-1}.$ Если это число целое, 3 делится на $2n-1,$ тогда $2n-1=\pm1$ или $2n-1=\pm3,$ тогда
$n\in\{-1; 0; 1; 2\}.$\\
8. $\cfrac{4n+5}{2n+1}=\cfrac{4n+2+3}{2n+1}=2+\cfrac{3}{2n+1}.$ Если это число целое, 3 делится на $2n+1,$ тогда $2n+1=\pm1$ или $2n+1=\pm3,$ тогда
$n\in\{-2; -1; 0; 1\}.$\\
9. Из каждой вершины можно провести по 4 диагонали, каждая диагональ соединяет 2 вершины, значит всего их $7\cdot4:2=14.$\\
10. Из каждой вершины можно провести по 5 диагоналей, каждая диагональ соединяет 2 вершины, значит всего их $8\cdot5:2=20.$\\
11. $\cfrac{3n+17}{n+4}=\cfrac{3n+12+5}{n+4}=3+\cfrac{5}{n+4}.$ Если дробь является сократимой, $n+4$ делится на 5. Наибольшим таким двузначным числом является 96.\\
12. $\cfrac{3n+16}{n+4}=\cfrac{3n+12+4}{n+4}=3+\cfrac{4}{n+4}.$ Если дробь является сократимой, $n+4$ делится на 4. Наибольшим таким двузначным числом является 98.\\
13. Если количество книг делится на 3, 4 и 5, оно делится на $3\cdot4\cdot5=60.$ Значит, книг было 60.\\
14. Если количество книг делится на 3, 4 и 7, оно делится на $3\cdot4\cdot7=84.$ Значит, книг было 84.\\
15. $2x^2+xy=x+7\Leftrightarrow x(2x+y-1)=7\Leftrightarrow \left[\begin{array}{l} \begin{cases}x=7,\\  2x+y-1=1.\end{cases}\\ \begin{cases}x=-7,\\  2x+y-1=-1.\end{cases}\\\begin{cases}x=1,\\  2x+y-1=7.\end{cases}\\\begin{cases}x=-1,\\  2x+y-1=-7.\end{cases} \end{array}\right.
\Leftrightarrow \left[\begin{array}{l} \begin{cases}x=7,\\  y=-12.\end{cases}\\ \begin{cases}x=-7,\\  y=14.\end{cases}\\\begin{cases}x=1,\\  y=6.\end{cases}\\\begin{cases}x=-1,\\  y=-4.\end{cases} \end{array}\right.$\\
16. $x^2-3xy=x-3y+2\Leftrightarrow x(x-3y)-(x-3y)=2\Leftrightarrow (x-1)(x-3y)=2\Leftrightarrow
\left[\begin{array}{l} \begin{cases}x-1=2,\\  x-3y=1.\end{cases}\\ \begin{cases}x-1=1,\\  x-3y=2.\end{cases}\\ \begin{cases}x-1=-1,\\  x-3y=-2.\end{cases}\\ \begin{cases}x-1=-2,\\  x-3y=-1.\end{cases} \end{array}\right.
\Leftrightarrow \left[\begin{array}{l} \begin{cases}x=2,\\  y=0.\end{cases}\\ \begin{cases}x=-1,\\  y=0.\end{cases} \end{array}\right.$\\
17. Это число 499.\\
18. Это число 599.\\
19. Выбрать команду на первое место 3 способа, а на второе 2, значит всего способов $3\cdot2=6.$\\
20. Выбрать получателя яблока 3 способа, а мандарина 2, значит всего способов $3\cdot2=6.$\\
21. Чётное число, не кратное 6, может иметь вид $6k+2$ или $6k+4.$ Тогда $(6k+2)^2=36k^2+24k+4=12(3k^2+2k)+4,\ (6k+4)^2=36k^2+48k+16=12(3k^2+4k+1)+4.$ В обоих случаях квадрат этого числа при делении на 12 даёт остаток 4.\\
22. Целое число, не кратное 3, может иметь вид $3k+1$ или $3k+2.$ Тогда $(3k+1)^2=9k^2+6k+1=3(3k^2+2k)+1,\ (3k+2)^2=9k^2+12k+4=3(3k^2+4k+1)+1.$ В обоих случаях квадрат этого числа при делении на 3 даёт остаток 1.\\
23. $n=5k+4,$ тогда $4n+3=20k+16+3=5(4k+3)+4,$ то есть остаток при делении на 5 равен 4.\\
24. $n=4k+3,$ тогда $3n+2=12k+9+2=4(3k+2)+3,$ то есть остаток при делении на 4 равен 3.\\
25. Если число делится на 4 и на 6, оно делится на НОК$(4,6)=12.$ Это числа от $12=12\cdot1$ до $96=12\cdot8,$ их 8.\\
26. Если число делится на 6 и на 9, оно делится на НОК$(6,9)=18.$ Это числа от $18=18\cdot1$ до $90=18\cdot5,$ их 5.\\
27. $x^{1024}+1\geqslant1,$ а $\cfrac{1}{2+x^{318}}\leqslant\cfrac{1}{2},$ значит у уравнения решений нет.\\
28. $x^{1024}+2\geqslant2,$ а $\cfrac{1}{1+x^{318}}\leqslant1,$ значит у уравнения решений нет.\\
29. Если $x<0,$ то возможны только два случая: $x=-1,\ y=-1$ или $x=-3,\ y=1.$\\
30. Если $x<0,$ то возможны только два случая: $x=-1,\ y=1$ или $x=-3,\ y=3.$\\
31. Из Петербурга в Братск можно проехать $2\cdot4=8$ способами.\\
32. Из Петербурга в Хабаровск можно проехать $2\cdot5=10$ способами.\\
33. Каждый шахматист сыграет 6 партий, при этом каждая партия играется двумя шахматистами, значит их $7\cdot6:2=21.$\\
34. В январе 31 день, то есть 4 полных недели и ещё 3 дня. В полных неделях всех дней по одному, значит пятый понедельник выпадает на один из трёх последних дней. Тогда четверг на эти дни выпасть не может и четвергов было 4.\\
35. В марте 31 день, то есть 4 полных недели и ещё 3 дня. В полных неделях всех дней по одному, значит пятый вторник выпадает на один из трёх последних дней. Тогда пятница на эти дни выпасть не может и пятниц было 4.\\
36. Кубики с ровно двумя разными окрашенными гранями находятся вдоль рёбер, без двух крайних на этом ребре. Значит, их $12\cdot8=96.$\\
37. Кубики с ровно двумя разными окрашенными гранями находятся вдоль рёбер, без двух крайних на этом ребре. Значит, их $12\cdot8=96.$\\
38. Всего королей 4. Выбрать первого короля 4 способа, а второго 3, при этом порядок, в котором они выбраны, не важен, поэтому всего вариантов $4\cdot3:2=6.$\\
39. Всего валетов 4. Выбрать первого валета 4 способа, а второго 3, при этом порядок, в котором они выбраны, не важен, поэтому всего вариантов $4\cdot3:2=6.$\\
40. Так как НОД$(a,b)\cdot$НОД$(a,b)=ab,$ НОД$(a,b)=3.$\\
41. Так как НОД$(a,b)\cdot$НОД$(a,b)=ab,$ НОД$(a,b)=5.$\\
42. Из каждой точки можно провести 6 хорд, при этом каждая хорда соединяет две точки, значит всего их $7\cdot6:2=21.$\\
43. Из каждой точки можно провести 5 хорд, при этом каждая хорда соединяет две точки, значит всего их $6\cdot5:2=15.$\\
44. а) $(3x^5+4x^3)'=(3x^5)'+(4x^3)'=3(x^5)'+4(x^3)'=3\cdot5x^4+4\cdot3x^2=15x^4+12x^2.$\\
б) $1'=(x^0)'=0.$\\
в) $T'(x)=(3x^4-2x^2)'=(3x^4)'-(2x^2)'=3(x^4)'-2(x^2)'=3\cdot4x^3-2\cdot2x=12x^3-4x.$\\
$(T'(x)+3xT(x))'=(12x^3-4x+3x(3x^4-2x^2))'=(9x^5+6x^3-4x)'=45x^4+18x^2-4.$ Тогда $45x^4+18x^2-4-18x^2-45x^4=-4.$\\
45. а) $(3x^3+4x)'=(3x^3)'+(4x)'=3(x^3)'+4(x)'=3\cdot3x^2+4\cdot 1=9x^2+4.$\\
б) $(x^6-2x^4)'=(x^6)'-2(x^4)'=(x^3\cdot x^3)'-2(x\cdot x^3)'=(x^3)'\cdot x^3+x^3\cdot (x^3)'-2x' \cdot x^3-2x\cdot (x^3)'=3x^2\cdot x^3+x^3\cdot3x^2-2x^3-2x\cdot3x^2=6x^5-8x^3.$\\
в) $T'(x)=(3x^3-2x)'=(3x^3)'-(2x)'=3(x^3)'-2(x)'=3\cdot3x^2-2\cdot1=9x^2-2.$\\
$(T(x)+3xT(x))'=T'(x)+(3x)'T(x)+3xT'(x)=9x^2-2+3(3x^3-2x)+3x(9x^2-2)=36x^3+9x^2-12x-2.$ Тогда $36x^3+9x^2-12x-2-9x^2-36x^3=-12x-2.$\\
46. На первое место можно поставить три цифры 1, 4 и 5. На второе место можно поставить все 4 цифры. На третье место можно поставить цифры 1 и 5. Тогда всего чисел $3\cdot4\cdot2=24.$\\
47. На первое место можно поставить три цифры 1, 4 и 5. На второе место можно поставить все 4 цифры. На третье место можно поставить цифры 0 и 4. Тогда всего чисел $3\cdot4\cdot2=24.$\\
48. Сумма выпавших очков 10 может быть в трёх случаях: 4 и 6, 6 и 4 или 5 и 5. Всего случаев $6\cdot6=36,$ значит вероятность того, что сумма выпавших очков равна 10, равна $\cfrac{3}{36}=\cfrac{1}{12}.$\\
49. Сумма выпавших очков 4 может быть в трёх случаях: 3 и 1, 1 и 3 или 2 и 2. Всего случаев $6\cdot6=36,$ значит вероятность того, что сумма выпавших очков равна 10, равна $\cfrac{3}{36}=\cfrac{1}{12}.$\\
50. Каждый шахматист сыграл 14 партий, при этом в каждой партии принимали участие 2 шахматиста, значит всего партий было  $15\cdot14:2=105.$\\
51. Каждый шахматист сыграл 13 партий, при этом в каждой партии принимали участие 2 шахматиста, значит всего партий было  $14\cdot13:2=91.$\\
52. Всего докладчиков было $15\cdot2+20=50.$ Профессор М. выступит в третий день, если будет одним из 20 докладчиков, вероятность чего равна $\cfrac{20}{50}=\cfrac{2}{5}=40\%.$\\
53. Всего докладчиков было $20\cdot2+10=50.$ Профессор М. выступит в третий день, если будет одним из 10 докладчиков, вероятность чего равна $\cfrac{10}{50}=\cfrac{1}{5}=20\%.$\\
54. $a=3k+1,$ тогда $a^2=(3k+1)^2=9k^2+6k+1=3(3k^2+2k)+1,$ то есть остаток равен 1.\\
55. $a=5k+1,$ тогда $a^2=(5k+1)^2=25k^2+10k+1=5(5k^2+2k)+1,$ то есть остаток равен 1.\\
56. $(4\cdot\Game-1)\cdot(4\cdot\Game+1)-17\cdot\Game\cdot(\Game-2)=16\Game^2-1-17\Game^2+34\Game=-16-1+17+34\Game=34\Game.$\\
57. $(3\cdot\Game-1)\cdot(3\cdot\Game+1)-10\cdot\Game\cdot(\Game-2)=9\Game^2-1-10\Game^2+20\Game=-9-1+10+20\Game=20\Game.$\\
58. При $k=1,\ n=1$ получим $1>\cfrac{2}{3}.$ При $k=1,\ n=2$ имеем $\cfrac{1}{2}=\cfrac{2}{4}.$ При $n=1,\ k=3$ имеем $\cfrac{1}{3}<\cfrac{2}{5}.$ значит, возможны все варианты.\\
59. $\cfrac{5x^2+10x+14}{x^2+2x+3}=\cfrac{5x^2+10x+15-1}{x^2+2x+3}=5-\cfrac{1}{x^2+2x+3}.$ Значение этого выражения будет наименьшим, когда значение дроби будет наибольшим. Значение дроби будет наибольшим, когда её знаменатель будет наименьшим. В знаменателе находится квадратичная функция, наименьшее значение которой достигается в вершине параболы $x=-\cfrac{2}{2}=-1.$ При $x=-1$ значение выражения равно $5-\cfrac{1}{1-2+3}=\cfrac{9}{2}.$\\
60. $\cfrac{x^2+2x+5}{x^2+2x+3}=\cfrac{x^2+2x+3+2}{x^2+2x+3}=1+\cfrac{2}{x^2+2x+3}.$ Значение этого выражения будет наибольшим, когда значение дроби будет наибольшим. Значение дроби будет наибольшим, когда её знаменатель будет наименьшим. В знаменателе находится квадратичная функция, наименьшее значение которой достигается в вершине параболы $x=-\cfrac{2}{2}=-1.$ При $x=-1$ значение выражения равно $1+\cfrac{2}{1-2+3}=2.$\\
61. а) Пусть куплено $x$ красных карандашей и $y$ синих. Тогда выполнены три условия\\ $\begin{cases}17x+13y\leqslant495,\\ x+y=32,\\ |x-y|\leqslant5.\end{cases}
\Leftrightarrow \begin{cases}17(32-y)+13y\leqslant495,\\ x=32-y,\\ |32-y-y|\leqslant5.\end{cases}
\Leftrightarrow \begin{cases}49\leqslant4y,\\ x=32-y,\\ |32-2y|\leqslant5.\end{cases}$
Этим условиям удовлетворяют, например, $x=y=16.$\\
б) Аналогично пункту а) запишем $\begin{cases}17x+13y\leqslant495,\\ x+y=35,\\ |x-y|\leqslant5.\end{cases}
\Leftrightarrow \begin{cases}17(35-y)+13y\leqslant495,\\ x=35-y,\\ |35-y-y|\leqslant5.\end{cases}
\Leftrightarrow$\\$ \begin{cases}100\leqslant4y,\\ x=35-y,\\ |35-2y|\leqslant5.\end{cases}
\Leftrightarrow \begin{cases}25\leqslant y,\\ x=35-y,\\ |35-2y|\leqslant5.\end{cases}$
В этом случае $|35-2y|\geqslant|35-50|=15,$ так что 35 карандашей купить нельзя.\\
в) Пусть куплено $n$ карандашей, тогда $\begin{cases}17x+13y\leqslant495,\\ x+y=n,\\ |x-y|\leqslant5.\end{cases}
\Leftrightarrow \begin{cases}17(n-y)+13y\leqslant495,\\ x=n-y,\\ |n-y-y|\leqslant5.\end{cases}
\Leftrightarrow$\\$ \begin{cases}17n-495\leqslant4y,\\ x=n-y,\\ |n-2y|\leqslant5.\end{cases}$
Так как $17n-495\leqslant4y,$ верно неравенство $y\geqslant \cfrac{17n-495}{4}.$ Тогда $|n-2y|=2y-n\geqslant \cfrac{17n-495}{2}-n=\cfrac{15n-495}{2}.$ Для выполнения третьего условия должно выполняться неравенство $\cfrac{15n-495}{2} \leqslant 5 \Leftrightarrow3n-99\leqslant 2\Leftrightarrow 3n\leqslant 101
\Leftrightarrow n\leqslant 33.$ При $n=33$ можно взять $x=16,\ y=17.$\\
62. $\cfrac{x^2-4x+7}{x^2-4x+5}=\cfrac{x^2-4x+5+2}{x^2-4x+5}=1+\cfrac{2}{x^2-4x+5}.$ Значение этого выражения будет наибольшим, когда значение дроби будет наибольшим. Значение дроби будет наибольшим, когда её знаменатель будет наименьшим. В знаменателе находится квадратичная функция, наименьшее значение которой достигается в вершине параболы $x=-\cfrac{-4}{2}=2.$ При $x=2$ значение выражения равно $1+\cfrac{2}{4-8+5}=3.$\\
63. Раз $19a=97b,\ a$ делится на 97. Пусть $a=97c,$ тогда $19\cdot97c=97b$ и $b=19c,$ поэтому $a+b=97c+19c=116c,$ что делится на 116.\\
64. $\left(\cfrac{1}{b+c}+\cfrac{1}{a+c}+\cfrac{1}{a+b}\right)(a+b+c)=\cfrac{a}{b+c}+\cfrac{a}{a+c}+\cfrac{a}{a+b}+
\cfrac{b}{b+c}+\cfrac{b}{a+c}+\cfrac{b}{a+b}+\cfrac{c}{b+c}+$\\$\cfrac{c}{a+c}+\cfrac{c}{a+b}=
\cfrac{a}{b+c}+\cfrac{b}{a+c}+\cfrac{c}{a+b}+\cfrac{a+b}{a+b}+\cfrac{a+c}{a+c}+\cfrac{b+c}{b+c}=\cfrac{a}{b+c}+\cfrac{b}{a+c}+\cfrac{c}{a+b}+3.$
Тогда $\cfrac{a}{b+c}+\cfrac{b}{a+c}+\cfrac{c}{a+b}=7\cdot\cfrac{7}{10}-3=\cfrac{19}{10}.$\\
65. Пусть $\sqrt{n^2-17}=m,$ тогда $n^2-17=m^2,\ (n-m)(n+m)=17.$ Тогда либо $\begin{cases}n+m=17,\\n-m=1.\end{cases}\Rightarrow n=9,$ либо $
\begin{cases}n+m=-1,\\n-m=-17.\end{cases}\Rightarrow n=-9.$ Другие случаи невозможны, так как $m$ должно быть натуральным числом.\\
66. $6^{15}\cdot21^{7}=2^{15}\cdot3^{15}\cdot3^{7}\cdot7^{7}=2^{15}\cdot3^{22}\cdot7^7.$ Любой делитель этого числа является некоторой комбинацией множителей 2, 3 и 7. Двоек можно взять от 0 до 15, трое от 0 до 22, семёрок от 0 до 7. Значит, всего делителей $16\cdot23\cdot8=2944.$\\
67. Пусть $a=\overline{xy},$ тогда $\overline{xxyy}=99\overline{xy},\ 1100x+11y=990x+99y,\ 110x=88y,\ 5x=4y\Rightarrow x=4,\ y=5,\ a=45.$\\
68. Пусть было $x$ строк и $y$ букв в каждой строке, тогда $(x+7)(y+7)-xy=476,\ xy+7(x+y)+49-xy=476,\ x+y=61.$ Тогда $xy-(x-4)(y-4)=xy-xy+4(x+y)-16=244-16=228.$\\
69. Выпишем, какие остатки дают степени 3 при делении на 7: 3, 2, 6, 4, 5, 1, 3, 2, ... Цикл состоит из 6 остатков, остаток 5 является в нём пятым. Значит, надо найти наибольшее двузначное число, дающее при делении на 6 остаток 5, это число 95.\\
70. Заметим, что $\cfrac{2}{m}\leqslant2,$ а $\cfrac{1}{n-1}\leqslant1.$ Значит, если их сумма равна 3, эти дроби в точности равны 2 и 1, тогда $m=1,\ n=2.$\\
71. Заметим, что $\cfrac{2}{m-1}\leqslant2,$ а $\cfrac{1}{n}\leqslant1.$ Значит, если их сумма равна 3, эти дроби в точности равны 2 и 1, тогда $m=2,\ n=1.$\\
72. $\cfrac{12n+70}{4n+11}=\cfrac{12n+33+37}{4n+11}=3+\cfrac{37}{4n+11}.$ Если это выражение является целым числом, 37 делится на $4n+11,$ тогда $4n+11=\pm37$ или $4n+11=\pm1,$ поэтому $n\in\{-12;-3\}.$\\
73. $\cfrac{15n+58}{5n+9}=\cfrac{15n+27+31}{5n+9}=3+\cfrac{31}{5n+9}.$ Если это выражение является целым числом, 31 делится на $5n+9,$ тогда $5n+9=\pm31$ или $5n+9=\pm1,$ поэтому $n\in\{-8;-2\}.$\\
74. Заметим, что $\cfrac{25}{4}=5\cdot\cfrac{5}{4},\ \cfrac{15}{4}=3\cdot\cfrac{5}{4},\ \cfrac{5}{2}=2\cdot\cfrac{5}{4}.$ Значит, модули всех чисел $x,\ y$ и $z$ увеличились в $\cfrac{5}{4}$ раза, поэтому значения их квадратов увеличились в  $\left(\cfrac{5}{4}\right)^2=\cfrac{25}{16}$ раз и значение выражения равно $16\cdot
\cfrac{25}{16}=25.$\\
75. Заметим, что $10=4\cdot\cfrac{5}{2},\ \cfrac{15}{2}=3\cdot\cfrac{5}{2},\ 5=2\cdot\cfrac{5}{2}.$ Значит, модули всех чисел $x,\ y$ и $z$ увеличились в $\cfrac{5}{2}$ раза, поэтому значения их квадратов увеличились в  $\left(\cfrac{5}{2}\right)^2=\cfrac{25}{4}$ раз и значение выражения равно $4\cdot
\cfrac{25}{4}=25.$\\
76. $(x-2)(y+3)=2\Leftrightarrow \left[\begin{array}{l}\begin{cases}x-2=2,\\ y+3=1.\end{cases}\\ \begin{cases}x-2=-2,\\ y+3=-1.\end{cases}\\
\begin{cases}x-2=1,\\ y+3=2.\end{cases}\\ \begin{cases}x-2=-1,\\ y+3=-2.\end{cases}\end{array}\right.\Leftrightarrow
\left[\begin{array}{l}\begin{cases}x=4,\\ y=-2.\end{cases}\\ \begin{cases} x=0,\\ y=-4.\end{cases}\\
\begin{cases}x=3,\\ y=-1.\end{cases}\\ \begin{cases}x=1,\\ y=-5.\end{cases}\end{array}\right.$\\
77. $(x+2)(y-3)=2\Leftrightarrow \left[\begin{array}{l}\begin{cases}x+2=2,\\ y-3=1.\end{cases}\\ \begin{cases}x+2=-2,\\ y-3=-1.\end{cases}\\
\begin{cases}x+2=1,\\ y-3=2.\end{cases}\\ \begin{cases}x+2=-1,\\ y-3=-2.\end{cases}\end{array}\right.\Leftrightarrow
\left[\begin{array}{l}\begin{cases}x=0,\\ y=4.\end{cases}\\ \begin{cases} x=-4,\\ y=2.\end{cases}\\
\begin{cases}x=-1,\\ y=5.\end{cases}\\ \begin{cases}x=-3,\\ y=1.\end{cases}\end{array}\right.$\\
78. $a\cdot29+b\cdot30+c\cdot31=366,$ тогда $a$ и $b$ не превосходят 12, а $c$ не превосходит 11 (так как все числа должны быть натуральными, а значит положительными). Перепишем уравнение в следующем виде: $30(a+b+c)=366+a-c.$ Если $a+b+c\geqslant 13,$ то $a-c\geqslant390-366=23,$ что противоречит тому, что все числа не превосходят 12. Если $a+b+c\leqslant 11,$ то $a-c\leqslant 330-366=-36,$ что также противоречит тому, что числа не превосходят 12. Значит, $a+b+c=12,$ тогда $a-c=-6,$ то есть $a=c-6.$ Подставим это в первое равенство: $b+2c-6=12,\ b+2c=18.$ Так как все числа натуральны, возможны только два случая: $c=7$ или $c=8,$ иначе $a$ либо $b$ окажутся неположительными. Таким образом, либо $a=1,\ b=4,\ c=7,$ либо $a=2,\ b=2,\ c=8.$\\
79. $1394=2\cdot17\cdot41,\ 1581=3\cdot17\cdot31,$ значит наибольший общий делитель равен 17.\\
80. $1378=2\cdot13\cdot53,\ 1599=3\cdot13\cdot41,$ значит наибольший общий делитель равен 13.\\
81. Первого дежурного выбрать 30 способов, а второго --- 29, при этом способ, когда мы выбрали сначала Васю, а потом Петю, и способ, когда мы выбрали сначала Петю, а потом Васю, является одним и тем же способом. Значит, всего способов $\cfrac{30\cdot29}{2}=435.$\\
82. Первого дежурного выбрать 20 способов, а второго --- 19, при этом способ, когда мы выбрали сначала Васю, а потом Петю, и способ, когда мы выбрали сначала Петю, а потом Васю, является одним и тем же способом. Значит, всего способов $\cfrac{20\cdot19}{2}=190.$\\
83. В этой алгебраической сумме 4 чётных и 5 нечётных чисел, поэтому её значение всегда будет нечётно и 20 быть равно не может.\\
84. В этой алгебраической сумме 4 чётных и 5 нечётных чисел, поэтому её значение всегда будет нечётно и 30 быть равно не может.\\
85. Пусть это было число $\overline{abc}=100a+10b+c,$ тогда $100a+10b+c-a-b-c=186,\ 99a+9b=186,\ 33a+3b=62.$ Левая часть делится на 3, а правая --- нет, значит такого числа не существует.\\
86. $x^2+2xy+y^2-z^2=5\Leftrightarrow (x+y)^2-z^2=5\Leftrightarrow (x+y-z)(x+y+z)=5\Leftrightarrow \begin{cases} x+y-z=1,\\ x+y+z=5.\end{cases}
\Leftrightarrow \begin{cases} 2(x+y)=6,\\ x+y+z=5.\end{cases}
\Leftrightarrow \begin{cases} x+y=3,\\ z=2.\end{cases}\Rightarrow
(x;y;z)\in \{(1;2;2),(2;1;2)\}.$\\
87. $x^2+2xy+y^2-z^2=7\Leftrightarrow (x+y)^2-z^2=7\Leftrightarrow (x+y-z)(x+y+z)=7\Leftrightarrow \begin{cases} x+y-z=1,\\ x+y+z=7.\end{cases}
\Leftrightarrow \begin{cases} 2(x+y)=8,\\ x+y+z=7.\end{cases}
\Leftrightarrow \begin{cases} x+y=4,\\ z=3.\end{cases}\Rightarrow
(x;y;z)\in \{(1;3;3),(3;1;3),(2;2;3)\}.$
88. а) $a=121=11^2,\ b=361=19^2,\ c=5041=71^2.$\\
б) $n=x(x+1)(x+2)(x+3)+1=(x^2+3x)(x^2+3x+2)+1=[t=x^2+3x]=t(t+2)+1=t^2+2t+1=(t+1)^2.$\\
89. а) $a=121=11^2,\ b=361=19^2,\ c=5041=71^2.$\\
б) $n=(x+3)(x+2)(x+1)x+1=(x^2+3x)(x^2+3x+2)+1=[t=x^2+3x]=t(t+2)+1=t^2+2t+1=(t+1)^2.$\\
90. Если противоядие в колбе №1 или №2, то правдивы надписи на 3 и 5 колбах, что невозможно. Если противоядие в колбе №3, то правдивы надписи на второй и пятой колбах, что невозможно. Если противоядие в колбе №5, то правдивы надписи на второй, третьей и четвёртой колбах, что невозможно. И, наконец, если противоядие в колбе №4, то правдива только надпись на второй колбе, что соответствует условию задачи.\\
91. Число делится на 6 тогда и только тогда, когда оно делится на 2 и на 3, поэтому цифра 7 может стоять только на первом или втором месте. При этом последняя цифра должна быть чётной, а сумма цифр должна делиться на 3. Подходящими являются числа с 7 на первом месте: 720, 750, 780, 702, 732, 762, 792, 714, 744, 774, 726, 756, 786, 708, 738, 768, 798 и числа с 7 на втором месте, получающиеся из всех вышеперечисленных чисел, кроме 702, 708 и 774, перестановкой первых двух цифр. Значит, всего подходящих чисел $17+14=31.$\\
92. Число делится на 6 тогда и только тогда, когда оно делится на 2 и на 3, поэтому цифра 5 может стоять только на первом или втором месте. При этом последняя цифра должна быть чётной, а сумма цифр должна делиться на 3. Подходящими являются числа с 5 на первом мест: 510, 540, 570, 522, 552, 582, 504, 534, 564, 594, 516, 546, 576, 528, 558, 588 и числа с 5 на втором месте, получающиеся из всех вышеперечисленных чисел, кроме 504, 552 и 558, перестановкой первых двух цифр. Значит, всего подходящих чисел $16+13=29.$\\
93. $x(x+2)=y^2+30\Leftrightarrow x^2+2x+1-y^2=31\Leftrightarrow (x+1)^2-y^2=31\Leftrightarrow (x+1-y)(x+1+y)=31.$ Так как число 31 является простым, а числа $x$ и $y$ натуральными, возможен только один случай $\begin{cases}x+1-y=1,\\ x+1+y=31\end{cases}\Leftrightarrow \begin{cases}y=x,\\ 2x=30\end{cases}\Leftrightarrow
x=y=15.$\\
94. $y(y-2)=x^2+28\Leftrightarrow y^2-2y+1-x^2=29\Leftrightarrow (y-1)^2-x^2=29\Leftrightarrow (y-1-x)(y-1+x)=29.$ Так как число 29 является простым, а числа $x$ и $y$ натуральными, возможен только один случай $\begin{cases}y-1-x=1,\\ y-1+x=29\end{cases}\Leftrightarrow \begin{cases}y=x+2,\\ 2x=28\end{cases}\Leftrightarrow
\begin{cases}y=16,\\ x=14.\end{cases}$\\
95. а) Да, например $2015+8=2023.$\\
б) Пусть $\overline{xyz}+\overline{abc}=799,$ где все цифры могут быть равны нулю (числа не обязаны быть оба трёхзначными). Тогда $100x+10y+z+100a+10b+c=799,\
100(x+a)+10(y+b)+z+c=799,$ откуда $z+c=9,\ y+b=9,\ x+a=7.$ Значит, $a+b+c=7-x+9-y+9-z=25-(x+y+z)=x+y+z,\ x+y+z=\cfrac{25}{2},$ что невозможно.\\
в) Так как любое число даёт при делении на 9 такой же остаток, как и его сумма цифр, остатки всех пяти чисел равны, поэтому каждое из чисел больше предыдущего хотя бы на 9. Если самое маленькое из чисел равно $x,$ то их сумма не меньше, чем $x+x+9+x+18+x+27+x+36=5x+90.$ При этом, если $x=1,$ то минимальное значение суммы равно $1+10+100+1000+10000=11111,$ если $x=2,$ то минимальное значение суммы равно $2+11+20+101+110=244,$ если $x=3,$ то минимальное значение суммы равно $3+12+21+30+102=168,$ а если $x=4,$ то минимальное значение суммы равно $4+13+22+31+40=110.$ При $x\geqslant5$ имеем неравенство $5x+90>110,$ значит число 110 является наименьшим натуральным числом, представимым в виде суммы пяти различных натуральных чисел с одинаковой суммой цифр.\\
96. а) Да, например $2014+7=2021.$\\
б) Пусть $\overline{xyz}+\overline{abc}=599,$ где все цифры могут быть равны нулю (числа не обязаны быть оба трёхзначными). Тогда $100x+10y+z+100a+10b+c=599,\
100(x+a)+10(y+b)+z+c=599,$ откуда $z+c=9,\ y+b=9,\ x+a=5.$ Значит, $a+b+c=5-x+9-y+9-z=23-(x+y+z)=x+y+z,\ x+y+z=\cfrac{23}{2},$ что невозможно.\\
в) Так как любое число даёт при делении на 9 такой же остаток, как и его сумма цифр, остатки всех пяти чисел равны, поэтому каждое из чисел больше предыдущего хотя бы на 9. Если самое маленькое из чисел равно $x,$ то их сумма не меньше, чем $x+x+9+x+18+x+27+x+36+x+45=6x+135.$ При этом, если $x=1,$ то минимальное значение суммы равно $1+10+100+1000+10000+100000=111111,$ если $x=2,$ то минимальное значение суммы равно $2+11+20+101+110+200=444,$ если $x=3,$ то минимальное значение суммы равно $3+12+21+30+102+120=288,$ если $x=4,$ то минимальное значение суммы равно $4+13+22+31+40+103=213,$ если $x=5,$ то минимальное значение суммы равно
$5+14+23+32+41+50=165.$ При $x\geqslant6$ имеем неравенство $6x+135>165,$ значит число 165 является наименьшим натуральным числом, представимым в виде суммы шести различных натуральных чисел с одинаковой суммой цифр.\\
97. а) Да, могло. Например, если числа записаны в порядке 9, 16, 15, 14, 13, 12, 11, 18, 17, 10.\\
б) Всего по кругу записано 10 чисел. Для каждой пары соседних чисел мы ищем наибольший общий делитель, следовательно, получим 10 наибольших общих делителей. Если они все попарно различны, то хотя бы один из них не меньше 10. Но такого быть не может, так как для данных чисел наибольший из всевозможных наибольших общих делителей есть НОД$(18,9) = 9.$\\
98. Пусть двузначное число записывается как $\overline{ab},$ тогда имеет место равенство $10a+b=4(a+b),\ 10a+b=4a+4b,\ 6a=3b,\ 2a=b.$ Так как цифры не могут быть больше 9, цифра $a$ не больше 4. Таким образом, условию удовлетворяют только числа 12, 24, 36, 48.\\
99. $\cfrac{7n-11}{8-n}=-7+\cfrac{45}{8-n},$ поэтому число 45 должно делиться на $8-n$ и при этом результат деления должен быть больше 7, чтобы итоговый результат был положительным. Это возможно при $8-n=1,\ 8-n=3,\ 8-n=5,$ то есть $n\in\{3;5;7\}.$\\
100. $\cfrac{8n-13}{11-n}=-8+\cfrac{75}{11-n},$ поэтому число 75 должно делиться на $11-n$ и при этом результат деления должен быть больше 8, чтобы итоговый результат был положительным. Это возможно при $11-n=1,\ 11-n=3,\ 11-n=5,$ то есть $n\in\{6;8;10\}.$\\
101. Во-первых, подходят все числа, в которых есть хотя бы одна цифра 9 (её можно поставить на первое место). Чтобы узнать их количество, вычтем из общего количества чисел те числа, в которых цифры 9 нет. Таких чисел $8\cdot9\cdot9=648$ (на первое место нельзя ставить 9 и 0, на два других --- только 9). Во-вторых, в числе может быть две цифры 8 (их можно поставить на первые 2 места). Сначала посчитаем числа, в которых ровно две 8: если они на первых двух местах, то на третье можно поставить 8 цифр (без 8 и 9), если на первом и третьем, то на второе место также можно поставить 8 цифр, а если на двух последних местах, то на первое можно поставить 7 цифр (без 8, 9 и 0). Также есть подходящее число 888.
Таким образом, всего подходящих чисел $900-648+8+8+7+1=276.$\\
102. Во-первых, подходят все числа, в которых есть хотя бы одна цифра 8 или 9 (её можно поставить на первое место). Чтобы узнать их количество, вычтем из общего количества чисел те числа, в которых цифр 8 и 9 нет. Таких чисел $7\cdot8\cdot8=448$ (на первое место нельзя ставить 8, 9 и 0, на два других --- только 8 и 9). Во-вторых, в числе может быть две цифры 7 (их можно поставить на первые 2 места). Сначала посчитаем числа, в которых ровно две 7: если они на первых двух местах, то на третье можно поставить 7 цифр (без 7, 8 и 9), если на первом и третьем, то на второе место также можно поставить 7 цифр, а если на двух последних местах, то на первое можно поставить 6 цифр (без 7, 8, 9 и 0). Также подходит число 777.
Таким образом, всего подходящих чисел $900-448+7+7+6+1=473.$
\newpage
