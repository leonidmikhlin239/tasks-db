78. $a\cdot29+b\cdot30+c\cdot31=366,$ тогда $a$ и $b$ не превосходят 12, а $c$ не превосходит 11 (так как все числа должны быть натуральными, а значит положительными). Перепишем уравнение в следующем виде: $30(a+b+c)=366+a-c.$ Если $a+b+c\geqslant 13,$ то $a-c\geqslant390-366=23,$ что противоречит тому, что все числа не превосходят 12. Если $a+b+c\leqslant 11,$ то $a-c\leqslant 330-366=-36,$ что также противоречит тому, что числа не превосходят 12. Значит, $a+b+c=12,$ тогда $a-c=-6,$ то есть $a=c-6.$ Подставим это в первое равенство: $b+2c-6=12,\ b+2c=18.$ Так как все числа натуральны, возможны только два случая: $c=7$ или $c=8,$ иначе $a$ либо $b$ окажутся неположительными. Таким образом, либо $a=1,\ b=4,\ c=7,$ либо $a=2,\ b=2,\ c=8.$\\
