62. $\cfrac{x^2-4x+7}{x^2-4x+5}=\cfrac{x^2-4x+5+2}{x^2-4x+5}=1+\cfrac{2}{x^2-4x+5}.$ Значение этого выражения будет наибольшим, когда значение дроби будет наибольшим. Значение дроби будет наибольшим, когда её знаменатель будет наименьшим. В знаменателе находится квадратичная функция, наименьшее значение которой достигается в вершине параболы $x=-\cfrac{-4}{2}=2.$ При $x=2$ значение выражения равно $1+\cfrac{2}{4-8+5}=3.$\\