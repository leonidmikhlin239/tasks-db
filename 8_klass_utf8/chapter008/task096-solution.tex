96. а) Да, например $2014+7=2021.$\\
б) Пусть $\overline{xyz}+\overline{abc}=599,$ где все цифры могут быть равны нулю (числа не обязаны быть оба трёхзначными). Тогда $100x+10y+z+100a+10b+c=599,\
100(x+a)+10(y+b)+z+c=599,$ откуда $z+c=9,\ y+b=9,\ x+a=5.$ Значит, $a+b+c=5-x+9-y+9-z=23-(x+y+z)=x+y+z,\ x+y+z=\cfrac{23}{2},$ что невозможно.\\
в) Так как любое число даёт при делении на 9 такой же остаток, как и его сумма цифр, остатки всех пяти чисел равны, поэтому каждое из чисел больше предыдущего хотя бы на 9. Если самое маленькое из чисел равно $x,$ то их сумма не меньше, чем $x+x+9+x+18+x+27+x+36+x+45=6x+135.$ При этом, если $x=1,$ то минимальное значение суммы равно $1+10+100+1000+10000+100000=111111,$ если $x=2,$ то минимальное значение суммы равно $2+11+20+101+110+200=444,$ если $x=3,$ то минимальное значение суммы равно $3+12+21+30+102+120=288,$ если $x=4,$ то минимальное значение суммы равно $4+13+22+31+40+103=213,$ если $x=5,$ то минимальное значение суммы равно
$5+14+23+32+41+50=165.$ При $x\geqslant6$ имеем неравенство $6x+135>165,$ значит число 165 является наименьшим натуральным числом, представимым в виде суммы шести различных натуральных чисел с одинаковой суммой цифр.\\