97. а) Да, могло. Например, если числа записаны в порядке 9, 16, 15, 14, 13, 12, 11, 18, 17, 10.\\
б) Всего по кругу записано 10 чисел. Для каждой пары соседних чисел мы ищем наибольший общий делитель, следовательно, получим 10 наибольших общих делителей. Если они все попарно различны, то хотя бы один из них не меньше 10. Но такого быть не может, так как для данных чисел наибольший из всевозможных наибольших общих делителей есть НОД$(18,9) = 9.$\\
