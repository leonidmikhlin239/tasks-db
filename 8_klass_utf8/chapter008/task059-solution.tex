59. $\cfrac{5x^2+10x+14}{x^2+2x+3}=\cfrac{5x^2+10x+15-1}{x^2+2x+3}=5-\cfrac{1}{x^2+2x+3}.$ Значение этого выражения будет наименьшим, когда значение дроби будет наибольшим. Значение дроби будет наибольшим, когда её знаменатель будет наименьшим. В знаменателе находится квадратичная функция, наименьшее значение которой достигается в вершине параболы $x=-\cfrac{2}{2}=-1.$ При $x=-1$ значение выражения равно $5-\cfrac{1}{1-2+3}=\cfrac{9}{2}.$\\