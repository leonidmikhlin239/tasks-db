\section{Нестандартные задачи}
1. Наибольший общий делитель целых чисел a и b равен 1. Доказать, что
НОД$(a, a+b)=1.$\\
2. Наибольший общий делитель целых чисел a и b равен 1. Доказать, что
НОД$(a, a-b)=1.$\\
3. Найдите все натуральные $n,$ при которых число $\cfrac{3n-1}{n+1}$ является целым.\\
4. Найдите все натуральные $n,$ при которых число $\cfrac{3n+1}{n-1}$ является целым.\\
5. В магазине продаются раки: маленькие --- по 5 рублей, большие --- по 7 рублей. Сколько маленьких и больших раков можно купить на 101 рубль ровно?\\
6. В магазине продаются раки: маленькие --- по 5 рублей, большие --- по 8 рублей. Сколько маленьких и больших раков можно купить на 116 рублей ровно?\\
7. При каких целых $n$ число $\cfrac{4n-5}{2n-1}$ будет целым?\\
8. При каких целых $n$ число $\cfrac{4n+5}{2n+1}$ будет целым?\\
9. Сколько различных диагоналей можно провести в выпуклом семиугольнике?\\
10. Сколько различных диагоналей можно провести в выпуклом восьмиугольнике?\\
11. Найти наибольшее двузначное натуральное число $n$ такое, что $\cfrac{3n+17}{n+4}$ --- сократимая дробь.\\
12. Найти наибольшее двузначное натуральное число $n$ такое, что $\cfrac{3n+16}{n+4}$ --- сократимая дробь.\\
13. На столе лежат книги, число которых меньше 100. Сколько лежит книг, если их можно без остатка связать в пачки как по 3, так и по 4 и по 5 книг?\\
14. На столе лежат книги, число которых меньше 100. Сколько лежит книг, если их можно без остатка связать в пачки как по 3, так и по 4 и по 7 книг?\\
15. Решить уравнение в целых числах: $2x^2+xy=x+7.$\\
16. Решить уравнение в целых числах: $x^2-3xy=x-3y+2.$\\
17. Найдите наименьшее трёхзначное число, сумма цифр которого равна 22.\\
18. Найдите наименьшее трёхзначное число, сумма цифр которого равна 23.\\
19. <<Зенит>>, <<Спартак>> и ЦСКА стали призёрами первенства России по футболу. Сколькими способами они могут расположиться на первых трёх местах?\\
20. Сколькими способами можно раздать яблоко, мандарин и грушу Пете, Саше и Гале так, чтобы каждому досталось по одному фрукту?\\
21. Пусть $a$ --- чётное число, не кратное 6. Найти остаток от деления числа $a^2$ на 12.\\
22. Пусть $a$ --- целое число, не кратное 3. Найти остаток от деления числа $a^2$ на 3.\\
23. Число $n$ при делении на 5 даёт в остатке 4. Найти остаток от деления числа $(4n+3)$ на 5.\\
24. Число $n$ при делении на 4 даёт в остатке 3. Найти остаток от деления числа $(3n+2)$ на 4.\\
25. Сколько двузначных чисел делятся без остатка на 4 и на 6?\\
26. Сколько двузначных чисел делятся без остатка на 6 и на 9?\\
27. Решить уравнение $x^{1024}+1=\cfrac{1}{2+x^{318}}.$\\
28. Решить уравнение $x^{1024}+2=\cfrac{1}{1+x^{318}}.$\\
29. Решить в целых числах $x\cdot (y-2)=3,$ если известно, что $x<0.$\\
30. Решить в целых числах $x\cdot (y-4)=3,$ если известно, что $x<0.$\\
31. Из Петербурга в Москву можно проехать двумя способами, а из Москвы в Братск четырьмя способами. Сколькими способами можно проехать из Петербурга в Братск?\\
32. Из Петербурга в Москву можно проехать двумя способами, а из Москвы в Хабаровск пятью способами. Сколькими способами можно проехать из Петербурга в Хабаровск?\\
33. В шахматном турнире принимают участие 7 игроков. Сколько нужно сыграть игр, чтобы каждый шахматист сыграл с каждым?\\
34. В январе было 5 понедельников. Какое наибольшее число четвергов могло быть в этом январе?\\
35. В марте было 5 вторников. Какое наибольшее число пятниц могло быть в этом марте?\\
36. В кубе с ребром 10 см все грани покрасили в разные цвета. Затем куб разрезали на 1000 кубиков со стороной в 1 см. Сколько получилось кубиков, у которых ровно две грани окрашены в разные цвета?\\
37. В кубе с ребром 1 м все грани покрасили в разные цвета. Затем куб разрезали на 1000 кубиков со стороной в 10 см. Сколько получилось кубиков, у которых ровно две грани окрашены в разные цвета?\\
38. В карточной колоде 36 карт, по девять каждой масти. Мы берём двух королей. Сколько различных пар мы можем получить?\\
39. В карточной колоде 36 карт, по девять каждой масти. Мы берём двух валетов. Сколько различных пар мы можем получить?\\
40. Наименьшее общее кратное чисел $a$ и $b$ равно $\cfrac{ab}{3}.$ Найдите их наибольший общий делитель.\\
41. Наименьшее общее кратное чисел $a$ и $b$ равно $\cfrac{ab}{5}.$ Найдите их наибольший общий делитель.\\
42. На окружности взяли 7 точек и провели через них всевозможные хорды. Сколько всего хорд провели?\\
43. На окружности взяли 6 точек и провели через них всевозможные хорды. Сколько всего хорд провели?\\
44. Произвольному многочлену $P(x)$ ставится в соответствие многочлен $P'(x)$ так, что выполнялись следующие правила:\\
1. Для любых двух многочленов $P_1(x)$ и $P_2(x): \left(P_1(x)+P_2(x)
ight)'=P_1'(x)+P_2'(x);$\\
2. Для любого числа $a$ и многочлена $P(x): (a\cdot P(x))'=a\cdot P'(x);$\\
3. Если $P(x)=x^n,$ то $P'(x)=n\cdot x^{n-1}.$\\
Найдите $P'(x),$ если:\\
а) $P(x)=3x^5+4x^3;$\\
б) $P(x)=1$ для любого $x,$ т.е. $P(x)\equiv 1;$\\
в) $P(x)=(T'(x)+3xT(x))'-18x^2-45x^4,$ где $T(x)=3x^4-2x^2.$\\
45. Произвольному многочлену $P(x)$ ставится в соответствие многочлен $P'(x)$ так, что выполнялись следующие правила:\\
1. Для любых двух многочленов $P_1(x)$ и $P_2(x): \left(P_1(x)+P_2(x)
ight)'=P_1'(x)+P_2'(x);$\\
2. Для любых двух многочленов $P_1(x)$ и $P_2(x): (P_1(x)\cdot P_2(x))'=P_1'(x)\cdot P_2(x)+P_1(x)\cdot P_2'(x);$\\
3. Для любого числа $a$ и многочлена $P(x): (a\cdot P(x))'=a\cdot P'(x);$\\
4. Если $P(x)=x,$ то $P'(x)=1.$\\
5. Если $P(x)=x^3,$ то $P'(x)=3\cdot x^2.$\\
а) Найдите $P'(x),$ если $P(x)=3x^3+4x;$\\
б) Найдите $P'(x),$ если $P(x)=x^6-2x^4;$\\
в) Найдите $P(x),$ если $P(x)=(T(x)+3xT(x))'-9x^2-36x^3,$ где $T(x)=3x^3-2x.$\\
46. Сколько трёхзначных нечётных чисел можно составить из цифр 0, 1, 4, 5, если цифры числа могут повторяться?\\
47. Сколько трёхзначных чётных чисел можно составить из цифр 0, 1, 4, 5, если цифры числа могут повторяться?\\
48. Бросаются 2 игральные кости. Найдите вероятность того, что сумма выпавших очков равна 10.\\
49. Бросаются 2 игральные кости. Найдите вероятность того, что сумма выпавших очков равна 4.\\
50. В шахматном турнире принимали участие 15 шахматистов, причём каждый из них сыграл только одну партию с каждым из остальных. Сколько всего партий было сыграно в этом турнире?\\
51. В шахматном турнире принимали участие 14 шахматистов, причём каждый из них сыграл только одну партию с каждым из остальных. Сколько всего партий было сыграно в этом турнире?\\
52. Конференция длится три дня. В первый и второй день выступают по 15 докладчиков, в третий день --- 20. Какова вероятность того, что доклад профессора М. выпадет на третий день, если порядок докладов определяется жеребьёвкой?\\
53. Конференция длится три дня. В первый и второй день выступают по 20 докладчиков, в третий день --- 10. Какова вероятность того, что доклад профессора М. выпадет на третий день, если порядок докладов определяется жеребьёвкой?\\
54. Остаток от деления числа $a$ на 3 равен 1. Найти остаток от деления числа $a^2$ на 3.\\
55. Остаток от деления числа $a$ на 5 равен 1. Найти остаток от деления числа $a^2$ на 5.\\
56. Введём новое число $\Game,$ такое, что $\Game^2=-1,$ а все остальные арифметические операции с ним и другими числами происходят как обычно. Говорят, что выражение имеет {\it не раздражающий} вид, если оно имеет вид $a+b\cdot\Game,$ где $a$ и $b$ вещественные числа.

Запишите в {\it не раздражающем} виде следующее выражение:
$$(4\cdot\Game-1)\cdot(4\cdot\Game+1)-17\cdot\Game\cdot(\Game-2).$$
57. Введём новое число $\Game,$ такое, что $\Game^2=-1,$ а все остальные арифметические операции с ним и другими числами происходят как обычно. Говорят, что выражение имеет {\it не раздражающий} вид, если оно имеет вид $a+b\cdot\Game,$ где $a$ и $b$ вещественные числа.

Запишите в {\it не раздражающем} виде следующее выражение:
$$(3\cdot\Game-1)\cdot(3\cdot\Game+1)-10\cdot\Game\cdot(\Game-2).$$
58. Числитель дроби $\cfrac{k}{n}$ (здесь $k$ и $n$ --- натуральные числа) увеличили на 1, а её знаменатель --- на 2. Выясните, будет ли полученная дробь меньше или же больше исходной.\\
59. Найдите наименьшее значение выражения $\cfrac{5x^2+10x+14}{x^2+2x+3}.$\\
60. Найдите наибольшее значение выражения $\cfrac{x^2+2x+5}{x^2+2x+3}.$\\
61. Красные карандаши стоят 17 руб за штуку, синие --- 13 руб. Нужно купить карандаши на сумму 495 рублей, причём число красных не должно отличаться от числа синих более чем на 5 штук.\\
а) Можно ли купить 32 карандаша? (ответ обосновать)\\
б) Можно ли купить 35 карандашей? (ответ обосновать)\\
в) Какое наибольшее количество карандашей можно купить при таких условиях? (ответ обосновать)\\
62. Найти наибольшее значение выражения $\cfrac{x^2-4x+7}{x^2-4x+5}.$\\
63. Натуральные числа $a$ и $b$ таковы, что $19a=97b.$ Докажите, что их сумма $(a+b)$ делится на 116.\\
64. Найдите $\cfrac{a}{b+c}+\cfrac{b}{a+c}+\cfrac{c}{a+b},$ если $a+b+c=7$ и $\cfrac{1}{b+c}+\cfrac{1}{a+c}+\cfrac{1}{a+b}=\cfrac{7}{10}.$\\
65. Найдите все целые значения $n$ таких, что $\sqrt{n^2-17}$ --- целое число.\\
66. Найдите количество различных делителей числа $6^{15}\cdot21^{7}.$\\
67. Если между цифрами двузначного числа $a$ вписать это же число, то полученное четырёхзначное число будет в 99 раз больше этого двузначного числа. Найдите двузначное число $a.$\\
68. На странице во всех строках одно и то же число букв. Если увеличить число строк и число букв в строке на 7, то число букв на странице увеличится на 476. На сколько уменьшится число букв на странице, если уменьшить число строк и число букв в строке на 4?\\
69. Найдите наибольшее двузначное число $n$ при котором остаток от деления числа $3^n$ на 7 равен 5, если такое число $n$ существует.\\
70. При каких натуральных $m$ и $n$ выполнено равенство $\cfrac{2}{m}+\cfrac{1}{n-1}=3?$\\
71. При каких натуральных $m$ и $n$ выполнено равенство $\cfrac{2}{m-1}+\cfrac{1}{n}=3?$\\
72. Найдите все целые значения $n,$ при каждом из которых значение выражения $\cfrac{12n+70}{4n+11}$ является целым числом.\\
73. Найдите все целые значения $n,$ при каждом из которых значение выражения $\cfrac{15n+58}{5n+9}$ является целым числом.\\
74. Значение выражения $ax^2+by^2+cz^2$ при $x=5,\ y=-3,\ z=-2$ равно 16. Найдите значение данного выражения при $x=\cfrac{25}{4},\ y=-\cfrac{15}{4},\ z=-\cfrac{5}{2}.$\\
75. Значение выражения $ax^2+by^2+cz^2$ при $x=4,\ y=-3,\ z=2$ равно 4. Найдите значение данного выражения при $x=10,\ y=-\cfrac{15}{2},\ z=5.$\\
76. Решите уравнение в целых числах: $(x-2)(y+3)=2.$\\
77. Решите уравнение в целых числах: $(x+2)(y-3)=2.$\\
78. Найдите такие натуральные $a,\ b$ и $c,$ что $a\cdot29+b\cdot30+c\cdot31=366.$\\
79. Найти наибольший общий делитель чисел 1394 и 1581.\\
80. Найти наибольший общий делитель чисел 1378 и 1599.\\
81. Сколькими способами можно выбрать двух дежурных из 30 человек в классе?\\
82. Сколькими способами можно выбрать двух дежурных из 20 человек в классе?\\
83. Звёздочкой обозначают знаки <<$+$>> или <<$-$>> совершенно произвольно. Может ли выполняться равенство $1*2*3*4*5*6*7*8*9=20?$\\
84. Звёздочкой обозначают знаки <<$+$>> или <<$-$>> совершенно произвольно. Может ли выполняться равенство $1*2*3*4*5*6*7*8*9=30?$\\
85. Из трёхзначного числа вычли сумму его цифр. Может ли разность оказаться равной 186?\\
86. Найдите все тройки натуральных чисел $(x;y;z),$ для которых $x^2+2xy+y^2-z^2=5.$\\
87. Найдите все тройки натуральных чисел $(x;y;z),$ для которых $x^2+2xy+y^2-z^2=7.$\\
88. а) Покажите, что данные числа являются квадратами натуральных чисел:\\
$a=2\cdot3\cdot4\cdot5+1;\qquad b=3\cdot4\cdot5\cdot6+1;\qquad c=7\cdot8\cdot9\cdot10+1.$\\
б) Обобщите имеющуюся закономерность и докажите её.\\
89. а) Покажите, что данные числа являются квадратами натуральных чисел:\\
$a=1+5\cdot4\cdot3\cdot2;\qquad b=1+6\cdot5\cdot4\cdot3;\qquad c=1+10\cdot9\cdot8\cdot7.$\\
б) Обобщите имеющуюся закономерность и докажите её.\\
90. На полке в шкафу стоят 5 колб. Известно, что только в одной из них противоядие, в остальных же --- яд. Но только на одной из колб надпись правдива. В какой колбе находится противоядие? Надписи на колбах по порядку: <<В №2 --- противоядие>>, <<Тут и в №1 --- яд>>, <<Тут и в №4 --- яд>>, <<Противоядие в №3 или в №5>>,
<<Тут и в №4 --- яд>>.\\
91. Найдите количество трёхзначных чисел, делящихся на 6, в записи которых встречается цифра 7.\\
92. Найдите количество трёхзначных чисел, делящихся на 6, в записи которых встречается цифра 5.\\
93. Решите уравнение в целых числах $x(x+2)=y^2+30.$\\
94. Решите уравнение в целых числах $y(y-2)=x^2+28.$\\
95. а) Можно ли число 2023 представить в виде суммы двух натуральных чисел, суммы цифр которых равны?\\
б) Можно ли число 799 представить в виде суммы двух натуральных чисел, суммы цифр которых равны?\\
в) Найдите наименьшее натуральное число, которое можно представить в виде суммы пяти различных натуральных чисел, суммы цифр которых равны.\\
96. а) Можно ли число 2021 представить в виде суммы двух натуральных чисел, суммы цифр которых равны?\\
б) Можно ли число 599 представить в виде суммы двух натуральных чисел, суммы цифр которых равны?\\
в) Найдите наименьшее натуральное число, которое можно представить в виде суммы шести различных натуральных чисел, суммы цифр которых равны.\\
97. По кругу в некотором порядке по одному разу написаны натуральные числа от 9 до 18. Для каждой из десяти пар соседних чисел нашли их наибольший общий делитель. а) Могло ли получиться так, что все наибольшие общие делители равны 1? б) Могло ли получиться так, что все наибольшие общие делители попарно различны?\\
98. Найти все двузначные числа, которые в четыре раза больше суммы своих цифр.\\
99. При каких натуральных $n$ число $\cfrac{7n-11}{8-n}$ будет натуральным?\\
100. При каких натуральных $n$ число $\cfrac{8n-13}{11-n}$ будет натуральным?\\
101. Сколько существует трёхзначных чисел, из которых перестановками цифр можно получить число большее 879?\\
102. Сколько существует трёхзначных чисел, из которых перестановками цифр можно получить число большее 769?

ewpage
