102. Во-первых, подходят все числа, в которых есть хотя бы одна цифра 8 или 9 (её можно поставить на первое место). Чтобы узнать их количество, вычтем из общего количества чисел те числа, в которых цифр 8 и 9 нет. Таких чисел $7\cdot8\cdot8=448$ (на первое место нельзя ставить 8, 9 и 0, на два других --- только 8 и 9). Во-вторых, в числе может быть две цифры 7 (их можно поставить на первые 2 места). Сначала посчитаем числа, в которых ровно две 7: если они на первых двух местах, то на третье можно поставить 7 цифр (без 7, 8 и 9), если на первом и третьем, то на второе место также можно поставить 7 цифр, а если на двух последних местах, то на первое можно поставить 6 цифр (без 7, 8, 9 и 0). Также подходит число 777.
Таким образом, всего подходящих чисел $900-448+7+7+6+1=473.$
\newpage
