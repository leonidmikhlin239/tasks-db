75. $\sqrt{x-2}=100|x-3|+1.$ Правая часть уравнения не меньше 1, значит  $x-2\geqslant1,\ x\geqslant3.$ При $x=3$ левая и правая часть равны. При $x>3$ имеем
$\sqrt{x-2}=100x-300+1,\ \sqrt{x-2}=100x-299,\ x-2=100^2x^2-200\cdot299x+299^2,\ 100^2x^2-(200\cdot299+1)x+299^2+2=0.$ Один из корней этого уравнения, как мы уже знаем, равен 3, тогда второй по теореме Виета равен $\cfrac{299^2+2}{3\cdot100^2}<3,$ что противоречит предположению $x>3.$ Таким образом, $x=3$ --- единственное решение.\\
