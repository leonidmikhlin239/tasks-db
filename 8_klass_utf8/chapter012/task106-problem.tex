106. а) Так как $x^2-(\sqrt{3}+\sqrt{5})x+\sqrt{15}=(x-\sqrt{3})(x-\sqrt{5})=0,$ имеем $x=\sqrt{3}$ или $x=\sqrt{5}.$\\
б) Понятно, что $\sqrt{3}<2,$ а $\sqrt{5}>2.$ Тогда произведём сравнение $2-\sqrt{3}\ ??\ \sqrt{4}-2,\ 4\ ??\ \sqrt{3}+\sqrt{5},\ 16\ ??\ 3+2\sqrt{15}+5,\ 8\ ??\ 2\sqrt{15},\ 64>60.$ Значит, $\sqrt{5}$ находится ближе к числу 2.\\
