74. $\sqrt{2x-x^2}+\sqrt{x^2-3x+2}+10x=20\Leftrightarrow \sqrt{x(2-x)}+\sqrt{(x-2)(x-1)}+10x=20.$ Если $x>2$ или $x<0,$ то первое подкоренное выражение является отрицательным, а если $x\in (1;2),$ то второе. При этом $x=2$ является решением уравнения, а  $x=0$ или $x=1$ --- нет. Осталось разобрать только случай $x\in(0;1).$ В этом случае первые два слагаемых меньше $\sqrt{2},$ а третье меньше 10, значит их сумма не может быть равна 20. Таким образом, единственное решение --- $x=2.$\\
