95. а) $M=\left(\cfrac{1}{\sqrt{c}+3\sqrt{d}}-\cfrac{1}{\sqrt{c}}
ight):\left(\cfrac{6}{3\sqrt{c}-2\sqrt{d}}-\cfrac{22\sqrt{d}}{3c+7\sqrt{cd}-6d}+\cfrac{3}{\sqrt{c}+3\sqrt{d}}
ight)=
[x=\sqrt{c},\ y=\sqrt{d}]=
\left(\cfrac{1}{x+3y}-\cfrac{1}{x}
ight):\left(\cfrac{6}{3x-2y}-\cfrac{22y}{3x^2+7xy-6y^2}+\cfrac{3}{x+3y}
ight)=$\\$
\cfrac{x-x-3y}{x(x+3y)}:\left(\cfrac{6x+18y}{(3x-2y)(x+3y)}-\cfrac{22y}{(3x-2y)(x+3y)}+\cfrac{9x-6y}{(3x-2y)(x+3y)}
ight)=$\\$
\cfrac{-3y}{x(x+3y)}:\cfrac{15x-10y}{(3x-2y)(x+3y)}=\cfrac{-3y}{x(x+3y)}\cdot\cfrac{(3x-2y)(x+3y)}{5(3x-2y)}=-\cfrac{3y}{5x}=-\cfrac{3\sqrt{d}}{5\sqrt{c}}.$\\
б) $x=\sqrt{99}=3\sqrt{11},\ y=\sqrt{275}=5\sqrt{11},$ тогда $M=-\cfrac{15\sqrt{11}}{15\sqrt{11}}=-1.$\\
в) $x=\sqrt{76}=2\sqrt{19},\ y=\sqrt{171}=3\sqrt{19},$ тогда $3x-2y=6\sqrt{19}-6\sqrt{19}=0,$ поэтому значение $M$ неопределено (присутствует деление на 0).\\
