63. $x^3+6x^2+ax=0\Leftrightarrow x(x^2+6x+a)=0.$ У этого уравнения точно всегда есть корень $x=0.$ Два корня у него может быть, если один из корней уравнения
$x^2+6x+a=0$ совпадает с $x=0$ или уравнение $x^2+6x+a=0$ имеет один корень (не совпадающий с $x=0$). В первом случае $0+0+a=0,\ a=0.$ Во втором случае $D=36-4a=0,\ a=9.$ Корень $x=-3$ с корнем $x=0$ не совпадает.\\
