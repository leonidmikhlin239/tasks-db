17. Сначала разберём случай, когда уравнение не является квадратным: $k-1=0,\ k=1,\ -x+2=0,\ x=2.$ Теперь разберём случай, когда у уравнения один корень:
$(2k-1)^2-4\cdot2\cdot(k-1)=0,\ 4k^2-4k+1-8k+8=0,\ (2k-3)^2=0,\ k=\cfrac{3}{2}.$ В этом случае $x=\cfrac{2k-1}{2(k-1)}=\cfrac{2}{1}=2.$ Во всех остальных случаях $x=\cfrac{2k-1\pm|2k-3|}{2(k-1)}=\left[\begin{array}{l}\cfrac{2k-1+2k-3}{2(k-1)}=2,\\ \cfrac{2k-1-2k+3}{2(k-1)}=\cfrac{1}{k-1}.\end{array}
ight.$\\
