76. Для того, чтобы число 1 находилось между корнями уравнения, необходимо и достаточно выполнение двух условий. Во-первых, эти корней должно быть два, то есть $D>0.$ Во-вторых, выражение $(x_1-1)(x_2-1)$ должно быть отрицательно (это гарантирует, что скобки разных знаков, а значит один корень больше 1, а другой меньше). Запишем эти условия в виде системы неравенств и решим её, воспользовавшись теоремой Виета:\\
$\begin{cases}(a+1)^2+4a^2>0,\\ (x_1-1)(x_2-1)<0.\end{cases}\Leftrightarrow x_1x_2-(x_1+x_2)+1<0\Leftrightarrow
-a^2+a+1+1<0\Leftrightarrow a^2-a-2>0\Leftrightarrow (a-2)(a+1)>0\Leftrightarrow a\in (-\infty;-1)\cup(2;+\infty).$\\
