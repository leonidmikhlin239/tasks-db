136. По теореме Виета имеем соотношения $x_1+x_2=\cfrac{3}{2},\ x_1x_2=-4.$ Пусть $y_1=\cfrac{x_1-1}{x_2},\ y_2=\cfrac{x_2-1}{x_1},$ тогда выполняются равенства
$y_1+y_2=\cfrac{x_1-1}{x_2}+\cfrac{x_2-1}{x_1}=\cfrac{x_1^2-x_1+x_2^2-x_2}{x_1x_2}=$\\$\cfrac{(x_1+x_2)^2-2x_1x_2-(x_1+x_2)}{x_1x_2}=
\cfrac{\cfrac{9}{4}+8-\cfrac{3}{2}}{-4}=-\cfrac{35}{16},\ y_1y_2=\cfrac{(x_1-1)(x_2-1)}{x_1x_2}=\cfrac{x_1x_2-(x_1+x_2)+1}{x_1x_2}=
\cfrac{-4-\cfrac{3}{2}+1}{-4}=\cfrac{9}{8}=\cfrac{18}{16}.$ По обратной теореме Виета числа $y_1$ и $y_2$ являются корнями квадратного уравнения $16x^2+35x+18=0.$\\
