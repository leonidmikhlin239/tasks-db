97. $\begin{cases}
(k+2)x+3y=9+kx,\\
x+(k+4)y=2.
\end{cases}\Leftrightarrow\begin{cases}
2x+3y=9,\\
x+(k+4)y=2.
\end{cases}\Leftrightarrow\begin{cases}
2x+3y=9,\\
(2k+5)y=-5.
\end{cases}$
У этой системы не может быть бесконечно много решений: из второго уравнения однозначно определяется $y,$ а потом из первого уравнения определяется $x.$ Если $2k+5
eq0,$ то у системы одно решение, а при $2k+5=0$ --- ни одного.\\
