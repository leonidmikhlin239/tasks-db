134. У этого уравнения может быть только один корень в двух случаях. Во-первых, у числителя может быть один корень, не равный $-3.$ Для этого уравнение $(2q+1)x^2+(3q-2)x+q+2=0$ должно быть линейным или его дискриминант должен быть равен нулю. Уравнение линейно при $2q+1=0,\ q=-\cfrac{1}{2},$ тогда $-\cfrac{7}{2}x+\cfrac{3}{2}=0,\ x=\cfrac{3}{7},$ это значение $q$ подходит. Дискриминант равен нулю при $(3q-2)^2-4(2q+1)(q+2)=0,\ q^2-32q-4=0,\ q=16\pm2\sqrt{65}.$ При этих значениях $q$ единственный корень также не равен $-3.$ Во втором случае, подставив $-3$ в числитель и приравняв его к нулю, получим $18q+9-9q+6+q+2=0,\ 10q+17=0,\ q=-\cfrac{17}{10}.$ Все найденные значения $q$ подходят, таким образом $p\in\left\{16\pm2\sqrt{65};-\cfrac{1}{2};-\cfrac{17}{10}
ight\}.$\\
