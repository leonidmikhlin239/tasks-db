103.а)$g(x)=\cfrac{4\sqrt{x^2+4x+4}}{\sqrt{16+8x+x^2}+x}=\cfrac{4\sqrt{(x+2)^2}}{\sqrt{(x+4)^2}+x}=\cfrac{4|x+2|}{|x+4|+x}=
\begin{cases} \cfrac{-4(x+2)}{-x-4+x}=x+2,\ x<-4,\\
\cfrac{-4(x+2)}{x+4+x}=-2, -4\leqslant x < -2,\\
\cfrac{4(x+2)}{x+4+x}=2,\ -2<x.\end{cases}$
$$\begin{tikzpicture}[scale=0.3]
\tikzset {line01/.style={line width =0.5pt}}
\tikzset{line02/.style={line width =1pt}}
\tikzset{line03/.style={dashed,line width =0.5pt}}
%\filldraw [black] (0,0) circle (1pt);
\draw [->] (-8,0) -- (6,0);
\draw [->] (0,-8) -- (0,6);
\draw[line01] (-8,-6) -- (-4,-2);
\draw[line01] (-4,-2) -- (-2.3,-2);
\draw[line01] (-1.72,2) -- (6,2);
\draw[line03] (-2,1.8) -- (-2,-1.8);
\draw[line03] (-4,0) -- (-4,-1.8);
\draw[line03] (0,-2) -- (-1.8,-2);
%\draw[line03] (1,0) -- (1,2);
%\draw[line03] (0,2) -- (1,2);
%\draw[line03] (0,-2) -- (1,-2);
\draw (6.2,0.7) node {\scriptsize $x$};
%\draw (-1.2,-2) node {\scriptsize $-2$};
%\draw (-0.7,2) node {\scriptsize $2$};
%\draw (-0.7,1.2) node {\scriptsize $1$};
\draw (-1.7,0.5) node {\tiny $-2$};
\draw (-0.5,2.55) node {\scriptsize $2$};
\draw (-4,0.7) node {\scriptsize $-4$};
\draw (0.6,-2) node {\tiny $-2$};
\draw (0.7,6.2) node {\scriptsize $y$};
\draw (-2,2) circle (8pt);
\draw (-2,-2) circle (8pt);
%\draw (-2,-3) circle (8pt);
\end{tikzpicture}$$
б) Прямая $y=k(x+2)$ при любом значении $k$ проходит через точку $(-2;0).$ При $k=1$ прямая $y=x+2$ совпадает с одной из прямых исходной функции, а значит имеет с данным графиком бесконечное количество точек пересечения. При уменьшении значения $k$ эта прямая пересекает данный график либо только в одной точке (при $x>0$), либо вообще его не пересекает (при $k\leqslant0$). При увеличении значения $k$ прямая будет пересекать данный график в 2 точках (до $x=-2$ и после $x=-2$). Значит, прямая $y=k(x+2)$ имеет с данным графиком более одной общей точки при $k\geqslant1.$\\
