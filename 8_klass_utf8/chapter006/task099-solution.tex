99. а) Пусть парабола задана уравнением $y=ax^2+bx+c,$ тогда имеем систему уравнений\\ $\begin{cases} 3=a+b+c,\\ 0=4a+2b+c,\\ 0=a-b+c.\end{cases}
\Leftrightarrow \begin{cases} c=3-a-b,\\ 3a+b+3=0,\\ 3-2b=0.\end{cases}
\Leftrightarrow \begin{cases} c=3,\\ a=-\cfrac{3}{2},\\ b=\cfrac{3}{2}.\end{cases}\Rightarrow y=-\cfrac{3}{2}x^2+\cfrac{3}{2}x+3.$\\
б) Пусть парабола задана уравнением $y=ax^2+bx+c,$ тогда имеем систему уравнений\\ $\begin{cases} 1=a+b+c,\\ 4=4a+2b+c,\\ 11=9a+3b+c.\end{cases}
\Leftrightarrow \begin{cases} c=1-a-b,\\ 3a+b=3,\\ 8a+2b=10.\end{cases}
\Leftrightarrow \begin{cases} c=1-a-b,\\ 3a+b=3,\\ 2a=4.\end{cases}
\Leftrightarrow \begin{cases} c=2\\ b=-3,\\ a=2.\end{cases}
\Rightarrow y=2x^2-3x+2,$ на графике этой квадратичной функции все три точки лежат.\\
