104.а) $g(x)=\cfrac{6\sqrt{x^2-2x+1}}{\sqrt{4-4x+x^2}-x}=\cfrac{6\sqrt{(x-1)^2}}{\sqrt{(x-2)^2}-x}=\cfrac{6|x-1|}{|x-2|-x}=
\begin{cases} \cfrac{-6(x-1)}{-x+2-x}=3,\ x<1,\\
\cfrac{6(x-1)}{-x+2-x}=-3,\ 1< x \leqslant 2,\\
\cfrac{6(x-1)}{x-2-x}=3-3x,\ 2<x.\end{cases}$
$$\begin{tikzpicture}[scale=0.3]
\tikzset {line01/.style={line width =0.5pt}}
\tikzset{line02/.style={line width =1pt}}
\tikzset{line03/.style={dashed,line width =0.5pt}}
%\filldraw [black] (0,0) circle (1pt);
\draw [->] (-8,0) -- (6,0);
\draw [->] (0,-8) -- (0,6);
\draw[line01] (2,-3) -- (3,-6);
\draw[line01] (-7,3) -- (0.7,3);
\draw[line01] (1.3,-3) -- (2,-3);
\draw[line03] (1,2.8) -- (1,-2.8);
\draw[line03] (2,0) -- (2,-2.8);
\draw[line03] (0,-3) -- (0.8,-3);
%\draw[line03] (1,0) -- (1,2);
%\draw[line03] (0,2) -- (1,2);
%\draw[line03] (0,-2) -- (1,-2);
\draw (6.2,0.7) node {\scriptsize $x$};
%\draw (-1.2,-2) node {\scriptsize $-2$};
%\draw (-0.7,2) node {\scriptsize $2$};
%\draw (-0.7,1.2) node {\scriptsize $1$};
\draw (1.3,0.5) node {\tiny $1$};
\draw (-0.5,2.55) node {\scriptsize $3$};
\draw (2,0.5) node {\tiny $2$};
\draw (-0.8,-3) node {\tiny $-3$};
\draw (0.7,6.2) node {\scriptsize $y$};
\draw (1,3) circle (8pt);
\draw (1,-3) circle (8pt);
%\draw (-2,-3) circle (8pt);
\end{tikzpicture}$$
б) Прямая $y=k(x-1)$ при любом значении $k$ проходит через точку $(1;0).$ При $k=-3$ прямая $y=3-3x$ совпадает с одной из прямых исходной функции, а значит имеет с данным графиком бесконечное количество точек пересечения. При увеличении значения $k$ эта прямая пересекает данный график либо только в одной точке (при $x<0$), либо вообще его не пересекает (при $k\geqslant0$). При уменьшении значения $k$ прямая будет пересекать данный график в 2 точках (до $x=1$ и после $x=1$). Значит, прямая $y=k(x-1)$ имеет с данным графиком более одной общей точки при $k\leqslant-3.$\\
