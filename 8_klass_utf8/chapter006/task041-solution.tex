41. Один из катетов этого треугольника равен 3, значит другой должен быть равен $9\cdot2:3=6.$ Тогда эта прямая может проходить через точку $(6;0)$ или $(-6;0).$ В обоих случаях $b=3$ (так как прямая проходит через точку $(0;3).)$ Пусть прямая задана уравнением $y=kx+b.$ В первом случае $6k+3=0,\ k=-\cfrac{1}{2},$ а во втором --- $-6k+3=0,\ k=\cfrac{1}{2}.$ Значит, прямая может иметь уравнение $y=3-\frac{1}{2}x$ или $y=3+\frac{1}{2}x.$\\
