113. Первые две прямые пересекаются в точке $(0;0).$ Найдём точку пересечения первой прямой и третьей: $\begin{cases}y=\cfrac{2}{3}x,\\ x+y=5.\end{cases}
\Leftrightarrow \begin{cases}y=\cfrac{2}{3}x,\\ x+\cfrac{2}{3}x=5.\end{cases}
\Leftrightarrow \begin{cases}y=2,\\ x=3.\end{cases}$ Найдём точку пересечения второй прямой и третьей: $\begin{cases}y=4x,\\ x+y=5.\end{cases}
\Leftrightarrow \begin{cases}y=4x,\\ x+4x=5.\end{cases}
\Leftrightarrow \begin{cases}y=4,\\ x=1.\end{cases}$ Таким образом, искомый треугольник имеет вершины в точках $(0;0),\ (3;2),\ (1;4).$\\
{{PIC:g9-1000.png}}\\
Достроив треугольник до прямоугольника, найдём его площадь: $S_{\Delta AOB}=S_{OMNK}-S_{\Delta OMA}-S_{\Delta ANB}-S_{\Delta OBK}=
3\cdot4-1\cdot4:2-2\cdot2:2-3\cdot2:2=5.$\\
