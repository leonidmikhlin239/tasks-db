86. Пусть парабола задана уравнением $y=ax^2+bx+c.$ Так как в точке $C(0;5)$ находится вершина, $-\cfrac{b}{2a}=0,\ b=0,\ 0+0+c=5,\ c=5.$ Так как парабола проходит через точку $B(4;-3),$ имеем уравнение $16a+5=-3, a=-\cfrac{1}{2}.$ Найдём точки пересечения с осью $Ox:\ -\cfrac{1}{2}x^2+5=0,\ x^2=10,\ x=\pm\sqrt{10}.$ Значит, это точки $(-\sqrt{10};0)$ и $(\sqrt{10};0)$\\
