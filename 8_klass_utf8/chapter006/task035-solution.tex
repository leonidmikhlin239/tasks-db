35. Пусть прямая задана уравнением $y=kx+b,$ тогда она отсекает на осях координат отрезки, имеющие длины $|b|$ и $\left|\cfrac{b}{k}\right|.$ Тогда
$|b|=\left|\cfrac{b}{k}\right|\Leftrightarrow |b|\left(1-\cfrac{1}{|k|}\right)=0\Leftrightarrow
\left[\begin{array}{l}b=0,\\ k=1,\\ k=-1.\end{array}\right.$ Если $b=0,$ то прямая не отсекает отрезков от осей координат. При $k=1$ имеем равенство $3+b=2,\ b=-1.$
При $k=-1$ имеем равенство $-3+b=2,\ b=5.$ Таким образом, уравнение прямой может иметь вид $y=x-1$ или $y=5-x.$\\
