94. а) $T=\left(\cfrac{4}{2\sqrt{a}-3\sqrt{b}}-\cfrac{14\sqrt{b}}{2a+\sqrt{ab}-6b}+\cfrac{7}{\sqrt{a}+2\sqrt{b}}\right):\left(\cfrac{1}{\sqrt{a}+2\sqrt{b}}-\cfrac{1}{\sqrt{a}}\right)=
[x=\sqrt{a},\ y=\sqrt{b}]=
\left(\cfrac{4}{2x-3y}-\cfrac{14y}{2x^2+xy-6y^2}+\cfrac{7}{x+2y}\right):\left(\cfrac{1}{x+2y}-\cfrac{1}{x}\right)=$\\$
\left(\cfrac{4x+8y}{(2x-3y)(x+2y)}-\cfrac{14y}{(2x-3y)(x+2y)}+\cfrac{14x-21y}{(2x-3y)(x+2y)}\right):\cfrac{x-x-2y}{x(x+2y)}=$\\$
\cfrac{18x-27y}{(2x-3y)(x+2y)}\cdot\cfrac{x(x+2y)}{-2y}=-\cfrac{9x}{2y}=-\cfrac{9\sqrt{a}}{2\sqrt{b}}.$\\
б) $x=\sqrt{92}=2\sqrt{23},\ y=\sqrt{207}=3\sqrt{23},$ тогда $T=-\cfrac{18\sqrt{2}}{6\sqrt{23}}=-3.$\\
в) $x=\sqrt{153}=3\sqrt{17},\ y=\sqrt{68}=2\sqrt{17},$ тогда $2x-3y=6\sqrt{17}-6\sqrt{17}=0,$ поэтому значение $T$ неопределено (присутствует деление на 0).\\