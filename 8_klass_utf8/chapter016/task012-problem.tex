12. Пусть на первой машине можно сделать копию пакета документов за $x$ минут, а на второй --- за $x+30.$ Тогда скорости их копирования равны $\cfrac{1}{x}$ и
$\cfrac{1}{x+30}$ (пакета в минуту) и $\cfrac{1}{x}+\cfrac{1}{x+30}=\cfrac{1}{20}\Leftrightarrow\cfrac{x+30+x}{x^2+30x}=\cfrac{1}{20}\Leftrightarrow
40x+600=x^2+30x\Leftrightarrow x^2-10x-600=0\Leftrightarrow x=30.$ Значит, одна машина может выполнить работу за 30 минут, а другая --- за 60.\\
