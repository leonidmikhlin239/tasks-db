\section{Стандартные задачи решения}
1. Пусть один насос наполняет бассейн за $x$ часов, а другой за $x+2.$ Тогда их скорости наполнения бассейна равны $\cfrac{1}{x}$ и $\cfrac{1}{x+2}$ (бассейна в час) и $\cfrac{1}{x}+\cfrac{1}{x+2}=\cfrac{1}{2\frac{55}{60}}\Leftrightarrow \cfrac{x+2+x}{x^2+2x}=\cfrac{12}{35}\Leftrightarrow
70x+70=12x^2+24x\Leftrightarrow6x^2-23x-35=0\Leftrightarrow x=5.$ Значит, насосы наполняют бассейны за 5 часов и за $5+2=7$ часов.\\
2. Пусть один трактор вспахивает поле за $x$ дней, а другой за $x+5.$ Тогда их скорости вспахивания равны $\cfrac{1}{x}$ и $\cfrac{1}{x+5}$ (поля в день) и $\cfrac{1}{x}+\cfrac{1}{x+5}=\cfrac{\cfrac{2}{3}}{4}\Leftrightarrow \cfrac{x+5+x}{x^2+5x}=\cfrac{1}{6}\Leftrightarrow
12x+30=x^2+5x\Leftrightarrow x^2-7x-30=0\Leftrightarrow x=10.$ Значит, трактора вспахивают поля за 10 дней и за $10+5=15$ дней.\\
3. $a=2a^2, a>0\Rightarrow a=\cfrac{1}{2}.$\\
4. $a=4a^2, a>0\Rightarrow a=\cfrac{1}{4}.$\\
5. $a\cdot1,1\cdot1,15=a\cdot1,265.$ Значит, товар подорожал на $26,5\%.$\\
6. $a\cdot1,15\cdot1,1=a\cdot1,265.$ Значит, товар подорожал на $26,5\%.$\\
7. Пусть скорость автомобиля равна $x\text{ км/ч}.$ Тогда $2,5x=2(x+20)-15,\ 2,5x=2x+40-15,\ x=50\text{км/ч}.$ Тогда расстояние от города до посёлка равно $50\cdot2,5=125$км.\\
8. Пусть скорость автобуса равна $x\text{ км/ч},$ тогда скорость легковой машины равна $x+20\text{ км/ч}.$ Тогда $5(x+20)-70=6x,\ 5x+100-70=6x,\ x=30\text{км/ч}.$\\
9. Скорость вскапывания Саши равна $\cfrac{1}{10}-\cfrac{1}{15}=\cfrac{1}{30}.$ Значит, Саша вскопает грядку за 30 минут.\\
10. Скорость пропалывания Тани равна $\cfrac{1}{12}-\cfrac{1}{20}=\cfrac{1}{30}.$ Значит, Таня прополет грядку за 30 минут.\\
11. Пусть на первой машине можно сделать копию пакета документов за $x$ минут, а на второй --- за $x+15.$ Тогда скорости их копирования равны $\cfrac{1}{x}$ и
$\cfrac{1}{x+15}$ (пакета в минуту) и $\cfrac{1}{x}+\cfrac{1}{x+15}=\cfrac{1}{10}\Leftrightarrow\cfrac{x+15+x}{x^2+15x}=\cfrac{1}{10}\Leftrightarrow
20x+150=x^2+15x\Leftrightarrow x^2-5x-150=0\Leftrightarrow x=15.$ Значит, одна машина может выполнить работу за 15 минут, а другая --- за 30.\\
12. Пусть на первой машине можно сделать копию пакета документов за $x$ минут, а на второй --- за $x+30.$ Тогда скорости их копирования равны $\cfrac{1}{x}$ и
$\cfrac{1}{x+30}$ (пакета в минуту) и $\cfrac{1}{x}+\cfrac{1}{x+30}=\cfrac{1}{20}\Leftrightarrow\cfrac{x+30+x}{x^2+30x}=\cfrac{1}{20}\Leftrightarrow
40x+600=x^2+30x\Leftrightarrow x^2-10x-600=0\Leftrightarrow x=30.$ Значит, одна машина может выполнить работу за 30 минут, а другая --- за 60.\\
13. $S=4\cdot6:0,75=32\text{ м}^2.$\\
14. $S=3\cdot4:0,6=20\text{ м}^2.$\\
15. Пусть скорость первого автомобиля равна $x\text{ км/ч},$ а весь путь равен $2S.$ Тогда верно соотношение $\cfrac{2S}{x}=\cfrac{S}{24}+\cfrac{S}{x+16}
\Leftrightarrow \cfrac{2}{x}=\cfrac{x+16+24}{24x+384}\Leftrightarrow 48x+768=x^2+40x \Leftrightarrow x^2-8x-768=0\Leftrightarrow x=32\text{ км/ч}.$\\
16. Пусть скорость первого автомобиля равна $x\text{ км/ч},$ а весь путь равен $2S.$ Тогда верно соотношение $\cfrac{2S}{x}=\cfrac{S}{33}+\cfrac{S}{x+22}
\Leftrightarrow \cfrac{2}{x}=\cfrac{x+22+33}{33x+726}\Leftrightarrow 66x+1452=x^2+55x \Leftrightarrow x^2-11x-1452=0\Leftrightarrow x=44\text{ км/ч}.$\\
17. $a\cdot1,1\cdot0,9=0,99a.$\\
18. $a\cdot0,9\cdot1,1=0,99a.$\\
19. Пусть собственная скорость парохода (она же скорость движения по озеру) равна $x\text{ км/ч},$ тогда $\cfrac{9}{x}+\cfrac{20}{x+3}=1\Leftrightarrow \cfrac{9x+27+20x}{x^2+3x}=1\Leftrightarrow
x^2+3x=29x+27\Leftrightarrow x^2-26x-27=0 \Leftrightarrow x=27\text{ км/ч}.$\\
20. Пусть собственная скорость парохода (она же скорость движения по озеру) равна $x\text{ км/ч},$ тогда $\cfrac{9}{x}+\cfrac{16}{x-3}=1\Leftrightarrow \cfrac{9x-27+16x}{x^2-3x}=1\Leftrightarrow
x^2-3x=25x-27\Leftrightarrow x^2-28x+27=0 \Leftrightarrow x=27\text{ км/ч}.$\\
21. Средняя скорость велосипедиста равна $\cfrac{58+19}{2,5+1}=22\text{ км/ч}.$\\
22. Средняя скорость велосипедиста равна $\cfrac{36+34}{1,5+2}=20\text{ км/ч}.$\\
23. $\text{Р}=0,8\text{П},$ тогда $\text{П}=\cfrac{1}{0,8}\text{Р}=1,25\text{Р},$ то есть пиджак дороже рубашки на $25\%.$\\
24. Скорость их сближения равна $60+80=140\text{ км/ч}.$ Значит, за час до встречи между ними было 140 километров.\\
25. В сиропе содержится $180\cdot0,25=45$г сахара. Если он составляет $20\%,$ то новая масса сиропа должна быть $45:0,2=225$г, то есть надо добавить $225-180=45$г воды.\\
26. В сиропе содержится $220\cdot0,25=55$г сахара. Если он составляет $20\%,$ то новая масса сиропа должна быть $55:0,2=275$г, то есть надо добавить $275-220=55$г воды.\\
27. Пусть изначально было $x$ посетителей, тогда сбор составлял $150x$ рублей. Новый сбор стал составлять $150x\cdot1,25=187,5x,$ а новое количество посетителей --- $1,5x.$ Тогда новая цена билета равна $\cfrac{187,5x}{1,5x}=125$ рублей.\\
28. Пусть изначально было $x$ посетителей, тогда сбор составлял $120x$ рублей. Новый сбор стал составлять $120x\cdot1,25=150x,$ а новое количество посетителей --- $1,5x.$ Тогда новая цена билета равна $\cfrac{150x}{1,5x}=100$ рублей.\\
29. Пусть скорость третьего мотоциклиста равна $x\text{ км/ч}.$ Первый мотоциклист за 20 минут проедет $75:3=25$ километров, а второй --- $60:3=20$ километров. Тогда $\cfrac{25}{x-75}-\cfrac{20}{x-60}=1\Leftrightarrow\cfrac{25x-1500-20x+1500}{x^2-135x+4500}=1\Leftrightarrow 5x=x^2-135x+4500\Leftrightarrow
x^2-140x+4500=0\Leftrightarrow x=90\text{ км/ч}$ (второй корень равен $50\text{ км/ч},$ но с такой скоростью третий мотоциклист никого бы не догнал).\\
30. Пусть скорость байдарки в стоячей воде равна $x\text{ км/ч},$ тогда $\cfrac{25}{x}+\cfrac{9}{x-2}=\cfrac{56}{x+2}\Leftrightarrow$\\$
\cfrac{25x-50+9x}{x^2-2x}=\cfrac{56}{x+2}\Leftrightarrow 34x^2-50x+68x-100=56x^2-112x\Leftrightarrow
22x^2-130x+100=0\Leftrightarrow 11x^2-65x+50=0 \Leftrightarrow x=5\text{ км/ч}$(второй корень равен $\cfrac{10}{11}\text{ км/ч},$ но с такой скоростью нельзя было бы плыть против течения).\\
31. Пусть скорость третьего велосипедиста равна $x\text{ км/ч}.$ Первый велосипедист за 20 минут проедет $15:3=5$ километров, а второй --- $12:3=4$ километров. Тогда $\cfrac{5}{x-15}-\cfrac{4}{x-12}=1\Leftrightarrow\cfrac{5x-60-4x+60}{x^2-27x+180}=1\Leftrightarrow x=x^2-27x+180\Leftrightarrow
x^2-28x+180=0\Leftrightarrow x=18\text{ км/ч}$ (второй корень равен $10\text{ км/ч},$ но с такой скоростью третий велосипедист никого бы не догнал).\\
32. Пусть скорость байдарки в стоячей воде равна $x\text{ км/ч},$ тогда $\cfrac{15}{x}+\cfrac{9}{x-2}=\cfrac{42}{x+2}\Leftrightarrow$\\$
\cfrac{15x-30+9x}{x^2-2x}=\cfrac{42}{x+2}\Leftrightarrow 24x^2-30x+48x-60=42x^2-84x\Leftrightarrow
18x^2-102x+60=0\Leftrightarrow 9x^2-51x+30=0 \Leftrightarrow x=5\text{ км/ч}$(второй корень равен $\cfrac{2}{3}\text{ км/ч},$ но с такой скоростью нельзя было бы плыть против течения).\\
33. Пусть закупочная цена альбома равна $x$ рублей. Покупатель заплатил $x\cdot2,3\cdot1,05\cdot0,6=1,449x.$ Тогда $1,449x=57,96,\ x=40$ рублей. Покупатель заплатил больше на $40\cdot0,449=17$ рублей 96 копеек.\\
34. Пусть закупочная цена альбома равна $x$ рублей. Покупатель заплатил $x\cdot1,2\cdot2,1\cdot0,7=1,764x.$ Тогда $1,764x=123,48,\ x=70$ рублей. Покупатель заплатил больше на $70\cdot0,764=53$ рубля 48 копеек.\\
35. Пусть закупочная цена альбома равна $x$ рублей. Покупатель заплатил $x\cdot1,05\cdot2,2\cdot0,6=1,386x.$ Тогда $1,386x=69,3,\ x=50$ рублей. Покупатель заплатил больше на $50\cdot0,386=19$ рублей 30 копеек.\\
36. Пусть закупочная цена альбома равна $x$ рублей. Покупатель заплатил $x\cdot2,1\cdot1,05\cdot0,7=1,5435x.$ Тогда $1,5435x=92,61,\ x=60$ рублей. Покупатель заплатил больше на $60\cdot0,5435=32$ рубля 61 копейку.\\
37. Пусть вклад каждый год увеличивается на $x\%,$ тогда его величина каждый год умножается на $\left(1+\cfrac{x}{100}
ight).$ Составим уравнение:
$50000\cdot\left(1+\cfrac{x}{100}
ight)^2=55125,\ \left(1+\cfrac{x}{100}
ight)^2=1,1025,\ 1+\cfrac{x}{100}=1,05,\ x=5\%.$\\
38. Пусть вклад каждый год увеличивается на $x\%,$ тогда его величина каждый год умножается на $\left(1+\cfrac{x}{100}
ight).$ Составим уравнение:
$2000\cdot\left(1+\cfrac{x}{100}
ight)^2=2420,\ \left(1+\cfrac{x}{100}
ight)^2=1,21,\ 1+\cfrac{x}{100}=1,1,\ x=10\%.$\\
39. Средняя скорость Пети равна $\cfrac{1+7+2}{15+12+13}=\cfrac{10}{40}=\cfrac{1}{4}\text{ км/мин}=15\text{ км/ч}.$\\
40. Средняя скорость спортсмена равна $\cfrac{1+10+29}{15+50+55}=\cfrac{40}{120}=\cfrac{1}{3}\text{ км/мин}=20\text{ км/ч}.$\\
41. Пусть один лыжник проезжает круг за $x$ минут, а другой --- з $x+3.$ Тогда их скорости равны $\cfrac{1}{x}$ и $\cfrac{1}{x+3}$ круга в минуту и $60\cdot\cfrac{1}{x}-60\cdot\cfrac{1}{x+3}=1,\ \cfrac{60x+180-60x}{x(x+3)}=1,\ x^2+3x-180=0,\ x=12$ минут. Значит, один лыжник проходит круг за 12 минут, а другой за $12+3=15$ минут.\\
42. Пусть один лыжник проезжает круг за $x$ минут, а другой --- з $x+5.$ Тогда их скорости равны $\cfrac{1}{x}$ и $\cfrac{1}{x+5}$ круга в минуту и $60\cdot\cfrac{1}{x}-60\cdot\cfrac{1}{x+5}=1,\ \cfrac{60x+300-60x}{x(x+5)}=1,\ x^2+5x-300=0,\ x=15$ минут. Значит, один лыжник проходит круг за 15 минут, а другой за $15+5=20$ минут.\\
43. В 1 килограмме изюма содержится $1\cdot0,95=0,95$кг сухого вещества. В свежем винограде его столько же, но оно составляет $20\%,$ значит его надо взять
$0,95:0,2=4,75$ килограмма.\\
44. Пусть Пятачок делает подарок за $x$ дней, а Винни-Пух --- за $x+12.$ Тогда их скорости изготовления подарка равны $\cfrac{1}{x}$ и $\cfrac{1}{x+12}$ и $\cfrac{1}{x}+\cfrac{1}{x+12}=\cfrac{1}{8},\ \cfrac{x+12+x}{x^2+12x}=\cfrac{1}{8},\ 16x+96=x^2+12x,\ x^2-4x-96=0,\ x=12$ дней.\\
45. Пусть Карлсон съедает банку варенья за $x$ минут, а Малыш --- за $x+10.$ Тогда их скорости поедания варенья равны $\cfrac{1}{x}$ и $\cfrac{1}{x+10}$ и $\cfrac{1}{x}+\cfrac{1}{x+10}=\cfrac{1}{12},\ \cfrac{x+10+x}{x^2+10x}=\cfrac{1}{12},\ 24x+120=x^2+10x,\ x^2-14x-120=0,\ x=20$ минут.\\
46. Пусть скорость курьера равна $x\text{ км/ч},$ а весь путь равен $2S.$ Тогда верно соотношение $\cfrac{2S}{x}=\cfrac{S}{102}+\cfrac{S}{x-17}
\Leftrightarrow \cfrac{2}{x}=\cfrac{x-17+102}{102x-1734}\Leftrightarrow 204x-3468=x^2+85x \Leftrightarrow x^2-119x+3468=0\Leftrightarrow x=51\text{ км/ч}$ (второй корень больше $60\text{ км/ч}).$\\
47. Пусть скорость генерала равна $x\text{ км/ч},$ а весь путь равен $2S.$ Тогда верно соотношение $\cfrac{2S}{x}=\cfrac{S}{x-13}+\cfrac{S}{78}
\Leftrightarrow \cfrac{2}{x}=\cfrac{78+x-13}{78x-1014}\Leftrightarrow 156x-2028=x^2+65x \Leftrightarrow x^2-91x+2028=0\Leftrightarrow x=52\text{ км/ч}$ (второй корень больше $48\text{ км/ч}).$\\
48. Возьмём задание рабочего за единицу, а скорость его выполнения за $x.$ Тогда $\cfrac{1}{x}-\cfrac{1}{1,16x}=1,\ 1,16-1=1,16x,\ x=\cfrac{4}{29}.$ Тогда он стал тратить на выполнение задания $\cfrac{1}{\cfrac{4}{29}\cdot1,16}=\cfrac{25}{4}\text{ ч}=6\text{ ч}15$мин.\\
49. Пусть концентрация соли во втором растворе выражается десятичной дробью $x.$ Тогда $30\cdot0,2+10\cdot x=(30+10)\cdot0,25,\ 6+10x=10,\ x=0,4.$ Значит, содержание соли во втором растворе составляет $40\%.$\\
50. Пусть скорость велосипедиста равна $x\text{ км/ч},$ а длина пути равна $S$ км. Тогда имеем систему уравнений
$\begin{cases}\cfrac{S}{x}=\cfrac{S}{x+5}+\cfrac{1}{5},\\ \cfrac{S}{x}=\cfrac{S}{x-8}-\cfrac{2}{3}.\end{cases}
\Leftrightarrow
\begin{cases}\cfrac{Sx+5S-Sx}{x(x+5)}=\cfrac{1}{5},\\ \cfrac{Sx-Sx+8S}{x(x-8)}=\cfrac{2}{3}.\end{cases}
\Leftrightarrow
\begin{cases}\cfrac{5S}{x(x+5)}=\cfrac{1}{5},\\ \cfrac{8S}{x(x-8)}=\cfrac{2}{3}.\end{cases}
\Leftrightarrow$\\$
\begin{cases}x^2+5x=25S,\\ 2x^2-16x=24S.\end{cases}
\Leftrightarrow
\begin{cases}x^2+5x=25S,\\ 26x=26S.\end{cases}
\Leftrightarrow
\begin{cases}x(x-20)=0,\\ x=S.\end{cases}
\Leftrightarrow
\begin{cases}x=20\text{ км/ч},\\ S=20\text{ км}.\end{cases}$\\
51.  Пусть в первом заповеднике было $x$ волков, а во втором --- $210-x.$ Тогда $1,1x+1,3(210-x)=251,\ 0,2x=273-251,\ x=110.$ Значит, в первом заповеднике было 110 волков, а во втором --- $210-110=100.$\\
52. Пусть взято $x$ граммов четырёхпроцентного раствора соли, тогда девятипроцентного взято $250-x$ граммов. Тогда $0,04x+0,09(250-x)=0,06\cdot250,\ 0,05x=7,5,\ x=150$ граммов. Значит, необходимо взять 150 граммов четырёхпроцентного раствора и $250-150=100$ граммов девятипроцентного.\\
53. Товар стал стоить $1000\cdot0,6\cdot0,6\cdot0,8=288$ рублей.\\
54. Товар стал стоить $1000\cdot0,8\cdot0,8\cdot0,4=256$ рублей.\\
55. Пусть мотоциклист ехал со скоростью $x\text{ км/ч},$ тогда $\cfrac{30}{x}+\cfrac{9}{60}=\cfrac{30}{x-10},\ \cfrac{30x-30x+300}{x(x-10)}=\cfrac{3}{20},\
\cfrac{300}{x^2-10x}=\cfrac{3}{20},\ 2000=x^2-10x,\ x^2-10x-2000=0,\ x=50\text{ км/ч}.$\\
56. За 20 минут Петя пройдёт $4,5\cdot\cfrac{1}{3}=1,5$км. Вася догоняет его со скоростью $12-4,5=7,5\text{ км/ч},$ значит ему потребуется $1,5:7,5=0,2$ч. В этот момент он будет на расстоянии $0,2\cdot12=2,4$км.\\
57. За 6 минут Нина проедет $15\cdot\cfrac{1}{10}=1,5$км. Брат догоняет его со скоростью $40-15=25\text{ км/ч},$ значит ему потребуется $1,5:25=0,06$ч. В этот момент он будет на расстоянии $0,06\cdot40=2,4$км.\\
58. Пусть взято $x$ граммов $15\%$-ного раствора соли, тогда $0,15x+0,6\cdot50=0,4(x+50),\ 0,25x=10,\ x=40$ граммов.\\
59. Пусть взято $x$ граммов $75\%$-ного раствора соли, тогда $0,75x+0,15\cdot30=0,5(x+30),\ 0,25x=10,5,\ x=42$ грамма.\\
60. Чтобы проехать мимо товарного поезда, пассажирскому поезду необходимо проехать больше, чем товарному, на сумму их длин. Пусть длина пассажирского поезда равна$x$ км, тогда $\cfrac{1}{20}\cdot70=\cfrac{1}{20}\cdot30+1,4+x,\ x=0,6$км$=600$ метров.\\
61. Чтобы проехать мимо товарного поезда, пассажирскому поезду необходимо проехать больше, чем товарному, на сумму их длин. Пусть длина пассажирского поезда равна$x$ км, тогда $\cfrac{1}{30}\cdot75=\cfrac{1}{30}\cdot45+0,8+x,\ x=0,2$км$=200$ метров.\\
62. Концентрация полученного раствора равна $7\cdot0,14:(7+21)=0,035=3,5\%.$\\
63. Концентрация полученного раствора равна $5\cdot0,12:(5+7)=0,05=5\%.$\\
64. Пусть первоначальная скорость поезда равна $x$ км/ч, тогда $\cfrac{720}{x}-\cfrac{720}{x+10}=1,$\\$ \cfrac{720x+7200-720x}{x^2+10x}=1,\ x^2+10x-7200=0,\
(x+90)(x-80)=0,\ x=80$км/ч.\\
65. Пусть первоначальная скорость поезда равна $x$ км/ч, тогда $\cfrac{560}{x}-\cfrac{560}{x+10}=1,$\\$ \cfrac{560x+5600-560x}{x^2+10x}=1,\ x^2+10x-5600=0,\
(x+80)(x-70)=0,\ x=70$км/ч.\\
66. Концентрация получившегося раствора составляет $\cfrac{8\cdot0,25+12\cdot0,2}{8+12}=0,22=22\%.$\\
67. Концентрация получившегося раствора составляет $\cfrac{4\cdot0,15+6\cdot0,25}{4+6}=0,21=21\%.$\\
68. Пусть скорость второго велосипедиста равна $x$км/ч, тогда скорость первого равна $x+3$км/ч и $\cfrac{88}{x}-\cfrac{88}{x+3}=3,\ \cfrac{88x+88\cdot3-88x}{x^2+3x}=3,\ x^2+3x-88=0,\ (x+11)(x-8)=0,\ x=8$км/ч.\\
69. Пусть скорость второго велосипедиста равна $x$км/ч, тогда скорость первого равна $x+4$км/ч и $\cfrac{96}{x}-\cfrac{96}{x+4}=4,\ \cfrac{96x+96\cdot4-96x}{x^2+4x}=4,\ x^2+4x-96=0,\ (x+12)(x-8)=0,\ x=8$км/ч.\\
70. Если рельс удлинится на 3 мм, его длина будет равна $10000+3=10003$мм. Запишем тогда закон изменения длины: $10003=10000\cdot(1+1,2\cdot10^{-5}t),\
1,0003=1+1,2\cdot10^{-5}t,\ 1,2\cdot10^{-5}t=0,0003,\ t=25^\circ C.$\\
71. В 40 кг морской воды содержится $40\cdot0,05=2$ кг соли. Чтобы она составляла $2\%,$ необходимо, чтобы воды было $2:0,02=100$кг, то есть добавить нужно $100-40=60$кг воды.\\
72. Пусть первого раствора изначально было взято $x$ литров, а второго --- $y.$ Тогда имеем систему уравнений $\begin{cases} 0,4x+0,75y=0,61(x+y),\\
0,4(x-40)+0,75(y-40)=0,65(x+y-80).\end{cases}\Leftrightarrow$\\$\begin{cases} 0,21x=0,14y,\\
0,4x+0,75y-46=0,65x+0,65y-52.\end{cases}\Leftrightarrow\begin{cases} y=\cfrac{3}{2}x,\\
0,25x-6=0,1y.\end{cases}\Leftrightarrow\begin{cases} y=\cfrac{3}{2}x,\\
0,25x-6=0,15x.\end{cases}\Leftrightarrow\begin{cases} y=90,\\
x=60.\end{cases}$ Значит, первого раствора было 60 литров, а второго --- 90.\\
73. Пусть первого раствора изначально было взято $x$ литров, а второго --- $y.$ Тогда имеем систему уравнений $\begin{cases} 0,3x+0,65y=0,55(x+y),\\
0,3(x+20)+0,65(y+20)=0,51(x+y+40).\end{cases}\Leftrightarrow$\\$\begin{cases} 0,25x=0,1y,\\
0,3x+0,65y+19=0,51x+0,51y+20,4.\end{cases}\Leftrightarrow\begin{cases} y=\cfrac{5}{2}x,\\
0,21x+1,4=0,14y.\end{cases}\Leftrightarrow$\\$\begin{cases} y=\cfrac{5}{2}x,\\
0,21x+1,4=0,35x.\end{cases}\Leftrightarrow\begin{cases} y=25,\\
x=10.\end{cases}$ Значит, первого раствора было 10 литров, а второго --- 25.\\
74. Пусть один тратит на забег внутри озера $x$ минут, а другой --- $x+7.$ Тогда их скорости равны $\cfrac{1}{x}$ и $\cfrac{1}{x+7}$ (кругов в минуту). В первый раз они встретятся, когда один обгонит другого на один круг, поэтому $\cfrac{1}{\cfrac{1}{x}-\cfrac{1}{x+7}}=84,\ \cfrac{x(x+7)}{7}=84,\
x^2+7x-588=0,\ (x+28)(x-21)=0,\ x=21$мин. Значит, скорость одного равна $\cfrac{1}{21},$ а другого --- $\cfrac{1}{28}.$ Если они побегут в противоположных направлениях, им  необходимо совместно преодолеть 1 круг, на это они потратят $\cfrac{1}{\cfrac{1}{21}+\cfrac{1}{28}}=12$ минут.\\
75. Пусть один тратит на забег внутри озера $x$ минут, а другой --- $x+9.$ Тогда их скорости равны $\cfrac{1}{x}$ и $\cfrac{1}{x+9}$ (кругов в минуту). В первый раз они встретятся, когда преодолеют вместе один круг, поэтому $\cfrac{1}{\cfrac{1}{x}+\cfrac{1}{x+9}}=20,\ \cfrac{x(x+9)}{2x+9}=20,\
x^2-31x-180=0,\ (x+5)(x-36)=0,\ x=36$мин. Значит, скорость одного равна $\cfrac{1}{36},$ а другого --- $\cfrac{1}{45}.$ Если они побегут в одном направлении, то впервые встретятся, когда один обгонит второго на один круг, на это они потратят $\cfrac{1}{\cfrac{1}{36}-\cfrac{1}{45}}=180$ минут или 3 ч.\\
76. Всего автомобиль проехал $2\cdot50+1\cdot100+2\cdot75=350$км за $2+1+2=5$ часов. Значит, его средняя скорость равна $350:5=70$км/ч.\\
77. Всего автомобиль проехал $190+180+170=540$км за $190:50+180:90+170:100=7,5$ часов. Значит, его средняя скорость равна $540:7,5=72$км/ч.\\
78. Пусть $40\%$-го и $60\%$-го растворов было $x$кг и $y$кг соответственно, тогда имеем систему уравнений $\begin{cases}
0,4x+0,6y=0,2(x+y+5),\\ 0,4x+0,6y+0,8\cdot5=0,7(x+y+5).\end{cases}\Leftrightarrow\begin{cases}
0,2x+0,4y=1,\\ 0,3x+0,1y=0,5.\end{cases}\Leftrightarrow\begin{cases}
2x+4y=10,\\ 3x+y=5.\end{cases}\Leftrightarrow\begin{cases}
2x+20-12x=10,\\ y=5-3x.\end{cases}\Leftrightarrow\begin{cases}
x=1,\\ y=2.\end{cases}$ Значит, $40\%$-го раствора было 1 кг, а $60\%$-го --- 2 кг.\\
79. Пусть $30\%$-го и $60\%$-го растворов было $x$кг и $y$кг соответственно, тогда имеем систему уравнений $\begin{cases}
0,3x+0,6y=0,36(x+y+10),\\ 0,3x+0,6y+0,5\cdot10=0,41(x+y+10).\end{cases}\Leftrightarrow\begin{cases}
0,24y-0,06x=3,6,\\ 0,11x-0,19y=0,9.\end{cases}\Leftrightarrow\begin{cases}
4y-x=60,\\ 11x-19y=90.\end{cases}$\\$\Leftrightarrow\begin{cases}
x=4y-60,\\ 44y-660-19y=90.\end{cases}\Leftrightarrow\begin{cases}
x=60,\\ y=30.\end{cases}$ Значит, $30\%$-го раствора было 60 кг, а $60\%$-го --- 30 кг.\\
80. Пусть в растворе было $m$г соли тогда имеем систему уравнений $\begin{cases} (m+60)\cdot\cfrac{x}{100}=m,\\ (m+80)\cdot\cfrac{x-5}{100}=m.\end{cases}\Leftrightarrow \begin{cases} mx+60x=100m,\\ mx-5m+80x-400=100m.\end{cases}\Rightarrow
5m-20x+400=0\Rightarrow m=4x-80.$ Значит, $4x^2-80x+60x=400x-8000,\ x^2-420x+8000=0,\ (x-25)(x-80)=0,\ x=25$ или $x=80.$ Значит, раствор мог содержать $4\cdot25-80=20$г соли или $4\cdot80-80=240$г соли.\\
81. Пусть скорость первого трактора была равна $x$ км/ч, тогда скорость второго трактора была равна $x+5$ км/ч. Переведя минуты в часы (второй трактор потратил на дорогу на $5+3=8$ минут меньше), составим уравнение: $\cfrac{20}{x}-\cfrac{20}{x+5}=\cfrac{2}{15},\
\cfrac{20x+100-20x}{x(x+5)}=\cfrac{2}{15},\ x^2+5x=750,\ x^2+5x-750=0,\ x=25$км/ч, тогда скорость второго трактора равна $25+5=30$ км/ч.\\
82. Пусть скорость первого трактора была равна $x$ км/ч, тогда скорость второго трактора была равна $x+5$ км/ч. Переведя минуты в часы (второй трактор потратил на дорогу на $5+4=9$ минут меньше), составим уравнение: $\cfrac{15}{x}-\cfrac{15}{x+5}=\cfrac{3}{20},\
\cfrac{15x+75-15x}{x(x+5)}=\cfrac{3}{20},\ x^2+5x=500,\ x^2+5x-500=0,\ x=20$км/ч, тогда скорость второго трактора равна $20+5=25$ км/ч.\\
83. Пусть зарплата была $x$ рублей, а цена на товар --- $y$ рублей. Тогда изначально можно было купить $\cfrac{x}{y}$ товара, а теперь --- $\cfrac{1,04x}{1,3y}=\cfrac{104x}{130y}=\cfrac{4x}{5y}.$ Новое количество составляет от старого $\cfrac{\cfrac{4x}{5y}}{\cfrac{x}{y}}\cdot 100\%=80\%.$ Таким образом, теперь можно купить на $20\%$ меньше товара.\\
84. Пусть зарплата была $x$ рублей, а цена на товар --- $y$ рублей. Тогда изначально можно было купить $\cfrac{x}{y}$ товара, а теперь --- $\cfrac{1,02x}{1,7y}=\cfrac{102x}{170y}=\cfrac{3x}{5y}.$ Новое количество составляет от старого $\cfrac{\cfrac{3x}{5y}}{\cfrac{x}{y}}\cdot 100\%=60\%.$ Таким образом, теперь можно купить на $40\%$ меньше товара.\\
85. Пусть из бака вылили $x$ литров спирта, тогда в нём стало $64-x$л спирта и $x$л воды. Когда из бака вылили $x$ литров смеси, в нём осталось $64-x-x\cdot \cfrac{64-x}{64}$ литров спирта. Значит, $64-x-\cfrac{64x-x^2}{64}=49,\
4096-64x-64x+x^2=3136,\ x^2-128x+960=0,\ (x-8)(x-120)=0,\ x=8,$ так как всего в баке было 64 литра. Значит, в первый раз вылили 8 литров спирта, а во второй --- $64-8-49=7$ литров.\\
86. В конце концов в первом сосуде оказалось $(30+2):2=16$л спирта, а во втором --- $30-16=14$ литров. Пусть изначально в первом сосуде было $x$л спирта, тогда во втором было $30-x$ литров. Доля спирта в первом сосуде после дополнения стала $\cfrac{x}{30},$ тогда во второй сосуд после переливания попало $x\cdot\cfrac{x}{30}=\cfrac{x^2}{30}$ литров спирта, и всего в нём стало $30-x+\cfrac{x^2}{30}$ спирта. Когда из него вылили 12 литров из 30,
спирта осталось $\cfrac{18}{30}\cdot\left(30-x+\cfrac{x^2}{30}
ight)$ литров спирта, значит $\cfrac{18}{30}\cdot\left(30-x+\cfrac{x^2}{30}
ight)=14,\
90-3x+\cfrac{x^2}{10}=70,\ x^2-30x+200=0,\ (x-20)(x-10)=0,\ x=10$л или $x=20$л. Значение $x=10$л не подходит, так как тогда из первого сосуда сначала выливают 10 литров, а потом доливают 12, отчего он должен переполниться. Таким образом, в первом сосуде было 20 литров спирта, а во втором --- 10 литров.\\
87. Пусть скорость второго пешехода равна $x$ км/ч, тогда скорость первого равна $x+1$ км/ч. Так как места встречи они достигли одновременно, верно соотношение $\cfrac{9}{x+1}+\cfrac{1}{2}=\cfrac{10}{x},\
18x+x^2+x=20x+20,\ x^2-x-20=0,\ (x-5)(x+4)=0,\ x=5$км/ч. Тогда скорость пешехода, шедшего из $A,$ равна $5+1=6$км/ч.\\
88. Пусть скорость второго пешехода равна $x$ км/ч, тогда скорость первого равна $x+2$ км/ч. Так как места встречи они достигли одновременно, верно соотношение $\cfrac{15}{x+2}+\cfrac{1}{2}=\cfrac{12}{x},\
30x+x^2+2x=24x+48,\ x^2+8x-48=0,\ (x-4)(x+12)=0,\ x=4$км/ч. Тогда скорость пешехода, шедшего из $A,$ равна $4+2=6$км/ч.

ewpage
