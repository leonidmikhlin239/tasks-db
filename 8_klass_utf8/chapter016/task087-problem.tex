87. Пусть скорость второго пешехода равна $x$ км/ч, тогда скорость первого равна $x+1$ км/ч. Так как места встречи они достигли одновременно, верно соотношение $\cfrac{9}{x+1}+\cfrac{1}{2}=\cfrac{10}{x},\
18x+x^2+x=20x+20,\ x^2-x-20=0,\ (x-5)(x+4)=0,\ x=5$км/ч. Тогда скорость пешехода, шедшего из $A,$ равна $5+1=6$км/ч.\\
