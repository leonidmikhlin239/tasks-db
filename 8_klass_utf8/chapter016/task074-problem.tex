74. Пусть один тратит на забег внутри озера $x$ минут, а другой --- $x+7.$ Тогда их скорости равны $\cfrac{1}{x}$ и $\cfrac{1}{x+7}$ (кругов в минуту). В первый раз они встретятся, когда один обгонит другого на один круг, поэтому $\cfrac{1}{\cfrac{1}{x}-\cfrac{1}{x+7}}=84,\ \cfrac{x(x+7)}{7}=84,\
x^2+7x-588=0,\ (x+28)(x-21)=0,\ x=21$мин. Значит, скорость одного равна $\cfrac{1}{21},$ а другого --- $\cfrac{1}{28}.$ Если они побегут в противоположных направлениях, им  необходимо совместно преодолеть 1 круг, на это они потратят $\cfrac{1}{\cfrac{1}{21}+\cfrac{1}{28}}=12$ минут.\\
