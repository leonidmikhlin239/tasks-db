72. Пусть первого раствора изначально было взято $x$ литров, а второго --- $y.$ Тогда имеем систему уравнений $\begin{cases} 0,4x+0,75y=0,61(x+y),\\
0,4(x-40)+0,75(y-40)=0,65(x+y-80).\end{cases}\Leftrightarrow$\\$\begin{cases} 0,21x=0,14y,\\
0,4x+0,75y-46=0,65x+0,65y-52.\end{cases}\Leftrightarrow\begin{cases} y=\cfrac{3}{2}x,\\
0,25x-6=0,1y.\end{cases}\Leftrightarrow\begin{cases} y=\cfrac{3}{2}x,\\
0,25x-6=0,15x.\end{cases}\Leftrightarrow\begin{cases} y=90,\\
x=60.\end{cases}$ Значит, первого раствора было 60 литров, а второго --- 90.\\
