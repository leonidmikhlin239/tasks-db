\section{Раздел 13: Ряды: решения.}
1. Начнём раскладывать так, чтобы на каждой тарелке было как можно меньше орехов: $1+2+3+4+5+6+7+8+9=45.$\\
2. Начнём раскладывать так, чтобы на каждой тарелке было как можно меньше орехов: $1+2+3+4+5+6+7+8+9+10=55.$\\
3. Чтобы распилить бревно на 6 частей, необходимо сделать $6-1=5$ распилов, значит уйдёт $5\cdot1\text{мин}30\text{с}=5\cdot90\text{с}=450\text{с}=7\text{мин}30\text{с}.$\\
4. Чтобы распилить бревно на 8 частей, необходимо сделать $8-1=7$ распилов, значит уйдёт $7\cdot1\text{мин}30\text{с}=7\cdot90\text{с}=630\text{с}=10\text{мин}30\text{с}.$\\
5. На первые 8 тарелок положим как можно меньше грибов, а на последнюю --- все оставшиеся: $1+2+3+4+5+6+7+8+19=55.$\\
6. На первые 9 тарелок положим как можно меньше орехов, а на последнюю --- все оставшиеся: $1+2+3+4+5+6+7+8+9+20=65.$\\
7. Раз кот будет крайним слева, значит он с самого начала сидит с левого краю. Шарик может сесть между ним и Фёдором, значит следующим сидит Фёдор, потом Печкин, а потом Шарик (так как Шарик должен сидеть крайним справа).\\
8. Раз папа будет крайним справа, значит он с самого начала сидит с правого краю. Мама может сесть между ним и бабушкой, значит слева от него сидит бабушка, потом девочка, а потом мама (так как мама должна сидеть крайней слева).\\
9. Между ними расположено $43-17-1=25$ натуральных чисел.\\
10. Между ними расположено $53-19-1=33$ натуральных числа.\\
11. Между крайними точками $200-1=199$ промежутков, а значит расстояние $199\cdot2=398$см.\\
12. Между крайними точками $180-1=179$ промежутков, а значит расстояние $179\cdot5=895$см.\\
13. С 39-го дня до 239-го пройдёт $239-39+1=201$ день.\\
14. От 200-й до 239-й расположено $239-200+1=40$ книг.\\
15. Сложим все числа, имеющие чётные цифры (их 6), не забыв при этом чётную цифру 0: $1235+3124+6876+10005+217860+1000000=1239100.$\\
16. Сложим все числа, имеющие чётные цифры (их 6), не забыв при этом чётную цифру 0: $2+101+2398+7039+9899+80800=100239.$\\
17. Делится на 30 каждое 30-е число, таких будет $900:30=30.$ Значит, $900-30=870$ чисел не делятся на 30.\\
18. Делится на 25 каждое 25-е число, таких будет $750:25=30.$ Значит, $750-30=720$ чисел не делятся на 25.\\
19. За один невисокосный год Вася съест $30+31+30+31+31+28+31+30=242$ йогурта. Из любых 4 лет подряд ровно один год является високосным (то есть имеет один дополнительный день), поэтому всего Вася съест $242\cdot4+1=969$ йогуртов.\\
20. За один невисокосный год Вася съест $31+30+31+31+28+31+30+31=243$ банана. Из любых 4 лет подряд ровно один год является високосным (то есть имеет один дополнительный день), поэтому всего Вася съест $243\cdot4+1=973$ банана.\\
21. Заметим, что красивыми являются и слова, в которых количества букв А и Б вообще не отличаются, и в которых этих букв вообще нет. Таким образом, красивыми являются 9 слов: БАБА, БРАМСОНКАРЛСЁ, НЯНЯ, ХЕНСВИКСТУВАСУНДВИК, БАНЯ, ОБАМА, АЛЬБРУОВЕРБУ, АТТЕСТЬТЪОСТЕРУП, СУНДЭККЕНУТНЭСФУНБУ.\\
22. Заметим, что красивыми являются и слова, в которых количества букв А и Б вообще не отличаются, и в которых этих букв вообще нет. Таким образом, красивыми являются 8 слов: КУКУ, РОДЬДЪСТОРПОВЕРУД, ПСАКИ, КУСАЧКИ, СЭРОТОСТЕРО, БРУКЁКЛАССЕН, КУМУШКА, ХЕНСВИКСТУВАСУНДВИК.\\
23. Почётными являются 8 чисел: 2239, 100, 31337779, 324577711189, 31415926, 1000, 2390, 111.\\
24. Зачётными являются 7 чисел: 2239, 1100, 77313379, 70010, 2390, 777, 771113245789.\\
25. Это 17 чисел: $163,\ 173,\ 183,\ 193,\ 203,\ 213,\ 223$ и весь десяток от 230 до 239.\\
26. Это 17 чисел: весь десяток от 240 до 249 и $254,\ 264,\ 274,\ 284,\ 294,\ 304,\ 314.$\\
27. Превосходными являются 4 числа: $1234,\ 2053,\ 4703519,\ 8956713.$\\
28. Превосходными являются 3 числа: $2345,\ 1027,\ 13247.$\\
29. Всего от 239 до 1001 расположено $1001-239+1=763$ числа. Чётным является каждое второе число. Поделим с остатком: $763:2=381$ (ост. 1). Последнее число 1001 чётным не является, значит их 381.\\
30. Всего от 501 до 1239 расположено $1239-501+1=739$ чисел. Чётным является каждое второе число. Поделим с остатком: $739:2=369$ (ост. 1). Последнее число 1239 чётным не является, значит их 369.\\
31. Третий вагон с конца является шестым с начала, а третья дверь с конца --- второй с начала.\\
32. Четвёртый вагон с конца является пятым с начала, а вторая дверь с конца --- третьей с начала.\\
33. Уберём одинаковую еду из обоих заказов, получим, что 4 порции картошки стоят как 1 чизбургер и 3 ролла, то есть также 4 порции другой еды. Поэтому картошка не может быть ни самой дешёвой едой (тогда 4 порции картошки стоили бы дешевле 4 порций другой еды), ни самой дорогой едой (тогда 4 порции картошки стоили бы дороже 4 порций другой еды). Значит, картошка находится посередине и в порядке возрастания блюда располагаются так: ролл, картошка, чизбургер.\\
34. Уберём одинаковую еду из обоих заказов, получим, что 3 порции картошки стоят как 1 хачапури и 2 бифштекса, то есть также 3 порции другой еды. Поэтому картошка не может быть ни самой дешёвой едой (тогда 3 порции картошки стоили бы дешевле 3 порций другой еды), ни самой дорогой едой (тогда 3 порции картошки стоили бы дороже 3 порций другой еды). Значит, картошка находится посередине и в порядке возрастания блюда располагаются так: бифштекс, картошка, хачапури.\\
35. Количество букв <<а>> не изменилось, значит в исходном слове их так и было 4. Поэтому букв <<р>> должно быть хотя бы 2 (1 отличается от 4 на $3>2$). С другой стороны, если их хотя бы 3, то в новом слове их будет $3+4=7,$ что отличается от количества букв <<а>> на $7-4=3>2.$ Значит, букв <<р>> в исходном слове было 2.\\
36. Букв <<р>> стало 7, значит в исходном слове их было $7-2=5.$ Поэтому букв <<а>> было хотя бы 3 (2 отличается от 5 на $3>2.$) С другой стороны, если их хотя бы 4, то в новом слове  их будет $4+6=10,$ что отличается от количества букв <<р>> на $10-7=3>2.$ Значит, букв <<а>> в исходном слове было 3.\\
37. На 4 начинается 1 такое число 465. Если число начинается на 5, то 6 может стоять на 2 или на 3 месте. Если 6 стоит на 2 месте, то это числа от 560 до 569, весь десяток нам подходит. Если 6 стоит на 3 месте, то это ещё 9 чисел 506, 516, 526, 536, 546, 556, 576, 586, 596 (число 566 мы уже посчитали). Если число начинается на 6, то 5 может стоять на 2 или на 3 месте. Если 5 стоит на 2 месте, то это числа от 650 до 659, весь десяток нам подходит. Если 5 стоит на 3 месте, то это ещё 6 чисел: 605, 615, 625, 635, 645, 665 (число 655 мы уже посчитали). Итого нам подходят $1+10+9+10+6=36$ чисел.\\
38. На 2 начинается 1 такое число 243. Если число начинается на 3, то 4 может стоять на 2 или на 3 месте. Если 4 стоит на 2 месте, то это числа от 340 до 349, весь десяток нам подходит. Если 4 стоит на 3 месте, то это ещё 9 чисел 304, 314, 324, 334, 354, 364, 374, 384, 394 (число 344 мы уже посчитали). Если число начинается на 4, то 3 может стоять на 2 или на 3 месте. Если 3 стоит на 2 месте, то это числа от 430 до 439, весь десяток нам подходит. Если 3 стоит на 3 месте, то это ещё 5 чисел: 403, 413, 423, 443, 453 (число 433 мы уже посчитали). Итого нам подходят $1+10+9+10+5=35$ чисел.\\
39. Анна обогнала Галину и ещё двоих, значит она находится на 3 месте. Вера решила меньше задач, чем Галина, но больше, чем Божена, значит после Анны девочки идут в таком порядке: Галина, Вера, Божена. Божена и Евгения вместе решили столько же задач, сколько Вера и Дарья вместе, при этом Божена решила меньше задач, чем Вера, значит Евгения решила больше задач, чем Дарья. Поэтому девочек следует расположить так: ЕДАГВБ.\\
40. Антон пропустил вперёд Георгия и ещё двоих, значит он находится на 4 месте. Дмитрий решил задач больше, чем Георгий, но меньше, чем Василий, значит перед Антоном мальчики идут в таком порядке: Василий, Дмитрий, Георгий. Борис и Дмитрий вместе решили задач столько же, сколько Василий и Евгений вместе, при этом Дмитрий решил меньше задач, чем Василий, значит, Борис решил задач больше, чем Евгений. Поэтому мальчиков следует расположить так: ВДГАБЕ.\\
41. $1;\ 1+14=15;\ 15\cdot3=45;\ 45+14=59;\ 59\cdot3=177;\ 177+14=191;\ 191\cdot3=573.$\\
42. $3;\ 3+16=19;\ 19\cdot3=57;\ 57+16=73;\ 73\cdot3=219;\ 219+16=235;\ 235\cdot3=705.$\\
43. Перед ними 4 четвёрки, за ними 3 четвёрки, значит всего $4+3+1=8$ четвёрок, то есть $8\cdot4=32$ человека.\\
44. Всего выпало $38-13+1=26$ страниц, то есть $26:2=13$ листов.\\
45. Между кабинами 4 и 19 с обеих сторон по $19-4-1=14$ кабин, значит всего кабин $14\cdot2+2=30.$\\
46. Между спицами 11 и 26 с обеих сторон по $26-11-1=14$ спиц, значит всего спиц $14\cdot2+2=30.$\\
47. $3; 2\cdot2=4;\ 3+6=9;\ 4\cdot4=16;\ 9+6=15;\ 6\cdot6=36;\ 15+6=21;\ 8\cdot8=64.$\\
48. $1\cdot1=1; 14;\ 3\cdot3=9;\ 14+14=28;\ 5\cdot5=25;\ 28+14=42;\ 7\cdot7=49;\ 42+14=56.$\\
49. $2;\ 2\cdot3=6;\ 6+1=7;\ 7\cdot3=21;\ 21+1=22;\ 22\cdot3=66;\ 66+1=67;\ 67\cdot3=201.$\\
50. $0;\ 0;\ 0\cdot0=0;\ 0+1=1;\ 0+3=3;\ 1\cdot1=1;\ 1+1=2;\ 3+3=6;\ 2\cdot2=4;\ 2+1=3;\ 6+3=9;\ 3\cdot3=9;\ 3+1=4;\ 9+3=12;\ 4\cdot4=16;\ 4+1=5;\ 12+3=15;\ 5\cdot5=25.$\\
51. Подходят 2 числа: 5767897, 4357563.\\
52. Это числа от 534 до 1001, значит их $1001-534+1=468.$ Поделим с остатком: $468:7=66$ (ост. 6). Среди последних 6 чисел есть делящееся на 7: число 1001. Значит, всего таких чисел $66+1=67.$\\
53. Между 10 кустами сирени он посадил $10-1=9$ кустов жасмина, значит всего кустов стало $10+9=19.$ Таким образом, кустов шиповника он посадил $19-1=18.$\\
54. Между 12 <<Мерседесами>> припаркуется $12-1=11$ <<Фольксвагенов>>, значит всего машин будет $12+11=23.$ Таким образом, велосипедов будет поставлено $23-1=22.$\\
55. Однозначных чисел выписано 9, цифр в них тоже 9. Двузначных чисел выписано 90, цифр в них $90\cdot2=180.$ Трёхзначных чисел выписано $299-100+1=200,$ цифр в них $200\cdot3=600.$ Значит, всего выписано $9+180+600=789$ цифр.\\
56. Если медных монет было $x,$ остальных монет было $x+1$ и $x+x+1=25,\ 2x+1=25,\ 2x=24,\ x=12.$ Значит, золотых и серебряных монет было $25-12=13.$ Если золотых монет было $y,$ то серебряных монет было $2(y-1)$ и $y+2(y-1)=12,\ y+2y-2=13,\ 3y-2=13,\ 3y=15,\ y=5.$\\
57. Раз после разрезания по красным линиям получится 7 кусков, а после разрезания по зелёным получится 13, красных линий было $7-1=6,$ а зелёных --- $13-1=12.$ Поэтому после разрезания по всем этим линиям получится $6+12+1=19$ кусков.\\
58. Чурбачков получилось $10+1=11.$\\
59. Нужно сделать $12-1=11$ распилов.\\
60. Между 16 чурбачками расположено $16-1=15$ промежутков. 10 из них --- это распилы, значит оставшиеся $15-10=5$ --- это промежутки между брёвнами, которых таким образом было $5+1=6.$\\
61. Между 113 чурбачками расположено $113-1=112$ промежутков. 98 из них --- это распилы, значит оставшиеся $112-98=14$ --- это промежутки между брёвнами, которых таким образом было $14+1=15.$\\
62. Кусков было $25\text{м}:1\text{дм}=250\text{дм}:1\text{дм}=250,$ значит распилов надо было сделать $250-1=249.$ Таким образом, на это уйдёт $249\cdot3=747\text{мин}=12\text{ч }27\text{мин}.$\\
63. Одно бревно необходимо распилить на $3\text{м}:5\text{дм}=30\text{дм}:5\text{дм}=6$ кусков, для чего необходимо сделать $6-1=5$ распилов. Значит, всего распилов надо сделать $60\cdot5=300.$\\
64. Наиболее выгодно будет купить два 10-метровых бревна (и сделать на каждом по 4 распила) и одно 7-метровое бревно (и сделать на нём 3 распила). Всего Суслики-строители заплатят $2\cdot1200+900+(4\cdot2+3)\cdot100=4400$ рублей.\\
65. В зависимости от того, с какой стороны от тополя расположен дуб, это расстояние может быть равно $47+23=70$ метров, либо $47-23=24$ метра.\\
66. За 10 секунда Вася пробежал $5-1=4$ промежутка между флажками. Чтобы добежать до 25-го флажка, ему необходимо пробежать $25-1=24$ промежутка, на что уйдёт $6\cdot20=120$ секунд.\\
67. Лестница на 3 этаж состоит из $3-1=2$ пролётов, а лестница на 7 этаж --- из $7-1=6,$ значит она длиннее в $6:2=3$ раза.\\
68. В первом ряду расстояние от первого до последнего камешка равно $(8-1)\cdot2=14$см, а во втором ряду оно равно $(15-1)\cdot1=14$ см, значит эти ряды имеют одинаковую длину.\\
69. Между ними $1234-485-1=748$ чисел.\\
70. Она живёт на $25-6+1=20$ этаже.\\
71. Получилось $5+1=6$ кусков.\\
72. Они сделали $10-1=9$ разрезов.\\
73. Между 13 кусками расположено $13-1=12$ промежутков. 8 из них --- это разрезы, значит оставшиеся $12-8=4$ --- это промежутки между батонами, которых таким образом было $4+1=5.$\\
74. За 6 секунд лифт преодолевает $3-1=2$ пролёта, значит $9-1=8$ пролётов он преодолеет за $6\cdot4=24$ секунды.\\
75. В 12-метровой колбасе 3-метровых кусков будет $12:3=4,$ значит распилов было сделано $4-1=3,$ поэтому один распил занимает $12:3=4$ минуты. Чтобы распилить её на 1-метровые куски, необходимо сделать $12-1=11$ распилов, что займёт $11\cdot4=44$ минуты.\\
76. Всего промежутков между уроками было $30-1=29,$ но 5 из них --- это промежутки между 6 учебными днями. Значит, оставшиеся $29-5=24$ промежутка --- это перемены, поэтому всего он съел 24 конфеты.\\
77. Между ними всего $700-300-1=399$ чисел. Поделим с остатком: $399:2=199$ (ост. 1). Последнее рассматриваемое число 699 --- нечётное, значит всего нечётных чисел $199+1=200.$\\
78. Чётные и нечётные страницы в книге чередуются, если номер первой страницы был нечётным, номер последней должен быть чётным, поэтому последняя страница имеет номер 934. Таким образом, выпало $934-439+1=496$ страниц.\\
79. Всего между коржами было $10-1=9$ промежутков, из которых $3-1=2$ --- это промежутки между тортами. Значит, между слоями тортов было $9-2=7$ промежутков, поэтому слоёв шоколадного крема было $7-5=2.$\\
80. Если провести 11 параллелей, они разделят глобус на $11+1=12$ частей, а 10 меридианов поделят каждую из этих частей ещё на 10 частей, таким образом всего частей будет $12\cdot10=120.$\\
81. Получится $5+4=9$ частей (каждое разрезание на 5 частей добавляет $5-1=4$ новых части).\\
82. Раз после разрезания по красным линиям получится 11 кусков, а после разрезания по жёлтым получится 9, красных линий было $11-1=10,$ а жёлтых --- $9-1=8.$ Поэтому после разрезания по всем этим линиям получится $10+8+1=19$ кусков.\\
83. Каждое разрезание на 3 части добавляет 2 новых части, поэтому общее количество частей всегда будет нечётным (изначально 1 часть) и не может быть равно 2018.\\
84. Если красных и синих фишек было $x,$ Принц положил $2(x-1)$ фишек и $x+2(x-1)=19,\ x+2x-2=19,\ 3x-2=19,\ 3x=21,\ x=7.$ Значит, красных и синих фишек было 7. Пусть Герда положила $y$ фишек, тогда Кай положил $y-1$ и $y+y-1=7,\ 2y-1=7,\ 2y=8,\ y=4.$\\
85. Лестница на 4 этаж состоит из $4-1=3$ пролёта, а лестница на 2 этаж --- из $2-1=1,$ значит лестница на 4 этаж в $3:1=3$ раза.\\
86. Это 4 столба: по 2 столба с числами 3 и 13 в каждом направлении (они не совпадают, так как между городами 20 километров).\\
87. Это 6 столбов: по 3 столба с числами 5, 15 и 25 в каждом направлении (они не совпадают, так как между городами 29 километров).\\
88. Всего от 10 до 110 расположено $110-10+1=101$ число, на 2 делится каждое второе. Поделим с остатком: $101:2=50$ (ост. 1). Проверим последнее 1 число: оно равно 110 и является чётным, поэтому всего чётных чисел $50+1=51.$\\
89. Всего от 30 до 330 расположено $330-30+1=301$ число, на 3 делится каждое третье. Поделим с остатком: $301:3=100$ (ост. 1). Проверим последнее 1 число: оно равно 330 и делится на 3, поэтому всего делящихся на 3 чисел $100+1=101.$\\
90. а) Всего между 59 и 1001 расположено $1001-59-1=941$ число. На 7 делится каждое седьмое число, поделим с остатком: $941:7=134$ (ост. 3). Среди последних 3 чисел (998, 999, 1000) делящихся на 7 нет, значит их 134.\\
б) Всего между 59 и 1002 расположено $1002-59-1=942$ числа. На 7 делится каждое седьмое число, поделим с остатком: $942:7=134$ (ост. 4). Среди последних 4 чисел (998, 999, 1000, 1001) есть делящееся на 7 (1001:7=143), значит их $134+1=135.$\\
91. а) Всего между 59 и 2002 расположено $2002-59-1=1942$ числа. На 13 делится каждое тринадцатое число, поделим с остатком: $1942:13=149$ (ост. 5). Среди последних 5 чисел (1997, 1998, 1999, 2000, 2001) делящихся на 13 нет, значит их 149.\\
б) Всего между 60 и 2003 расположено $2003-60-1=1942$ числа. На 13 делится каждое седьмое число, поделим с остатком: $1942:13=149$ (ост. 5). Среди последних 5 чисел (1998, 1999, 2000, 2001, 2002) есть делящееся на 13 (2002:13=154), значит их $149+1=150.$\\
92. Если бы тарелок было 8, то на них было бы как минимум $1+2+3+4+5+6+7+8=36>29$ орехов. Значит, максимум тарелок может быть 7, например $1+2+3+4+5+6+8=29.$\\
93. Если бы тарелок было 9, то на них было бы как минимум $1+2+3+4+5+6+7+8+9=45>37$ орехов. Значит, максимум тарелок может быть 8, например $1+2+3+4+5+6+7+9=37.$\\
94. Единицу в разряде тысяч содержит $1999-1239+1=761$ число. Единицу в разряде сотен (но не в разряде тысяч) содержат $2199-2100+1=100$ чисел. Единицу в разряде десятков (но не в разрядах тысяч и сотен) содержат $2019-2010+1=10$ чисел. Единицу в разряде единиц (но не в разрядах тысяч и сотен) содержат ещё 9 чисел: 2001, 2021 2031, 2041, 2051, 2061, 2071, 2081, 2091. Всего подходящих чисел $761+100+10+9=880.$\\
95. Двойку в разряде тысяч содержит $2999-2239+1=761$ число. Двойку в разряде сотен (но не в разряде тысяч) содержат $3299-3200+1=100$ чисел. Двойку в разряде десятков (но не в разрядах тысяч и сотен) содержат $3029-3020+3129-3120+1=20$ чисел. Двойку в разряде единиц (но не в разрядах тысяч, сотен и десятков) содержат ещё 18 чисел: 3002, 3012, 3032, 3042, 3052, 3062, 3072, 3082, 3092 и такие же числа с 1 вместо 0 в разряде сотен. Всего подходящих чисел $761+100+20+18=899.$\\
96. Между первым и последним поездом находится $8-1=7$ интервалов. Если интервал равен 6 минут или меньше, то между первым и последним поездом пройдёт не более, чем $6\cdot7=42$ минуты и за 60 минут Марк точно должен увидеть ещё хотя бы один поезд. Если интервал равен 9 минут или больше, то между первым и последним поездом пройдёт хотя бы $9\cdot7=63$ минуты и Марк не успеет за 60 минут увидеть 8 поездов. Покажем, как интервал может быть 7 или 8 минут. Пусть интервал равен 7 минут, тогда разобьём 60 минут на три интервала: $5+49+6=60.$ Если первый поезд проехал мимо Марка через 5 минут после начала наблюдения, он увидит ровно 8 поездов за $7\cdot7=49$ минут. Пусть интервал равен 8 минут, тогда разобьём 60 минут на три интервала: $2+56+2=60.$ Если первый поезд проехал мимо Марка через 2 минуты после начала наблюдения, он увидит ровно 8 поездов за $8\cdot7=56$ минут.\\
97. Между первым и последним поездом находится $12-1=11$ интервалов. Если интервал равен 6 минут или меньше, то между первым и последним поездом пройдёт не более, чем $6\cdot11=66$ минут и за 90 минут Марк точно должен увидеть ещё хотя бы один поезд. Если интервал равен 9 минут, то между первым и последним поездом пройдёт хотя бы $9\cdot11=99$ минут и Марк не успеет за 90 минут увидеть 8 поездов. Покажем, как интервал может быть 7 или 8 минут. Пусть интервал равен 7 минут, тогда разобьём 90 минут на три интервала: $6,5+77+6,5=90.$ Если первый поезд проехал мимо Марка через 6,5 минут после начала наблюдения, он увидит ровно 8 поездов за $7\cdot11=77$ минут. Пусть интервал равен 8 минут, тогда разобьём 90 минут на три интервала: $1+88+1=90.$ Если первый поезд проехал мимо Марка через 1 минуту после начала наблюдения, он увидит ровно 8 поездов за $8\cdot11=88$ минут.\\
98. Если сделать 15 распилов, то получится $15+1=16$ частей. Значит, всего получилось $16\cdot3=48$ брёвнышек.\\
99. Всего пятизначных чисел $99999-10000+1=90000.$ На 15 делится каждое пятнадцатое, значит их $90000:15=6000.$\\
100. Всего пятизначных чисел $99999-10000+1=90000.$ На 21 делится каждое двадцать первое и $90000:21=4285$ (ост. 5). Так как среди чисел от 99995 до 99999 нет делящихся на 21, всего их 4285.\\
101. Эта сумма равна $12+10+123458=123480.$\\
102. Эта сумма равна $21+101+123458=123580.$\\
103. Продолжим последовательность до момента, когда она зациклится:
3, 4, 7, 1, 8, 9, 7, 6, 3, 9, 2, 1, 3, 4, 7, 1, ... Итак, цикл имеет вид
3, 4, 7, 1, 8, 9, 7, 6, 3, 9, 2, 1, в нём 12 цифр и он начинается с первой цифры последовательности. Так как $2022:12=168$ (ост. 6), на 2022-м месте стоит шестая цифра цикла, равная 9.\\
104. Продолжим последовательность до момента, когда она зациклится:
9, 2, 1, 3, 4, 7, 1, 8, 9, 7, 6, 3, 9, 2, 1, 3, ... Итак, цикл имеет вид
9, 2, 1, 3, 4, 7, 1, 8, 9, 7, 6, 3, в нём 12 цифр и он начинается с первой цифры последовательности. Так как $2022:12=168$ (ост. 6), на 2022-м месте стоит шестая цифра цикла, равная 7.\\
105. К сгущённому молоку Винни-Пух спустился с полки номер $4+7=11.$ Значит, мёд стоял на полке номер $11-3=8,$ так как эта полка средняя, всего полок $7+1+7=15.$\\
106. К бутерброду Матроскин спустился с этажа номер $5+8=13.$ Значит, Дядя Фёдор живёт на этаже номер $13-4=9,$ так как этот этаж средний, всего этажей $8+1+8=17.$\\
107. Раз расстояние между крайними точками равно 12 см, между первой и последней точкой $12:3=4$ промежутка, а значит Олег поставил $4+1=5$ точек.\\
108. Олеся, Инна и Надя ниже Ани, а Аня ниже Кати, значит Катя самая высокая.\\
109. Сумма всех однообразных чисел равна $104060+1003+694936=799999.$\\
110. Сумма всех однообразных чисел равна $504020+3001+392978=899999.$\\
111. Между 239 и 2339 находится $2339-239-1=2099$ чисел. Нечётными являются половина из них. Так как $2099:2=1049$ (ост. 1), а последним рассматриваемым числом является чётное число 2338, таких чисел 1049.\\
112. Между 239 и 2139 находится $2139-239-1=1899$ чисел. Нечётными являются половина из них. Так как $1899:2=949$ (ост. 1), а последним рассматриваемым числом является чётное число 2138, таких чисел 949.\\
113. Всего получилось $(11+1)\cdot5=60$ кусков.\\
114. Всего получилось $(4+1)\cdot14=70$ кусков.\\
115. С момента рождения Джонатана прошло $2024-1856=168$ лет. Так как первый год был годом дракона, а $168:12=14,$  всего таких лет было $14+1=15.$\\
116. С момента рождения Джонатана прошло $2024-1832=192$ года. Так как первый год был годом дракона, а $192:12=16,$  всего таких лет было $16+1=17.$\\
117. Всего между числами 239 и 777 находится $777-239-1=537$ чисел. Так как $537:2=268$ (ост. 1), и последнее рассматриваемое число 776 является чётным и не подходит, ответом в задаче является число 268.\\
118. Всего между числами 329 и 777 находится $777-329-1=447$ чисел. Так как $447:2=223$ (ост. 1), и последнее рассматриваемое число 776 является чётным и не подходит, ответом в задаче является число 223.\\
119. Перечислим все подходящие числа: 241, 223, 205, 421, 403, 601. Таких чисел 6.\\
120. Перечислим все подходящие числа: 251, 233, 215, 431, 413, 611. Таких чисел 6.

ewpage
