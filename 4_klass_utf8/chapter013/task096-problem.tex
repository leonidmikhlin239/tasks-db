96. Между первым и последним поездом находится $8-1=7$ интервалов. Если интервал равен 6 минут или меньше, то между первым и последним поездом пройдёт не более, чем $6\cdot7=42$ минуты и за 60 минут Марк точно должен увидеть ещё хотя бы один поезд. Если интервал равен 9 минут или больше, то между первым и последним поездом пройдёт хотя бы $9\cdot7=63$ минуты и Марк не успеет за 60 минут увидеть 8 поездов. Покажем, как интервал может быть 7 или 8 минут. Пусть интервал равен 7 минут, тогда разобьём 60 минут на три интервала: $5+49+6=60.$ Если первый поезд проехал мимо Марка через 5 минут после начала наблюдения, он увидит ровно 8 поездов за $7\cdot7=49$ минут. Пусть интервал равен 8 минут, тогда разобьём 60 минут на три интервала: $2+56+2=60.$ Если первый поезд проехал мимо Марка через 2 минуты после начала наблюдения, он увидит ровно 8 поездов за $8\cdot7=56$ минут.\\
