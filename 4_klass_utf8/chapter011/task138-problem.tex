138. Если периметр равностороннего треугольника равен 21дм, его сторона равна $21:3=7\text{дм}.$ Тогда вторая сторона прямоугольника равна $3500\text{см}^2:7\text{дм}=35\text{дм}^2:7\text{дм}=5\text{дм}$. Сторона квадрата с площадью $64\text{дм}^2$ равна 8дм (так как $8\cdot8=64.$). Тогда его периметр (как и периметр прямоугольника) равен $8\cdot4=32$ дм. Возможны два случая: новый прямоугольник построен на стороне, равной 7дм, или на стороне, равной 5дм. В первом случае вторая сторона построенного прямоугольника равна $32:2-7=9$дм, а значит его площадь равна $7\cdot9=63\text{дм}^2.$ Во втором случае его вторая сторона равна $32:2-5=11$дм, а значит его площадь равна $5\cdot11=55\text{дм}^2.$\\
