235. Раз они встретились, возможны два варианта: автомобили двигались навстречу друг другу или более быстрый догонял более медленный. В первом случае скорость их сближения была равна $90+60=150$км/ч, а во втором --- $90-60=30$км/ч. Так как 20 минут составляют одну треть от часа, расстояние между автомобилями за 20 минут до встречи могло быть равно $150:3=50$км или $30:3=10$км.\\
