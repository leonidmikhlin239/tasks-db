290. Первый и второй сближались со скоростью $4+3=7$км/ч, значит расстояние между домами первого и второго равно $7\cdot3=21$км. Первый и третий сближались со скоростью $4+5=9$км/ч, значит расстояние между домами первого и третьего равно $9\cdot3+9:3\cdot2=33$км (так как 40 минут составляют две трети от часа). Поэтому расстояние между домами второго и третьего равно $33-21=12$км. Третий догоняет второго со скоростью $5-3=2$км/ч, значит он догонит его через $12:2=6$ часов после выхода из дома. После встречи с первым пройдёт $6:00-3:40=2$часа 20 минут.\\
