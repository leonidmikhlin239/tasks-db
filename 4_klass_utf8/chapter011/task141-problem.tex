141. Раз расстояние между зайчатами через 30 секунд после столкновения стало равно 21 метру, скорость их удаления (равная сумме их скоростей) равна 42м/мин. Если скорость более медленного зайчонка равна $x,$ то скорость более быстрого --- $x+6$ и $x+x+6=42,\ 2x=36,\ x=18$м/мин. Значит, более быстрый зайчонок бежит со скоростью $18+6=24$м/мин. Он удалялся от своей норки 3 минуты и приближался к ней 30 секунд, то есть можно считать, что он удалялся 2 минуты и 30 секунд. Тогда он окажется на расстоянии $2\cdot24+24:2=60$м.\\
