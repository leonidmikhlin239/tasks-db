236. Раз они встретились, возможны два варианта: автомобили двигались навстречу друг другу или более быстрый догонял более медленный. В первом случае скорость их сближения была равна $80+40=120$км/ч, а во втором --- $80-40=40$км/ч. Так как 15 минут составляют одну четверть от часа, расстояние между автомобилями за 15 минут до встречи могло быть равно $120:4=30$км или $40:4=10$км.\\
