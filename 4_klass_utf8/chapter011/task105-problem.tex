105. Периметр цветника составляет $50\cdot4=200$м. Ширина составляет 2/8 длины, то есть длина больше в 4 раза. Пусть ширина равна $x,$ тогда длина равна $4x$ и верно равенство $(x+4x)\cdot2=200,\ 5x=100,\ x=20$м. Тогда длина равна $4\cdot20=80$м, а площадь цветника $S=20\cdot80=1600\text{м}^2.$ Розами засадили 3/8 цветника, то есть $1600:8\cdot3=600\text{м}^2.$ Гладиолусами засадили 3/5 остатка, то есть $(1600-600):5\cdot3=600\text{м}^2.$ Значит, под незабудки осталось $1000-600=400\text{м}^2.$\\
