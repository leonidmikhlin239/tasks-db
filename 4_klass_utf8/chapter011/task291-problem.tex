291. Первый и второй сближались со скоростью $3+4=7$км/ч, значит расстояние между домами первого и второго равно $7\cdot3=21$км. Первый и третий сближались со скоростью $3+5=8$км/ч, значит расстояние между домами первого и третьего равно $8\cdot3+8:4=26$км (так как 15 минут составляют одну четверть от часа). Поэтому расстояние между домами второго и третьего равно $26-21=5$км. Третий догоняет второго со скоростью $5-4=1$км/ч, значит он догонит его через $5:1=5$ часов после выхода из дома. После встречи с первым пройдёт $5:00-3:15=1$час 45 минут.\\
