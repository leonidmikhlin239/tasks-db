139. Если периметр равностороннего треугольника равен 9дм, его сторона равна $9:3=3\text{дм}.$ Тогда вторая сторона прямоугольника равна $2400\text{см}^2:3\text{дм}=24\text{дм}^2:3\text{дм}=8\text{дм}$. Сторона квадрата с площадью $36\text{дм}^2$ равна 6дм (так как $6\cdot6=36.$). Тогда его периметр (как и периметр прямоугольника) равен $6\cdot4=24$ дм. Возможны два случая: новый прямоугольник построен на стороне, равной 3дм, или на стороне, равной 8дм. В первом случае вторая сторона построенного прямоугольника равна $24:2-3=9$дм, а значит его площадь равна $3\cdot9=27\text{дм}^2.$ Во втором случае его вторая сторона равна $24:2-8=4$дм, а значит его площадь равна $8\cdot4=32\text{дм}^2.$\\
