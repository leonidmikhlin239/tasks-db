243. Собака удаляется от хозяина со скоростью $30-5=25$км/ч, а сближается с ним со скоростью $15+5=20$км/ч. На какое расстояние собака от хозяина убежала, на такое же расстояние она с ним в итоге и сблизилась. Пусть убегала она $x$ минут, а сближалась $y$ минут, тогда $25x=20y,\ 5x=4y,$ при этом $x+y=9.$ Число 9 в таком соотношении можно поделить единственным способом: $x=4,\ y=5.$ Тогда всего она пробежала $30\text{км/ч}\cdot4\text{мин}+15\text{км/ч}\cdot5\text{мин}=500\text{м/мин}\cdot4\text{мин}+
250\text{м/мин}\cdot5\text{мин}=3250$м.\\
