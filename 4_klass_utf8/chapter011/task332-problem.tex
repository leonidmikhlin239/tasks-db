332. Май содержит 4 полных недели и ещё три дня. Раз суббот было больше, чем четвергов, суббота в эти три дня попала, а четверг --- нет. Это могло произойти только если суббота была 29.05 или 30.05. Тогда в первом случае 1.05 было также субботой, а во втором --- пятницей (прошло 28 и 29 дней соответственно). Построим цепочку для невисокосного и високосного года: $1.05\stackrel{-30}{
ightarrow}1.04\stackrel{-31}{
ightarrow}1.03
\stackrel{-28(-29)}{
ightarrow}1.02\stackrel{-2}{
ightarrow}30.01.$
В невисокосном году надо "отмотать назад" $30+31+28+2=91$ день, а в високосном --- $30+31+29+2=92.$ Так как 91 делится на 7, а 92 даёт остаток 1, в первом случае 30.01 могло быть субботой или пятницей, а во втором --- пятницей или четвергом. Значит, все возможные ответы --- это четверг, пятница и суббота.\\
