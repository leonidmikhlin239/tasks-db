72. В задаче не дано, каким был год: високосным или нет. Поэтому необходимо разобрать два случая. Рассмотрим следующую цепочку:
$10.01\stackrel{+31}{
ightarrow}10.02\stackrel{+28(29)}{
ightarrow}10.03.$ В невисокосном году до 10.03 пройдёт $31+28=59$ дней, а в високосном --- $31+29=60$ дней. Число 59 даёт при делении на 7 остаток 3, а число 60 --- остаток 4. Значит, в невисокосном году к понедельнику надо прибавить 3 дня, а в високосном --- 4. Поэтому в невисокосном году 10.03 будет четвергом, а в високосном --- пятницей. Тогда ближайшая суббота в невисокосном году будет 12.03, а в високосном --- 11.03. Последняя суббота марта наступит через 2 недели, то есть 26.03 в невисокосном году и 25.03 в високосном.\\
