73. В задаче не дано, каким был год: високосным или нет. Поэтому необходимо разобрать два случая. Рассмотрим следующую цепочку:
$9.01\stackrel{+31}{
ightarrow}9.02\stackrel{+28(29)}{
ightarrow}9.03.$ В невисокосном году до 9.03 пройдёт $31+28=59$ дней, а в високосном --- $31+29=60$ дней. Число 59 даёт при делении на 7 остаток 3, а число 60 --- остаток 4. Значит, в невисокосном году ко вторнику надо прибавить 3 дня, а в високосном --- 4. Поэтому в невисокосном году 9.03 будет пятницей, а в високосном --- субботой. Тогда ближайшая пятница в невисокосном году так и будет 9.03, а в високосном --- 8.03. Последняя пятница марта наступит через 3 недели, то есть 30.03 в невисокосном году и 29.03 в високосном.\\
