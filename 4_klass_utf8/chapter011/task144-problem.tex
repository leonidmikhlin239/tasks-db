144. Раз площадь квадратной дощечки равна $64\text{см}^2,$ её сторона равна 8см (так как $8\cdot8=64$). Площадь полученного приложением прямоугольника равна $96\text{см}^2,$ а одна из его сторон равна стороне квадрата, то есть 8см. Поэтому вторая его сторона (полученная как сумма сторон квадрата и прямоугольника) равна $96\text{см}^2:8\text{см}=12$см, а значит это и есть большая сторона, которая равна одной из сторон треугольной дощечки. Периметр полученного прямоугольника равен $P=(12+8)\cdot2=40$см. У треугольной дощечки равны какие-то две стороны, а значит возможны два случая: две стороны по 12 см, а значит третья сторона равна $40-12\cdot2=16$см или две стороны по $x$ см, тогда $2x+12=40,\ 2x=28,\ x=14$см.\\
