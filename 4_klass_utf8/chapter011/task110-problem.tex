110. Оля потратила 12г краски на $12\cdot12\cdot6=864\text{см}^2.$ Вася распилил большой куб на $12\cdot12\cdot12=1728$ маленьких кубиков, общая площадь поверхности которых равна $1\cdot1\cdot6\cdot1728=10368\text{см}^2,$ из которых $864\text{см}^2$ уже покрашены, а значит окрасить надо ещё $10368-864=9504\text{см}^2.$ Эта площадь больше в $9504:864=11$ раз, а значит краски потребуется больше в 11 раз: $11\cdot12=132$г.\\
