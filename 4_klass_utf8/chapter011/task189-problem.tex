189. Пусть ширина сада была равна $x$ метров, тогда длина была равна $x+1000\text{ дм}=x+100$м. Тогда после расчистки пустыря длина станет равна $x+120,$ а ширина --- $x+20$ метров и верно соотношение $x+120=2\cdot(x+20),\ x+120=2x+40,\ 120=x+40,\ x=80$м. Тогда ширина и длина сада были равны 80 и 180 метров, а стали равны 100 и 200 метров. Таким образом, площадь увеличится на $100\cdot200-80\cdot180=5600\text{ м}^2.$\\
