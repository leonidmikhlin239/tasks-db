307. Преобразуем скорость грузовика: 60км/ч$=1$км/мин. Тогда за 1 ч 40 мин$=100$мин он проехал $100$км, а значит автобус проехал $20+100=120$км. Автобус был в пути 1 час 41 минуту, но сделал 11 остановок, то есть ехал 1 час 30 минут. Так как он проехал за это время $120$км, его скорость равна $80$км/ч ($80+80:2=120$). Автобус догнал грузовик на $121$км, но остановился на 1 мин. За это время грузовик уехал на 1 км. Скорость сближения равна $80-60=20$км/ч или 1км/3мин, значит автобус снова догонит грузовик через 3 минуты. Во время первой встречи автобус был на расстоянии 120 км от города и остановился на 1 минуту. Далее за каждые 30 минут он проезжает $80:2=40$км и 3 раза останавливается на 1 минуту. Так как от момента первой встречи до прибытия прошло как раз 34 минуты, расстояние между городом и деревней равно
$120+40=160$км.\\
