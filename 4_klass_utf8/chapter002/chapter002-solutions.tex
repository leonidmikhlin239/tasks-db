\section{Раздел 9: Примеры с приёмами рациональных вычислений решения.}
1.$281\cdot323+281\cdot227+119\cdot550=281\cdot(323+227)+119\cdot550=281\cdot550+119\cdot550=550\cdot(281+119)=550\cdot400=220000.$\\
2.$319\cdot233+319\cdot217+181\cdot450=319\cdot(233+217)+181\cdot450=319\cdot450+181\cdot450=450\cdot(319+181)=450\cdot500=225000.$\\
3.$215\cdot407+92\cdot193-407\cdot123=407\cdot(215-123)+92\cdot193=407\cdot92+92\cdot193=92\cdot(407+193)=92\cdot600=55200.$\\
4.$317\cdot308+83\cdot192-308\cdot234=308\cdot(317-234)+83\cdot192=308\cdot83+83\cdot192=83\cdot(308+192)=83\cdot500=41500.$\\
5.$239\cdot367-600\cdot111+233\cdot239-128\cdot489=239\cdot(367+233)-600\cdot111-128\cdot489=239\cdot600-600\cdot111-128\cdot489=600\cdot(239-111)-128\cdot489=600\cdot128-
128\cdot489=128\cdot(600-489)=128\cdot111=14208.$\\
6. У Вани: $239\cdot600-128\cdot489-600\cdot111=600\cdot(239-111)-128\cdot489=600\cdot128-128\cdot489=128\cdot(600-489)=128\cdot111.$ А у Тани $129\cdot112,$ больше оба множителя, а значит больше и результат.\\
7. $239\cdot329-500\cdot112+171\cdot239-127\cdot389=239\cdot(329+171)-500\cdot112-127\cdot389=239\cdot500-500\cdot112-127\cdot389=500\cdot(239-112)-127\cdot389=
500\cdot127-127\cdot389=127\cdot(500-389)=127\cdot111=14097.$\\
8. У Димы: $239\cdot500-127\cdot389-500\cdot112=500\cdot(239-112)-127\cdot389=500\cdot127-127\cdot389=127\cdot(500-389)=127\cdot111.$ А Катя высчитала $128\cdot109,$ сравним их результаты. У Димы: $127\cdot111=127\cdot(109+2)=127\cdot109+127\cdot2=127\cdot109+254.$ У Кати: $128\cdot109=(127+1)\cdot109=127\cdot109+1\cdot109=127\cdot109+109.$ Таким образом, результат Димы больше на $254-109=145.$\\
9. $239\cdot135+112\cdot234-366\cdot127+239\cdot231=239\cdot(135+231)+112\cdot234-366\cdot127=239\cdot366+112\cdot234-366\cdot127=366\cdot(239-127)+112\cdot234=
366\cdot112+112\cdot234=112\cdot(366+234)=112\cdot600=67200.$\\
10. $1239\cdot478=1239\cdot2\cdot239=2478\cdot239>2478\cdot238.$ Значит, у Кости результат больше, чем у Жени.\\
11. $239\cdot181+124\cdot302-398\cdot115+217\cdot239=239\cdot(181+217)+124\cdot302-398\cdot115=239\cdot398+124\cdot302-398\cdot115=398\cdot(239-115)+124\cdot302=
398\cdot124+124\cdot302=124\cdot(398+302)=124\cdot700=86800.$\\
12. $239\cdot2478=239\cdot2\cdot1239=478\cdot1239<478\cdot1241.$ Значит, у Жени результат больше, чем у Кости.\\
13. $143\cdot239-239-39\cdot350+239\cdot207=239\cdot(143+207)-239-39\cdot350=239\cdot350-239-39\cdot350=350\cdot(239-39)-239=350\cdot200-239=70000-239=69761.$\\
14. $237\cdot239-39\cdot450-239+239\cdot213=239\cdot(237+213)-39\cdot450-239=239\cdot450-39\cdot450-239=450\cdot(239-39)-239=450\cdot200-239=90000-239=89761.$\\
15. $624\times17+625\times 19+626\times14=624\times17+(624+1)\times19+(624+2)\times14=624\times17+624\times19+1\times19+624\times14+2\times14=
624\times(17+19+14)+19+28=624\times50+47=31200+47=31247.$\\
16. $526\times18+525\times19+524\times13=(524+2)\times18+(524+1)\times19+524\times13=524\times18+2\times18+524\times19+1\times19+524\times13=524\times(18+19+13)+36+19=
524\times50+55=26255.$\\
17. Первое произведение: $239\times243=239\times(242+1)=239\times242+239\times1=239\times242+239.$ Второе произведение: $240\times242=(239+1)\times242=239\times242+1\times242=239\times242+242.$ Таким образом, произведения отличаются на $242-239=3.$\\
18. Первое произведение: $237\times240=237\times(239+1)=237\times239+237\times1=237\times239+237.$ Второе произведение: $238\times239=(237+1)\times239=237\times239+1\times239=237\times 239+239.$ Таким образом, произведения отличаются на $239-237=2.$\\
19. $1002+499\times243-998+501\times239=501\times2+499\times243-499\times2+501\times239=501\times(2+239)+499\times(243-2)=501\times241+499\times241=
241\times(501+499)=241\times1000=241000.$\\
20. $1004+498\times243-996+502\times239=502\times2+498\times243-498\times2+502\times239=502\times(2+239)+498\times(243-2)=502\times241+498\times241=
241\times(502+498)=241\times1000=241000.$\\
21. $279\times3\times137-93\times3\times410+632\times373-631\times372=279\times411-279\times410+(631+1)\times373-631\times372=279\times(411-410)+631\times373+1\times373-
631\times372=279\times1+631\times(373-372)+373=279+631+373=1283.$\\
22.$189\times3\times307-63\times3\times920+737\times293-736\times292=189\times921-189\times920+(736+1)\times293-736\times292=189\times(921-920)+736\times293+1\times293-
736\times292=189\times1+293+736\times(293-292)=189+293+736=1218.$\\
23. $1) 6547-5983=564.\ 2) 487+569=\textbf{1}056.\ 3)3000\times900=2700000<3415\times926<3500\times1000=3500000,$ значит это произведение точно не начинается с 1.
$4) 34000\times37000=1258000000<34789\times37483<35000\times40000=1400000000,$ значит это произведение точно начинается с $1.$ В примерах $5)$ и $6)$ надо определить первое неполное делимое и начать делить в столбик. С цифры $1$ начинается ответ в примере $5)$ (вторую цифру искать не надо). Пример $7)$ также необходимо начать считать в столбик, при этом можно сначала найти самую первую цифру ответа, она не равна $1.$ Пример $8): 217\times342-342\times146+71\times158=342\times(217-146)+71\times158=342\times71+71\times158=71\times(342+158)=71\times500=35500.$\\
24. $1)4385+5892=\textbf{1}0277.\ 2) 763-677=86.\ 3)3700\times300=1110000<3711\times358<4000\times400=1600000,$ значит это произведение точно начинается с 1.
$4)30000\times90000=2700000000<32592\times98734<33000\times100000=3300000000,$ значит это произведение точно не начинается с 1.
В примерах $5)$ и $6)$ надо определить первое неполное делимое и начать делить в столбик. С цифры $1$ начинается ответ в примере $6)$ (вторую цифру искать не надо). Пример $7)$ также необходимо начать считать в столбик, при этом можно сначала найти самую первую цифру ответа, она не равна $1.$ Пример $8):
329\times268-268\times273+56\times132=268\times(329-273)+56\times132=268\times56+56\times132=56\times(268+132)=56\times400=22400.$\\
25. $345\cdot73 + 23\cdot25 + 345\cdot27 + 77\cdot25=345\cdot(73+27)+25\cdot(23+77)=345\cdot100+25\cdot100=100\cdot(345+25)=100\cdot370=37000.$\\
26. $162\cdot54+12\cdot18 + 88\cdot18+ 162\cdot46=162\cdot(54+46)+18\cdot(12+88)=162\cdot100+18\cdot100=100\cdot(162+18)=100\cdot180=18000.$\\
27. $15\cdot34-15\cdot14+15\cdot80=15\cdot(34-14+80)=15\cdot100=1500.$\\
28. $(84\cdot92+14\cdot53\cdot6-7\cdot5\cdot12):4:21:5+(473\cdot25\cdot0:11)-(36:3+36:6)+(26-10):(13-5)=(84\cdot92+53\cdot84-5\cdot84):(4\cdot21):5+0-(12+6)+16:8=
84\cdot(92+53-5):84:5-18+2=140:5-18+2=28-18+2=12.$\\
29. $(72\cdot99+12\cdot68\cdot6-3\cdot7\cdot24):9:8:5+(728:13\cdot25\cdot0)-(30:2+30:3)+(28-8):(7-2)=(72\cdot99+72\cdot68-7\cdot72):(9\cdot8):5+0-(15+10)+20:5=
72\cdot(99+68-7):72:5-25+4=160:5-25+4=32-25+4=11.$\\
30. $c=48\cdot9:2-(12\cdot4):2+608=9\cdot48:2-48:2+608=(48:2)\cdot(9-1)+608=192+608=800.$\\
$p=56\cdot4:7\cdot25=56:7\cdot(4\cdot25)=8\cdot100=800=c.$ Значит, четверть от $c$ меньше удвоенного $p$ в $4\cdot2=8$ раз.\\
31. $a=54\cdot7:2-(18\cdot6):2+565=54\cdot(7-2):2+565=135+565=700.$ $d=63\cdot25:9\cdot4=63:9\cdot(25\cdot4)=7\cdot 100=700=a.$
Значит, утроенное число $a$ больше половины числа $d$ в $3\cdot2=6$ раз.\\
32. $(390\cdot2\cdot85+13\cdot11\cdot60-1560:2):13:20:3+115=(780\cdot85+780\cdot11-780\cdot1):(13\cdot20\cdot3)+115=780\cdot(85+11-1):780+115=95+115=210.$\\
33. $155\cdot208-35\cdot92+155\cdot92-35\cdot208=155\cdot(208+92)-35\cdot(92+208)=155\cdot300-35\cdot300=300\cdot(155-35)=300\cdot120=36000.$\\
34. $178\cdot144-38\cdot56+178\cdot56-38\cdot144=178\cdot(144+56)-38\cdot(56+144)=178\cdot200-38\cdot200=200\cdot(178-38)=200\cdot140=28000.$\\
35. $144\cdot321+72\cdot4-144\cdot123=144\cdot321+144\cdot2-144\cdot123=144\cdot(321+2-123)=144\cdot200=28800.$\\
36. $k=(36\cdot14):3-24\cdot9:2+102=12\cdot14-12\cdot9+102=12\cdot(14-9)+102=60+102=162.$ $p=72\cdot4:8\cdot25=72:8\cdot(4\cdot25)=9\cdot100=900.$ Тогда
$(900:2):(162:9)=450:18=25.$\\
37. $x=54\cdot25:9\cdot4=54:9\cdot(25\cdot4)=6\cdot100=600.$ $a=24+(48\cdot13):4-36\cdot8:3=24+12\cdot13-12\cdot8=24+12\cdot(13-8)=24+60=84.$ Тогда
$(600:2):(84:7)=300:12=25.$\\
38. Первое число:$1239\times238=(1238+1)\times238=1238\times238+1\times238=1238\times238+238.$ Второе число: $1238\times239=1238\times(238+1)=1238\times238+1238\times1=1238\times238+1238.$ Значит, второе число больше на $1238-238=1000.$\\
39. Первое число:$2238\times239=2238\times(238+1)=2238\times238+2238\times1=2238\times238+2238.$ Второе число: $2239\times238=(2238+1)\times238=
2238\times238+1\times238=2238\times238+238.$ Значит, первое число больше на $2238-238=2000.$\\
40. Первое число: $238\times239\times1240=238\times239\times(240+1000)=238\times239\times240+238\times239\times1000.$ Второе число: $1238\times239\times240=(238+1000)\times239\times240=238\times239\times240+239\times240\times1000=238\times239\times240+239\times(238+2)\times1000=
238\times239\times240+239\times238\times1000+239\times2\times1000=238\times239\times240+239\times238\times1000+478000.$ Значит, второе число больше на 478000.\\
41. Первое число: $1238\times239\times240=(238+1000)\times239\times240=238\times239\times240+239\times240\times1000=238\times239\times240+239\times(238+2)\times1000=
238\times239\times240+239\times238\times1000+239\times2\times1000=238\times239\times240+239\times238\times1000+478000.$ Второе число: $238\times239\times1240=238\times239\times(240+1000)=238\times239\times240+238\times239\times1000.$ Значит, первое число больше на 478000.\\
42. Первое число: $2238\times239\times1240=(1238+1000)\times239\times1240=
1238\times239\times1240+1000\times239\times1240.$ Второе число:
$1238\times239\times2240=(1240+1000)\times239\times1238=1240\times239\times1238+
1000\times239\times1238.$ Первое число больше на $1000\times239\times1240-1000\times239\times1238=1000\times239\times(1240-1238)=478000.$\\
43. Первое число: $1239\times240\times2241=(1241+1000)\times240\times1239=1241\times240\times1239+1000\times240\times1239.$ Второе число: $2239\times240\times1241=(1239+1000)\times240\times1241=
1239\times240\times1241+1000\times240\times1241.$ Второе число больше на
 $1000\times240\times1241-1000\times240\times1239=1000\times240\times(1241-1239)=480000.$\\
44. $A=125\cdot39\cdot8\cdot11:250:13=(125\cdot8):250\cdot(39:13)\cdot11=1000:250\cdot3\cdot11=4\cdot33=132,\ B=24\cdot17:2-2\cdot7\cdot12+3\cdot45\cdot4-(5+7)\cdot37=17\cdot12-14\cdot12+45\cdot12-37\cdot12=12\cdot(17-14+45-37)=12\cdot11=132.$ Значит, $A=B.$\\
45. Преобразуем первое выражение: $1239\times2038=(1039+200)\times2038=1039\times2038+200\times2038.$ Аналогично преобразуем второе:
$2238\times1039=(2038+200)\times1039=2038\times1039+200\times1039.$ Первое выражение больше на $200\times2038-200\times1039=
200\times(2038-1039)=200\times(1000-1)=200000-200=199800.$\\
46. Преобразуем первое выражение: $1239\times2138=(1139+100)\times2138=1139\times2138+100\times2138.$ Аналогично преобразуем второе:
$2238\times1139=(2138+100)\times1139=2138\times1139+100\times1139.$ Первое выражение больше на $100\times2138-100\times1139=
100\times(2138-1139)=100\times(1000-1)=100000-100=99900.$\\
47. $339 \times 340 + 340 \times 341 + 341 \times 320=
340\cdot(339+341)+341\cdot320=340\cdot680+(340+1)\cdot320=
340\cdot680+340\cdot320+1\cdot320=340\cdot(680+320)+320=
340\cdot1000+320=340320.$\\
48. $339 \times 340 + 340 \times 342 + 341 \times 319=
340\cdot(339+342)+341\cdot319=340\cdot681+(340+1)\cdot319=
340\cdot681+340\cdot319+1\cdot319=340\cdot(681+319)+319=
340\cdot1000+319=340319.$
\newpage
