\section{Раздел 2: примеры с приёмами рациональных вычислений}
Во всех задачах необходимо применять вышеописанные приёмы, возможно по несколько раз.\\
1. $281\cdot323+281\cdot227+119\cdot550.$\\
2. $319\cdot233+319\cdot217+181\cdot450.$\\
3. $215\cdot407+92\cdot193-407\cdot123.$\\
4. $317\cdot308+83\cdot192-308\cdot234.$\\
5. $239\cdot367-600\cdot111+233\cdot239-128\cdot489.$\\
6. Ваня вычислил $239\cdot600-128\cdot489-600\cdot111,$ а Таня вычислила $129\cdot112.$ У кого из них результат получился больше?\\
7. $239\cdot329-500\cdot112+171\cdot239-127\cdot389.$\\
8. Дима вычислил $239\cdot500-127\cdot389-500\cdot112,$ а Катя вычислила $128\cdot109.$ У кого из них результат получился больше?\\
9. $239\cdot135+112\cdot234-366\cdot127+239\cdot231.$\\
10. Костя вычислил $1239\cdot 478,$ а Женя вычислила $2478\cdot238.$ У кого из них результат получился больше?\\
11. $239\cdot181+124\cdot302-398\cdot115+217\cdot239.$\\
12. Костя вычислил $239\cdot2478,$ а Женя вычислила $478\cdot1241.$ У кого из них результат получился больше?\\
13. $143\cdot239-239-39\cdot350+239\cdot207.$\\
14. $237\cdot239-39\cdot450-239+239\cdot213.$\\
15. $624\times17+625\times 19+626\times14.$\\
16. $526\times18+525\times19+524\times13.$\\
17. На сколько отличаются числа $239\times243$ и $240\times242?$\\
18. На сколько отличаются числа $237\times240$ и $238\times239?$\\
19. $1002+499\times243-998+501\times239.$\\
20. $1004+498\times243-996+502\times239.$\\
21. $279\times3\times137-93\times3\times410+632\times373-631\times372.$\\
22. $189\times3\times307-63\times3\times920+737\times293-736\times292.$\\
23. Какие из результатов данных действий начинаются с цифры 1? Выпишите в ответ номера нужных примеров.\\
$1)6547-5983;\qquad 2)487+569;\qquad 3)3415\times926;\qquad 4)34789\times37483;\qquad5) 67014910068636:347968524;$
$6)6633327517568932:192589;\quad7)10457852355532-932381476923;\quad8)217\times342-342\times146+71\times158.$\\
24. Какие из результатов данных действий начинаются с цифры 1? Выпишите в ответ номера нужных примеров.\\
$1)4385+5892;\qquad 2)763-677;\qquad 3)3711\times358;\qquad4)32592\times98734;\qquad5) 74173787591475:356894725;$
$6)4651722726829701:235681;\quad7)10736528548858-962332147692;\quad8)329\times268-268\times273+56\times132.$\\
25. $345\cdot73 + 23\cdot25 + 345\cdot27 + 77\cdot25.$\\
26. $162\cdot54+12\cdot18 + 88\cdot18+ 162\cdot46.$\\
27. $15\cdot34-15\cdot14+15\cdot80.$\\
28. $(84\cdot92+14\cdot53\cdot6-7\cdot5\cdot12):4:21:5+(473\cdot25\cdot0:11)-(36:3+36:6)+(26-10):(13-5).$\\
29. $(72\cdot99+12\cdot68\cdot6-3\cdot7\cdot24):9:8:5+(728:13\cdot25\cdot0)-(30:2+30:3)+(28-8):(7-2).$\\
30. Даны два числа $c=48\cdot9:2-(12\cdot4):2+608$ и $p=56\cdot4:7\cdot25.$ Вычислите эти числа и узнайте, во сколько раз четверть числа $c$ меньше удвоенного числа $p.$\\
31. Даны два числа $a=54\cdot7:2-(18\cdot6):2+565$ и $d=63\cdot25:9\cdot4.$ Вычислите эти числа и узнайте, во сколько раз утроенное число $a$ больше половины числа $d.$\\
32. $(390\cdot2\cdot85+13\cdot11\cdot60-1560:2):13:20:3+115.$\\
33. $155\cdot208-35\cdot92+155\cdot92-35\cdot208.$\\
34. $178\cdot144-38\cdot56+178\cdot56-38\cdot144.$\\
35. $144\cdot321+72\cdot4-144\cdot123.$\\
36. Даны два числовых выражения:\\
$k=(36\cdot14):3-24\cdot9:2+102$ и $p=72\cdot4:8\cdot25.$\\
1) Вычислите $k$ и $p.$\\
2) Узнайте, во сколько раз девятая часть числа $k$ меньше половины числа $p.$\\
37. Даны два числовых выражения:\\
$x=54\cdot25:9\cdot4$ и $a=24+(48\cdot13):4-36\cdot8:3.$\\
1) Вычислите $x$ и $a.$\\
2) Узнайте, во сколько раз половина числа $x$ больше седьмой части числа $a.$\\
38. Какое число больше и на сколько: $1239\times238$ или $1238\times239?$\\
39. Какое число больше и на сколько: $2238\times239$ или $2239\times238?$\\
40. Какое число больше и на сколько: $238\times239\times1240$ (первое) или $1238\times239\times240$ (второе)?\\
41. Какое число больше и на сколько: $1238\times239\times240$ (первое) или $238\times239\times1240$ (второе)?\\
42. Какое из чисел больше и на сколько: $2238\times239\times1240$ (первое) или $1238\times239\times2240$ (второе)?\\
43. Какое из чисел больше и на сколько: $1239\times240\times2241$ (первое) или $2239\times240\times1241$ (второе)?\\
44. Даны два числовых выражения:
$$\begin{array}{l} A=125\cdot39\cdot8\cdot11:250:13;\\ B=24\cdot17:2-2\cdot7\cdot12+3\cdot45\cdot4-(5+7)\cdot37.\end{array}$$
Вычислите значения этих выражений и сравните их.\\
45. Какое из чисел больше и на сколько: $1239\times2038$ (первое) или $2238\times1039$ (второе)?\\
46. Какое из чисел больше и на сколько: $1239\times2138$ (первое) или $2238\times1139$ (второе)?\\
47. $339 \times 340 + 340 \times 341 + 341 \times 320.$ \
48. $339 \times 340 + 340 \times 342 + 341 \times 319.$
\newpage
