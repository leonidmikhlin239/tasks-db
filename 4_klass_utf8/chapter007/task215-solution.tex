217. Взвесим 2 монеты: если они равны, то фальшивая найдена. Если они не равны, то оставшаяся монета точно настоящая, взвесим её с одной из первых и найдём фальшивую. Итого понадобится 3 действия.

Взвесим 2 монеты: если они равны, то фальшивая среди оставшихся 2, а первые 2 монеты --- настоящие. Взвесим настоящую с одно из оставшихся: если они равны, то последняя монета --- фальшивая, в противном случае фальшивая взвешиваемая. Если первые 2 монеты не равны, то оставшиеся точно настоящие, взвесим одну из них с любой из первых 2 и аналогично найдём фальшивую. Итого понадобится 3 действия.

Разделим монеты на 3 группы по 3 монеты и взвесим 2 из них. Если они равны, то в последней группе все монеты настоящие. Взвесим её с одной из первых двух и определим, в какой группе фальшивая монета и легче она или тяжелее, далее действуем как в задаче 215. Если изначальные группы не равны, то в последней группе все монеты настоящие и взвесив её с одной из первых двух, найдём и группу с фальшивой монетой и определим, легче она или тяжелее. Далее действуем как в задаче 215. Итого понадобится 4 действия.\\
