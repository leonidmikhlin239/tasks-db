72. Если бы дырки не было, то в прямоугольнике $20\times30$ было бы $20\cdot29+30\cdot19=1150$ перегородок (см. задачи 61-62). Дырка имеет размеры $(20-2\cdot4)\times(30-2\cdot4)=12\times22.$ Внутри неё <<пропадут>> $12\cdot21+22\cdot11=494$ перегородки. Также не будет считаться перегородками периметр дырки: $(12+22)\cdot2=68.$ Итого в букве О останется $1150-494-68=588$ перегородок.\\
