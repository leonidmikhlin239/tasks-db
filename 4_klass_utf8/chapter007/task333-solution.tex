336. Если первая и последняя цифры Петиного числа равны, то их разница не будет трёхзначной. Значит, его первая цифра больше последней. Но тогда если записать вычитание этих чисел столбиком, в разряде десятков будут стоять одинаковые цифры, а для разряда единиц из этого разряда придётся занимать. Тогда надо занимать и из разряда сотен, а в разряде десятков будет $x+10-1-x=9.$\\
