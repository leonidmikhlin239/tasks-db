71. Если бы дырки не было, то в прямоугольнике $30\times40$ было бы $30\cdot39+40\cdot29=2330$ перегородок (см. задачи 61-62). Дырка имеет размеры $(30-2\cdot4)\times(40-2\cdot4)=22\times32.$ Внутри неё <<пропадут>> $22\cdot31+32\cdot21=1354$ перегородки. Также не будет считаться перегородками периметр дырки: $(22+32)\cdot2=108.$ Итого в букве О останется $2330-1354-108=868$ перегородок.\\