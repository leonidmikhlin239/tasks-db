338. Посмотрим на разряд сотен. Так как $2+3=5,$ а в результате в разряде сотен стоит 6, сумма двузначных чисел, которыми кончаются слагаемые, равна 191. Это могут быть числа 99 и 92, 98 и 93, 97 и 94 или 96 и 95. Сумма их цифр в любом случае равна 29. Так как $6+7=13,$ в разряде тысяч у суммы в любом случае стоит цифра 3. Так как сумма начинается с 11, сумма цифр, стоящих в разряде сотен тысяч должна быть равна 10. Значит, сумма всех цифр, заменённых звёздочками, равна $29+3+10=42.$\\
