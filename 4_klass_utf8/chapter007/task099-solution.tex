101. На нижней белой грани должна стоять цифра 2, так как она окажется напротив 5. Значит, на чёрных гранях могут стоять только цифры 3, 4, 5 и 6 (при этом цифры, дающие в сумме 7, не могут стоять на них обе). Наибольшая разность получится, если поставить на них цифры 3 и 6, она равна $6-3=3.$\\