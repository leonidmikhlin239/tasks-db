280. Поворот налево (вместе со следующим движением) занимает у Феди $5+7=12$ секунд, а поворот направо --- $3+7=10$ секунд. Всё путешествие заняло $2\cdot60+2=122$ секунды. После первого поворота налево у Феди останется $122-12=110$ секунд, которые надо разбить на несколько слагаемых по 10 и по 12 секунд. Так как сумма заканчивается на 0, поворотов налево должно быть 0, 5, 10 и так далее. Если поворотов налево было 0, то поворотов направо было $110:10=11.$ Если поворотов налево было 5, то поворотов направо было $(110-5\cdot12):10=5.$ Больше поворотов быть не могло, так как $10\cdot12>110.$\\
