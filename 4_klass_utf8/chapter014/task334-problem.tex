337. Посмотрим на разряд сотен. Так как $4+2=6,$ а в результате в разряде сотен стоит 7, сумма двузначных чисел, которыми кончаются слагаемые, равна 193. Это могут быть числа 99 и 94, 98 и 95 или 97 и 96. Сумма их цифр в любом случае равна 31. Так как $5+8=13,$ в разряде тысяч у суммы в любом случае стоит цифра 3. Так как сумма начинается с 10, сумма цифр, стоящих в разряде сотен тысяч должна быть равна 9. Значит, сумма всех цифр, заменённых звёздочками, равна $31+3+9=43.$\\
