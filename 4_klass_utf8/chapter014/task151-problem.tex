153. Общая сторона у переднего и верхнего слоя должна быть равна 11 (общий делитель 44 и 77). Значит, изначально коробка имела размеры $4\times8\times11.$ После того, как съели передний слой, размеры стали $4\times7\times11,$ а после съедения верхнего --- $3\times7\times11.$ Таким образом, всего дети съедят $44+77+3\cdot7=142$ кусочка сахара, а изначально их было $4\cdot8\cdot11=352.$\\
