26. Если первым бал заяц, то в высказывании второй белки оба утверждения ложны (заяц не мог быть ещё и вторым, а лось не мог тоже быть первым). Если первой была лиса, то в утверждении первой белки ложны оба утверждения (заяц не мог быть тоже первым, лиса не могла быть ещё и второй). Значит, первым был лось и верно второе утверждение первой белки, поэтому лиса была второй.\\
