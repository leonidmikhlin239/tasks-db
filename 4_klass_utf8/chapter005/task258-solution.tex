258. На распиливание длинного бревна уходит $6-1=5$ распилов, а на распиливание короткого --- $5-1=4.$ Предположим, что все брёвна короткие, тогда на распиливание коротких брёвен ушло $4\cdot36=144$ распила, а на распиливание длинных --- 0. Разница между распилами коротких брёвен и распилами длинных составляет $144-0=144.$ При замене одного короткого бревна на длинное количество распилов коротких брёвен уменьшается на 4, а количество распилов длинных увеличивается на 5, таким образом разница уменьшается на $4+5=9.$ Значит, необходимо сделать $144:9=16$ замен (разница должна стать равной 0), поэтому длинных брёвен должно быть 16, а следовательно общее количество распилов равно $16\cdot5\cdot2=160.$\\
