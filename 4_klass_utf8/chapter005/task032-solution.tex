32. Очевидно, что Нина носит единственную женскую фамилию --- Петрова. Пусть после того, как она отдаст Сидорову три ручки, у всех станет по $x$ ручек. Тогда изначально у неё было $x+3$ ручки, у Сидорова $x-3,$ а у Иванова --- $x.$ Если Лёня был Иванов, то верно равенство $x+3+x=20,\ 2x+3=20,\ 2x=17,$ которое невозможно. Значит, Лёня носит фамилию Сидоров и $x+3+x-3=20,\ 2x=20,\ x=10.$ Значит, у Нины Петровой было $10+3=13$ ручек, у Лёни Сидорова было $10-3=7$ ручек и у Максима Иванова было 10 ручек.\\
