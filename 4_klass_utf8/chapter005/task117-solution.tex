117. Пусть только синих клеток С, только жёлтых Ж, сине-жёлтых СЖ, а чистых --- Ч. Синие и сине-жёлтые клетки составляют ровно треть, значит их в 2 раза меньше, чем остальных: $2(\text{С}+\text{СЖ})=\text{Ж}+\text{Ч},\ 2\text{С}+2\text{СЖ}=\text{Ж}+\text{Ч}.$ Также известно, что $\text{Ж}=2\text{С},$ поэтому $2\text{СЖ}=\text{Ч},$ что и требовалось доказать.\\