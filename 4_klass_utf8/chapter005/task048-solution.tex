48. Пусть один слон за один день выпивает $C$ (литров) воды, родник за один день <<набивает>> $P$ воды, а озеро (без учёта родника) содержит $O$ воды. Тогда верны следующие соотношения: $158C\cdot1=O+P\cdot1,\ 23C\cdot7=O+7\cdot P,$ то есть $158C=O+P,\ 161C=O+7P.$ Левые части равенств отличаются на $3C,$ а правые --- на $6P,$ откуда $3C=6P,\ C=2P.$ Подставим полученное соотношение в первое равенство для нахождения количества воды в озере: $158\cdot2P=O+P,\ 316P=O+P,\ 315P=O.$ Если к озеру пришёл один слон, он за день выпивает $2P$ воды, но родник за то же время добавляет $P$ воды. Значит, количество воды в озере будет каждый день уменьшаться на $P,$ и слону понадобится 315 дней, чтобы выпить озеро полностью.\\