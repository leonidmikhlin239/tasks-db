47. Пусть один слон за один день выпивает $C$ (литров) воды, родник за один день <<набивает>> $P$ воды, а озеро (без учёта родника) содержит $O$ воды. Тогда верны следующие соотношения: $163C\cdot1=O+P\cdot1,\ 33C\cdot5=O+5\cdot P,$ то есть $163C=O+P,\ 165C=O+5P.$ Левые части равенств отличаются на $2C,$ а правые --- на $4P,$ откуда $2C=4P,\ C=2P.$ Подставим полученное соотношение в первое равенство для нахождения количества воды в озере: $163\cdot2P=O+P,\ 326P=O+P,\ 325P=O.$ Если к озеру пришёл один слон, он за день выпивает $2P$ воды, но родник за то же время добавляет $P$ воды. Значит, количество воды в озере будет каждый день уменьшаться на $P,$ и слону понадобится 325 дней, чтобы выпить озеро полностью.\\
