118. Пусть только синих клеток С, только жёлтых Ж, сине-жёлтых СЖ, а чистых --- Ч. Синие и сине-жёлтые клетки составляют ровно четверть, значит их в 3 раза меньше, чем остальных: $3(\text{С}+\text{СЖ})=\text{Ж}+\text{Ч},\ 3\text{С}+3\text{СЖ}=\text{Ж}+\text{Ч}.$ Также известно, что $\text{Ж}=3\text{С},$ поэтому $3\text{СЖ}=\text{Ч},$ что и требовалось доказать.\\
