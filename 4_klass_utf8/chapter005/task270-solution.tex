270. Разделим год пользования ноутбуком на 4 части, тогда всего частей было $4\cdot10=40$ и одна часть стоит $200000:40=5000$ рублей. Если Алина заплатила $x$ рублей, то Кирилл заплатил $3x$ и $x+3x=200000,\ 4x=200000,\ x=50000,\ 3x=150000.$ За первый год Кирилл использовал только одну часть (а Алина три), значит у него остались права ещё на $150000-5000=145000$ рублей, которые и должна отдать ему Алина.\\