307. Пусть в 2020 году Кирилл потратил $2x$ минут на решение задач сидя, $2y$ минут на решение задач бегом и $2z$ минут на решение задач на велосипеде. Тогда
$2x+2y+2z=3$ч, откуда $x+y+z=180:2=90$мин. В 2021 году он потратил $x+2y+2z=150$мин, значит $y+z=150-90=60$мин, $x=90-60=30$мин. В 2022 году он потратил $2x+y+2z=165$мин, значит $z=165-90-30=45$мин, а $y=60-45=15$мин. Тогда в 2023 году он потратит $2x+2y+z=2\cdot30+2\cdot15+45=135$мин или 2 часа 15 минут.\\
