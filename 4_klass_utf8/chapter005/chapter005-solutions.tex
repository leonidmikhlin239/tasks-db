\section{Раздел 12: Стандартные задачи: решения}
1. Всего учебных недель было $150:6=25,$ значит столько было и суббот. Остальных дней было $150-25=125,$ раз классов было 4 и они решали по 4 задачи, всего было решено $125\cdot4\cdot4=2000$ задач.\\
2. Всего учебных недель было $150:6=25,$ значит столько было и суббот. Остальных дней было $150-25=125,$ раз классов было 5 и они решали по 3 задачи, всего было решено $125\cdot5\cdot3=1875$ задач.\\
3. Пусть крокодил получил $x$ таблеток, тогда носорог получил $x+1,$ бегемот $x+2,$ слон $x+3$ и верно равенство $x+x+1+x+2+x+3=2010,\ 4x+6=2010,\ 4x=2004,\ x=501.$ Значит, слон съел $501+3=504$ таблетки.\\
4. Пусть рысь получила $x$ таблеток, тогда медведь получил $x+1,$ волк $x+2,$ лиса $x+3,$ и верно равенство $x+x+1+x+2+x+3=2014,\ 4x+6=2014,\ 4x=2008,\ x=502.$ Значит, рысь съела 502 таблетки.\\
5. Составим уравнение: $x\cdot10=x+18,\ 10x=x+18,\ 9x=18,\ x=2.$\\
6. Составим уравнение: $x\cdot9=x+24,\ 9x=x+24,\ 8x=24,\ x=3.$\\
7. И то, и другое приводит к уменьшению разности, значит она уменьшится на $8+5=13.$\\
8. И то, и другое приводит к увеличению разности, значит она увеличится на $11+5=16.$\\
9. Если бы у хозяйки были только куры, у них было бы $35\cdot2=70$ ног. Не хватает $94-70=24$ ноги. Замена куры на кролика увеличивает количество ног на $4-2=2.$ Значит, необходимо произвести $24:2=12$ замен и будет 12 кроликов и $35-12=23$ куры.\\
10. Если бы на ферме имелись только гуси, у них было бы  $30\cdot2=60$ ног. Не хватает $86-60=26$ ног. Замена гуся на корову увеличивает количество ног на $4-2=2.$
Значит, необходимо произвести $26:2=13$ замен и гусей будет $30-13=17.$\\
11. Если бы на ферме имелись только гуси, у них было бы  $40\cdot2=80$ ног. Не хватает $98-80=18$ ног. Замена гуся на корову увеличивает количество ног на $4-2=2.$
Значит, необходимо произвести $18:2=9$ замен и коров будет 9.\\
12. Если бы все лодки были четырёхместные, в них было бы $10\cdot4=40$ мест. Не хватает $46-40=6$ мест. Замена четырёхместной лодки на шестиместную
добавляет $6-4=2$ места. Значит, необходимо произвести $6:2=3$ замены и шестиместных лодок будет 3, а четырёхместных --- $10-3=7.$\\
13. Задумали число $36\cdot6:4=54.$\\
14. Задумали число $24\cdot12:8=36.$\\
15. Поделим с остатком: $23:3=5$ (ост. 3). Значит, место №23 расположено в 6-м купе.\\
16. Если мы сложим школьников, изучающих английский, со школьниками, изучающими немецкий, получим $17+11=28.$ Результат оказался больше, чем общее количество школьников, так как изучающие оба языка школьники были посчитаны 2 раза, значит их было $28-23=5.$\\
17. Поделим с остатком: $(750+25):70=11$ (ост. 5). Значит, понадобится $11+1=12$ шлюпок.\\
18. Поделим с остатком: $(218+26):45=5$ (ост. 19). Значит, понадобится $5+1=6$ автобусов.\\
19. Вычтем тех детей, которые не едят: $35-10=25.$ Теперь сложим тех, кто ест бутерброды, с теми, кто посещает кафе: $20+11=31.$ Результат оказался больше, чем общее количество детей, так съедающие бутерброды сидя в кафе были посчитаны 2 раза, значит их было $31-25=6.$\\
20. Сложим одноруких и одноглазых: $32+29=61.$ При этом одноруких пиратов с одним глазом мы посчитали 2 раза, значит на самом деле травмированных пиратов $61-15=46.$ Таким образом, оставшиеся $50-46=4$ пирата являются здоровыми.\\
21. Обратим все действия: $40-15+7-6+11=37$ человек было в автобусе изначально.\\
22. Обратим все действия: $40-8+11-12+7=38$ человек было в автобусе изначально.\\
23. На одного туриста было израсходовано $32\text{ кг}:100=32000\text{ г}:100=320$г мяса. Значит, масла было израсходовано $320:8=40$г, а хлеба --- $320\cdot2=640$г.\\
24. На одного пятиклассника пришлось $24\text{ кг}:100=24000\text{ г}:100=240$г мороженого. Значит, чая пришлось $240:8=30$г, а карамели --- $240\cdot2=480$г.\\
25. За 7 недель в школе будет израсходовано $1300\cdot7=9100$ листов. Поделим с остатком: $9100:500=18$ (ост. 100). Значит, всего потребуется $18+1=19$ пачек.\\
26. За 9 недель в школе будет израсходовано $1100\cdot9=9900$ листов. Поделим с остатком: $9900:500=19$ (ост. 400). Значит, всего потребуется $19+1=20$ пачек.\\
27. Составим уравнение: $x+4=x\cdot3,\ x+4=3x,\ 4=2x,\ x=2.$ Тогда $2\cdot6=2+10.$\\
28. Составим уравнение: $x+6=x\cdot3,\ x+6=3x,\ 6=2x,\ x=3.$ Тогда $3\cdot5=3+12.$\\
29. $2,\ 3,\ 2\cdot3+2=8,\ 3\cdot8+2=26,\ 2\cdot26+2=210,\ 26\cdot210+2=5462.$\\
30. $2,\ 5,\ 2\cdot5-3=7,\ 5\cdot7-3=32,\ 7\cdot32-3=221,\ 32\cdot221-3=7069.$\\
31. Очевидно, что Аня носит единственную женскую фамилию --- Иванова. Пусть после того, как она отдаст Петрову один карандаш, у всех станет по $x$ карандашей. Тогда изначально у неё был $x+1$ карандаш, у Петрова $x-1,$ а у Сидорова --- $x.$ Если Витя был Сидоров, то верно равенство $x+1+x=10,\ 2x+1=10,\ 2x=9,$ которое невозможно. Значит, Витя носит фамилию Петров и $x+1+x-1=10,\ 2x=10,\ x=5.$ Значит, у Ани Ивановой было $5+1=6$ карандашей, у Вити Петрова было $5-1=4$ карандаша и у Бори Сидорова было 5 карандашей.\\
32. Очевидно, что Нина носит единственную женскую фамилию --- Петрова. Пусть после того, как она отдаст Сидорову три ручки, у всех станет по $x$ ручек. Тогда изначально у неё было $x+3$ ручки, у Сидорова $x-3,$ а у Иванова --- $x.$ Если Лёня был Иванов, то верно равенство $x+3+x=20,\ 2x+3=20,\ 2x=17,$ которое невозможно. Значит, Лёня носит фамилию Сидоров и $x+3+x-3=20,\ 2x=20,\ x=10.$ Значит, у Нины Петровой было $10+3=13$ ручек, у Лёни Сидорова было $10-3=7$ ручек и у Максима Иванова было 10 ручек.\\
33. Пусть во втором здании расходуется $x$ листов бумаги, тогда в первом их расходуется $x+6$ и верно равенство $x+x+6=260,\ 2x+6=260,\ 2x=254,\ x=127.$ Значит, в первом здании расходуется 127 листов в день, а во втором --- $127+6=133.$ Тогда на 12 дней в первом здании будет израсходовано $12\cdot127=1524$ листа, а во втором --- $12\cdot133=1596$ листов. Теперь поделим с остатком: $1524:500=3$ (ост. 24), $1596:500=3$ (ост. 96). Значит, в каждое здание потребуется по $3+1=4$ пачки, а всего их нужно будет $4+4=8.$\\
34. Пусть во втором здании расходуется $x$ листов бумаги, тогда в первом их расходуется $x+6$ и верно равенство $x+x+6=360,\ 2x+6=360,\ 2x=354,\ x=177.$ Значит, в первом здании расходуется 177 листов в день, а во втором --- $177+6=183.$ Тогда на 12 дней в первом здании будет израсходовано $12\cdot177=2124$ листа, а во втором --- $12\cdot183=2196$ листов. Теперь поделим с остатком: $2124:500=4$ (ост. 124), $2196:500=4$ (ост. 196). Значит, в каждое здание потребуется по $4+1=5$ пачек, а всего их нужно будет $5+5=10.$\\
35. Мандарин и апельсин весят 200 г, значит необходимо найти только вес груши. Для этого сложим апельсин с грушей и яблоко с грушей. Получатся 2 груши, апельсин и яблоко, а весить они будут $176+159=335$ граммов. При этом апельсин с яблоком весят 167 граммов, а значит 2 груши весят $335-167=168$г, поэтому одна груша весит $168:2=84$г. Таким образом, мандарин, апельсин и груша вместе весят $200+84=284$ грамма.\\
36. Яблоко и груша весят 300 г, значит необходимо найти только вес мандарина. Для этого сложим мандарин с грушей и мандарин с апельсином. Получатся 2 мандарина, апельсин и груша, а весить они будут $175+168=343$ грамма. При этом апельсин с грушей весят 157 граммов, а значит 2 мандарина весят $343-157=186$г, поэтому один мандарин весит $186:2=93$г. Таким образом, мандарин, яблоко и груша вместе весят $300+93=393$ грамма.\\
37. Мальчики всех продуктов едят одинаковое количество, значит разница между количеством съеденных конфет и количеством съеденных котлет образовалась только за счёт девочек. Одна девочка ест конфет на $10-3=7$ больше, чем котлет, а всего конфет было съедено на $714-371=343$ больше. Значит, девочек было $343:7=49.$ Разницу между конфетами и козинаками также обеспечивают девочки, причём одна девочка съела козинаков на $24-10=14$ больше. Значит, всего их было съедено $714+49\cdot14=1400$ штук.\\
38. Мальчики всех продуктов едят одинаковое количество, значит разница между количеством съеденных мандаринов и количеством съеденных манго образовалась только за счёт девочек. Одна девочка ест мандаринов на $11-8=3$ больше, чем манго, а всего мандаринов было съедено на $813-669=144$ больше. Значит, девочек было $144:3=48.$ Разницу между марципанами и мандаринами также обеспечивают девочки, причём одна девочка съела марципанов на $17-11=6$ больше. Значит, всего их было съедено $813+48\cdot6=1101$ штука.\\
39. Пусть олимпиада длится $x$ часов. Тогда с момента начала олимпиады в городе А до момента начала олимпиады в городе Б прошёл $x-1$ час, а с момента начала в городе А до окончания в городе Б прошло соответственно $x-1+x=2x-1$ часов. Отсюда $2x-1=9,\ 2x=10,\ x=5$ часов.\\
40. Пусть олимпиада длится $x$ часов. Тогда с момента начала олимпиады в городе А до момента начала олимпиады в городе Б прошёл $x+1$ час, а с момента начала в городе А до окончания в городе Б прошло соответственно $x+1+x=2x+1$ часов. Отсюда $2x+1=7,\ 2x=6,\ x=3$ часа.\\
41. Обратим внимание, что блоки пилить нельзя, поэтому необходимо понять, сколько блоков можно поместить в одну машину. Для этого поделим с остатком:
$1\text{ т}:33\text{ кг}=1000\text{ кг}:33\text{ кг}=30$ (ост. 10). Значит, в одну машину помещается 30 блоков и останутся свободными ещё 10 кг, которые можно заполнить 5 кирпичами. Поэтому на перевозку всех блоков понадобится $30000:30=1000$ машин, они же смогут перевезти ещё $5\cdot1000=5000$ кирпичей, значит дополнительные машины для кирпичей не понадобятся.\\
42. Обратим внимание, что блоки пилить нельзя, поэтому необходимо понять, сколько блоков можно поместить в одну машину. Для этого поделим с остатком:
$1\text{ т}:32\text{ кг}=1000\text{ кг}:32\text{ кг}=31$ (ост. 8). Значит, в одну машину помещается 31 блок и останутся свободными ещё 8 кг, которые можно заполнить 4 кирпичами. Поэтому на перевозку всех блоков понадобится $31000:31=1000$ машин, они же смогут перевезти ещё $4\cdot1000=4000$ кирпичей, значит дополнительные машины для кирпичей не понадобятся.\\
43. В третьем ящике будет $1001+2999+239-4-1007=3228$ гвоздиков.\\
44. Во втором ящике будет $2002+1999+239-7-1003=3230$ гвоздиков.\\
45. Пусть в понедельник ученик решил $x$ задач. Тогда во вторник он решил $x+2,$ в среду $x+4,$ в четверг $x+6,$ в пятницу $x+8,$ в субботу $x+10,$ а в воскресенье --- $x+12$ задач. С другой стороны, в воскресенье он должен был решить $3x$ задач, а значит $3x=x+12,\ 2x=12,\ x=6.$ Тогда в пятницу он решил $6+8=14$ задач.\\
46. Пусть в понедельник ученик решил $x$ задач. Тогда во вторник он решил $x+2,$ в среду $x+4,$ в четверг $x+6,$ в пятницу $x+8,$ в субботу $x+10,$ а в воскресенье --- $x+12$ задач. С другой стороны, в воскресенье он должен был решить $4x$ задач, а значит $4x=x+12,\ 3x=12,\ x=4.$ Тогда в пятницу он решил $4+8=12$ задач.\\
47. Пусть один слон за один день выпивает $C$ (литров) воды, родник за один день <<набивает>> $P$ воды, а озеро (без учёта родника) содержит $O$ воды. Тогда верны следующие соотношения: $163C\cdot1=O+P\cdot1,\ 33C\cdot5=O+5\cdot P,$ то есть $163C=O+P,\ 165C=O+5P.$ Левые части равенств отличаются на $2C,$ а правые --- на $4P,$ откуда $2C=4P,\ C=2P.$ Подставим полученное соотношение в первое равенство для нахождения количества воды в озере: $163\cdot2P=O+P,\ 326P=O+P,\ 325P=O.$ Если к озеру пришёл один слон, он за день выпивает $2P$ воды, но родник за то же время добавляет $P$ воды. Значит, количество воды в озере будет каждый день уменьшаться на $P,$ и слону понадобится 325 дней, чтобы выпить озеро полностью.\\
48. Пусть один слон за один день выпивает $C$ (литров) воды, родник за один день <<набивает>> $P$ воды, а озеро (без учёта родника) содержит $O$ воды. Тогда верны следующие соотношения: $158C\cdot1=O+P\cdot1,\ 23C\cdot7=O+7\cdot P,$ то есть $158C=O+P,\ 161C=O+7P.$ Левые части равенств отличаются на $3C,$ а правые --- на $6P,$ откуда $3C=6P,\ C=2P.$ Подставим полученное соотношение в первое равенство для нахождения количества воды в озере: $158\cdot2P=O+P,\ 316P=O+P,\ 315P=O.$ Если к озеру пришёл один слон, он за день выпивает $2P$ воды, но родник за то же время добавляет $P$ воды. Значит, количество воды в озере будет каждый день уменьшаться на $P,$ и слону понадобится 315 дней, чтобы выпить озеро полностью.\\
49. Изначально Максим должен был заплатить $5600:2=2800$ рублей. Он заплатил на $3200-2800=400$ рублей больше, значит эти деньги он вернул и остался должен $10000-400=9600$ рублей.\\
50. Изначально Вова должен был заплатить $7800:2=3900$ рублей. Он заплатил на $3900-3200=700$ рублей меньше, значит эти деньги Максим ему вернул и остался должен $3800-700=3100$ рублей.\\
51. После того как один из хамелеонов перекрасился из красного цвета в зелёный, красных животных стало $17-1=16.$ Так как их стало поровну, красных попугаев изначально было $16:2=8.$ Значит, красных хамелеонов изначально было $17-8=9,$ зелёных хамелеонов было $17-9=8,$ зелёных попугаев было $14-8=6.$\\
52. После того как один из хамелеонов перекрасился из красного цвета в зелёный, зелёных животных стало $15+1=16.$ Так как их стало поровну, зелёных попугаев изначально было $16:2=8.$ Значит, зелёных хамелеонов изначально было $15-8=7,$ красных хамелеонов было $15-7=8,$ красных попугаев было $12-8=4.$\\
53. Изначально расстояние между Гулливером и лилипутом было равно 8 шагам Гулливера, то есть $8\cdot11=88$ шагам лилипута. За 1 свой шаг Гулливер догоняет лилипута на $11-7=4$ шага лилипута, значит ему потребуется $88:4=22$ шага. За это время лилипут сделает $22\cdot7=154$ шага.\\
54. Изначально расстояние между Гулливером и лилипутом было равно 6 шагам Гулливера, то есть $6\cdot10=60$ шагам лилипута. За 1 свой шаг Гулливер догоняет лилипута на $10-7=3$ шага лилипута, значит ему потребуется $60:3=20$ шагов. За это время лилипут сделает $20\cdot7=140$ шагов.\\
55. Пусть у Максима $x$ монет, тогда у Ани их $4x,$ а у Димы --- $4x-3$ и верно равенство $x+4x+4x-3=1410,\ 9x-3=1410,\ 9x=1413,\ x=157.$ Тогда у Ани $157\cdot4=628$ монет.\\
56. Пусть у Ани $x$ монет, тогда у Максима их $3x,$ а у Димы --- $x-4$ и верно равенство $x+3x+x-4=1401,\ 5x-4=1401,\ 5x=1405,\ x=281.$ Значит, у Ани 281 монета.\\
57. Если дописать к числу лишний 0, оно увеличится в 10 раз. Если Саша хотел сложить числа $a$ и $b,$ то верны равенства $a+b=1222,\ a+10b=5551.$ Левая часть второго равенства больше на $9b,$ а правая --- на $5551-1222=4329,$ откуда $9b=4329,\ b=481.$ Тогда первое число $a=1222-481=741.$\\
58. Если дописать к числу лишний 0, оно увеличится в 10 раз. Если Паша хотел сложить числа $a$ и $b,$ то верны равенства $a+b=1331,\ a+10b=6641.$ Левая часть второго равенства больше на $9b,$ а правая --- на $6641-1331=5310,$ откуда $9b=5310,\ b=590.$ Тогда первое число $a=1331-590=741.$\\
59. Посчитаем количество игр девочек друг с другом. Первая сыграла со всеми остальными 3 игры, вторая --- ещё 2 (игра с первой девочкой уже посчитана), аналогично третья ещё 1, а все игры четвёртой девочки уже окажутся посчитаны. Значит, всего девочки сыграли друг с другом $3+2+1=6$ игр, в которых они независимо от результатов этих игр набрали на всех $6\cdot2=12$ очков (если одна девочка выигрывает у другой, они получают суммарно $2+0=2$ очка, а если они играют вничью, то также получают суммарно $1+1=2$ очка). Каждый из 6 мальчиков сыграл со всеми 4 девочками, значит всего мальчики с девочками провели $6\cdot4=24$ партии. В них было разыграно $24\cdot2=48$ очков, из которых девочки получили $40-12=28$ очков, а мальчики --- оставшиеся $48-28=20.$ Когда мальчик с девочкой играют вничью, разница между количеством очков, набранных всеми девочками, и количеством очков, набранных всеми мальчиками, не меняется. Победа девочки увеличивает эту разницу на 2 очка, а победа мальчика на те же 2 очка уменьшает. Так как итоговая разница оказалась равна $28-20=8$ очкам, девочки одержали на $8:2=4$ победы больше.\\
60. Посчитаем количество игр мальчиков друг с другом. Первый сыграл со всеми остальными 5 игр, второй --- ещё 4 (игра с первым мальчиком уже посчитана), аналогично третий ещё 3 и так далее. Значит, всего мальчики сыграли друг с другом $5+4+3+2+1=15$ игр, в которых они независимо от результатов этих игр набрали на всех $15\cdot2=30$ очков (если один мальчик выигрывает у другого, они получают суммарно $2+0=2$ очка, а если они играют вничью, то также получают суммарно $1+1=2$ очка). Каждый из 6 мальчиков сыграл со всеми 4 девочками, значит всего мальчики с девочками провели $6\cdot4=24$ партии. В них было разыграно $24\cdot2=48$ очков, из которых мальчики получили $40-30=10$ очков, а девочки --- оставшиеся $48-10=38.$ Когда мальчик с девочкой играют вничью, разница между количеством очков, набранных всеми девочками, и количеством очков, набранных всеми мальчиками, не меняется. Победа девочки увеличивает эту разницу на 2 очка, а победа мальчика на те же 2 очка уменьшает. Так как итоговая разница оказалась равна $38-10=28$ очкам, девочки одержали на $28:2=14$ побед больше.\\
61. Суммарный возраст увеличился на $90-62=28$ лет, при этом каждый друг постарел на 4 года. Значит, всего их было $28:4=7$ человек.\\
62. Узнаем, сколько досок для забора можно выпилить из одной магазинной: $4\text{ м}:75\text{ см}=400\text{ см}:75\text{ см}=5$ (ост. 25). Значит, из одной магазинной доски можно получить 5 досок для забора. Так как $112:5=22$ (ост. 2), для постройки забора надо купить $22+1=23$ доски.\\
63. Узнаем, сколько досок для забора можно выпилить из одной магазинной: $6\text{ м}:80\text{ см}=600\text{ см}:80\text{ см}=7$ (ост. 40). Значит, из одной магазинной доски можно получить 7 досок для забора. Так как $120:7=17$ (ост. 1), для постройки забора надо купить $17+1=18$ досок.\\
64. Так как у Светы с Машей столько же конфет, сколько у Оли, у Оли их половина от общего количества, то есть $80:2=40.$ У Светы с Олей конфет в 4 раза больше конфет, чем у Маши, значит у Маши их пятая часть от общего количества, то есть $80:5=16.$ Таким образом, у Светы остальные $80-40-16=24$ конфеты.\\
65. Так как у Светы с Машей столько же конфет, сколько у Оли, у Оли их половина от общего количества, то есть $60:2=30.$ У Светы с Олей конфет в 5 раз больше конфет, чем у Маши, значит у Маши их шестая часть от общего количества, то есть $60:6=10.$ Таким образом, у Светы остальные $60-30-10=20$ конфет.\\
66. Пусть на преодоление одного этажа тратится Э секунд, а на остановку --- О секунд. Тогда для поездок Пети и Тани верны следующие равенства: $12\text{Э}+3\text{О}=57\text{ с},\ 6\text{Э}+1\text{О}=25\text{ с}.$ Повторим поездку Тани два раза, получим равенство $12\text{Э}+2\text{О}=50\text{ с}.$ Его левая часть отличается от левой части первого равенства на одну остановку, а правая часть --- на 7 секунд. Значит, одна остановка занимает 7 секунд и
$6\text{Э}+7\text{ с}=25\text{ с},$ поэтому проезд одного этажа занимает 3 секунды. Тогда Коля будет ехать $10\cdot3+2\cdot7=44$ секунды.\\
67. Пусть на преодоление одного этажа тратится Э секунд, а на остановку --- О секунд. Тогда для поездок Пети и Тани верны следующие равенства: $10\text{Э}+3\text{О}=54\text{ с},\ 5\text{Э}+1\text{О}=23\text{ с}.$ Повторим поездку Тани два раза, получим равенство $10\text{Э}+2\text{О}=46\text{ с}.$ Его левая часть отличается от левой части первого равенства на одну остановку, а правая часть --- на 8 секунд. Значит, одна остановка занимает 8 секунд и
$5\text{Э}+8\text{ с}=23\text{ с},$ поэтому проезд одного этажа занимает 3 секунды. Тогда Коля будет ехать $8\cdot3+2\cdot8=40$ секунд.\\
68. После покупки 3 ручек останется $120-3\cdot12=84$  рубля. Переведём эту сумму в копейки и поделим с остатком: $8400:750=11$ (ост. 150). Значит, Олег сможет купить 11 тетрадей.\\
69. После покупки 4 линеек останется $140-4\cdot14=84$ рубля. Переведём эту сумму в копейки и поделим с остатком: $8400:950=8$ (ост. 800). Значит, Оля сможет купить 8 блокнотов.\\
70. Всего верблюдов $44:4=11.$ Если бы они все были одногорбые, у них было бы 11 горбов. Замена одногорбого верблюда на двугорбого добавляет 1 горб. Так как горбов должно быть 14, необходимо сделать $14-11=3$ замены. Значит, одногорбых верблюдов останется $11-3=8.$\\
71. Всего верблюдов $48:4=12.$ Если бы они все были одногорбые, у них было бы 12 горбов. Замена одногорбого верблюда на двугорбого добавляет 1 горб. Так как горбов должно быть 19, необходимо сделать $19-12=7$ замены. Значит, двугорбых верблюдов должно быть 7.\\
72. Енот весит $25-17=8$кг, значит собака весит $19-8=11$кг.\\
73. Петух весит $30-22=8$кг, значит кролик весит $17-8=9$кг.\\
74. Разница между половиной кладки и четвертью кладки --- это четверть кладки. Значит, четверть стены составляет $160\text{ см}-1\text{ м}=160\text{ см}-100\text{ см}=60\text{ см}.$ Поэтому высота всей стены составляет $4\cdot60=240\text{ см},$ а высота фундамента --- $100\text{ см}-60\text{ см}=40\text{ см}.$ Тогда высота законченной на 2/3 стены вместе с фундаментом составляет $240:3\cdot2+40=200$см.\\
75. Разница между половиной пальмы и четвертью пальмы --- это четверть пальмы. Значит, четверть пальмы составляет $190\text{ см}-120\text{ см}=70\text{ см}.$
Поэтому высота всей пальмы составляет $4\cdot70=280\text{ см},$ а высота кадки --- $120\text{ см}-70\text{ см}=50\text{ см}.$ Тогда ленточка, отмечающая 2/5 высоты пальмы, будет находиться на высоте $280:5\cdot2+50=162$см от пола.\\
76. Число 8 стоит $29-17=12$ динаров. Число 56 стоит $33-12=21$ динар. Число 1 стоит $47-2\cdot21=5$ динаров. Число 435 стоит $36-12=24$ динара. Тогда число 745231 стоит $24+17+5=46$ динаров.\\
77. Число 7 стоит $22-14=8$ гульденов. Число 94 стоит $17-8=9$ гульденов. Число 8 стоит $22-2\cdot9=4$ гульдена. Число 326 стоит $36-8=28$ гульденов. Тогда число 623158 стоит $28+14+4=46$ гульденов.\\
78. За 1 час Балда с Емелей съедают $15\cdot2+15\cdot3=75$ пирожков. Чудо-печка за то же время выпекает $60:2\cdot3=90$ пирожков. Значит, Балда и Емеля не успеют за 2 часа съесть те пирожки, которые за это время испекла Чудо-печка.\\
79. За 1 час Малыш с Карлсоном съедают $4\cdot4+16\cdot2=48$ плюшек. Фрекен Бок за то же время выпекает $60:4\cdot3=45$ плюшек. Значит, Малыш и Карлсон успеют за 2 часа съесть те плюшки, которые за это время испекла Фрекен Бок.\\
80. Пусть верблюмот весит $x$кг, тогда слонопотам с одной стороны весит $5x$кг, а с другой стороны --- $x+100$кг. Значит, $5x=x+100,\ 4x=100,\ x=25$кг. Значит, верблюмот весит 25 кг, а слонопотам --- $5\cdot25=125$ кг, вместе они весят $25+125=150$кг. Пусть кошкалот весит $y$кг, тогда антиконда весит $y+16$кг и верно равенство $y+y+16=150,\ 2y+16=150,\ 2y=134,\ y=67$кг. Таким образом, антиконда весит $67+16=83$кг и в порядке убывания животные располагаются следующим образом: слонопотам 125 кг, антиконда 83 кг, кошкалот 67 кг, верблюмот 25 кг.\\
81. Пусть тётел весит $x$кг, тогда мамугай с одной стороны весит $6x$кг, а с другой стороны --- $x+50$кг. Значит, $6x=x+50,\ 5x=50,\ x=10$кг. Значит, тётел весит 10 кг, а мамугай --- $6\cdot10=60$кг, вместе они весят $10+60=70$кг. Пусть хвалибри весит $y$кг, тогда пеликот весит $y+34$кг и верно равенство $y+y+34=70,\ 2y+34=70,\ 2y=36,\ y=18$кг. Таким образом, пеликот весит $18+34=52$кг и в порядке возрастания птиц располагаются следующим образом: тётел 10 кг, хвалибри 18 кг, пеликот 52 кг, мамугай 60 кг.\\
82. Пусть $a-b=c,$ тогда $a+b+c=a+a=2a=36,\ a=18.$\\
83. Пусть у Оли было $x$ фантиков, тогда у Зои их $x+200$ и если она отдаст Оле 100 фантиков, у обеих девочек их станет по $x+100.$\\
84. После второй остановки в автобусе было $8-5+7=10$ человек, после третьей --- $10-5+7=12$ человек.\\
85. Зайцев в доме $486\cdot5,$ а зайчиков --- $486\cdot5\cdot3,$ значит всего зайцев и зайчиков $486\cdot5+486\cdot5\cdot3.$\\
86. Пчёл в парке $4008:2=2004,$ мух $4008:3=1338,$ тогда гусениц в парке $(4008-2004-1336):4=167.$\\
87. Пусть все нрутасианцы были первого вида, тогда их было $11:1=11,$ и у них было $11\cdot4=44$ руки. Для сохранения общего количества голов можно заменять двух нрутасианцев первого вида на одного нрутасианца второго вида, при этом общее количество рук уменьшается на $4\cdot2-3=5.$ Всего рук должно быть меньше на  $44-29=15,$ значит необходимо сделать $15:5=3$ замены, после чего останется $11-2\cdot3=5$ нрутасианцев первого вида и $3\cdot1=3$ нрутасианца второго вида. Таким образом, всего нрутасианцев в делегации $5+3=8.$\\
88. Пусть все нутпенсианцы были первого вида, тогда их было $11:1=11,$ и у них было $11\cdot3=33$ ноги. Для сохранения общего количества голов можно заменять трёх нутпенсианцев первого вида на одного нутпенсианца второго вида, при этом общее количество ног уменьшается на $3\cdot3-2=7.$ Всего ног должно быть меньше на  $33-19=14,$ значит необходимо сделать $14:7=2$ замены, после чего останется $11-3\cdot2=5$ нутпенсианцев первого вида и $2\cdot1=2$ нутпенсианца второго вида. Таким образом, всего нутпенсианцев в делегации $5+2=7.$\\
89. Если сложить синие машинки с грузовиками, то мы получим $12+11=23,$ что на $23-19=4$ больше, чем общее количество машинок, значит как минимум 4 машинки были посчитаны 2 раза и являются синими грузовиками.\\
90. Если сложить книги со сказками с книгами с картинками, то мы получим $10+17=27,$ что на $27-23=4$ больше, чем общее количество книг, значит как минимум 4 книги были посчитаны 2 раза и являются книгами сказок с картинками.\\
91. Пусть корюшки и ряпушки было $x$ рыбок, тогда плоты и уклейки было $3x$ и $x+3x=256,\ 4x=256,\ x=64,$ а $3x=192.$ Пусть ряпушек было $y,$ тогда корюшек было $y+12$ и $y+y+12=64,\ 2y+12=64,\ 2y=52,\ y=26,$ а $y+12=38.$ Пусть уклеек было $z,$ тогда плотвы было $7z$ и $z+7z=192,\ 8z=192,\ z=24.$ Таким образом, корюшек было поймано больше, чем уклеек, на $38-24=14$ штук.\\
92. Пусть изумрудов и сапфиров было $x$ штук, тогда рубинов и алмазов было $2x$ и $x+2x=351,\ 3x=351,\ x=117,$ а $2x=234.$ Пусть алмазов было $y,$ тогда рубинов было $5y$ и $y+5y=234,\ 6y=234,\ y=39.$ Пусть изумрудов было $z,$ тогда сапфиров было $z+7$ и $z+z+7=117,\ 2z+7=117,\ 2z=110, z=55.$ Таким образом, изумрудов было больше, чем алмазов, на $55-39=16$ штук.\\
93. Всего каждой девочке (в составе компота) досталось $36:4=9$ груш, значит одна груша стоит $216:9=24$ рубля. Катя принесла за Нину $11-9=2$ груши и должна получить $2\cdot24=48$ рублей, Лена принесла за Нину $10-9=1$ грушу и должна получить $1\cdot24=24$ рубля, а Маша должна получить оставшиеся $216-48-24=144$ рубля.\\
94. Всего каждому мальчику (в составе компота) досталось $32:4=8$ яблок, значит одно яблоко стоит $128:8=16$ рублей. Антон принёс за Гену $9-8=1$ яблоко и должен получить $1\cdot16=16$ рублей, Борис принёс за Гену $10-8=2$ яблока и должен получить $2\cdot16=32$ рубля, а Вадим должен получить оставшиеся $128-16-32=80$ рублей.\\
95. Всего раскрасили $27-4=23$ тарелки. Если сложить синие и красные тарелки, то получится $18+21=39$ тарелок, что на $39-23=16$ больше, чем общее количество раскрашенных тарелок. Значит, 16 тарелок были посчитаны 2 раза и раскрашены обоими цветами.\\
96. Пусть рыцарь весит Р кг, а дракон --- Д. Тогда $\text{Р}+5\text{Д}=\text{Д}+8\text{Р},\ 4\text{Д}=7\text{Р}.$ Значит, $8\text{Д}=14\text{Р}.$\\
97. Сложим диван и чемодан с корзиной и собачонкой: $52+11=63$кг.\\
98. Пусть жёлтых карандашей было $x,$ тогда синих $14x,$ красных $2x,$ зелёных $14x+6$ и верно равенство $x+14x+2x+14x+6=68,\ 31x+6=68,\ 31x=62,\ x=2.$ Значит, жёлтых карандашей у Коли 2, синих $14\cdot2=28,$ красных $2\cdot2=4,$ зелёных $14\cdot2+6=34.$\\
99. Зелёных попугаев было $18-10=8,$ что на $9-8=1$ меньше, чем крикливых. Раз эти количества сравнялись, прилетевший попугай был зелёным, но не был крикливым.\\
100. На тетрадки по 60 рублей Даша потратила $480:2=240$ рублей, поэтому их было $240:60=4$ штуки. На тетрадки по $60:2=30$ рублей Даша потратила остальные 240 рублей, значит их было $240:30=8$ штук. Таким образом, средняя цена тетрадок составляет $480:(4+8)=40$ рублей.\\
101. $2\cdot1\cdot9+5=23,\ 2\cdot3+5=11,\ 1\cdot1+5=6.$\\
102. $2\cdot1\cdot7+5=19,\ 1\cdot9+5=14,\ 1\cdot4+5=9.$\\
103. Торт весил $5\cdot200+4\cdot25=1100$г.\\
104. В горшочке было $6\cdot300+5\cdot20=1900$г мёда.\\
105. Поделим с остатком: $(174+148+14):45=336:45=7$(ост. 21). Значит, потребуется $7+1=8$ автобусов.\\
106. Поделим с остатком: $(136+194+15):45=345:45=7$(ост. 30). Значит, потребуется $7+1=8$ автобусов.\\
107. Пусть коз на ферме К, гусей Г, а лошадей --- Л. Голов столько же, сколько крыльев, значит $\text{К}+\text{Г}+\text{Л}=2\text{Г},\ \text{К}+\text{Л}=\text{Г}.$ Рогов на 8 меньше, чем голов, значит $2\text{К}+8=\text{К}+\text{Г}+\text{Л},\ \text{К}+8=\text{Г}+\text{Л}=\text{К}+\text{Л}+\text{Л}=\text{К}+2\text{Л},$ откуда $8=2\text{Л},\ \text{Л}=4.$\\
108. Пусть коров на ферме К, уток У, а лошадей --- Л. Голов столько же, сколько рогов, значит $\text{К}+\text{У}+\text{Л}=2\text{К},\ \text{У}+\text{Л}=\text{К}.$ Крыльев на 10 меньше, чем голов, значит $2\text{У}+10=\text{К}+\text{У}+\text{Л},\ \text{У}+10=\text{К}+\text{Л}=\text{У}+\text{Л}+\text{Л}=\text{У}+2\text{Л},$ откуда $10=2\text{Л},\ \text{Л}=5.$\\
109. Пусть когда Маше было 5 лет, Косте было $x$ лет, а прошло с тех пор $y$ лет. Тогда сейчас Тане $x+1$ год, Косте $x+y,$ а Маше --- $5+y.$ Тогда сумма возрастов Тани и Маши больше возраста Кости на $x+1+5+y-(x+y)=x+y+6-x-y=6$ лет.\\
110. Пусть когда Насте было 7 лет, Антону было $x$ лет, а прошло с тех пор $y$ лет. Тогда сейчас Ксюше $x-1$ год, Антону $x+y,$ а Насте --- $7+y.$ Тогда сумма возрастов Насти и Ксюши больше возраста Антона на $x-1+7+y-(x+y)=x+y+6-x-y=6$ лет.\\
111. Каждый год сумма возрастов Андрея и Тани увеличивается на 2 года, значит суммарно 15 лет им было $(21-15):2=3$ года назад. Так как Андрею было столько, сколько Тане сейчас, 3 года назад, он старше на 3 года. Если ему было $x$ лет, Тане было $x-3$ и $x+x-3=15,\ 2x-3=15,\ 2x=18,\ x=9$лет.\\
112. Каждый год сумма возрастов Даши и Ксюши увеличивается на 2 года, значит суммарно 10 лет им было $(18-10):2=4$ года назад. Так как Даше было столько, сколько Ксюше сейчас, 4 года назад, она старше на 4 года. Если Ксюше было $x$ лет, Даше было $x+4$ и $x+x+4=10,\ 2x+4=10,\ 2x=6,\ x=3$года.\\
113. Пусть Тане $x$ лет, тогда бабушке $9x$ лет, а мама должна находиться ровно посередине между ними, значит ей $5x$ лет. Тогда $x+9x+5x=90,\ 15x=90,\ x=6,\ 9x=54$ года.\\
114. За неделю Хоттабыч тратит $5\cdot3+2\cdot5=25$ волосков. Поделим с остатком: $2019:25=80$ (ост. 19). Оставшихся 19 волосков хватит на 5 будних дней, а в субботу они закончатся. Значит, волоски закончатся через $80\cdot7+6=566$ дней.\\
115. За неделю Хоттабыч тратит $5\cdot2+2\cdot7=24$ волоска. Поделим с остатком: $2019:24=84$ (ост. 3). Оставшихся 3 волосков хватит на понедельник, а во вторник они закончатся. Значит, волоски закончатся через $84\cdot7+2=590$ дней.\\
116. За один цикл путник преодолевает $10\cdot2+3=23$ ступеньки. Поделим с остатком: $2019:23=87$ (ост. 18). Последние 18 ступенек путник преодолеет за $18:2=9$ шагов, а за один цикл он делает $10+3=13$ шагов. Значит, всего он сделает $87\cdot13+9=1140$ шагов.\\
117. Пусть только синих клеток С, только жёлтых Ж, сине-жёлтых СЖ, а чистых --- Ч. Синие и сине-жёлтые клетки составляют ровно треть, значит их в 2 раза меньше, чем остальных: $2(\text{С}+\text{СЖ})=\text{Ж}+\text{Ч},\ 2\text{С}+2\text{СЖ}=\text{Ж}+\text{Ч}.$ Также известно, что $\text{Ж}=2\text{С},$ поэтому $2\text{СЖ}=\text{Ч},$ что и требовалось доказать.\\
118. Пусть только синих клеток С, только жёлтых Ж, сине-жёлтых СЖ, а чистых --- Ч. Синие и сине-жёлтые клетки составляют ровно четверть, значит их в 3 раза меньше, чем остальных: $3(\text{С}+\text{СЖ})=\text{Ж}+\text{Ч},\ 3\text{С}+3\text{СЖ}=\text{Ж}+\text{Ч}.$ Также известно, что $\text{Ж}=3\text{С},$ поэтому $3\text{СЖ}=\text{Ч},$ что и требовалось доказать.\\
119. Нарисуем круги Эйлера для множеств марок Дениски, Мишки и Алёнки. У Дениски и Мишки нет одинаковых марок, значит в пересечении их множеств и в пересечении всех трёх множеств марок 0, как и уникальных марок у Алёнки. Пусть у Дениски и Алёнки одинаковых марок $x,$ тогда столько же уникальных марок у Мишки. Пусть у Дениски уникальных марок $y,$ тогда у Алёнки с Мишкой одинаковых марок тоже должно быть $y$ (так как у Дениски и Мишки марок поровну). Таким образом, у всех детей по $x+y$ марок, значит у Алёнки столько же марок, сколько у Дениски.\\
120. Повторим покупку Коли 3 раза, а покупку Антона --- 2 раза. Получим, что 6 пирожных, 12 йогуртов и 6 шоколадных батончиков стоят $440\cdot3=1320$ рублей, а 6 пирожных, 2 йогурта и 6 шоколадных батончиков стоят $560\cdot2=1120$ рублей. Значит, $12-2=10$ йогуртов стоят $1320-1120=200$ рублей. Таким образом, один йогурт стоит $200:10=20$ рублей.\\
121. Этих друзей $(80-62):3=6$ человек.\\
122. Численность войска составляет $1+10+10\cdot10+10\cdot10\cdot10=1111$ человек.\\
123. Если дописать к числу ноль, оно увеличится в 10 раз. Пусть он складывал числа $a$ и $b,$ тогда $a+b=1331,\ a+10b=8000,$ откуда $9b=8000-1331=6669,\ b=741,\ a=1331-741=590.$\\
124. Он проехал $(204-104):4=25$км и сломался у столба с надписью $104+25=129$км.\\
125. Пусть у Ани 12 частей, тогда у Бори $12:4=3$ части, а у Вити $12:3=4$ части. Всего частей $12+3=4=19,$ значит одна часть составляет $38:19=2$ монеты. Таким образом, у Ани $12\cdot2=24$ монеты.\\
126. Французы живут в 3 номерах, значит после японцев осталось $3\cdot3=9$ номеров, тогда после китайцев осталось $9\cdot2=18$ номеров, а всего в гостинице их было $18\cdot2=36.$\\
127. Разница между 7 шоколадками и 6 шоколадками составляет $35+15=50$ рублей, значит именно столько одна шоколадка. Поделим с остатком: $610:50=12$ (ост. 10), значит Саша сможет купить 12 шоколадок.\\
128. Разница между 9 открытками и 8 открытками составляет $14+26=40$ рублей, значит именно столько стоит одна открытка. Поделим с остатком: $700:40=17$ (ост. 20), значит Ира сможет купить 17 открыток.\\
129. Всего деревьев было $34+34\cdot2+34\cdot2+17=187.$\\
130. Вторая бочка больше на $83-70=13$ литров, что составляют одно ведро мальчика и одно ведро девочки. Поделим с остатком: $90:13=6$ (ост. 12), значит им нужно сходить за водой $6+1=7$ раз.\\
131. Вместе девочки собрали $22+(22-16)+(22-16)\cdot2=40$ марок.\\
132. Вместе хозяйки засолили $27+27\cdot2+(27+27\cdot2+4)=166$ банок.\\
133. Пусть на суп Гена затратил $x$ минут, тогда $x+3+x-12=19,\ 2x-9=19,\ 2x=28,\ x=14$ минут. На сборку и разборку Никита тратит 3 мин 44 с$+$1 мин 16 с$=5$мин. Значит, за 10 минут он успеет 2 раза собрать и разобрать робота, а за оставшиеся 4 минуты он соберёт робота в третий раз.\\
134. 8 кг груш занимают $8:4=2$ ящика. Если было продано $x$ ящиков, то осталось $x-2$ и $x+x-2=28,\ 2x-2=28,\ 2x=30,\ x=15.$\\
135. 14 кг моркови занимают $14:7=2$ мешка. Если было продано $x$ мешков, то осталось $x-2$ и $x+x-2=26,\ 2x-2=26,\ 2x=28,\ x=14.$\\
136. Всего саженцев вырастили $287+287:7\cdot(14+12)=1353$ штуки.\\
137. Пусть сначала Татьяна слепит всех лошадок, а Ольга в это время лепит павлинов. Тогда Татьяна потратит на это $40\cdot4=160$ минут, за это время Ольга слепит $160:5=32$ павлина и останется слепить ещё $54-32=22.$ За 30 минут мастерицы слепят вместе $30:5+30:6=11$ павлинов, значит закончат они работу через $22:11\cdot30=60$ минут. Всего у них уйдёт $160+60=220$ минут.\\
138. Всего в одном подъезде расположено $16\cdot6=96$ квартир. Поделим с остатком: $274:96=2$ (ост. 82). Это значит, что квартира Егора находится в 3 подъезде и является там 82-ой по счёту. Поделим с остатком ещё раз: $82:6=13$ (ост. 4), значит Егор живёт на 14-ом этаже.\\
139.  Пусть Лиза разложила по пять апельсинов в каждую корзину. Доложим три оставшихся лишними апельсина шестыми в три корзины. Если мы опустошим другие 3 корзины, то в оставшиеся как раз можно будет доложить по шестому апельсину. Таких <<дополненных>> корзин будет $3\cdot5=15,$ итого получится $3+15=18$ корзин, в которых лежит по 6 апельсинов. Значит, всего апельсинов $18\cdot6=108.$\\
140. Если сложить количество мам и пап, получится $24+18=42,$ что на $42-28=14$ больше, чем количество учеников в классе. Значит, у 14 человек на собрание одновременно пришли и мама, и папа.\\
141. Пусть пройдёт $x$ лет, тогда должно выполняться равенство $(12+x)\cdot2=42+x,\ 24+2x=42+x,\ 24+x=42,\ x=18$ лет.\\
142. Позавчера и сегодня Вовочка решил на $32-11=21$ задачу больше, чем вчера, а значит сегодня он решил 21 задачу.\\
143. Без учёта капитана на корабле $15-1=14$ голов и $41-1=40$ ног. Если бы на корабле были только матросы, у них было бы $14\cdot2=28$ ног, а должно быть на $40-28=12$ ног больше. При замене матроса на кошку становится на $4-2=2$ ноги больше, значит необходимо сделать $12:2=6$ замен. Таким образом, на корабле было 6 кошек.\\
144. Пусть всего участников забега --- 6 частей. Тогда перед ними бежит $6:2=3$ части, а позади --- $6:3=2$ части. Тогда Тилли Вилли и Дилли составляют $6-3-2=1$ часть от общего количества участником, которое равно $3\cdot6=18.$\\
145. Сделаем обратные действия: $2\cdot7=14,\ 14+6=20,\ 20:4=5,\ 5\cdot3=15,\ 15-5=10.$\\
146. Из 17 выстрелов 5 были доступны сразу, значит остальные $17-5=12$ он получил за точные попадания, которых было $12:2=6.$\\
147. Часовая стрелка делает полный оборот за $12\text{ ч}=720\text{ мин},$ а секундная --- за 1 мин, значит она движется в $720:1=720$ раз быстрее.\\
148. $8=3+5,\ 9=3+3+3,\ 10=5+5.$ Прибавляя к этим суммам купюры по 3 рубля, мы сможем получить любую сумму.\\
149. Если во дворе были бы одни куры, у них было бы $11\cdot2=22$ ноги. Это на $34-22=12$ ног меньше, чем должно быть. При замене куры на собаку становится на $4-2=2$ ноги больше. Значит, необходимо произвести $12:2=6$ замен. Таким образом, на дворе $11-6=5$ кур и 6 собак.\\
150. $\cfrac{2}{7}$ от 560 г --- это $560:7\cdot2=160$ г. Значит, у Феди осталось $160+29=189$ г орешков. Поэтому съел он $560-189=371$ г.\\
151. Пусть год назад сыну было $x$ месяцев, тогда маме было $x$ лет, то есть $12x$ месяцев. Тогда в через год сыну стало $x+12$ месяцев, а маме $12x+12$ месяцев
и $x+12+12x+12=41\cdot12,\ 13x+24=492,\ 13x=468,\ x=36.$ Значит, год назад сыну было $36:12=3$ года, а маме --- 36 лет. В этом году сыну стало $3+1=4$ года, а маме --- $36+1=37$ лет.\\
152. Так как Винни начал есть вдвое быстрее, за половину оставшегося пути (то есть четверть всего пути) он съест столько же мёда, сколько уже успел съесть (так как успел пройти он ровно в два раза больше --- половину пути). Значит, он съест ещё одну треть мёда и останется у него также одна треть. Теперь он идёт в два раза быстрее и ест также в два раза быстрее, значит последней трети мёда ему хватит на половину пути, а до дома Кролика надо пройти половину и ещё четверть. Таким образом, мёда не хватит.\\
153. Для нахождения минимального количества баллов, идущих в зачёт, необходимо найти максимально возможную стоимость худшей задачи. Если бы худшая задача стоила 15 баллов, Йумпага набрал бы как минимум $15\cdot5=75$ баллов, что больше, чем набранные им 72 балла. При этом стоить 14 баллов она может (например, если Йумпага набрал $14,\ 14,\ 14,\ 14$ и 16 баллов), а таком случае в зачёт пойдёт $72-14=58$ баллов.\\
154. Первый вносит половину суммы, вносимой остальными, то есть треть от стоимости лодки. Аналогично второй вносит четверть, а третий пятую часть стоимости лодки.
Разделим стоимость лодки на $3\cdot4\cdot5=60$ частей. Тогда первый вносит $60:3=20$ частей, второй $60:4=15$ частей, а третий --- $60:5=12$ частей. Таким образом, четвёртый вносит $60-20-15-12=13$ оставшихся частей, составляющих 130 рублей. Тогда одна часть равна $130:13=10$ рублей, а вся лодка стоит $60\cdot10=600$ рублей.\\
155. В третьем шкафу книг в 3 раза больше, чем в первых двух, значит в них одна четверть книг, а в третьем шкафу --- три четверти, то есть $2580:4\cdot3=1935$ книг. Тогда во втором шкафу $2580-1935-350=295$ книг.\\
156. Аня собрала в 2 раза больше васильков, чем Оля и Ира вместе взятые, значит они собрали одну треть, а Аня --- две трети, то есть $150:3\cdot2=100$ васильков. Тогда Оля собрала $150-100-28=22$ василька.\\
157. Если цветочного мёда $x$кг, то липового --- $x+15$ и $x+x+15=95,\ 2x+15=95,\ 2x=80,\ x=40,\ x+15=55$кг, поэтому липового мёда $55:5=11$ бочонков, а цветочного $40:5=8$ бочонков.\\
158. Если клубничного варенья $x$л, то малинового --- $x+9$ и $x+x+9=39,\ 2x+9=39,\ 2x=30,\ x=15,\ x+9=24$кг, поэтому клубничного варенья $15:3=5$ банок, а малинового $24:3=8$ банок.\\
159. Митя получил столько же, сколько Женя и Коля вместе, значит он получил половину всех денег, то есть $176:2=88$ рублей. Если Женя получил $x$ рублей, то Коля получил $x+34$ рубля и  $x+x+34=88,\ 2x+34=88,\ 2x=54,\ x=27,\ x+34=61$ рубль. Таким образом, Женя получил 27 рублей, Коля получил 61 рубль, а Митя получил 88 рублей.\\
160. Сложим все известные равенства: $\text{М}+\text{Г}+\text{Г}+\text{А}+\text{М}+\text{А}=36+42+50=128,\ 2(\text{М}+\text{Г}+\text{А})=128,\ \text{М}+\text{Г}+\text{А}=64$ гриба. Значит, Миша собрал $64-42=22$ гриба, Гриша собрал $64-50=14$ грибов, а Алёша собрал $64-36=28$ грибов.\\
161. Пусть второй помощник отремонтировал $x$ чайников, тогда первый помощник отремонтировал $x+3$ чайника, а мастер с одной стороны отремонтировал $x+3+6=x+9$ чайников, а с другой стороны --- $4x$ чайников, поэтому $4x=x+9,\ 3x=9,\ x=3$ чайника. Значит, первый помощник отремонтировал $3+3=6$ чайников, а мастер --- $3+9=12$ чайников. Таким образом, первый помощник потратил на ремонт одного чайника $3\text{ч}12\text{мин}:6=192\text{мин}:6=32\text{мин},$ второй потратил $192\text{мин}:3=64\text{мин},$ а мастер --- $192\text{мин}:12=16\text{мин}.$\\
162. 5 больших птиц и 3 маленьких стоят как $5\cdot2+3=13$ маленьких, а 3 больших и 5 маленьких --- как $3\cdot2+5=11$ маленьких. Значит, одна маленькая птица стоит $20:(13-11)=10$ рублей, а большая --- $10\cdot2=20$ рублей.\\
163. Всего учебников в библиотеке $273+27\cdot273=7644.$ Чистыми остались $7644:6=1274$ из них.\\
164. Всего в библиотеке $360:3\cdot4=480$ книг, исторических романов из них $480:5\cdot2=192.$\\
165. Лыжников было $696:3=232$ человека, а конькобежцев --- $(696-232):4=116$ человек. Значит, в хоккей играли $696-232-116=348$ человек.\\
166. В первый день Кум Тыква выложил $320:4+12=92$ кирпича, у него осталось $320-92=228$ кирпичей. Во второй день он выложил $228:3+7=83$ кирпича, у него осталось $228-83=145$ кирпичей, которые он и выложил в третий день.\\
167. Белых грибов было $80:5=16,$ подосиновиков было $(80-16):4=16,$ значит подберёзовиков было $80-16-16=48$ штук.\\
168. Пусть всего ребят было $4x,$ тогда $4x:2=2x$ ребят решили 2 задачи, $4x:4=x$ ребят решили 3 задачи, а остальные $4x-2x-x=x$ ребят решили 5 задач, поэтому $x=10,$ а всего участников было $4\cdot10=40$ человек.\\
169. За 30 часов <<Винтик>> соберёт $30:15\cdot16=32$ пылесоса, а <<Шпунтик>> соберёт 16 пылесосов за 24 часа и ещё $16:4=4$ пылесоса за оставшиеся 6 часов (так как 6 составляет четверть от 24). Значит, всего они соберут $32+16+4=52$ пылесоса. Поэтому 104 пылесоса они соберут за $104:52\cdot30=60$ часов.\\
170. За 1 час Юра делает $120:3=40$ корабликов, а Вася --- $120:2=60$ корабликов, вместе за 1 час они делают $40+60=100$ корабликов. Так как 25 составляет одну четверть от 100, им понадобится четверть часа, то есть $60:4=15$ минут. За 5 часов они сделают $5\cdot100=500$ корабликов.\\
171. За 1 час мама почистит $60:10=6$ вёдер, папа $60:12=5$ вёдер, а Вова --- 1 ведро. То есть вместе они за 1 час почистят $6+5=1=12$ вёдер. Поэтому 1 ведро они почистят за $60:12=5$ минут.\\
172. За 30 минут они вместе съели бы $30:10+30:15=5$ банок. Значит, 1 банку они вместе съедят за $30:5=6$ минут, а 7 банок --- за $7\cdot6=42$ минуты.\\
173. Пончик и Торопыжка вместе пекут за минуту $90:45+90:30=5$ блинов. Значит, Незнайка испечёт 90 блинов за $90:5=18$ минут.\\
174. Малыш ест $600:6=100$г варенья в минуту, а Карлсон --- $100\cdot2=200$г, значит вместе они за 1 минуту едят $100+200=300$г варенья и съедят 600 г за $600:300=2$ минуты.\\
175. За 20 минут они вместе выпьют $1+20:5=5$ мисок молока значит 1 миску они выпьют за $20:5=4$ минуты.\\
176. За 30 минут из обоих отверстий вместе вылилось бы $1+30:6=6$ баков, значит 1 бак выльется за $30:6=5$ минут.\\
177. Васька съедает банку <<Вискас>> за $6:2=3$ минуты, значит за 6 минут они вместе съели бы $1+6:3=3$ банки, поэтому 1 банку они съедят за $6:3=2$ минуты.\\
178. Малыш съедает $900:9=100$г варенья в минуту, а Карлсон --- $100\cdot2=200$г. Значит, вместе они съедают $100+200=300$г варенья в минуту и им понадобится $1800:300=6$ минут.\\
179. Старшая сестра собирает корзину клубники за $60:2=30$ минут, значит за час они вместе соберут $1+60:30=3$ корзины клубники, поэтому на 1 корзину им понадобится $60:3=20$ минут.\\
180. Токарь вытачивает $72:3=24$ детали за час, а его ученик --- $24:2=12$ деталей, вместе они вытачивают $24=12=36$ деталей, поэтому им понадобится $72:36=2$ часа.\\
181. За 56 дней вместе они съедят $1+56:8=8$ вагонов, поэтому 1 вагон они съедят за $56:8=7$ дней.\\
182. Если бы на празднике были только мальчики, у них было бы $22\cdot3=66$ шаров, что на $86-66=20$ меньше, чем должно быть. Замена одного мальчика на девочку увеличивает количество шаров на $5-3=2,$ поэтому необходимо сделать $20:2=10$ замен, при этом мальчиков останется $22-10=12.$ Таким образом, мальчиков больше на $12-10=2.$\\
183. Если бы в гараже были только легковые автомобили, у них было бы $750\cdot4=3000$ колёс, что на $3024-3000=24$ меньше, чем должно быть. Замена одного легкового автомобиля на грузовой увеличивает количество колёс на $6-4=2,$ поэтому необходимо сделать $24:2=12$ замен, при этом легковых автомобилей останется $750-12=738.$ Таким образом, легковых автомобилей больше на $738-12=726.$\\
184. Если бы в порту были только яхты, у них было бы $100\cdot1=100$ мачт, что на $146-100=46$ меньше, чем должно быть. Замена одной яхты на шхуну увеличивает количество мачт на $2-1=1,$ поэтому необходимо сделать $46:1=46$ замен, при этом яхт останется $100-46=54.$\\
185. Если бы в комнате были только столы с одним ящиком, у них было бы $10\cdot1=10$ ящиков, что на $14-10=4$ меньше, чем должно быть. Замена одного стола с одним ящиком на стол с двумя ящиками увеличивает количество ящиков на $2-1=1,$ поэтому необходимо сделать $4:1=4$ замены, при этом столов с одним ящиком останется $10-6=6.$\\
186. Пусть босых четвероклассников было $x,$ тогда босых пятиклассников было $16-x,$ а значит всего пятиклассников было $x+16-x=16.$ Поэтому всего на прогулку пошли $24+16=40$ учеников.\\
187. Даже если каждый борец и футболист также занимается плаванием, таких детей будет только $15+10=25,$ что меньше, чем 27. Значит, описанная ситуация невозможна.\\
188. Сложим всех спортсменов, получится $17+6+13=36$ человек, при этом каждого спортсмена мы посчитали по 2 раза, значит всего их $36:2=18.$\\
189. Витя полил $100:2=50$ кустов, а Аня полили новых кустов $100:2-3=47.$ Значит, $100-50-47=3$ куста так и остались неполитыми.\\
190. Будем ставить человеку галочку, если он сходил в поход. Тогда всего было поставлено $20\cdot4=80$ галочек. С другой стороны люди получили $10\cdot4+9\cdot3+5\cdot2=77$ галочек. Значит, оставшиеся $80-77=3$ галочки получили те 3 человека, которые сходили в 1 поход.\\
191. Девочек в секции $48:3=16$ человек, а мальчиков --- $48-16=32.$ На сборы ездили $48:4=12$ человек, при этом девочек ездило $16-9=7$ человек. Тогда мальчиков на сборах было $12-7=5$ человек, а остальные $32-5=27$ на сборы не ездили.\\
192. Читателями обеих библиотек являются $25+20-35=10$ человек.\\
193. Всего решать задачи любят $24-3=21$ человек. Только олимпиадные задачи любят решать $6-2=4$ человека, значит только обычные задачи любят решать $21-4-2=15$ человек.\\
194. Пусть ящик с апельсинами весит $x$кг, тогда ящик с бананами весит $x+2$кг. Раз вес ящиков с бананами на 24 кг больше, чем с апельсинами, их должно быть $24:2=12$ штук (так как каждый ящик весит больше на 2 кг) и тогда $12(x+x+2)=96,\ 2x+2=8,\ 2x=6,\ x=3,\ x+2=5.$ Таким образом, ящиков было по 12 штук, ящик с апельсинами весит 3 кг, а ящик с бананами --- 5 кг.\\
195. В третьей рукописи на $480-320=160$ страниц больше, чем во второй, а значит печатали со скоростью $160:4=40$ страниц в день. Тогда на все три рукописи ушло $(240+320+480):40=26$ дней.\\
196. Одна шоколадка стоила $300:6=50$ рублей, значит всего было потрачено $50\cdot10=500$ рублей. Если Варя потратила $x$ рублей, то Маша потратила $x+300$ и $x+x+300=500,\ 2x+300=500,\ 2x=200,\ x=100,\ x+300=400$ рублей.\\
197. Сделаем действия в обратном порядке: $105-21=84,\ 84:3=28,\ 28+17=45,\ 360:45=8.$\\
198. Сделаем действия в обратном порядке: $18\cdot4=72,\ 72\rightarrow27,\ 27:9=3,\ 3+8=11.$\\
199. Сделаем действия в обратном порядке: $4\cdot11=44,\ 44+10=54,\ 54:2=27,\ 27-19=8$ лет.\\
200. Сделаем действия в обратном порядке: $200:5=40,\ 40-12=28,\ 364:28=13,\ 13+49=62.$\\
201. Другая половина репки весит 4 кг, значит репка весит $4\cdot2=8$кг.\\
202. Значит $\cfrac{1}{6}$ кирпича весит 2 кг, тогда кирпич весит $2\cdot6=12$кг.\\
203. Пусть весь путь был $6x,$ тогда $3x+1+2x+1=6x,\ 5x+2=6x,\ 2=x,\ 6x=6\cdot2=12$км.\\
204. Если стакан наполнен бактериями наполовину, через 1 секунду он наполнится полностью. Значит, наполовину он был наполнен через 59 секунд.\\
205. Если кувшинка за 30 дней покрывает весь пруд, за 29 дней она покрывает половину пруда. Тогда две кувшинки за 29 дней покроют весь пруд.\\
206. Повторим покупку Коли 3 раза, а покупку Антона 2 раза, получим: 6 пирожных, 12 кексов и 6 шоколадок стоят $3\cdot440=1320$ рублей, а 6 пирожных, 2 кекса и 6 шоколадок стоят $2\cdot560=1120$ рублей. Поэтому $12-2=10$ кексов стоят $1320-1120=200$ рублей, то есть 1 кекс стоит $200:10=20$ рублей.\\
207. Сложим её покупки: 35 больших и 35 маленьких бусинок стоят $528+522=1050$ рублей, поэтому 5 больших и 5 маленьких бусинок стоят $1050:7=150$ рублей, а 20 больших и 20 маленьких бусинок стоят $4\cdot150=600$ рублей.\\
208. Повторим покупку Оли 5 раз: 5 блокнотов и $5\cdot8=40$ ручек стоят $5\cdot121=605$ рублей. Значит, $40-6=34$ ручки стоят $605-129=476$ рублей, поэтому 1 ручка стоит $476:34=14$ рублей, а 1 блокнот --- $121-8\cdot14=9$ рублей.\\
209. 1 пирожное стоит $300-200=100$ рублей, значит 1 булочка стоит $(200-100):4=25$ рублей. Значит, Ваня сможет купить $(1000-8\cdot100):25=8$ булочек.\\
210. Представим себе двухчашечные весы и снимем с каждой из сторон по 2 красных кирпича. Получим, что 1 белый кирпич легче 1 красного кирпича на 1 кг.\\
211. Пусть в третьем племени живёт $x$ человек, тогда в первом племени живёт $3x$ человек, а во втором --- $21-x.$ Значит, $3x+21-x=47,\ 2x+21=47,\ 2x=26,\ x=13$ человек. Значит, всего на острове живёт $47+13=60$ человек. Если бы у всех были бы рубины, их было бы $60\cdot2=120$ штук, что на $180-120=60$ меньше, чем должно быть. Одна замена рубинов на алмазы делает на $3-2=1$ камень больше, значит таких замен должно быть $60:1=60.$ То есть должно быть 60 человек с алмазами, у них будет $60\cdot3=180$ алмазов.\\
212. В 2 раза больше денег --- это $2\cdot2=4$ конфеты и ещё $2\cdot3=6$ рублей. С другой стороны, на эти 6 рублей можно купить пятую конфету и ещё рубль останется, поэтому одна конфета стоит $6-1=5$ рублей.\\
213. У Димы остаётся 1 рубль, и если добавить к нему 3 рубля, то можно будет купить вторую конфету. Значит, конфета стоит $1+3=4$ рубля.\\
214. На $7+3=10$ рублей можно у=купить недостающие  $20-15=5$ конфет, значит одна конфета стоит $10:5=2$ рубля.\\
215. Если поросят стало в 2 раза больше, то и домиков за то же время они построят в 2 раза больше. Значит, 6 поросят построят 6 домиков за 3 дня.\\
216. Один землекоп за 2 часа выкопает одну яму, тогда 6 землекопов за 2 часа выкопают 6 ям, а за 5 часов выкопают $6\cdot2+6:2=15$ ям.\\
217. 4 яблока весят как 6 груш и 1 яблоко, значит 3 яблока весят как 6 груш, значит 1 яблоко весит как 2 груши.\\
218. Уберём по 1 яблоку с обеих чаш: 4 яблока весят как 2 груши, значит 2 яблока весят как груша, значит груша весит $100\cdot2=200$г.\\
219. Петя весит $89-63=26$кг, Миша весит $89-58=31$кг, а Коля весит $89-26-31=32$кг.\\
220. Сложим все фотографии: $8+6+3+7=24.$ Так как на каждой фотографии по 3 девочки, каждая фотография была посчитана 3 раза и на самом деле их $24:3=8.$ Таня присутствует на 6 снимках, значит сделала она $8-6=2$ снимка.\\
221. Сложим все песни: $8+6+3+7=24.$ Так как каждую песню пели 3 девочки, каждая песня была посчитана 3 раза и на самом деле их $24:3=8.$ Таня пела 6 песен, значит аккомпанировала она $8-6=2$ раза.\\
222. Сложим все подарки: $7+9+6=22.$ Каждый подарок был посчитан 2 раза, значит всего их $22:2=11$ штук.\\
223. Если у Оли $x$ конфет, то у Миши их с одной стороны $7x,$ а с другой стороны $x+42$ (если он отдаст 21 конфету, у обоих станет по $x=21$). Значит, $7x=x+42,\ 6x=42,\ x=7$ конфет. Значит, вместе у них было $7+7\cdot7=56$ конфет.\\
224. Если у Оли $x$ рублей, у Вари их $x+100$ и надо отдать 50 рублей (у обеих станет по $x+50$).\\
225. Всего у них 5 кружек крупы на троих, поделим их на 15 частей, каждому лесорубу достанется по $15:3=5$ частей. Тогда первый продал третьему $2\cdot3-5=1$ часть, а второй --- $3\cdot3=5=4$ части, поэтому первому надо дать 1 рубль, а второму --- 4 рубля.\\
226. Пусть Гек с одной ветки перевесит игрушку на пустую ветку, а с другой игрушку отложит. Получится в точности ситуация Чука (3 ветки по одной игрушке и 1 игрушка в стороне), значит всего веток и было 3, а игрушек --- 4.\\
227. Пусть во второй ситуации 1 галка пересядет на пустую палку, а другая отлетит в сторону. Получится в точности первая ситуация (3 палки по одной галке и 1 галка в стороне), значит всего палок и было 3, а галок --- 4.\\
228. Наибольший возможный остаток при делении на 13 равен 12, значит это число $13\cdot17+12=233.$\\
229. Лилий осталось $45-9=36.$ Пусть белых лилий $x,$ тогда жёлтых лилий $3x$ и $x+3x=36,\ 4x=36,\ x=9.$ Значит, белых лилий изначально было $9+3=12,$ а жёлтых --- $45-12=33.$\\
230. Если у Маши $x$ рублей, у Вари их $x+300$ и надо отдать 150 рублей (у обеих станет по $x+150$).\\
231. Пусть у него было число $2x,$ тогда $2x:2+3=2x\cdot2-3,\ x+3=4x-3,\ 3=3x-3,\ 3x=6,\ x=2,\ 2x=4.$ Значит, получил он $4:2+3=5.$\\
232. Если зелёный был 1, то синих было 6, а красных 13, что не подходит. если зелёных было 3 или больше, то синих было 18 или больше, тогда в сумме их уже больше 20. Значит, зелёных карандашей было 2, тогда синих было 12, а красных было 6.\\
233. В одном подъезде $12\cdot3=36$ квартир. Поделим с остатком: $189:36=5$ (ост. 9), значит эта квартира расположена в 6-м  подъезде.\\
234. Если бы все переходы были в переулке, на них было бы $8\cdot7=56$ полос, что на $81-56=25$ меньше, чем должно быть. При замене переулка на улицу становится на $12-7=5$ полос больше, значит необходимо произвести $25:5=5$ замен и переходов в переулке останется $8-5=3.$\\
235. Пусть лип было посажено $x,$ тогда берёз было посажено $x+30$ и $x+x+30=84,\ 2x+30=84,\ 2x=54,\ x=27,\ x+30=57.$ Пусть елей было посажено $y,$ тогда сосен посажено $6y$ и $y+6y=84:2,\ 7y=42,\ y=6,\ 6y=36.$ Значит, сосен и берёз вместе было посажено $36+57=93$ дерева.\\
236. Пусть тигров было $x,$ тогда волков было $7x$ и $x+7x=56,\ 8x=56,\ x=7,\ 7x=49.$ Пусть косуль было $y,$ тогда зайцев $y+20$ и $y+y+20=56\cdot2,\ 2y+20=112,\ 2y=92,\ y=46.$ Тогда волков и косуль было $49+46=95$ особей.\\
237. Объединим данные: из $3+4=7$ тонн яблок и $4+3=7$ тонн винограда получится $1130+1250=2380$кг сухофруктов, поэтому из 1 т яблок и 1 т винограда получится $2380:7=340$кг сухофруктов, а значит из 5 т яблок и 5 т винограда получится $340\cdot5=1700$кг сухофруктов. Из 3 тонн винограда и 3 тонн яблок получится $340\cdot3=1020$кг сухофруктов, значит из 1 тонны винограда получится $1130-1020=110$кг изюма.\\
238. Объединим данные: из $5+4=9$ тонн слив и $4+5=9$ тонн груш получится $1530+1620=3150$кг сухофруктов, поэтому из 1 т слив и 1 т груш получится $3150:9=350$кг сухофруктов, а значит из 7 т слив и 7 т груш получится $350\cdot7=2450$кг сухофруктов. Из 4 тонн слив и 4 тонн груш получится $350\cdot4=1400$кг сухофруктов, значит из 1 тонны слив получится $1530-1400=130$кг чернослива.\\
239. Уберём по 3 карандаша: 3 карандаша стоят на 30 рублей дешевле, чем 3 ручки, значит карандаш стоит на $30:3=10$ рублей дешевле ручки.\\
240. Уберём по 4 тетради: 4 тетради стоят на 40 рублей дешевле, чем 4 блокнота, значит тетрадь стоит на $40:4=10$ рублей дешевле блокнота.\\
241. Саша и Миша весят половину общего веса, то есть $200:2=100$кг.\\
242. Саша и Миша весят половину общего веса, то есть $100:2=50$кг.\\
243. Подмастерьям необходимо изготовить $110-3\cdot10=80$ ключей. Один подмастерье за 10 дней сделает $10\cdot2-10\cdot2:5=16$ хороших ключей, значит необходимо взять $80:16=5$ подмастерьев.\\
244. Подмастерьям необходимо изготовить $180-4\cdot9=144$ ключа. Один подмастерье за 9 дней сделает $9\cdot3=27$ ключей, из которых 5 бракованные. Поделим с остатком: $144:(27-5)=6$ (ост. 12), значит необходимо взять 7 подмастерьев.\\
245. Пусть за первые 4 дня рождения Паша получил $y$ подарков, а на 5 день рождения он получил $x$ подарков. Тогда всего у него после 10 дня рождения будет $y+x+x-1+x-2+x-3+x-4+x-5=y+6x-15$ подарков, поэтому $y+6x-15=55,\ y+6x=70.$ Минимально возможное значение $x$ равно 5 (он не мог получить на 10 день рождения меньше 0 подарков), а максимальное --- 11 ($12\cdot6=72>70$). Значит, на 10 день рождения он мог получить от $5-5=0$ до $11-5=6$ подарков.\\
246. Пусть за первые 4 дня рождения Паша получил $y$ подарков, а на 5 день рождения он получил $x$ подарков. Тогда всего у него после 11 дня рождения будет $y+x+x-2+x-4+x-6+x-8+x-10+x-12=y+7x-42$ подарка, поэтому $y+7x-42=112,\ y+7x=154.$ Минимально возможное значение $x$ равно 12 (он не мог получить на 11 день рождения меньше 0 подарков), а максимальное --- 22 ($23\cdot7=161>154$). Значит, на 10 день рождения он мог получить от $12-10=2$ до $22-10=12$ подарков.\\
247. Количество баскетболистов каждую минуту либо увеличивалось на 1, либо уменьшалось на 2. Значит, максимум баскетболистов могло быть $10\cdot2=20$ человек (если количество только уменьшалось). Если заменить одно уменьшение на увеличение, изначальное количество баскетболистов уменьшится на $2+1=3.$ Значит, их могло быть $20,\ 17,\ 14,\ 11,\ 8,\ 5,\ 2.$\\
248. Количество баскетболистов каждую минуту либо увеличивалось на 1, либо уменьшалось на 2. Значит, максимум баскетболистов могло быть $11\cdot2=22$ человека (если количество только уменьшалось). Если заменить одно уменьшение на увеличение, изначальное количество баскетболистов уменьшится на $2+1=3.$ Значит, их могло быть $22,\ 19,\ 16,\ 13,\ 10,\ 7,\ 4,\ 1.$\\
249. В первый день здоровых не более 327 детей, во второй не более 163, в третий не более 81, на четвёртый не более 40, на пятый не более 19, на шестой меньше 10.\\
250. В первый день здоровых не более 362 детей, во второй не более 180, в третий не более 89, на четвёртый не более 44, на пятый не более 21, на шестой меньше 11.\\
251. Для нахождения минимального количества баллов, идущих в зачёт, необходимо найти максимально возможную стоимость худшего рисунка. Если бы худший рисунок стоил 15 баллов, Вася набрал бы как минимум $15\cdot5=75$ баллов, что больше, чем набранные им 72 балла. При этом стоить 14 баллов он может (например, если Вася набрал $14,\ 14,\ 14,\ 14$ и 16 баллов), а таком случае в зачёт пойдёт $72-14=58$ баллов.\\
252. Для нахождения минимального количества баллов, идущих в зачёт, необходимо найти максимально возможную стоимость худшего рисунка. Если бы худший рисунок стоил 17 баллов, Петя набрал бы как минимум $17\cdot5=85$ баллов, что больше, чем набранные им 82 балла. При этом стоить 16 баллов он может (например, если Петя набрал $16,\ 16,\ 16,\ 16$ и 18 баллов), а таком случае в зачёт пойдёт $82-16=66$ баллов.\\
253. Весь подарок стоил $5695+1405=7100$ рублей, значит изначально Вадим должен был заплатить $7100:2=3550$ рублей. Он заплатил больше и именно эту разницу ему должен отдать Кирилл, чтобы все оказались в расчёте: $5695-3550=2145$ рублей.\\
254. Весь подарок стоил $5695+1445=7140$ рублей, значит изначально Вадим должен был заплатить $7140:2=3570$ рублей. Он заплатил больше и именно эту разницу ему должен отдать Кирилл, чтобы все оказались в расчёте: $5695-3570=2125$ рублей.\\
255. 4 весёлых обезьяны и 2 грустных за $3\cdot20=60$ минут съедят $1\cdot3=3$ ящика бананов, а 1 весёлая обезьяна и 2 грустных за то же время съедают 1 ящик бананов. Значит, $4-1=3$ весёлых обезьяны за 60 минут съедают $3-1=2$ ящика бананов. Тогда 1 весёлая обезьяна съедает 2 ящика бананов за $60\cdot3=180$ минут, поэтому 1 ящик бананов она съест за $180:2=90$ минут.\\
256. 8 весёлых обезьяны и 4 грустных за $40:4=10$ минут съедят 1 ящик бананов, значит за $10\cdot3=30$ минут они съедят $1\cdot3=3$ ящика бананов. Поэтому $8-2=6$ весёлых обезьян за 30 минут съедают $3-1=2$ ящика бананов. Значит, 1 весёлая обезьяна съест 2 ящика бананов за $30\cdot6=180$ минут, а 1 ящик --- за $180:2=90$ минут.\\
257. На распиливание длинного бревна уходит $5-1=4$ распила, а на распиливание короткого --- $4-1=3.$ Предположим, что все брёвна короткие, тогда на распиливание коротких брёвен ушло $3\cdot35=105$ распилов, а на распиливание длинных --- 0. Разница между распилами коротких брёвен и распилами длинных составляет $105-0=105.$ При замене одного короткого бревна на длинное количество распилов коротких брёвен уменьшается на 3, а количество распилов длинных увеличивается на 4, таким образом разница уменьшается на $3+4=7.$ Значит, необходимо сделать $105:7=15$ замен (разница должна стать равной 0), поэтому длинных брёвен должно быть 15, а следовательно общее количество распилов равно $20\cdot3\cdot2=120.$\\
258. На распиливание длинного бревна уходит $6-1=5$ распилов, а на распиливание короткого --- $5-1=4.$ Предположим, что все брёвна короткие, тогда на распиливание коротких брёвен ушло $4\cdot36=144$ распила, а на распиливание длинных --- 0. Разница между распилами коротких брёвен и распилами длинных составляет $144-0=144.$ При замене одного короткого бревна на длинное количество распилов коротких брёвен уменьшается на 4, а количество распилов длинных увеличивается на 5, таким образом разница уменьшается на $4+5=9.$ Значит, необходимо сделать $144:9=16$ замен (разница должна стать равной 0), поэтому длинных брёвен должно быть 16, а следовательно общее количество распилов равно $16\cdot5\cdot2=160.$\\
259. Это число $100:5:10=2.$\\
260. Это число $10\cdot2\cdot5=100.$\\
261. Поделим с остатком: $38:4=9$ (ост. 2). Значит, квартира №38 находится на 10-м этаже.\\
262. Если мы сложим школьников, изучающих английский, со школьниками, изучающими немецкий, получим $12+14=26.$ Результат оказался больше, чем общее количество школьников, так как изучающие оба языка школьники были посчитаны 2 раза, значит их было $26-18=8.$\\
263. Пусть у Максима изначально $a$ денег, а на второе мороженое ему не хватает $b$ денег. Тогда на $a+b$ денег он может купить 2 мороженых, а на $a+4b$ денег он может купить 3 мороженых, а значит 1 мороженое стоит $4b-b=3b$ денег. Поэтому для покупки 4 мороженых надо дать на $3b$ денег больше, чем на 3 мороженых, то есть $4b+3b=7b$ денег, что в 7 раз больше, чем ему не хватает на второе.\\
264. Пусть у Кирилла изначально $a$ денег, а на второе мороженое ему не хватает $b$ денег. Тогда на $a+b$ денег он может купить 2 мороженых, а на $a+5b$ денег он может купить 3 мороженых, а значит 1 мороженое стоит $5b-b=4b$ денег. Поэтому для покупки 4 мороженых надо дать на $4b$ денег больше, чем на 3 мороженых, то есть $5b+4b=9b$ денег, что в 9 раз больше, чем ему не хватает на второе.\\
265. Сначала определим, какие номера для предпринимателя выгоднее. Номер <<люкс>> приносит $5000:40=125$ рублей с квадратного метра, а обычный номер приносит 4000 рублей с 30 квадратных метров, что выгоднее, так как $30\cdot125=3750<4000.$ Значит, надо постараться сделать как можно больше обычных номеров. Поделим с остатком: $940:30=31$ (ост. 10). Максимум можно сделать 31 обычный номер, но 10 квадратных метров остаются свободными, значит один из номеров можно заменить на <<люкс>>. Тогда со всех трёх зданий предприниматель заработает $3\cdot(30\cdot4000+1\cdot5000)=375000$ рублей.\\
266. Сначала определим, какие номера для предпринимателя выгоднее. Номер <<люкс>> приносит $5000:40=125$ рублей с квадратного метра, а обычный номер приносит 4000 рублей с 30 квадратных метров, что выгоднее, так как $30\cdot125=3750<4000.$ Значит, надо постараться сделать как можно больше обычных номеров. Поделим с остатком: $820:30=27$ (ост. 10). Максимум можно сделать 27 обычных номеров, но 10 квадратных метров остаются свободными, значит один из номеров можно заменить на <<люкс>>. Тогда со всех трёх зданий предприниматель заработает $3\cdot(26\cdot4000+1\cdot5000)=327000$ рублей.\\
267. Если бы все числа были трёхзначными (хоть такого быть и не может, так как их всего 900), в них было бы $1000\cdot3=3000$ цифр. Не хватает ещё $3900-3000=900$ цифр. При замене трёхзначного числа на четырёхзначное добавляется $4-3=1$ цифра, значит необходимо сделать $900:1=900$ замен и искомое число будет 900-м четырёхзначным числом, которое равно $1000+900-1=1899$ (от 1000 до 1899 ровно $1899-1000+1=900$ чисел).\\
268. Если бы все числа были трёхзначными (хоть такого быть и не может, так как их всего 900), в них было бы $1000\cdot3=3000$ цифр. Не хватает ещё $3800-3000=800$ цифр. При замене трёхзначного числа на четырёхзначное добавляется $4-3=1$ цифра, значит необходимо сделать $800:1=800$ замен и искомое число будет 800-м четырёхзначным числом, которое равно $1000+800-1=1799$ (от 1000 до 1799 ровно $1799-1000+1=800$ чисел).\\
269. Разделим год пользования ноутбуком на 4 части, тогда всего частей было $4\cdot5=20$ и одна часть стоит $100000:20=5000$ рублей. Если Анна заплатила $x$ рублей, то Максим заплатил $3x$ и $x+3x=100000,\ 4x=100000,\ x=25000,\ 3x=75000.$ За первый год Максим использовал только одну часть (а Анна три), значит у него остались права ещё на $75000-5000=70000$ рублей, которые и должна отдать ему Анна.\\
270. Разделим год пользования ноутбуком на 4 части, тогда всего частей было $4\cdot10=40$ и одна часть стоит $200000:40=5000$ рублей. Если Алина заплатила $x$ рублей, то Кирилл заплатил $3x$ и $x+3x=200000,\ 4x=200000,\ x=50000,\ 3x=150000.$ За первый год Кирилл использовал только одну часть (а Алина три), значит у него остались права ещё на $150000-5000=145000$ рублей, которые и должна отдать ему Алина.\\
271. Остаток равен $6:2=3.$ Значит, делили число $6\cdot4+3=27.$\\
272. Пусть Маша съела $x$ конфет, тогда осталось  $7x$ конфет и $x+7x=56,\ 8x=56,\ x=56:8=7.$\\
273. Так как $135:4=33$ (ост. 3), могли украсить максимум 33 пирожных. Действительно, если бы украсили 34, на них бы потратили $4\cdot34=136$ ягод, а $136>135.$\\
274. Количество выпиваемых чашек увеличивается на единицу каждую минуту. Значит, количество чашек на столе через 45 минут будет равно $7+1+2+\ldots+44+45-1-2-\ldots-43-44=7+45=52.$\\
275. Масса бананов равна $36000-4950=31050\text{ г}=31\text{ кг }50\text{ г}.$ Масса апельсинов равна $36:3\cdot2=24$кг. Масса винограда равна $(22000-1300):2=10350\text{ г}=10\text{ кг }350\text{ г}.$ Масса киви равна $10350+1300=11650\text{ г}=11\text{ кг }650\text{ г}.$ Масса бананов больше массы винограда в $31050:10350=3$ раза.\\
276. Сын делает 5 шагов из их совместных $3+5=8,$ значит он сделает $6400:8\cdot5=4000$ шагов.\\
277. Пока Петя ел 18 ложек, Вася съел $18:2\cdot5=45.$ Значит, банка вмещает в себя $18+45=63$ ложки.\\
278. В третий отель могли приехать 1 или 2 туриста. В первом случае в первый отель приехали $1+10=11$ туристов, во второй $11\cdot2=22$ туриста, а в четвёртый $3-1=2$ туриста, всего $11+22+1+2=36$ туристов. Во втором случае в первый отель приехали $2+10=12$ туристов, во второй $12\cdot2=24$ туриста, в четвёртый $3-2=1$ турист, всего $12+24+2+1=39$ туристов.\\
279. В третий отель могли приехать 1 или 2 туриста. В первом случае в первый отель приехали $1+11=12$ туристов, во второй $12\cdot2=24$ туриста, а в четвёртый $3-1=2$ туриста, всего $12+24+1+2=39$ туристов. Во втором случае в первый отель приехали $2+11=13$ туристов, во второй $13\cdot2=26$ туристов, в четвёртый $3-2=1$ турист, всего $13+26+2+1=42$ туриста.\\
280. За 70 минут Витя съест 6 тортов, а Миша --- 5, при этом оба отдохнут и будут готовы продолжать есть. Значит, за $70\cdot4=280$ минут они съедят $(6+5)\cdot4=44$ торта. Через 30 минут Витя съест 3 торта и начнёт отдыхать, а Миша съест 2 и будет доедать третий. Он доест его через 6 минут и тогда будет съедено $44+3+2+1=50$ тортов. Всего пройдёт $280+30+6=316$ минут.\\
281. В каждой паре $2+3=5$ клешней, значит осталось $533-40\cdot5=333$ клешни. Так как на 2 это число не делится, остались марсиане с 3 клешнями и их $333:3=111.$ Таким образом, всего было $40\cdot2+111=191$ марсианин.\\
282. Если после 457446 он получил 458403, то он прибавлял $458403-457446=957.$ Значит, после этого он прибавит 958 и получит $458403+958=459361.$\\
283. Если после 466095 он получил 467061, то он прибавлял $467061-466095=966.$ Значит, после этого он прибавит 967 и получит $467061+967=468028.$\\
284. Если первым выписали число 100, то выписаны числа от 100 до 999 и от 1000 до 1099, всего в них $900\cdot3+100\cdot4=3100$ цифр. Если начать выписывать числа позже, то цифр будет становиться только больше (вместо трёхзначных будут выписаны четырёхзначные). Если начать выписывать раньше, то вместо четырёхзначных будут выписываться двузначные числа, в которых на $4-2=2$ цифры меньше, значит надо начать выписывать на $(3100-3092):2=4$ числа раньше, то есть первым выписать $100-4=96.$\\
285. Если первым выписали число 100, то выписаны числа от 100 до 999 и от 1000 до 1099, всего в них $900\cdot3+100\cdot4=3100$ цифр. Если начать выписывать числа позже, то цифр будет становиться только больше (вместо трёхзначных будут выписаны четырёхзначные). Если начать выписывать раньше, то вместо четырёхзначных будут выписываться двузначные числа, в которых на $4-2=2$ цифры меньше, значит надо начать выписывать на $(3100-3090):2=5$ числа раньше, то есть первым выписать $100-5=95.$\\
286. Если бы в деревне были одни куры, то они бы сказали, что у них $53088:2\cdot3=79632$ ноги. До заявленного количества не хватает $123456-79632=43824$ ноги. При замене куры на корову становится на $5-3=2$ ноги больше, значит надо сделать $43824:2=21912$ замен и в деревне 21912 коров и $53088:2-21912=4632$ куры, у них $21912\cdot4+4632\cdot2=96912$ ног.\\
287. Если бы в деревне были одни куры, то они бы сказали, что у них $53328:2\cdot3=79992$ ноги. До заявленного количества не хватает $123456-79992=43464$ ноги. При замене куры на корову становится на $5-3=2$ ноги больше, значит надо сделать $43824:2=21732$ замен и в деревне 21732 коров и $53328:2-21732=4932$ куры, у них $21732\cdot4+4932\cdot2=96792$ ноги.\\
288. Всего у детей дети получат $(100+100+150+200+200+500)\cdot4=5000$ юаней, из которых 50 уйдут на оплату комиссии, то есть каждый сотый полученный юань. Тогда тот, у кого было 150 долларов, получит $150\cdot4-150\cdot4:100=594$ юаня.\\
289. Всего у детей дети получат $(500+500+450+400+400+250)\cdot4=10000$ юаней, из которых 50 уйдут на оплату комиссии, то есть каждый двухсотый полученный юань. Тогда тот, у кого было 450 долларов, получит $450\cdot4-450\cdot4:200=1791$ юань.\\
290. Так как $5\cdot4=20,$ а также три нуля есть на конце этих чисел, произведение оканчивается на 4 нуля.\\
291. Так как $5\cdot6=30,$ а также пять нулей есть на конце этих чисел, произведение оканчивается на 6 нулей.\\
292. Задумали число $21012:412=51.$\\
293. Задумали число $17013:321=53.$\\
294. 10 тетрадей и 12 ручек стоили бы $2\cdot120=240$ рублей, 12 блокнотов стоят дороже 12 ручек на $12\cdot4=48$ рублей, поэтому за 10 тетрадей и 12 блокнотов Антон заплатил $240+48=288$ рублей.\\
295. 14 блокнотов и 18 линеек стоили бы $2\cdot160=320$ рублей, 14 альбомов стоят дороже 14 блокнотов на $14\cdot6=84$ рубля, поэтому ща 14 альбомов и 18 линеек Ксюша заплатила $320+84=404$ рубля.\\
296. Три четверти куртки стоят 3600 рублей, значит куртка стоит $3600:3\cdot4=4800$ рублей.\\
297. Две трети брюк стоят 2400 рублей, значит брюки стоят $2400:2\cdot3=3600$ рублей.\\
298. К цепи добавилась часть звена без сцепки и она стала длиннее на $18-13=5$см. Цепь из двух звеньев состоит из двух частей без сцепок и сцепки, значит длина сцепки равна $13-2\cdot5=3$см, тогда длина звена между двумя сцепками равна $5-3=2$см. Цепь из 7 звеньев состоит из 6 сцепок, 2 частей звена без сцепок и 5 частей между сцепками, поэтому её длина равна $6\cdot3+2\cdot5+5\cdot2=38$см.\\
299. К цепи добавилась часть звена без сцепки и она стала длиннее на $22-17=5$см. Цепь из трёх звеньев состоит из четырёх частей без сцепок и сцепки, значит длина сцепки равна $22-4\cdot5=2$см, тогда длина звена между двумя сцепками равна $5-2=3$см. Цепь из 10 звеньев состоит из 9 сцепок, 2 частей звена без сцепок и 8 частей между сцепками, поэтому её длина равна $9\cdot2+2\cdot5+8\cdot3=52$см.\\
300. Пусть на второй ферме живёт $x$ капибар, тогда на первой их $3x,$ а на третьей --- $3\cdot(x+3x)=12x$ и $x+3x+12x=784,\ 16x=784,\ x=49.$ Тогда на первой ферме живёт $3\cdot49=147$ капибар.\\
301. Пусть на второй ферме живёт $x$ капибар, тогда на первой их $3x,$ а на третьей --- $3\cdot(x+3x)=12x$ и $x+3x+12x=768,\ 16x=768,\ x=48.$ Тогда на первой ферме живёт $3\cdot48=144$ капибары.\\
302. Нюша задумала число $(2\cdot7+6):4\cdot3-1=14.$\\
303. Нюша задумала число $(3\cdot6+2):5\cdot4-1=15.$\\
304. Лис пришло 8 пар, то есть $8\cdot2=16.$ Волков пришло на 1 пару меньше, то есть $16-2=14.$ Значит, зайцев и ежей пришло $(16+14):2=15.$ Пусть ежей пришло $x,$ тогда зайцев пришло $x+3$ и $x+x+3=15,\ 2x+3=15,\ x=6.$\\
305. Если биографий и сказок вместе $x,$ то фантастики и энциклопедий вместе $2x$ и $x+2x=216,\ 3x=216,\ x=216:3=72.$ Если фантастики $y$ книг, то энциклопедий $3y$ и $y+3y=2\cdot72,\ 4y=144,\ y=36.$ Значит, фантастики 36 книг, а энциклопедий --- $3\cdot36=108$ книг. Если биографий $z$ книг, то книг со сказками $z+6$ и $z+z+6=72,\ 2z+6=72,\ 2z=66,\ z=33.$ Значит, биографий 33 книги, а книг со сказками --- $33+6=39.$\\
306. Если бы все группы были бы группами богатырей, то понадобилось бы $20\cdot3=60$ значков, что на $96-60=36$ значков меньше, чем было выдано. При замене одной группы богатырей на группу гномов становится нужно на $7-3=4$ значка больше, значит необходимо сделать $36:4=9$ замен, и групп гномов было 9, а групп богатырей --- $20-9=11.$\\
307. Пусть в 2020 году Кирилл потратил $2x$ минут на решение задач сидя, $2y$ минут на решение задач бегом и $2z$ минут на решение задач на велосипеде. Тогда
$2x+2y+2z=3$ч, откуда $x+y+z=180:2=90$мин. В 2021 году он потратил $x+2y+2z=150$мин, значит $y+z=150-90=60$мин, $x=90-60=30$мин. В 2022 году он потратил $2x+y+2z=165$мин, значит $z=165-90-30=45$мин, а $y=60-45=15$мин. Тогда в 2023 году он потратит $2x+2y+z=2\cdot30+2\cdot15+45=135$мин или 2 часа 15 минут.\\
308. Пусть в 2020 году Кирилл потратил $2x$ минут на решение задач сидя, $2y$ минут на решение задач бегом и $2z$ минут на решение задач на велосипеде. Тогда
$2x+2y+2z=3$ч, откуда $x+y+z=180:2=90$мин. В 2021 году он потратил $x+2y+2z=140$мин, значит $y+z=140-90=50$мин, $x=90-50=40$мин. В 2022 году он потратил $2x+y+2z=155$мин, значит $z=155-90-40=25$мин, а $y=50-25=25$мин. Тогда в 2023 году он потратит $2x+2y+z=2\cdot40+2\cdot25+25=155$мин или 2 часа 35 минут.\\
309. Ни одно из получившихся чисел не делится на 5 (а значит, и на 25), поэтому ошибся Вася. Колино число точно больше Петиного, значит он мог получить только 2016 или 2028. Но 2016 не делится на 13, значит Коля получил 2028 и исходное число было равно $2028:13=156.$ Поэтому Вася должен был получить $156\cdot25=3900.$\\
310. Ни одно из получившихся чисел не делится на 5 (а значит, и на 25), поэтому ошибся Вася. Петино число точно больше Колиного, значит он мог получить только 2184 или 2197. Если Петино число равно 2184, то изначальное число равно $2184:13=168$ и Коля должен был получить $168\cdot12=2016,$ а такого числа нет. Значит, Петя получил $2197$ и исходное число равно $2197:169.$ Тогда Вася должен был получить $169\cdot25=4225.$\\
311. Пусть первый отрезок имеет длину $2a,$ второй $2b,$ третий $2c$ и четвёртый $2d.$ Тогда $2a+2b+2c+2d=48$см, $a+b+c+d=24$см. Кроме того, $a+2b+c=16$см, $b+2c+d=14$см, поэтому $(a+b)-(c+d)=2$см. Значит, $2(a+b)=24+2=26$см, $a+b=13$см, $b+c=16-13=3$см. Сумма длин 2-го и 3-го отрезков равна $2(b+c)=2\cdot3=6$см.\\
312. Пусть первый отрезок имеет длину $2a,$ второй $2b,$ третий $2c$ и четвёртый $2d.$ Тогда $2a+2b+2c+2d=56$см, $a+b+c+d=28$см. Кроме того, $a+2b+c=18$см, $b+2c+d=24$см, поэтому $(c+d)-(a+b)=6$см. Значит, $2(c+d)=28+6=34$см, $c+d=17$см, $b+c=24-17=7$см. Сумма длин 2-го и 3-го отрезков равна $2(b+c)=2\cdot7=14$см.\\
313. За первые 6 дней Петька изготовил $6\cdot4=24$ детали. Василий Иванович каждый день делает на $7-4=3$ детали больше, значит он сравняется с Петькой через $24:3=8$ дней (именно столько дней они и работали вместе). За это время они вместе сделали $8\cdot7\cdot2=112$ деталей.\\
314. За первые 6 дней Петька изготовил $6\cdot10=60$ деталей. Василий Иванович каждый день делает на $13-10=3$ детали больше, значит он сравняется с Петькой через $60:3=20$ дней (именно столько дней они и работали вместе). За это время они вместе сделали $20\cdot13\cdot2=520$ деталей.\\
315. Пусть обычных пассажиров было $x,$ тогда курсантов было $2x,$ а студентов --- $3x.$ Студентов было на 14 больше, чем обычных пассажиров, поэтому $3x-x=14,\ 2x=14,\ x=7.$ Значит, обычных пассажиров было 7, курсантов $2\cdot7=14,$ студентов $3\cdot7=21.$ Тогда обычных пассажиров и курсантов было $7+14=21,$ значит лицеистов и студентов было $21\cdot3=63,$ поэтому лицеистов было $63-21=42$ человека.\\
316. Пусть всем изначально выдали по $x$ задач, тогда $5\cdot(x-3)=2x,\ 5x-15=2x,\ 3x=15,\ x=5,$ то есть изначально каждый участник получил по 5 задач. Если первую задачу участник решил за $y$ минут, то вторую он решил за $y+2$ минуты, третью за $y+4,$ поэтому $y+y+2+y+4=15,\ 3y+6=15,\ 3y=9,\ y=3$мин. Тогда всего на 5 задач он потратит $3+5+7+9+11=35$ минут.\\
317. Пусть карточек с нечётными номерами вытащили $x,$ тогда карточек с чётными номерами вытащили $x+13$ и $x+x+13=41,\ 2x=28,\ x=14.$\\
318. Пусть домов с чётными номерами было $x,$ тогда домов с нечётными номерами было $x+15$ и $x+x+15=43,\ 2x=28,\ x=14.$\\
319. Пять четвертей числа равны 4000, значит одна четверть числа равна $4000:5=800,$ значит число равно $800\cdot4=3200.$ Если прибавить к нему его пятую часть, получится $3200+3200:5=3840.$\\
320. Четыре трети числа равны 6000, значит одна треть числа равна $6000:4=1500,$ значит число равно $1500\cdot3=4500.$ Если прибавить к нему его четверть, получится $4500+4500:4=5625.$\\
321. Если бы привезли только задачники, в них было бы $50\cdot23=1150$ картинок, то есть $1304-1150=154$ картинок не хватает. Если заменить задачник учебником по математике, станет на $30-23=7$ картинок больше. Значит, необходимо сделать $154:7=22$ замены, и задачников останется $50-22=28.$\\
322. Если бы привезли только учебники, в них было бы $50\cdot26=1300$ стихотворений, то есть $1552-1300=252$ стихотворений не хватает. Если заменить учебник хрестоматией, станет на $40-26=14$ хрестоматий больше. Значит, необходимо сделать $252:14=18$ замен.\\
323. Сложим две покупки, получится, что набор из 8 карандашей, 8 линеек и 8 тетрадей стоит $132+204=336$ рублей, тогда набор, в котором всех предметов по 2, стоит $336:4=84$ рубля.\\
324. Сложим две покупки, получится, что набор из 9 огурцов, 9 помидоров и 9 свёкл стоит $99+207=306$ рублей, тогда набор, в котором всех предметов по 3, стоит $306:4=102$ рубля.\\
325. В обоих наборах яблок в 2 раза больше, чем груш и апельсинов вместе, поэтому всего их куплено $260\cdot2=520.$\\
326. В обоих наборах ручек в 2 раза больше, чем точилок и линеек вместе, поэтому всего их куплено $270\cdot2=540.$\\
327. Каждый выпил $(5+7):3=4$ стакана компота, значит Лёша продал $5-4=1$ стакан, а Олег --- $7-4=3$ стакана. Поэтому Олегу должно достаться в 3 раза больше денег, чем Лёше. Если Лёше досталось $x$ рублей, то Олегу досталось $3x$ и $x+3x=24,\ 4x=24,\ x=24:4=6,\ 3x=3\cdot6=18$ рублей.\\
328. Каждый съел $(6+9):3=5$ пирожков, значит Лена продала $9-5=4$ пирожка, а Оля --- $6-5=1$ пирожок. Поэтому Лене должно достаться в 4 раза больше денег, чем Оле. Если Оле досталось $x$ рублей, то Лене досталось $4x$ и $x+4x=45,\ 5x=45,\ x=45:5=9,\ 4x=4\cdot9=36$ рублей.\\
329. Перед покупкой третьего покупателя в магазине оставалось $\left(3+\cfrac{1}{3}\right):2\cdot3=5$ чашек. Значит, изначально в магазине было $\left(5+\cfrac{1}{3}\right):2\cdot3=8$ чашек.\\
330. Перед покупкой третьего покупателя в магазине оставалось $\left(7+\cfrac{1}{3}\right):2\cdot3=11$ блюдец. Значит, изначально в магазине было $\left(11+\cfrac{1}{3}\right):2\cdot3=17$ блюдец.\\
331. Длина общей границы равна $(18000-13500):150=30$ метров.\\
332. Длина общей границы равна $(22400-16000):160=40$ метров.\\
333. Один человек без присмотра начальника потратил бы на работу в $8\cdot2=16$ раз больше времени. Значит, за такое же время с работой справились бы 16 человек без присмотра. Так как было всего 12 человек $16-12=4$ из них должны были быть во второй группе, которая копает быстрее ($8+4\cdot2=16$). Тогда первая группа из $12-4=8$ человек прокопала $8/16,$ то есть ровно половину траншеи, $1000:2=500$ метров.\\
334. Один человек без присмотра начальника потратил бы на работу в $6\cdot2=12$ раз больше времени. Значит, за такое же время с работой справились бы 12 человек без присмотра. Так как было всего 10 человек $12-10=2$ из них должны были быть во второй группе, которая копает быстрее ($8+2\cdot2=12$). Тогда первая группа из $10-2=8$ человек прокопала $8/12,$ то есть две трети траншеи, $900:3\cdot2=600$ метров.\\
335. Переходы между классами на общее количество учащихся в трёх классах не влияют, значит во всех трёх классах вместе теперь учится $29+32+30-3=88$ человек.\\
336. Переходы между классами на общее количество учащихся в трёх классах не влияют, значит во всех трёх классах вместе теперь учится $28+31+32-3=88$ человек.\\
337. Всего необходимо сыграть $5+4+3+2+1=15$ партий. При этом по 2 партии можно играть одновременно, поэтому сначала можно сыграть 14 партий одновременно на двух досках (за то время, которое занимают 7 партий), а потом сыграть последнюю. Значит, 8 партий занимают 6 часов, то есть на одну партию уходит $6\text{ ч}:8= 360\text{ мин}:8=45$мин. Таким образом, на то, чтобы сыграть все партии на одном поле, потребуется $15\cdot45=675$мин или 11 ч 15 мин.\\
338. Всего необходимо сыграть $5+4+3+2+1=15$ партий. При этом по 2 партии можно играть одновременно, поэтому сначала можно сыграть 14 партий одновременно на двух досках (за то время, которое занимают 7 партий), а потом сыграть последнюю. Значит, 8 партий занимают 10 часов, то есть на одну партию уходит $10\text{ ч}:8= 600\text{ мин}:8=75$мин. Таким образом, на то, чтобы сыграть все партии на одном поле, потребуется $15\cdot75=1125$мин или 18 ч 45 мин.
\newpage
