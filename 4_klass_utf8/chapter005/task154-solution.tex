154. Первый вносит половину суммы, вносимой остальными, то есть треть от стоимости лодки. Аналогично второй вносит четверть, а третий пятую часть стоимости лодки.
Разделим стоимость лодки на $3\cdot4\cdot5=60$ частей. Тогда первый вносит $60:3=20$ частей, второй $60:4=15$ частей, а третий --- $60:5=12$ частей. Таким образом, четвёртый вносит $60-20-15-12=13$ оставшихся частей, составляющих 130 рублей. Тогда одна часть равна $130:13=10$ рублей, а вся лодка стоит $60\cdot10=600$ рублей.\\
