\section{Раздел 5: стандартные задачи}
1. В году 150 учебных дней. В каждом из четырёх классов решали по 4 задачи за один урок математики. Сколько всего задач было решено за год, если по субботам уроков математики не было, а во все остальные дни их было ровно по одному? (суббота также является учебным днём)\\
2. В году 150 учебных дней. В каждом из пяти классов решали по 3 задачи за один урок математики. Сколько всего задач было решено за год, если по субботам уроков математики не было, а во все остальные дни их было ровно по одному? (суббота также является учебным днём)\\
3. Доктор Айболит раздал четырём заболевшим зверям 2010 чудодейственных таблеток. Носорог получил на одну больше, чем крокодил, бегемот на одну больше, чем носорог, а слон --- на одну больше, чем бегемот. Сколько таблеток придётся съесть слону?\\
4. Доктор Айболит раздал четырём заболевшим зверям 2014 чудодейственных таблеток. Волк получил на одну меньше, чем лиса, медведь на одну меньше, чем волк, а рысь --- на одну меньше, чем медведь. Сколько таблеток придётся съесть рыси?\\
5. Какое число надо умножить на 10, чтобы результат был таким же, как при прибавлении к этому числу 18?\\
6. Какое число надо умножить на 9, чтобы результат был таким же, как при прибавлении к этому числу 24?\\
7. Как изменится разность, если уменьшаемое уменьшить на 8, а вычитаемое увеличить на 5?\\
8. Как изменится разность, если уменьшаемое увеличить на 11, а вычитаемое уменьшить на 5?\\
9. Хозяйка развела кур и кроликов. Всего у них 35 голов и 94 ноги. Сколько у хозяйки кур и сколько кроликов?\\
10. На ферме имеются гуси и коровы общим числом 30, а общее количество ног у них равно 86. Сколько на ферме гусей?\\
11. На ферме имеются гуси и коровы общим числом 40, а общее количество ног у них равно 98. Сколько на ферме коров?\\
12. Для похода 46 лицеистов приготовили шестиместные и четырёхместные лодки. Сколько было тех и других лодок, если все ребята разместились в 10 лодках и свободных мест не осталось?\\
13. Задумали число, увеличили его в 4 раза, а результат уменьшили в 6 раз. Получили 36. Какое число задумали?\\
14. Задумали число, увеличили его в 8 раз, а результат уменьшили в 12 раз. Получили 24. Какое число задумали?\\
15. В купейном вагоне 36 мест, по 4 в каждом купе. Укажите номер купе, в котором расположено место №23.\\
16. Из 23 школьников 17 изучают английский язык, а 11 --- немецкий язык. Сколько школьников изучают два языка, если известно, что каждый изучает хотя бы один язык?\\
17. Теплоход рассчитан на 750 пассажиров и 25 членов команды. Каждая спасательная шлюпка может вместить 70 человек. Какое наименьшее количество шлюпок должно быть на теплоходе, чтобы в случае необходимости в них можно было разместить всех пассажиров и членов команды?\\
18. В летнем лагере 218 детей и 26 воспитателей. В автобус помещается не более 45 пассажиров. Сколько автобусов потребуется, чтобы перевезти всех из лагеря в город?\\
19. В 4 классе 35 учеников. В течение учебного дня 20 человек питаются бутербродами, 11 человек посещают кафе, 10 человек голодают. Сколько человек съедает бутерброды, сидя в кафе?\\
20. В пиратской шайке 50 человек. Из них 32 одноруких, 29 одноглазых, 15 --- одноруких с одним глазом. Сколько здоровых пиратов в шайке?\\
21. В автобусе было несколько пассажиров. На первой остановке вышло 11 и вошло 6, а на второй вышло 7 и вошло 15 пассажиров. Сколько пассажиров было в автобусе до первой остановки, если после второй остановки автобуса их стало 40?\\
22. В автобусе было несколько пассажиров. На первой остановке вышло 7 и вошло 12, а на второй вышло 11 и вошло 8 пассажиров. Сколько пассажиров было в автобусе до первой остановки, если после второй остановки автобуса их стало 40?\\
23. Во время туристского слёта на питание 100 туристов израсходовали 32 кг мяса, что оказалось в 8 раз больше, чем масла, и в 2 раза меньше, чем хлеба. Сколько граммов каждого продукта в отдельности пришлось на питание одного туриста?\\
24. Во время школьного праздника на угощение 100 пятиклассников приготовили 24 кг мороженого, что оказалось в 8 раз больше, чем чая, и в 2 раза меньше, чем карамели. Сколько граммов каждого продукта в отдельности пришлось на угощение одного школьника?\\
25. В пачке 500 листов бумаги. За неделю в школе расходуется 1300 листов. Какое наименьшее количество пачек бумаги нужно купить в школу на 7 недель?\\
26. В пачке 500 листов бумаги. За неделю в школе расходуется 1100 листов. Какое наименьшее количество пачек бумаги нужно купить в школу на 9 недель?\\
27. Некоторое число таково, что прибавить к нему 4 --- то же самое, что умножить его на 3. Тогда умножить его на 6 --- это то же самое, что прибавить к нему...\\
28. Некоторое число таково, что прибавить к нему 6 --- то же самое, что умножить его на 3. Тогда умножить его на 5 --- это то же самое, что прибавить к нему...\\
29. Запишите шесть чисел подряд, если первое число равно 2, второе 3, а каждое следующее равно произведению двух предыдущих, увеличенному на 2.\\
30. Запишите шесть чисел подряд, если первое число равно 2, второе 5, а каждое следующее равно произведению двух предыдущих, уменьшенному на 3.\\
31. Аня, Боря и Витя обсуждали, у кого сколько карандашей. Петров с Сидоровым посовещались, и после этого Витя сказал: <<Если Иванова отдаст Петрову один карандаш, то у нас троих будет их поровну>>. На что Боря заметил: <<Если Аня отдаст все свои карандаши Вите, то у него будет ровно 10 карандашей>>. Определите фамилию и количество карандашей каждого из ребят.\\
32. Лёня, Максим и Нина обсуждали, у кого сколько ручек. Иванов с Сидоровым посовещались, и после этого Лёня сказал: <<Если Петрова отдаст Сидорову три ручки, то у нас троих будет их поровну>>. На что Максим заметил: <<Если Нина отдаст все свои ручки Лёне, то у него будет ровно 20 ручек>>. Определите фамилию и количество ручек каждого из ребят.\\
33. В лицее №239 два здания. Каждый день в лицее расходуется 260 листов бумаги, причём в первом здании на 6 листов больше, чем во втором. Сколько пачек бумаги надо закупить на 12 дней, если переносить открытые пачки и листы по улице нельзя, а в каждой пачке 500 листов бумаги?\\
34. В лицее №239 два здания. Каждый день в лицее расходуется 360 листов бумаги, причём в первом здании на 6 листов больше, чем во втором. Сколько пачек бумаги надо закупить на 12 дней, если переносить открытые пачки и листы по улице нельзя, а в каждой пачке 500 листов бумаги?\\
35. На столе лежат фрукты. Известно, что яблоко и апельсин вместе весят 167 граммов, апельсин и груша весят вместе 176 граммов, мандарин и апельсин весят вместе 200 граммов, а груша и яблоко весят вместе 159 граммов. Сколько весят вместе взятые мандарин, апельсин и груша?\\
36. На столе лежат фрукты. Известно, что апельсин и груша вместе весят 157 граммов, груша и мандарин весят вместе 175 граммов, яблоко и груша весят вместе 300 граммов, а мандарин и апельсин весят вместе 168 граммов. Сколько весят вместе взятые яблоко, мандарин и груша?\\
37. На дне рождения каждый мальчик съел 4 конфеты, 4 котлеты и 4 козинака, а каждая девочка съела 10 конфет, 3 котлеты и 24 козинака. Всего было съедено 714 конфет и 371 котлета Сколько было съедено козинаков?\\
38. На празднике каждый мальчик съел 5 мандаринов, 5 манго и 5 марципанов, а каждая девочка съела 11 мандаринов, 8 манго и 17 марципанов. Всего было съедено 813 мандаринов и 669 манго. Сколько было съедено марципанов?\\
39. В городах А и Б проходит олимпиада по математике. Известно, что в обоих городах олимпиада началась в 10 часов утра по местному времени и продолжалась одинаковое время. Оказалось, что в городе А олимпиада закончилась на час позже, чем началась в Б. При этом в Б она закончилась на 9 часов позже, чем началась в А. Сколько времени длится олимпиада?\\
40. В городах А и Б проходит олимпиада по математике. Известно, что в обоих городах олимпиада началась в 11 часов утра по местному времени и продолжалась одинаковое время. Оказалось, что в городе А олимпиада закончилась на час раньше, чем началась в Б. При этом в Б она закончилась на 7 часов позже, чем началась в А. Сколько времени длится олимпиада?\\
41. На складе лежат 30000 блоков массой по 33 кг каждый, и 3456 кирпичей по 2 кг каждый. Какое наименьшее количество машин грузоподъёмностью одна тонна надо, чтобы увезти это со склада?\\
42. На складе лежат 31000 блоков массой по 32 кг каждый, и 2345 кирпичей по 2 кг каждый. Какое наименьшее количество машин грузоподъёмностью одна тонна надо, чтобы увезти это со склада?\\
43. В трёх ящиках лежали 1001, 2999 и 239 гвоздиков. Мальчик Вася стал перекладывать гвоздики из ящика в ящик, при этом ничего не теряя. Когда Вася закончил, его мама выяснила, что теперь в первом ящике 4 гвоздика, а во втором 1007 гвоздиков. Сколько гвоздиков в третьем ящике?\\
44. В трёх ящиках лежали 2002, 1999 и 239 винтиков. Мальчик Вася стал перекладывать винтики из ящика в ящик, при этом ничего не теряя. Когда Вася закончил, его мама выяснила, что теперь в первом ящике 7 винтиков, а во третьем 1003 винтика. Сколько винтиков во втором ящике?\\
45. В течение недели ученик каждый день решал на две задачи больше, чем в предыдущий день, при этом в воскресенье он решил втрое больше задач, чем в понедельник. Сколько задач он решил в пятницу?\\
46. В течение недели ученик каждый день решал на две задачи больше, чем в предыдущий день, при этом в воскресенье он решил вчетверо больше задач, чем в понедельник. Сколько задач он решил в пятницу?\\
47. На дне озера бьёт родник. Стадо из 163 слонов могло бы выпить озеро за 1 день, а стадо из 33 слонов --- за 5 дней. За сколько дней выпьет озеро один слон?\\
48. На дне озера бьёт родник. Стадо из 158 слонов могло бы выпить озеро за 1 день, а стадо из 23 слонов --- за 7 дней. За сколько дней выпьет озеро один слон?\\
49. Дима и Максим хотят вместе заплатить 5600 рублей, разделив затраты поровну. Дима дал Максиму взаймы 10000 рублей, Максим заплатил 3200 рублей, а Дима --- оставшиеся деньги. Сколько денег должен дать Максим Диме, чтобы никто никому не был должен?\\
50. Вова и Максим хотят вместе заплатить 7800 рублей, разделив затраты поровну. Вова дал Максиму взаймы 3800 рублей, Вова заплатил 3200 рублей, а Максим --- оставшиеся деньги. Сколько денег должен дать Максим Вове, чтобы никто никому не был должен?\\
51. На дереве сидело несколько красных и несколько зелёных хамелеонов, а также несколько красных и несколько зелёных попугаев. Хамелеонов было 17, красных животных --- тоже 17, а зелёных животных было 14. После того как один из хамелеонов перекрасился из красного цвета в зелёный, красных попугаев и красных хамелеонов стало поровну. Сколько зелёных попугаев было на дереве?\\
52. На дереве сидело несколько красных и несколько зелёных хамелеонов, а также несколько красных и несколько зелёных попугаев. Хамелеонов было 15, зелёных животных --- тоже 15, а красных животных было 12. После того как один из хамелеонов перекрасился из красного цвета в зелёный, зелёных попугаев и зелёных хамелеонов стало поровну. Сколько красных попугаев было на дереве?\\
53. Гулливер погнался за лилипутом, когда расстояние между ними было равно 8 шагам Гулливера. Пока Гулливер делает 1 шаг, лилипут пробегает 7 шагов, но 1 шаг Гулливера равен 11 шагам лилипута. Сколько шагов пробежал лилипут до момента, когда Гулливер его догнал?\\
54. Гулливер погнался за лилипутом, когда расстояние между ними было равно 6 шагам Гулливера. Пока Гулливер делает 1 шаг, лилипут пробегает 7 шагов, но 1 шаг Гулливера равен 10 шагам лилипута. Сколько шагов пробежал лилипут до момента, когда Гулливер его догнал?\\
55. У Ани, Максима и Димы вместе 1410 монет, у Ани монет в 4 раза больше, чем у Максима, и на 3 монеты больше, чем у Димы. Сколько монет у Ани?\\
56. У Ани, Максима и Димы вместе 1401 монета, у Ани монет в 3 раза меньше, чем у Максима, и на 4 монеты больше, чем у Димы. Сколько монет у Ани?\\
57. Саша складывал два числа на калькуляторе, но, набирая второе число, случайно нажал в конце лишний ноль. Поэтому вместо 1222 он получил 5551. Какие числа хотел сложить Саша?\\
58. Паша складывал два числа на калькуляторе, но, набирая второе число, случайно нажал в конце лишний ноль. Поэтому вместо 1331 он получил 6641. Какие числа хотел сложить Паша?\\
59. Шесть мальчиков и четыре девочки организовали турнир в крестики-нолики. Каждый участник сыграл с каждым по одной партии. За выигрыш присуждали 2 очка, за ничью --- 1 очко, за проигрыш --- 0 очков. Девочки вместе набрали 40 очков. На сколько игр, в которых выиграла девочка у мальчика, больше, чем игр, в которых выиграл мальчик у девочки?\\
60. Шесть мальчиков и четыре девочки организовали турнир в крестики-нолики. Каждый участник сыграл с каждым по одной партии. За выигрыш присуждали 2 очка, за ничью --- 1 очко, за проигрыш --- 0 очков. Мальчики вместе набрали 40 очков. На сколько игр, в которых выиграла девочка у мальчика, больше, чем игр, в которых выиграл мальчик у девочки?\\
61. Возраст нескольких друзей составляет в сумме 62 года. Через 4 года он будет составлять 90 лет. Сколько этих друзей?\\
62. Для забора нужны доски длиной 75 сантиметров в количестве 112 штук. В магазине продаются доски длиной 4 метра. Сколько досок надо купить, чтобы построить забор?\\
63. Для забора нужны доски длиной 80 сантиметров в количестве 120 штук. В магазине продаются доски длиной 6 метра. Сколько досок надо купить, чтобы построить забор?\\
64. Света, Маша и Оля разделили между собой 80 конфет. Света заметила, что если она отдаст все свои конфеты Маше, то у Маши и Оли станет поровну конфет, а если она отдаст все свои конфеты Оле, то у Оли станет в четыре раза больше конфет, чем у Маши. Сколько конфет было у Светы?\\
65. Света, Маша и Оля разделили между собой 60 конфет. Света заметила, что если она отдаст все свои конфеты Маше, то у Маши и Оли станет поровну конфет, а если она отдаст все свои конфеты Оле, то у Оли станет в пять раз больше конфет, чем у Маши. Сколько конфет было у Светы?\\
66. Лифт в доме ездит с постоянной скоростью, а на каждом этаже, куда вызван, стоит одинаковое время. Время поездки в лифте считается от момента отправления с начального этажа до момента прибытия на конечный. Петя ехал вниз с 13 этажа, на 11 этаже к нему подсел Коля, на 7 этаже Таня, а на 5 этаже Витя. На первом этаже все вышли. Петя ехал 57 секунд, а Таня 25 секунд. Сколько секунд ехал Коля?\\
67. Лифт в доме ездит с постоянной скоростью, а на каждом этаже, куда вызван, стоит одинаковое время. Время поездки в лифте считается от момента отправления с начального этажа до момента прибытия на конечный. Петя ехал вниз с 11 этажа, на 9 этаже к нему подсел Коля, на 6 этаже Таня, а на 3 этаже Витя. На первом этаже все вышли. Петя ехал 54 секунды, а Таня 23 секунды. Сколько секунд ехал Коля?\\
68. У Олега есть 120 рублей. Тетрадь стоит 7 рублей 50 копеек, а ручка 12 рублей. Олегу обязательно нужно купить 3 ручки. Какое наибольшее количество тетрадей сможет купить Олег?\\
69. У Оли есть 140 рублей. Блокнот стоит 9 рублей 50 копеек, а линейка 14 рублей. Оле обязательно нужно купить 4 линейки. Какое наибольшее количество блокнотов сможет купить Оля?\\
70. У бедуина есть двугорбые и одногорбые верблюды. У всех верблюдов вместе 14 горбов и 44 ноги. Сколько одногорбых верблюдов у бедуина?\\
71. В цирке есть двугорбые и одногорбые верблюды. У всех верблюдов вместе 19 горбов и 48 ног. Сколько двугорбых верблюдов в цирке?\\
72. Кошка, собака и енот вместе весят 25 кг. Кошка и собака вместе весят 17 кг, енот и собака вместе весят 19 кг. Сколько весит собака?\\
73. Кролик, индюк и петух вместе весят 30 кг. Кролик и индюк вместе весят 22 кг, петух и кролик вместе весят 17 кг. Сколько весит кролик?\\
74. На фундаменте строят стену из кирпича. Когда четверть кирпичной кладки была завершена, высота её вместе с фундаментом была 1 м, когда половина всей кладки была закончена, высота была 160 см. Какой будет высота стены (с учётом фундамента), законченной на 2/3? Ответ выразите в сантиметрах.\\
75. В кадке, стоящей на полу, живёт пальма. Если привязать ленточку на высоте четверти пальмы от уровня кадки, то ленточка окажется на высоте 1 м 20 см от пола; если привязать ленточку в середине пальмы, то ленточка будет на высоте 190 см от пола. На какой высоте от пола будет ленточка, если она отмечает 2/5 высоты пальмы от уровня кадки? Ответ выразите в сантиметрах.\\
76. В магазине продаются свечки в виде цифр. Одинаковые цифры имеют одинаковую цену, а разные --- разную. Число 4385 стоит 36 динаров; число 27 стоит 17 динаров; число 56165 стоит 47 динаров; число 782 стоит 29 динаров; число 586 стоит 33 динара. Сколько стоит число 745231?\\
77. В магазине продаются свечки в виде цифр. Одинаковые цифры имеют одинаковую цену, а разные --- разную. Число 3276 стоит 36 гульденов; число 51 стоит 14 гульденов; число 49894 стоит 22 гульдена; число 715 стоит 22 гульдена; число 794 стоит 17 гульденов. Сколько стоит число 623158?\\
78. Балда и Емеля едят пирожки. Емеля может съесть 15 пирожков за полчаса, а Балда --- 15 пирожков за 20 минут. Чудо-печка каждые 2 минуты выпекает 3 пирожка. Печка пекла пирожки 2 часа. Успеют ли Емеля и Балда вместе съесть все пирожки за 2 часа?\\
79. Малыш и Карлсон едят плюшки. Малыш может съесть 4 плюшки за 15 минут, а Карлсон --- 16 плюшек за полчаса. Фрекен Бок каждые 4 минуты выпекает 3 плюшки. Она пекла плюшки 2 часа. Успеют ли Малыш и Карлсон вместе съесть все плюшки за 2 часа?\\
80. Кристофер Робин рассказывал Пятачку о диковинных животных, на которых ему приходилось охотиться: {\it <<Слонопотам в 5 раз тяжелее верблюмота, а кошкалот на 16 кг легче антиконды. При этом кошкалот и антиконда вместе весят столько же, сколько вместе слонопотам и верблюмот, хотя верблюмот на центнер легче слонопотама>>.} Сколько весит каждое из диковинных животных? Расположите их массы в порядке убывания.\\
81. Барон Мюнхгаузен рассказывал друзьям об удивительных птицах, обитающих за дальними морями:{\it <<Тётел в 6 раз легче мамугая, а пеликот на 34 кг тяжелее хвалибри. При этом тётел и мамугай вместе весят столько же, сколько вместе пеликот и хвалибри, хотя тётел на полцентнера легче мамугая>>.} Сколько весит каждая из удивительных птиц? Расположите в ответе их массы в порядке возрастания.\\
82. Сумма уменьшаемого, вычитаемого и разности равна 36. Найдите уменьшаемое.\\
83. У Зои было на 200 фантиков больше, чем у Оли. Сколько фантиков Зоя должна дать Оле, чтобы у них стало одинаковое количество фантиков?\\
84. На первой остановке в автобус вошли 8 человек. А на каждой следующей выходили 5 человек, а входили 7 человек. Сколько пассажиров было в  автобусе между третьей и четвёртой остановкой?\\
85. На острове 486 домов. В каждом доме живёт по 5 зайцев, у каждого зайца по 3 зайчика. Сколько зайцев и зайчиков живёт в этих домах? Напишите только выражение, вычислять его значение НЕ надо.\\
86. В парке 4008 насекомых. Половина из них пчёлы, одна треть --- мухи, четвёртая часть остальных ---  гусеницы. Сколько в парке гусениц?\\
87. Жители планеты Нрутас имеют либо 1 голову и 4 руки, либо 2 головы и 3 руки. Делегация с этой планеты на Землю имеет на всех 11 голов и 29 рук. Сколько всего нрутасианцев в делегации?\\
88. Жители планеты Нутпен имеют либо 1 голову и 3 ноги, либо 3 головы и 2 ноги. Делегация с этой планеты на Марс имеет на всех 11 голов и 19 ног. Сколько всего нутпенсианцев в делегации?\\
89. Олег хочет купить другу Мише подарок на день рождения. Он знает, что Миша хочет синий грузовик. В магазине продаётся 19 машинок, 12 из которых синие, 11 --- грузовики. Из какого минимального количества синих грузовиков Олег точно может выбрать подходящий подарок?\\
90. Алиса хочет купить подруге Тане подарок на день рождения. Она знает, что Таня хочет книгу сказок с картинками. В магазине продаётся 23 книги, 10 из которых со сказками, 17 --- с картинками. Из какого минимального количества книг сказок с картинками Алиса точно может выбрать подходящий подарок?\\
91. Чайка ловила рыбу --- корюшку, ряпушку, плотву и уклейку. За неделю она поймала 256 рыбок, причём корюшки и ряпушки вместе втрое меньше, чем плотвы и уклейки вместе. Корюшки она поймала на 12 рыбок больше, чем ряпушки, а уклеек в 7 раз меньше, чем плотвы. Выясните, каких рыбок --- корюшки или уклеек чайка поймала больше и на сколько?\\
92. Дракон таскал в своё логово драгоценные камни --- рубины, алмазы, изумруды и сапфиры. Рубинов и алмазов вместе оказалось вдвое больше, чем изумрудов и сапфиров вместе. Всего камней было 351. Алмазов было в 5 раз меньше, чем рубинов, а сапфиров на 7 больше, чем изумрудов. Выясните, чего было больше --- алмазов или изумрудов и на сколько?\\
93. Четыре подруги (Катя, Лена, Маша, Нина) решили сварить компот из 36 груш. Катя купила 11 груш, Лена --- 10 груш, Маша --- 15 груш (все груши стоили одинаково). Нина принесла 216 рублей. Сколько рублей кому из девочек она должна отдать?\\
94. Четверо друзей (Антон, Вадим, Борис, Гена) решили сварить компот из 32 яблок. Антон купил 9 яблок, Борис --- 10 яблок, Вадим --- 13 яблок (все яблоки стоили одинаково). Гена принёс 128 рублей. Сколько рублей кому из мальчиков он должен отдать?\\
95. Собирались раскрасить 27 тарелок. Четыре тарелки разбили, пока везли. Все остальные были раскрашены красным, синим или обоими цветами сразу. Синим раскрасили 18 тарелок, красным --- 21 тарелку. Сколько тарелок раскрасили обоими цветами одновременно?\\
96. Рыцарь и 5 драконов весят столько же, сколько дракон и 8 рыцарей. Сколько рыцарей потребуется, чтобы уравновесить 8 драконов?\\
97. Диван и собачонка вместе весят 46 кг, чемодан и корзина --- 17 кг, собачонка и корзина --- 11 кг, диван и чемодан --- 52 кг. Сколько весят вместе диван, чемодан, корзина и собачонка?\\
98. Четвероклассник Коля Рисовалкин коллекционирует карандаши. У него в коллекции 68 --- синих, зелёных, жёлтых и красных. Синих карандашей в 14 раз больше, чем жёлтых, и на 6 штук меньше, чем зелёных. Количество красных карандашей составляет седьмую часть синих. Сколько карандашей какого цвета есть у Коли?\\
99. На дереве сидело 18 попугаев. Из них 10 --- не зелёные, а 9 --- крикливые. Когда прилетел ещё один попугай, зелёных оказалось столько же, сколько крикливых. Был ли прилетевший попугай зелёным? Крикливым?\\
100. Средней ценой купленного товара называют количество потраченных на весь товар денег, делённое на количество купленного товара. Даша купила на 480 руб. тетрадок. Половину денег она потратила на тетрадки по 60 руб. за штуку, а другую половину на тетради, цена которых вдвое меньше. Какова средняя цена тетрадок, купленных Дашей?\\
101. У числа 2019 перемножьте все ненулевые цифры и прибавьте 5 (например, для числа 10234 эта операция даст $1\cdot2\cdot3\cdot4+5=29).$ С получившимся числом проделайте те же операции. И так до тех пор, пока не получится однозначное число. Какое это будет число?\\
102. У числа 2017 перемножьте все ненулевые цифры и прибавьте 5 (например, для числа 10234 эта операция даст $1\cdot2\cdot3\cdot4+5=29).$ С получившимся числом проделайте те же операции. И так до тех пор, пока не получится однозначное число. Какое это будет число?\\
103. Карлсон ест торт со скоростью 200 граммов в минуту,а Малыш --- 25 граммов в минуту. Карлсон начал есть торт. Через минуту к нему присоединился Малыш, а ещё через четыре минуты торт закончился. Сколько весил этот торт?\\
104. Винни Пух ест мёд со скоростью 300 граммов в минуту, а Пятачок --- 20 граммов в минуту. Винни Пух и Пятачок начали есть горшочек с мёдом вместе. Через пять минут Пятачка отозвал Кролик, и Винни Пух продолжил обедать один. А ещё через минуту мёд закончился. Сколько мёда было в горшочке?\\
105. На автобусную экскурсию захотели поехать 174 мальчика, 148 девочек и 14 учителей. Каждый автобус вмещает 45 пассажиров. Сколько автобусов потребуется, чтобы смогли поехать все желающие?\\
106. На автобусную экскурсию захотели поехать 136 мальчика, 194 девочек и 15 учителей. Каждый автобус вмещает 45 пассажиров. Сколько автобусов потребуется, чтобы смогли поехать все желающие?\\
107. На ферме содержатся козы, гуси и лошади. Оказалось, что у них голов столько же, сколько крыльев, а рогов на восемь меньше, чем голов. Сколько лошадей на ферме?\\
108. На ферме содержатся коровы, лошади и утки. Оказалось, что у них голов столько же, сколько рогов, а крыльев на десять меньше, чем голов. Сколько лошадей на ферме?\\
109. Когда Маше было пять лет, Косте было столько же лет, сколько Тане год назад. Что больше --- возраст Кости или сумма возрастов Тани и Маши, и на сколько?\\
110. Через год Ксюше будет столько же лет, сколько исполнилось Антону, когда Насте было семь лет. Что больше --- возраст Антона или сумма возрастов Ксюши и Насти, и на сколько?\\
111. Тане и Андрею вместе 21 год. Когда Андрею было столько, сколько Тане сейчас, им вместе было 15 лет. Сколько лет было тогда Андрею?\\
112. Даше и Ксюше 18 лет. Когда Даше было столько лет, сколько Ксюше сейчас, им вместе было 10 лет. Сколько лет тогда было Ксюше?\\
113. Бабушка Оля в 9 раз старше внучки Тани, а мама Аня младше бабушки Оли на столько же лет, на сколько мама Аня старше внучки Тани. Вместе бабушке, маме и внучке 90 лет. Сколько лет бабушке Оле?\\
114. Хоттабыч --- древний волшебник. Когда он творит заклинание, он выдёргивает из своей бороды волосок, рвёт его на две части и говорит <<Трах-тибидох!>>. В будние дни Хоттабыч творит по 3 заклинания, а в субботу и воскресенье --- по 5. Сейчас утро понедельника, а в бороде Хоттабыча 2019
волосков. Через сколько дней эти волоски закончатся?\\
115. Хоттабыч --- древний волшебник. Когда он творит заклинание, он выдёргивает из своей бороды волосок, рвёт его на две части и говорит <<Трах-тибидох!>>. В будние дни Хоттабыч творит по 2 заклинания, а в субботу и воскресенье --- по 7. Сейчас утро понедельника, а в бороде Хоттабыча 2019
волосков. Через сколько дней эти волоски закончатся?\\
116. Путник поднимается в гору по лестнице, в которой 2019 ступеней. Как правило, он шагает через ступеньку, но после десяти таких шагов он отдыхает и делает три шага, не перешагивая ступеньки. Сколько шагов сделал путник?\\
117. На уроке рисования Алёна вырезала из клетчатой бумаги прямоугольник $20\times18$ клеток. Сидевший с ней Богдан случайно капнул на него синей краской, после чего ровно треть клеток оказались запачканными. После этого, снова случайно, он пролил на него жёлтую краску. В результате оказалось, что клеток только с синими пятнами ровно в два раза меньше, чем клеток только с жёлтыми пятнами. Докажите, что чистых клеток ровно в два раза больше, чем сине-жёлтых (чистые клетки --- это клетки, на которых нет пятен краски, а сине-жёлтые --- это клетки, на которых есть и синие, и жёлтые пятна).\\
118.  На уроке рисования Алёна вырезала из клетчатой бумаги прямоугольник $20\times20$ клеток. Сидевший с ней Богдан случайно капнул на него синей краской, после чего ровно четверть клеток оказались запачканными. После этого, снова случайно, он пролил на него жёлтую краску. В результате оказалось, что клеток только с синими пятнами ровно в три раза меньше, чем клеток только с жёлтыми пятнами. Докажите, что чистых клеток ровно в три раза больше, чем сине-жёлтых (чистые клетки --- это клетки, на которых нет пятен краски, а сине-жёлтые --- это клетки, на которых есть и синие, и жёлтые пятна).\\
119. Дениска, Мишка и Алёнка собрали марки. Изучая свои альбомы, дети обнаружили: 1) у каждого коллекционера в альбоме нет одинаковых марок; 2) у Дениски и Мишки марок поровну, но нет ни одной одинаковой; 3) у Дениски и Алёнки одинаковых марок ровно столько, сколько уникальных марок у Мишки; 4) у Алёнки нет уникальных марок. Докажите, что у Алёнки марок столько же, сколько у Дениски.\\
120. Коля купил 2 пирожных, 4 йогурта и 2 шоколадных батончика, а Антон купил 1 йогурт, 3 шоколадных батончика и 3 пирожных. Сколько стоит йогурт, если Коля за свои покупки заплатил 440 рублей, а Антон --- 560?\\
121. Возраст нескольких друзей в сумме составляет 62 года. Через 3 года он будет составлять 80 лет. Сколько этих друзей?\\
122. У полковника 10 майоров, у каждого майора 10 лейтенантов, у каждого лейтенанта 10 солдат. Какова численность войска (всех званий)?\\
123. Саша складывал два числа на калькуляторе, но, набирая второе число, случайно нажал в конце лишний ноль. Поэтому вместо 1331 он получил 8000. Какие числа хотел сложить Саша?\\
124. Автомобиль ехал по дороге Москва-Петербург, начало его пути у столба <<104-й км>>, конец пути у столба <<204-й км>>. Проехав четверть пути, автомобиль сломался. Около какого столба сломался автомобиль?\\
125. У Ани, Бори и Вити вместе 38 монет, у Ани монет в 4 раза больше, чем у Бори, и в 3 раза больше, чем у Вити. Сколько монет у Ани?\\
126. После ремонта в пустую гостиницу приехали китайские туристы и заняли половину всех номеров. Затем в половине оставшихся поселились японские туристы, а последними в гостиницу приехали французы и заняли треть номеров, оставшихся после заселения японцев. Сколько номеров в гостинице, если французы живут в трёх номерах?\\
127. Для покупки семи шоколадок Саше не хватает 35 рублей. Если он купит шесть шоколадок, то у него останется 15 рублей. Сколько шоколадок он сможет купить на 610 рублей?\\
128. Чтобы купить девять открыток, Ире не хватает 14 рублей. Если она купит восемь открыток, то у неё останется 26 рублей. Сколько открыток она сможет купить на 700 рублей?\\
129. Поспорили как-то три зимних месяца: кто из них самый холодный. Ударили месяцы посохами оземь и заморозили все деревья в лесу. Январь заморозил сразу 34 дерева, что оказалось в 2 раза меньше, чем декабрь, а Февраль на 17 больше, чем декабрь. Посчитайте, сколько всего деревьев было в лесу.\\
130. Мальчик и девочка носят воду вёдрами из колодца. Бочка в 70 л наполнится, если мальчик выльет в неё 5 своих полных вёдер, а девочка добавит к ним 6 своих.
Бочка в 83 л наполнится, если мальчик выльет а неё 6 своих полных вёдер, а девочка добавит к ним 7 своих. Сколько раз им надо вместе сходить за водой, чтобы бочка в 90 л оказалась полной?\\
131. Три подружки решили коллекционировать марки. Лена собрала 22 марки, что на 16 больше, чем Соня. Соня собрала марок в 2 раза меньше, чем Юля. Сколько марок собрали девочки вместе?\\
132. Три хозяйки Маша, Катя и Надя солили огурцы на зиму в одинаковых банках. Маша засолила 27 банок, что в 2 раза меньше, чем засолила Катя. Надя заготовила на 4 банки больше, чем Маша и Катя вместе. Сколько банок огурцов засолили хозяйки вместе?\\
133. У Никиты есть детали только для одного робота. Никита собирает робота за 3 мин 44 с, а разбирает его за 1 мин 16 с. Его младший брат Гена нехотя съел суп, посмотрел 3 мин в окно, потом выпил компот, затратив на всё вместе 19 мин. Причём на суп Гена затратил на 12 мин больше, чем на компот. Сколько раз Никита успел бы собрать своего робота, пока Гена ел суп, если бы работал без остановки?\\
134. У продавца было 28 ящиков с грушами, по 4 кг в каждом. К концу рабочего дня оказалось, что продано на 8 кг груш больше, чем осталось. Сколько ящиков груш было продано?\\
135. У фермера было 26 мешков с морковью по 7 кг в каждом. К концу недели оказалось, что фермер продал на 14 кг моркови больше, чем у него осталось. Сколько мешков моркови было продано?\\
136. В питомнике вырастили саженцы деревьев: клёнов было 287, а на кажые 7 клёнов приходилось 14 елей и 12 дубов. Сколько всего саженцев вырастили в питомнике?\\
137. Двум мастерицам фабрики игрушек поручили сделать 40 глиняных лошадок и 54 павлина. Ольга тратит на любую игрушку по 5 минут. Татьяна лепит лошадку за 4 минуты, а павлина за 6 минут. Придумайте, как мастерицам распределить работу, чтобы, работая вдвоём, закончить её как можно скорее. Найдите, сколько времени потребует ваш способ. Объяснять, почему он самый лучший, не надо.\\
138. Егор живёт в шестнадцатиэтажном доме в квартире №274. В каждом подъезде на каждом этаже расположено по шесть квартир. На каком этаже живёт Егор?\\
139. Лиза раскладывает апельсины по корзинам. Если она положит по пять апельсинов в каждую корзину, останется три лишних апельсина. А если класть по шесть апельсинов в корзину, останутся три лишние корзины. Сколько апельсинов у Лизы?\\
140. У 28 человек 5 <<Ы>> класса на собрание пришли папы и мамы. Мам было 24, пап 18. У скольких учеников на собрание пришли одновременно и папа, и мама?\\
141. Коле Гераскину 12 лет, а профессору Селезнёву 42. Через сколько лет Коля будет вдвое младше профессора?\\
142. Ученик Вовочка любит решать математические задачи. Известно, что вчера он решил на 11 задач меньше, чем позавчера, и на 32 задачи меньше, чем позавчера и сегодня вместе. Сколько задач решил Вовочка сегодня?\\
143. На корабле <<Пиратское счастье>> несколько кошек, матросов, как и одноногий капитан. У всех вместе взятых 15 голов и 41 нога. Сколько на корабле было кошек?\\
144. Тилли, Вилли и Дилли участвовали в легкоатлетическом забеге. В какой-то момент времени оказалось, что они бегут рядом друг с другом, впереди них бежит половина участников забега и позади них треть участников забега. Сколько спортсменов участвовало в забеге?\\
145. Алёша задумал число. Он прибавил к нему 5, потом разделил сумму на 3, умножил на 4, отнял 6, разделил на 7 и получил число 2. Какое число задумал Алёша?\\
146. Гриша с папой ходил в тир. Уговор был такой: Гриша делает 5 выстрелов и за каждое попадание в цель получает право сделать ещё два выстрела. Всего Гриша сделал 17 выстрелов. Сколько раз Гриша попал в цель?\\
147. Во сколько раз секундная стрелка движется быстрее часовой?\\
148. Банк имеет неограниченное число купюр достоинством 3 и 5 рублей.  Докажите, что Банк может выдать без сдачи любое число рублей, начиная с 8 руб.\\
149. Во дворе ходят собаки и куры. У всех животных вместе 34 ноги и 11 голов. Сколько на дворе кур и сколько собак?\\
150. У Феди было 560 г орешков. После того, как мальчик съел часть из них, он посчитал, что если бы он съел ещё 29 г орешков, то у него бы осталось $\cfrac{2}{7}$ всех орешков. Найдите, сколько граммов орешков съел Федя.\\
151. Год назад маме было столько же лет, сколько месяцев было сыну. Сейчас им в сумме 41 год. Сколько лет каждому из них?\\
152. Винни шёл от Кролика домой  ел мёд. На середине пути он обнаружил, что съел треть мёда, и решил, что может есть вдвое быстрее. Пройдя ещё половину оставшегося пути, Винни решил развернуться и скорее пойти к Кролику за ещё одним горшком мёда. Винни пошёл вдвое быстрее, чем раньше. Хватит ли ему мёда до домика Кролика, если скорость поедания мёда он не менял? Поясните ответ.\\
153. В математической олимпиаде для марсиан 5 задач, за каждую дают от 1 до 20 баллов. То есть 0 баллов набрать нельзя! При этом в зачёт идёт только 4 лучших задачи из 5. Марсианин Йумпага набрал за 5 задач 72 балла. Какое наименьшее количество баллов ему может пойти в зачёт?\\
154. Четверо товарищей покупают лодку. Первый вносит половину суммы, вносимой остальными, второй треть суммы, вносимой остальными, третий четверть суммы, вносимой остальными, а четвёртый 130 рублей. Сколько стоит лодка?\\
155. В трёх шкафах 2580 книг. В первом шкафу 350 книг. В третьем шкафу книг в 3 раза  больше, чем в первом и во втором шкафу, вместе взятых. Сколько книг во втором шкафу?\\
156. Три девочки – Оля, Ира и Аня – собирали на поле васильки. Ира собрала 28 васильков, Аня собрала в 2 раза больше васильков, чем Оля и Ира вместе взятые. Сколько васильков собрала Оля, если всего девочки собрали 150 васильков?\\
157. У Винни-пуха в погребе стоят бочонки с липовым и цветочным мёдом. В них 95 кг мёда, причём цветочного на 15 кг меньше, чем липового. Каждый бочонок вмещает 5 кг мёда. Сколько бочонков с липовым мёдом в погребе? А сколько с цветочным?\\
158. Бабушка наварила летом 39 литров варенья --- малинового и клубничного. Клубничного варенья оказалось на 9 литров меньше, чем малинового. Всё варенье разлито в трёхлитровые банки. Сколько банок с клубничным вареньем получилось у бабушки? И сколько с малиновым?\\
159. Трое ребят поделили между собой 176 рублей. Причём Коля получил на 34 рубля больше Жени, а Митя столько же, сколько Женя и Коля вместе. Сколько кому досталось?\\
160. Миша, Гриша и Алёша собирали грибы. Миша и Гриша вместе набрали 36 грибов. Гриша и Алёша на двоих собрали 42 гриба. А Миша и Алёша вдвоем набрали ровно 50 грибов. Сколько грибов набрал каждый мальчик по отдельности?\\
161. Мастер и два его помощника 3 часа 12 минут одновременно ремонтировали чайники. Первый помощник отремонтировал на 3 чайника больше второго и на 6 чайников меньше мастера. Сколько времени потратил каждый помощник и мастер на ремонт одного чайника, если второй помощник отремонтировал в 4 раза меньше чайников, чем мастер?\\
162. В зоомагазине продают больших и маленьких птиц. Большая птица стоит вдвое дороже маленькой. Одна дама купила 5 больших птиц и 3 маленьких, а другая – 5 маленьких и 3 больших. При этом первая дама заплатила на 20 рублей больше. Сколько стоит каждая птица?\\
163. В школьной библиотеке имеется 273 учебника по алгебре, что в 27 раз меньше, чем учебников по остальным предметам. Невоспитанные дети разрисовали $\cfrac{5}{6}$  учебников. Сколько чистых учебников осталось в библиотеке?\\
164. Библиотекарь протёр пыль с 360 книг, что составляет $\cfrac{3}{4}$ всех книг библиотеки. Сколько в библиотеке исторических романов, если они составляют $\cfrac{2}{5}$ от всего числа книг?\\
165. В школьных соревнованиях по лыжным гонкам, беге на коньках и игре в хоккей участвовали 696 человек. Лыжники составляли $\cfrac{1}{3}$ всех участников, конькобежцы --- $\cfrac{1}{4}$ от числа остальных спортсменов. Сколько детей играли в хоккей?\\
166. Кум Тыква накопил к старости 320 кирпичей и решил построить из них дом. В первый день он выложил четверть всех кирпичей и ещё 12 штук. Во второй день он выложил треть оставшихся кирпичей и ещё 7 штук. В третий день, выложив оставшиеся кирпичи, он достроил дом. Сколько кирпичей положил он в последний день?\\
167. Вася принес из леса 80 грибов: подберезовики, подосиновики и белые. Белых была $\cfrac{1}{5}$ часть, четверть остальных составляли подосиновики, а все прочие --- это подберёзовики. Сколько подберёзовиков нашел Вася?\\
168. На олимпиаде половина участников решила ровно 2 задачи, четверть участников --- ровно 3 задачи, а остальные 10 человек решили по 5 задач. Сколько всего ребят участвовало в олимпиаде?\\
169. Два робота круглосуточно собирают пылесосы. Робот <<Винтик>> собирает 16 пылесосов за 15 часов, а робот <<Шпунтик>> собирает 16 пылесосов за сутки. Роботов включили одновременно. Сколько пылесосов они соберут вместе за 30 часов? За какое время они вместе соберут 104 пылесоса?\\
170. Юра делает 120 бумажных корабликов за 3 часа, а Вася делает 120 бумажных корабликов за 2 часа. Сколько времени им нужно, чтобы вместе сделать 25 бумажных корабликов? Сколько бумажных корабликов они вместе сделают за 5 часов?\\
171. Мама чистит ведро картошки за 10 минут, папа за 12 минут, а Вася за 1 час. За сколько времени они почистят ведро картошки, работая вместе?\\
172. Винни-Пух съедает банку меда за 10 минут, а Пятачок --- за 15. За какое время они съедят 7 банок меда, если будут есть одновременно?\\
173. Коротышки пекли блины. Пончик может испечь 90 блинов за 45 минут, а Торопыжка --- за 30 минут. Незнайка может испечь столько же блинов, сколько за минуту пекут Пончик и Торопыжка вместе. За сколько минут Незнайка может испечь 90 блинов?\\
174. Малыш может съесть 600 г варенья за 6 минут, а Карлсон --- в 2 раза быстрее. За какое время они съедят это варенье вместе?\\
175. Собака выпьет миску молока за 5 минут, а кошка за 20 минут. За сколько минут они выпьют молоко вместе?\\
176. Через маленькое отверстие вода из бака выльется за 30 минут, а через большое --- за 6 минут. За сколько минут выльется вода из бака через оба отверстия одновременно?\\
177. Кошка Мурка съедает банку <<Вискас>> за 6 минут, а кот Васька --- в 2 раза быстрее. За какое время они съедят банку <<Вискас>> вместе?\\
178. Малыш съедает 900 г варенья за 9 минут. Карлсон делает это вдвое быстрее. За сколько минут они вместе съедят  1 кг 800 г варенья?\\
179. Лена  может собрать корзину клубники за час, а её старшая сестра в 2 раза быстрее, за сколько минут сестры соберут корзину ягод вместе?\\
180. Токарь вытачивает 72 одинаковые детали за 3 ч, а его ученику на выполнение этой работы требуется в 2 раза больше времени. За сколько часов они выточат 72 детали, работая вместе?\\
181. Чебурашка съест вагон апельсинов за 56 дней, а крокодил Гена за 8 дней. За сколько дней они съедят вагон апельсинов вместе?\\
182. На празднике 22 ученика. У каждого мальчика по 3 шара, у каждой девочки по 5 шаров. Всего надули 86 шаров. Кого на празднике больше --- девочек или мальчиков и на сколько?\\
183. В гараже стоят 750 автомобилей. Грузовые автомобили имеют по 6 колёс, а легковые --- по 4 колеса. Каких автомобилей в гараже больше и на сколько, если колёс всего 3024?\\
184. В порту стоят яхты с одной мачтой и шхуны с 2 мачтами. Старенький смотритель порта забыл, сколько шхун и сколько яхт находится в порту. Только помнит, что всего пришло ровно 100 кораблей. Помоги ему восстановить данные, если он насчитал в порту всего 146 мачт.\\
185. В комнате 10 столов. Часть из них с одним выдвижным ящиком, а часть с двумя. Всего в столах 14 ящиков. Сколько столов с одним ящиком и сколько с двумя?\\
186. На прогулку пошли четвероклассники и пятиклассники. Все они были либо босиком, либо в тапочках. Четвероклассников было 24, а босых учеников 16. Обутых пятиклассников было столько же, сколько босых четвероклассников. Сколько учеников  ходили на прогулку?\\
187. В классе 27 пловцов, 10 борцов и 15 футболистов. Каждый спортсмен занимается ровно двумя из этих видов спорта. Сколько всего в классе спортсменов?\\
188. В классе 17 пловцов, 6 борцов и 13 футболистов. Каждый спортсмен занимается ровно двумя из этих видов спорта. Сколько всего в классе спортсменов?\\
189. В саду у Ани и Вити росло сто розовых кустов. Витя полил половину всех кустов, и Аня полила половину всех кустов. При этом оказалось, что ровно три куста, самые красивые, были политы и Аней, и Витей. Сколько розовых кустов остались неполитыми?\\
190. В каждый из четырех походов ходила группа из 20 человек. Во все 4 похода ходили 10 человек. Ровно в 3 похода ходили 9 человек. Ровно в 2 похода ходили 5 человек. Сколько человек ходило только в 1 поход?\\
191. В секции плавания занимается 48 детей. Треть из них --- девочки. Ровно четверть детей ездила на сборы. Известно, что ровно 9 девочек на сборы не ездили. Сколько мальчиков не ездили на сборы?\\
192. Каждый из 35 четвероклассников является читателем по крайней мере одной из двух библиотек: школьной и районной. 25 человек берут книги в школьной библиотеке, 20 --- в районной. Сколько четвероклассников являются читателями обеих библиотек?\\
193.  К репетитору по математике ходит 24 учеников. Из них олимпиадные задачи любят решать 6 человек, обычные и олимпиадные – 2 человека, а 3 ученика вообще не любят решать задачки. Сколько у репетитора по математике тех учеников, которые любят решать только обычные задачи?\\
194. В магазин привезли одинаковое количество ящиков с бананами и апельсинами. Всего 96 кг.  Один ящик с бананами тяжелее, чем ящик с апельсинами на 2 кг, и их общий вес на 24 кг больше, чем вес апельсинов. Сколько ящиков бананов и сколько  ящиков апельсинов привезли? Сколько килограммов бананов в одном ящике? Сколько килограммов апельсинов?\\
195. В одной рукописи было 240 страниц, в другой --- 320 страниц, в третьей --- 480 страниц. Третью рукопись пришлось перепечатывать на 4 дня дольше, чем вторую. Сколько потребовалось дней, чтобы при одной и той же производительности труда перепечатать все три рукописи?\\
196. Маша купила больше Вари на 6 шоколадок и потратила на 300 рублей больше. Всего было куплено 10 шоколадок. Сколько потратила каждая девочка на шоколад?\\
197. Иван разделил 360 на задуманное число. Из результата вычел 17, затем умножил на 3 и прибавил 21. Получилось 105. Какое число задумал Иван?\\
198. Игорь задумал число и вычел из него 8. Полученную разность умножил на 9. В получившемся числе поменял цифры местами, итог разделил на 4 и получил 18. Какое число задумал Игорь?\\
199.  Если число лет Кати увеличить сначала на 19, а потом ещё в 2 раза, затем полученный результат уменьшить на 10 и разделить на 11, то будет 4. Сколько лет Кате?\\
200. Миша задумал число и вычел из него 49. Потом он разделил 364 на результат, прибавил к частному 12, умножил на 5 и получил 200. Какое число задумал Миша?\\
201. Сколько весит репка, если она весит 4 кг и ещё половину репки?\\
202. Сколько весит кирпич, если он весит 2 кг и ещё $\cfrac{5}{6}$ своего веса?\\
203. После того, как пешеход прошел половину пути и 1 км, ему осталось пройти треть пути и 1 км. Чему равен весь путь?\\
204. Каждую секунду бактерия делится на две новые бактерии. Известно, что весь объём одного стакана бактерии заполняют за 1 минуту. За сколько секунд стакан будет заполнен бактериями наполовину?\\
205. На поверхности пруда плавает одна кувшинка, которая постоянно делится и разрастается. Таким образом, каждый день площадь, которую занимают кувшинки, увеличивается в два раза. Через 30 дней покрытой оказывается вся поверхность пруда. За сколько дней покроется кувшинками вся поверхность пруда, если изначально на поверхности будут плавать две кувшинки?\\
206. Коля купил 2 пирожных, 4 кекса и 2 шоколадки, а Антон купил 1 кекс, 3 шоколадки и 3 пирожных. Сколько стоит кекс, если Коля за свои покупки заплатил 440 рублей, а Антон 560?\\
207. 17 маленьких бусинок и 18 больших бусинок стоят вместе 528 рублей. А 18 маленьких бусинок и 17 больших бусинок стоят 522 рубля. Сколько заплатит девочка за 20 больших и 20 маленьких бусинок?\\
208. Оля купила 1 блокнот и 8 ручек за 121 рубль. Ира купила 5 блокнотов и 6 ручек за 129 рублей. Сколько стоит блокнот? Сколько стоит ручка?\\
209. Одно пирожное и 4 булочки стоят 200 рублей, а 2 пирожных и 4 булочки стоят 300 рублей. Сколько булочек сможет купить Ваня, если у него есть 1000 рублей и он уже купил 8 пирожных?\\
210. 2 красных кирпича и 1 белый кирпич легче, чем 3 красных кирпича на 1 кг. Что тяжелее, красный или белый кирпич, и на сколько?\\
211. На острове Трям живут 3 разных племени. В первом и втором племенах живут 47 человек. В третьем и во втором вместе живут 21 человек, причём в третьем человек в 3 раза меньше, чем в первом. У всех островитян есть или 2 рубина, или 3 алмаза. Всего у них 180 драгоценных камней. Сколько алмазов есть у жителей острова?\\
212. Васе хватает денег на 2 конфеты и ещё остаётся три рубля. Если бы денег у Васи было в два раза больше, он смог бы купить пять конфет, и у него остался бы один рубль. Сколько стоит одна конфета?\\
213. Если Дима купит 1 конфету, у него останется 1 рубль, а на 2 конфеты ему не хватит 3 рублей. Сколько стоит конфета?\\
214. Слава собирался купить 20 конфет, но ему не хватало для этого 3 руб. Тогда Слава купил 15 конфет, и у него осталось 7 руб. сдачи. Сколько стоит одна конфета?\\
215. Три поросенка за три дня построили 3 домика. За сколько дней шесть таких же поросят построят себе шесть таких же домиков?\\
216. Три землекопа за два часа выкопали три ямы. Сколько ям выкопают шесть землекопов за пять часов?\\
217. На одной чаше весов лежит 4 яблока, на другой --- 6 груш. Если добавить одно такое же яблоко к грушам, то весы будут уравновешены. Сколько груш уравновешивают одно яблоко?\\
218. На левой чаше весов 5 одинаковых яблок, на правой чаше --- 1 яблоко и 2 одинаковые груши. Груша больше, чем яблоко. Весы находятся в равновесии. Яблоко весит 100 г. Сколько весит груша?\\
219. Миша, Коля и Петя вместе имеют массу 89 кг. Миша с Колей вместе имеют массу 63 кг, Коля с Петей --- 58 кг. Сколько весит каждый мальчик?\\
220. Четыре девочки пошли гулять и решили сделать фотографии друг с другом. Для каждого снимка какие-нибудь три из них становились в группу, а четвертая фотографировала их. Вечером они посчитали, что Аня присутствует на 8 снимках, Таня на 6 снимках, Оля на 3 снимках, Катя на 7 снимках. Сколько фотографий сделала Таня?\\
221. Четыре девочки поют песни, аккомпанируя друг другу. Каждый раз одна из них играет на фортепьяно, а остальные три поют. Вечером они посчитали, что Аня
спела 8 песен, Таня --- 6 песен, Оля --- 3 песни, а Катя --- 7 песен. Сколько раз аккомпанировала Таня?\\
222. Котёнок Гав получил  подарки от друзей:  тортов и кексов вместе 7 штук, пирогов и кексов --- 9, а тортов и пирогов --- 6. Сколько всего подарков?\\
223. У Миши в 7 раз больше конфет, чем у Оли. Если Миша отдаст Оле 21 конфету, то конфет у детей станет поровну. Сколько конфет было у Миши и Оли вместе?\\
224. У Вари на 100 рублей больше, чем у Оли. Сколько нужно отдать Варе Оле, чтобы у них стало поровну?\\
225. Два лесоруба варили на обед кашу. Один всыпал в котёл 2 кружки крупы, второй --- 3 кружки. Когда каша была готова, подходит к ним третий лесоруб и говорит: позвольте мне с вами пообедать, я заплачу вам 5 монет. Как им поделить монеты?\\
226. Чук и Гек вместе с мамой наряжали ёлку. Чтобы они не подрались, мама выделила каждому из братьев по одинаковому числу веточек и по одинаковому числу игрушек. Чук попробовал на каждую ветку повесить по одной игрушке, но ему не хватило для этого одной ветки. Гек попробовал на каждую ветку повесить по две игрушки, но одна ветка у него оказалась пустой. Как вы думаете, сколько веток и сколько игрушек выделила мама сыновьям?\\
227. Летели галки, сели на палки. Сядут по одной --- галка лишняя, сядут по две --- палка лишняя. Сколько было палок и сколько было галок?\\
228. Найдите наибольшее целое число, дающее при делении на 13 с остатком частное 17.\\
229. На озере росли 45 белых и жёлтых лилий. После того, как 9 жёлтых лилий облетели, а 3 белых лилии пожелтели, то белых лилий стало втрое меньше, чем жёлтых. Сколько белых и жёлтых лилий вначале было на озере?\\
230. У Вари на 300 рублей больше, чем у Маши. Сколько денег должна Варя отдать Маше, чтобы у них стало поровну?\\
231. Вася должен был разделить число на 2 и к результату прибавить 3, а он по ошибке умножил число на 2 и от полученного произведения отнял 3. Ответ всё равно получился правильный! Какой?\\
232. В коробке лежат синие, красные и зелёные карандаши. Всего 20 штук. Синих в 6 раз больше, чем зелёных, красных меньше, чем синих. 
Сколько в коробке красных карандашей?\\
233. В 12-этажном доме на один этаж  приходится 3 квартиры. В каком подъезде будет расположена квартира 189?\\
234. Миша обратил внимание, что на улице на каждом пешеходном переходе 12 полос, а в переулке на каждом пешеходном переходе 7 полос. Всего есть 8 переходов и 81 полоса. Сколько пешеходных переходов в переулке?\\
235. При устройстве городского парка посадили деревья. Сосен посадили в 6 раз больше, чем елей, а лип на 30 меньше, чем берёз. Сколько посажено сосен и берёз вместе, если лиственных деревьев было 84, и это вдвое больше, чем хвойных?\\
236. В заповеднике водятся волки, тигры, зайцы и косули. Волков в 7 раз больше, чем тигров, а зайцев на 20 больше, чем косуль. Сколько в заповеднике косуль и волков вместе, если хищных животных 56 особей, и это вдвое меньше, чем травоядных?\\
237. Из 4 тонн винограда и 3 тонн яблок получается 1130 кг сухофруктов. Из 3 тонн винограда и 4 тонн яблок получается 1250 кг сухофруктов. Сколько килограммов сухофруктов получится из 5 тонн винограда и 5 тонн яблок? Сколько килограммов изюма получается из 1 тонны винограда?\\
238. Из 5 тонн слив и 4 тонн груш получается 1530 кг сухофруктов. Из 4 тонн слив и 5 тонн груш получается 1620 кг сухофруктов. Сколько килограммов сухофруктов получится из 7 тонн слив и 7 тонн груш? Сколько килограммов чернослива получается из 1 тонны слив?\\
239. Шесть карандашей стоят на 30 рублей дешевле, чем три ручки и три карандаша. На сколько рублей карандаш дешевле ручки?\\
240. Восемь тетрадей стоят на 40 рублей дешевле, чем четыре тетради и четыре блокнота. На сколько рублей тетрадь дешевле блокнота?\\
241. Маша, Миша, Саша и Даша вместе  весят 200 кг. Маша и Саша вместе в два раза тяжелее Миши, Даша с Машей весят столько же, сколько и Саша с  Мишей. Сколько весят Саша и Миша вместе?\\
242. Маша, Миша, Саша и Даша вместе  весят 100 кг. Маша и Саша вместе в два раза тяжелее Миши, Даша с Машей весят столько же, сколько и Саша с  Мишей. Сколько весят Саша и Миша вместе?\\
243. Мастер изготавливает три ключа в день, подмастерье делает два ключа в день, но каждый пятый выходит бракованный. Сколько нужно взять подмастерьев мастеру, если ему надо изготовить 110 хороших ключей за 10 дней?\\
244. Мастер изготавливает четыре ключа в день, подмастерье делает три ключа в день, но каждый пятый выходит бракованный. Сколько нужно взять подмастерьев мастеру, если ему надо изготовить 180 хороших ключей за 9 дней?\\
245. Начиная с шестого дня рождения Паша получал на один подарок меньше, чем в прошлый год. После своего десятого дня рождения у него оказалось в сумме 55 подарков. Сколько подарков он получил на десятый день рождения?\\
246. Начиная с шестого дня рождения Паша получал на два подарка меньше, чем в прошлый год. После своего одиннадцатого дня рождения у него оказалось в сумме 112 подарков. Сколько подарков он получил на десятый день рождения?\\
247. На школьном стадионе встретились футболисты и баскетболисты. Каждую минуту либо один футболист уходил играть в баскетбол, либо два баскетболиста присоединялись к футболистам. Через 10 минут ребята заметили, что на поле остались только футболисты. Сколько могло быть баскетболистов изначально?\\
248. На школьном стадионе встретились футболисты и баскетболисты. Каждую минуту либо один футболист уходил играть в баскетбол, либо два баскетболиста присоединялись к футболистам. Через 11 минут ребята заметили, что на поле остались только футболисты. Сколько могло быть баскетболистов изначально?\\
249. В школе, в которой учатся 655 детей, началась эпидемия. Каждый день заболевает более половины от ещё не заболевших (то есть в первый день здоровых не более 327 детей). На какой день уже точно можно утверждать, что здоровых детей меньше 10?\\
250. В школе, в которой учатся 725 детей, началась эпидемия. Каждый день заболевает более половины от ещё не заболевших (то есть в первый день здоровых не более 362 детей). На какой день уже точно можно утверждать, что здоровых детей меньше 11?\\
251. На конкурсе живописи каждый рисунок оценивается в целое число баллов от 1 до 20, но в окончательном подсчёте участнику засчитывается только 4 лучших рисунка. За 5 рисунков Вася набрал 72 балла. Какой наименьший результат может получиться при окончательном подсчёте?\\
252. На конкурсе живописи каждый рисунок оценивается в целое число баллов от 1 до 20, но в окончательном подсчёте участнику засчитывается только 4 лучших рисунка. За 5 рисунков Петя набрал 82 балла. Какой наименьший результат может получиться при окончательном подсчёте?\\
253. Вадим, Кирилл и Алина покупали подарок из двух частей. Вадим заплатил за первую часть подарка 5695 рублей, а Алина 1405 рублей за вторую. Изначально они договаривались, что Вадим заплатит половину всей суммы за подарок, а Кирилл и Алина --- поровну за оставшуюся часть. Понятно, что Кирилл и Алина должны Вадиму, но Кирилл решил заплатить за Алину остаток. Сколько теперь Кирилл должен Вадиму?\\
254. Вадим, Кирилл и Алина покупали подарок из двух частей. Вадим заплатил за первую часть подарка 5695 рублей, а Алина 1445 рублей за вторую. Изначально они договаривались, что Вадим заплатит половину всей суммы за подарок, а Кирилл и Алина --- поровну за оставшуюся часть. Понятно, что Кирилл и Алина должны Вадиму, но Кирилл решил заплатить за Алину остаток. Сколько теперь Кирилл должен Вадиму?\\
255. Одна весёлая и две грустных обезьяны съедают ящик бананов за час, а четыре весёлых и две грустных обезьяны съедают ящик бананов за 20 минут. Сколько времени одна весёлая обезьяна будет есть ящик бананов? (Все грустные обезьяны едят с одной скоростью, и все весёлые тоже с одной скоростью.)\\
256. Две весёлых и четыре грустных обезьяны съедают ящик бананов за 30 минут, а две весёлых и одна грустная обезьяна съедают ящик бананов за 40 минут. Сколько времени одна весёлая обезьяна будет есть ящик бананов? (Все грустные обезьяны едят с одной скоростью, и все весёлые тоже с одной скоростью.)\\
257. Имеется 35 брёвен --- длинных и коротких. Длинные распиливают на 5 частей, а короткие --- на 4 части. Чтобы распилить все короткие брёвна, потребовалось сделать столько же распилов, сколько чтобы распилить все длинные. Сколько было сделано распилов?\\
258. Имеется 36 брёвен --- длинных и коротких. Длинные распиливают на 6 частей, а короткие --- на 5 частей. Чтобы распилить все короткие брёвна, потребовалось сделать столько же распилов, сколько чтобы распилить все длинные. Сколько было сделано распилов?\\
259. Какое число надо умножить на 10, чтобы получилось частное чисел 100 и 5?\\
260. Какое число надо разделить на 5, чтобы получилось произведение чисел 10 и 2?\\
261. В тринадцатиэтажном доме на каждом этаже по 4 квартиры. На каком этаже расположена квартира №38?\\
262. Из 18 школьников 12 изучают английский язык, а 14 --- немецкий язык. Сколько школьников изучают два языка, если известно, что каждый изучает хотя бы один язык?\\
263. У Максима есть немного карманных денег, на которые он может купить себе одно мороженое, но на второе мороженое денег уже не хватает. Мама обнаружила, что если дать Максиму в четыре раза больше денег, чем ему не хватает до второго мороженого, то он сможет купить ровно три мороженых. А во сколько раз больше, чем ему не хватает на второе, надо дать ему денег, чтобы хватило ровно на четыре мороженых?\\
264. У Кирилла есть немного карманных денег, на которые он может купить себе одно мороженое, но на второе мороженое денег уже не хватает. Мама обнаружила, что если дать Кириллу в пять раз больше денег, чем ему не хватает до второго мороженого, то он сможет купить ровно три мороженых. А во сколько раз больше, чем ему не хватает на второе, надо дать ему денег, чтобы хватило ровно на четыре мороженых?\\
265. Предприниматель купил три здания и собирается открыть в них отель. В отеле могут быть стандартные номера площадью 30 квадратных метров и номера <<люкс>> площадью 40 квадратных метров. Общая площадь, которую можно отвести под номера в каждом здании, составляет 940 квадратных метров. Предприниматель может распределить эту площадь между номерами различных типов, как хочет. Обычный номер будет приносить отелю 4000 рублей в сутки, а номер <<люкс>> --- 5000 рублей в сутки. Какую наибольшую сумму денег сможет заработать в сутки на своём отеле предприниматель?\\
266. Предприниматель купил три здания и собирается открыть в них отель. В отеле могут быть стандартные номера площадью 30 квадратных метров и номера <<люкс>> площадью 40 квадратных метров. Общая площадь, которую можно отвести под номера в каждом здании, составляет 820 квадратных метров. Предприниматель может распределить эту площадь между номерами различных типов, как хочет. Обычный номер будет приносить отелю 4000 рублей в сутки, а номер <<люкс>> --- 5000 рублей в сутки. Какую наибольшую сумму денег сможет заработать в сутки на своём отеле предприниматель?\\
267. В ряд выписали 1000 подряд идущих натуральных чисел. Алина заметила, что выписано 3900 цифр. Какое число было выписано последним?\\
268. В ряд выписали 1000 подряд идущих натуральных чисел. Алина заметила, что выписано 3800 цифр. Какое число было выписано последним?\\
269. Максим и Анна решили вместе купить ноутбук Nenovo за 100 тысяч рублей, считая, что он будет работать 5 лет. Они изначально договаривались заплатить поровну и пользоваться поровну. Однако, через год оказалось, что Максим заплатил в три раза больше Анны, а пользуется в три раза меньше Анны. Тогда Анна решила выкупить у Максима долю и пользоваться ноутбуком единолично. Сколько Анна должна Максиму? Не забудьте, что они покупали ноутбук на 5 лет и каждый год он становится дешевле на одинаковую сумму!\\
270. Кирилл и Алина решили вместе купить ноутбук Pear за 200 тысяч рублей, считая, что он будет работать 10 лет. Они изначально договаривались заплатить поровну и пользоваться поровну. Однако, через год оказалось, что Кирилл заплатил в три раза больше Алины, а пользуется в три раза меньше Алины. Тогда Алина решила выкупить у Кирилла долю и пользоваться ноутбуком единолично. Сколько Алина должна Кириллу? Не забудьте, что они покупали ноутбук на 10 лет и каждый год он становится дешевле на одинаковую сумму!\\
271. Число разделили на 6. В частном получили 4, а остаток оказался в 2 раза меньше делителя. Найдите делимое.\\
272. Утром на столе мама оставила 56 конфет. После того как Маша съела некоторое количество конфет, на столе осталось конфет в 7 раз больше, чем она съела. Сколько конфет она съела?\\
273. В кондитерской каждое пирожное украшают не меньше, чем 4 ягодами. Всего на украшение пирожных потратили 135 ягод. Какое наибольшее число пирожных могли украсить?\\
274. Для чаепития Алиса подготовила 7 чашек с чаем. В первую минуту чаепития она налила ещё одну чашку чая, во вторую минуту --- 2, в третью --- 3 и так далее. Начиная со второй минуты, к чаепитию стали присоединяться разные гости, каждый из которых выпивал 1 чашку чая в минуту. Со второй минуты пришёл Мартовский Заяц, с третьей к ним присоединился Соня, с четвёртой  --- Болванщик и так далее, с каждой минутой к чаепитию присоединялся ещё один герой. Сколько полных чашек чая будет стоять на столе через 45 минут такого безумного чаепития?\\
275. В фруктовой лавке продавали яблоки, бананы, апельсины, киви и виноград. Яблок было 36 кг. Бананов было на 4950 г меньше, чем яблок, а половина апельсинов равна одной трети всех яблок. Киви и винограда было всего 22 кг, причём киви было на 1300 г больше, чем винограда.\\
а) Сколько в лавке было фруктов каждого вида?\\
б) Во сколько раз бананов было больше, чем винограда?\\
276. Мама с сыном шли вместе в школу. Пока мама делает три шага, сын делает пять. Вместе они сделали 6400 шагов. Сколько шагов сделал сын?\\
277. Петя с Васей едят мороженое из банки. Пока Петя съедает две ложки, Вася успевает проглотить 5. Петя смог съесть 18 ложек мороженого. Сколько всего ложек вмещает в себя банка, если мальчики закончили есть одновременно и ложки у них одинаковые?\\
278. Первого мая в первый отель приехало в два раза меньше туристов, чем во второй, и на 10 человек больше, чем в третий. А в третий и четвёртый отели в сумме приехало 3 туриста. Сколько всего человек могло приехать в отели, если в каждый отель приехал хотя бы один турист?\\
279. Первого мая в первый отель приехало в два раза меньше туристов, чем во второй, и на 11 человек больше, чем в третий. А в третий и четвёртый отели в сумме приехало 3 туриста. Сколько всего человек могло приехать в отели, если в каждый отель приехал хотя бы один турист?\\
280. Витя съедает торт за 10 минут, а Миша за 12. При этом после трёх тортов Вите надо сделать перерыв на 5 минут, а Миша после 60 минут жевания делает 10-минутный перерыв. Сколько им нужно времени, чтобы съесть 50 тортов? Есть один торт вдвоём нельзя.\\
281. У каждого марсианина 2 или 3 клешни. На собрании встретились марсиане, у которых в сумме было 533 клешни. Когда встречаются два марсианина с разным количеством клешней, они в обнимку уходят с собрания. Через некоторое время с собрания ушли 40 пар, и оказалось, что у всех остальных одинаковое количество клешней. Сколько марсиан было на собрании изначально?\\
282. Кирилл выписал по порядку все натуральные числа от 1 до 1000 и считает сумму всех чисел, каждый раз прибавляя следующее число к полученной сумме. То есть он получает: 3, 6, 10, 15, 21 и так далее. В какой-то момент он получил 457446 и сразу после него 458403. Какое число он получит следующим?\\
283. Кирилл выписал по порядку все натуральные числа от 1 до 1000 и считает сумму всех чисел, каждый раз прибавляя следующее число к полученной сумме. То есть он получает: 3, 6, 10, 15, 21 и так далее. В какой-то момент он получил 466095 и сразу после него 467061. Какое число он получит следующим?\\
284. В ряд выписали 1000 подряд идущих натуральных чисел. Алина заметила, что выписано 3092 цифры. Какое число было выписано первым?\\
285. В ряд выписали 1000 подряд идущих натуральных чисел. Алина заметила, что выписано 3090 цифр. Какое число было выписано первым?\\
286. В деревню приехал гипнотизёр и провёл анкетирование о наличии ног и голов среди всех кур и коров. Все куры утверждали, что у них две головы и три ноги, а все коровы --- что у них две головы и пять ног. Согласно полученным данным, в деревне 123456 ног и 53088 голов. А сколько на самом деле ног у всех животных?\\
287. В деревню приехал гипнотизёр и провёл анкетирование о наличии ног и голов среди всех кур и коров. Все куры утверждали, что у них две головы и три ноги, а все коровы --- что у них две головы и пять ног. Согласно полученным данным, в деревне 123456 ног и 53328 голов. А сколько на самом деле ног у всех животных?\\
288. За обмен долларов на юани взимается 50 юаней комиссии за каждый факт обмена. У 6 детей было 100, 100, 150, 200, 200, 500 долларов. Один доллар стоит 4 юаня. Чтобы избежать лишних комиссий, дети сложили все свои деньги и обменяли их вместе. После этого справедливо разделили. Сколько юаней получит тот, у кого было 150 долларов?\\
289. За обмен долларов на юани взимается 50 юаней комиссии за каждый факт обмена. У 6 детей было 500, 500, 450, 400, 400, 250 долларов. Один доллар стоит 4 юаня. Чтобы избежать лишних комиссий, дети сложили все свои деньги и обменяли их вместе. После этого справедливо разделили. Сколько юаней получит тот, у кого было 450 долларов?\\
290. На сколько нулей оканчивается произведение чисел 308500 и 43940?\\
291. На сколько нулей оканчивается произведение чисел 3685000 и 4039600?\\
292. Задуманное число умножили на 412 и получили 21012. Какое число задумали?\\
293. Задуманное число умножили на 321 и получили 17013. Какое число задумали?\\
294. Миша купил 5 тетрадей и 6 ручек и заплатил за них 120 рублей. Антон купил 10 тетрадей и 12 блокнотов. Сколько заплатил Антон, если блокнот на 4 рубля дороже ручки?\\
295. Алиса купила 7 блокнотов и 9 линеек и заплатила за них 160 рублей. Ксюша купила 14 альбомов и 18 линеек. Сколько заплатила Ксюша, если альбом на 6 рублей дороже блокнота?\\
296. Куртка стоит 3600 рублей и ещё четверть своей стоимости. Сколько всего рублей стоит куртка?\\
297. Брюки стоят 2400 рублей и ещё треть своей стоимости. Сколько всего рублей стоят брюки?\\
298. Цепь из двух звеньев имеет длину 13 см, а цепь из трёх таких же звеньев имеет длину 18 см. Какую длину имеет цепь из семи таких же звеньев?\\
299. Цепь из трёх звеньев имеет длину 17 см, а цепь их четырёх таких же звеньев имеет длину 22 см. Какую длину имеет цепь из десяти таких же звеньев?\\
300. На трёх фермах выращивают 784 капибары. Известно, что на первой ферме живёт втрое больше грызунов, чем на второй, а на третьей втрое больше, чем на первых двух вместе взятых. Сколько капибар найдёт Вова на первой ферме?\\
301. На трёх фермах выращивают 768 капибар. Известно, что на первой ферме живёт втрое больше грызунов, чем на второй, а на третьей втрое больше, чем на первых двух вместе взятых. Сколько капибар найдёт Вова на первой ферме?\\
302. Нюша задумала число, прибавила к нему 1, потом уменьшила сумму в 3 раза, увеличила в 4, уменьшила на 6, разделила на 7 и получила число 2. Какое число она задумала?\\
303. Нюша задумала число, прибавила к нему 1, потом уменьшила сумму в 4 раза, увеличила в 5, уменьшила на 2, разделила на 6 и получила число 3. Какое число она задумала?\\
304. Лесные звери пришли на день рождения медведя. Лис и волков вместе было в 2 раза больше, чем зайцев и ежей вместе. Лис пришло 8 пар, что на 1 пару больше, чем волков. Количество ежей на 3 меньше, чем зайцев. Сколько ежей пришло на день рождения к медведю?\\
305. В библиотеке есть книги следующих жанров: биографии, фантастика, энциклопедии и сказки. Биографий и сказок вместе вдвое меньше, чем фантастики и энциклопедий вместе. При этом количество фантастики составляет треть от количества энциклопедий. Книг со сказками на 6 больше, чем биографий. Сколько книг какого жанра в библиотеке, если общее количество книг 216?\\
306. В музыкальном фестивале участвуют несколько рок-групп. В каждой рок-группе либо только 7 гномов, либо только 3 богатыря. Организаторы фестиваля подготовили сувенирную продукцию --- по одному значку для каждого участника и по одному плакату для каждой рок-группы. Всего было выдано 96 значков и 20 плакатов. Сколько групп гномов и сколько групп богатырей участвовало в фестивале?\\
307. В соревновании Ironmath нужно решить 7 задач сидя, 7 --- бегом и 7 --- на велосипеде. В 202 году Кирилл выполнил все задания за 3 часа. В 2021 году он решал задачи сидя в два раза быстрее, чем в 2020, и пришёл к финишу на 30 минут раньше. В 2022 году он решал сидя с той же скоростью, что и в 2020, но бегом стал решать в два раза быстрее, чем ранее, и решил на 15 минут быстрее, чем в 2020. В 2023 году он стал решать на велосипеде в два раза быстрее, чем ранее, а бегом и сидя с той же скоростью, что и в 2020. За какое время он закончит соревнование в этом году?\\
308. В соревновании Ironmath нужно решить 7 задач сидя, 7 --- бегом и 7 --- на велосипеде. В 202 году Кирилл выполнил все задания за 3 часа. В 2021 году он решал задачи сидя в два раза быстрее, чем в 2020, и пришёл к финишу на 40 минут раньше. В 2022 году он решал сидя с той же скоростью, что и в 2020, но бегом стал решать в два раза быстрее, чем ранее, и решил на 25 минут быстрее, чем в 2020. В 2023 году он стал решать на велосипеде в два раза быстрее, чем ранее, а бегом и сидя с той же скоростью, что и в 2020. За какое время он закончит соревнование в этом году?\\
309. Учитель выписал на доску число. Вася должен был умножить число на 25, Коля --- на 13, а Петя --- на 12. В итоге у них получились ответы 1872, 2016, 2028 в каком-то порядке. Известно, что ровно один из мальчиков ошибся. Кто ошибся, и какой у него должен быть верный ответ? В ответ запишите имя и число (например: Игорь, 239).\\
310. Учитель выписал на доску число. Вася должен был умножить число на 25, Коля --- на 12, а Петя --- на 13. В итоге у них получились ответы 2197, 2184, 2028 в каком-то порядке. Известно, что ровно один из мальчиков ошибся. Кто ошибся, и какой у него должен быть верный ответ? В ответ запишите имя и число (например: Игорь, 239).\\
311. Отрезок, равный 48 см, разделён на четыре (возможно, неравных) отрезка. Расстояние между серединами 1-го и 3-го отрезков равно 16 см, 2-го и 4-го отрезков --- 14. Найдите сумму длин 2-го и 3-го отрезков.\\
312. Отрезок, равный 56 см, разделён на четыре (возможно, неравных) отрезка. Расстояние между серединами 1-го и 3-го отрезков равно 18 см, 2-го и 4-го отрезков --- 24. Найдите сумму длин 2-го и 3-го отрезков.\\
313. Рабочий Василий Иванович изготавливает в день 7 деталей, а его ученик Петька --- 4 детали. В течение первых шести дней работал только Петька, после чего ему на помощь пришел Василий Иванович. По окончании работы оказалось, что оба изготовили по одинаковому количеству деталей.\\
a) Сколько дней они работали вместе?\\
b) Сколько всего деталей они изготовили?\\
314. Рабочий Василий Иванович изготавливает в день 13 деталей, а его ученик Петька --- 10 деталей. В течение первых шести дней работал только Петька, после чего ему на помощь пришел Василий Иванович. По окончании работы оказалось, что оба изготовили по одинаковому количеству деталей.\\
a) Сколько дней они работали вместе?\\
b) Сколько всего деталей они изготовили?\\
315. Поднимаясь по эскалатору станции метро <<Василеостровская>>, четвероклассник заметил, что навстречу ему едут ученики одного известного физико-математического лицея (в галстуках), студенты одного известного университета (в пиджаках) и курсанты одной известной академии (в форме). А ещё там ехали совершенно обычные пассажиры. Четвероклассник стал всех считать. Оказалось, что лицеистов и студентов вместе ехало втрое больше, чем обычных пассажиров и курсантов вместе. При этом количество обычных пассажиров составляло половину от числа курсантов и треть от числа студентов. Сколько людей было в каждой из ехавших групп, если студентов было на 14 больше, чем обычных пассажиров?\\
316. Семь четвероклассников на математической турнире получили по одинаковому количеству задач. Через 15 минут пятеро из них решили по 3 задачи. Оказалось, что теперь у этих пятерых вместе столько же нерешённых задач, сколько было выдано вместе двум остальным в начале турнира. Сколько задач получил изначально каждый участник? Один из участников заметил, что каждую следующую задачу он решает на 2 минуты дольше предыдущей. За какое время он решит все задачи, если этот прирост продолжится?\\
317. Во время игры в лото вытащили 41 карточку. Карточек с чётными номерами вытащили на 13 больше, чем с нечётными. Сколько карточек с нечётными номерами вытащили?\\
318. Почтальон отнёс письма в 43 дома. В дома с нечётными номерами пришло на 15 писем больше, чем с чётными. Сколько писем пришло в дома с чётными номерами?\\
319. Если к задуманному числу прибавить его четверть, получится 4000. Сколько получится, если к этому числу прибавить его пятую часть?\\
320. Если к задуманному числу прибавить его треть, получится 6000. Сколько получится, если к этому числу прибавить его четверть?\\
321. В учебнике математики 30 картинок, а в задачнике 23 картинки. Всего привезли 50 книг (учебников и задачников), во всех этих книгах вместе 1304 картинки.
Сколько задачников привезли?\\
322. В учебнике литературы 26 стихотворений, а в хрестоматии 40 стихотворений. Всего привезли 50 книг (учебников и хрестоматий), во всех этих книгах вместе
1552 стихотворения. Сколько хрестоматий привезли?\\
323. 3 карандаша, 6 линеек и 1 тетрадь стоят 132 рубля, а 5 карандашей, 7 тетрадей и 2 линейки стоят 204 рубля. Сколько стоит набор, где каждого предмета по два?\\
324. 4 огурца, 2 помидора и 3 свёклы стоят 99 рублей, а 5 огурцов, 6 свёкл и 7 помидоров стоят 207 рублей. Сколько стоит набор, где каждого овоща по три?\\
325. В наборах первого вида по 8 яблок, 3 груши и 1 апельсину, в наборах второго вида по 12 яблок, 4 груши и 2 апельсина. Для формирования наборов куплено
260 груш и апельсинов вместе. Сколько всего куплено яблок?\\
326. В наборах первого вида по 14 ручек, 4 линейки и 3 точилки, в наборах второго вида по 6 ручек, 1 точилке и 2 линейки. Для формирования наборов куплено
270 точилок и линеек вместе. Сколько всего куплено ручек?\\
327. Лёша купил 5 стаканов компота, Олег купил 7 стаканов компота. Женя тоже захотел купить себе компот, но тот уже кончился. Лёша и Олег поделились с Женей компотом так, что каждому из троих досталось поровну. Женя отдал ребятам 24 рубля за свою долю. Сколько рублей достанется Олегу, если расходы у всех одинаковы?\\
328. Лена купила 9 пирожков, Оля купила 6 пирожков. Ира тоже захотела купить себе пирожки, но они уже закончились. Лена и Оля поделились с Ирой пирожками так,
что каждой из трёх досталось поровну. Ира отдала девочкам 45 рублей за свою долю. Сколько рублей достанется Лене, если расходы у всех одинаковы?\\
329. В магазине продавали чашки. Первому покупателю продали треть всех чашек и ещё треть чашки. Второму покупателю продали треть оставшихся чашек и ещё
треть чашки. Третьему покупателю продали 3 оставшихся чашки. Сколько чашек было в магазине изначально?\\
330. В магазине продавали блюдца. Первому покупателю продали треть всех блюдец и ещё треть блюдца. Второму покупателю продали треть оставшихся блюдец
и ещё треть блюдца. Третьему покупателю продали 7 оставшихся блюдец. Сколько блюдец было в магазине изначально?\\
331. Петя и Вася владеют соседними дачными участками одинаковой формы и размера. Петя решил
обнести свой участок забором и потратил на это 18000 рублей. Затем Вася захотел достроить забор
вокруг своего участка и потратил на это 13500 рублей. Какова длина общей границы двух участков,
если стоимость одного метра забора составляет 150 рублей?\\
332.  Петя и Вася владеют соседними дачными участками одинаковой формы и размера. Петя решил
обнести свой участок забором и потратил на это 22400 рублей. Затем Вася захотел достроить забор
вокруг своего участка и потратил на это 16000 рублей. Какая длина общей границы двух участков,
если стоимость одного метра забора составляет 160 рублей?\\
333.  Бригаде рабочих, состоящей из 12 человек, нужно было вырыть траншею длиной 1000 метров.
Для этого они разделились на две не обязательно равные группы и начали копать с разных сторон,
пока не встретились. Известно, что за второй группой приглядывал начальник, поэтому каждый
рабочий в ней копал в два раза быстрее рабочего из первой группы. Бригада справилась с работой
в 8 раз быстрее, чем если бы всю траншею выкопал один человек под присмотром начальника.\\
1) Сколько людей было в первой группе?\\
2) Сколько метров прокопала первая группа?\\
334. Бригаде рабочих, состоящей из 10 человек, нужно было вырыть траншею длиной 900 метров.
Для этого они разделились на две не обязательно равные группы и начали копать с разных сторон,
пока не встретились. Известно, что за второй группой приглядывал начальник, поэтому каждый
рабочий в ней копал в два раза быстрее рабочего из первой группы. Бригада справилась с работой
6 раз быстрее, чем если бы всю траншею выкопал один человек под присмотром начальника.\\
1) Сколько людей было в первой группе?\\
2) Сколько метров прокопала первая группа?\\
335. В нашем Лицее классы нумеруются цифрами, а не буквами, как обычно в школе. Например, это вступительная работа в классы 5-1 и 5-2. В 8-1 классе училось 29 человек, в 8-2 --- 32 человека, в 8-3 --- 30. После контрольной по физике 3 человека перешли из 8-1 в 8-2, 2 человека из 8-2 в 8-3, а 2 человека из 8-3 в 8-1. После контрольной по алгебре из 8-1 в 8-3 ушло 4 человека, из 8-3 в 8-2 --- 3, из 8-2 в 8-1 --- 7. После контрольной по литературе из 8-3 ушло в другую школу 3 человека. Сколько человек учится теперь вместе во всех трёх классах?\\
336. В нашем Лицее классы нумеруются цифрами, а не буквами, как обычно в школе. Например, это вступительная работа в классы 5-1 и 5-2. В 8-1 классе училось 28 человек, в 8-2 --- 31 человек, в 8-3 --- 32. После контрольной по физике 4 человека перешли из 8-1 в 8-2, 2 человека из 8-2 в 8-3, а 4 человека из 8-3 в 8-1. После контрольной по алгебре из 8-1 в 8-3 ушло 3 человека, из 8-3 в 8-2 --- 1, из 8-2 в 8-1 --- 7. После контрольной по литературе из 8-3 ушло в другую школу 3 человека. Сколько человек учится теперь вместе во всех трёх классах?\\
337. Шесть специалистов по новой игре смогли сыграть каждый с каждым по одному разу за 6 часов (все игры продолжаются одинаковое время). При этом у них было два поля для игры. А сколько им потребовалось бы времени, если бы поле было одно?\\
338. Шесть специалистов по новой игре смогли сыграть каждый с каждым по одному разу за 10 часов (все игры продолжаются одинаковое время). При этом у них было два поля для игры. А сколько им потребовалось бы времени, если бы поле было одно?
\newpage
