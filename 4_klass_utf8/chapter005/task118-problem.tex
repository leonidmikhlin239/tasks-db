118.  На уроке рисования Алёна вырезала из клетчатой бумаги прямоугольник $20\times20$ клеток. Сидевший с ней Богдан случайно капнул на него синей краской, после чего ровно четверть клеток оказались запачканными. После этого, снова случайно, он пролил на него жёлтую краску. В результате оказалось, что клеток только с синими пятнами ровно в три раза меньше, чем клеток только с жёлтыми пятнами. Докажите, что чистых клеток ровно в три раза больше, чем сине-жёлтых (чистые клетки --- это клетки, на которых нет пятен краски, а сине-жёлтые --- это клетки, на которых есть и синие, и жёлтые пятна).\\
