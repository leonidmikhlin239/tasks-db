316. Семь четвероклассников на математической турнире получили по одинаковому количеству задач. Через 15 минут пятеро из них решили по 3 задачи. Оказалось, что теперь у этих пятерых вместе столько же нерешённых задач, сколько было выдано вместе двум остальным в начале турнира. Сколько задач получил изначально каждый участник? Один из участников заметил, что каждую следующую задачу он решает на 2 минуты дольше предыдущей. За какое время он решит все задачи, если этот прирост продолжится?\\
