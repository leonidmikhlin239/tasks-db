257. На распиливание длинного бревна уходит $5-1=4$ распила, а на распиливание короткого --- $4-1=3.$ Предположим, что все брёвна короткие, тогда на распиливание коротких брёвен ушло $3\cdot35=105$ распилов, а на распиливание длинных --- 0. Разница между распилами коротких брёвен и распилами длинных составляет $105-0=105.$ При замене одного короткого бревна на длинное количество распилов коротких брёвен уменьшается на 3, а количество распилов длинных увеличивается на 4, таким образом разница уменьшается на $3+4=7.$ Значит, необходимо сделать $105:7=15$ замен (разница должна стать равной 0), поэтому длинных брёвен должно быть 15, а следовательно общее количество распилов равно $20\cdot3\cdot2=120.$\\
