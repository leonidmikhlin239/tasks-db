315. Поднимаясь по эскалатору станции метро <<Василеостровская>>, четвероклассник заметил, что навстречу ему едут ученики одного известного физико-математического лицея (в галстуках), студенты одного известного университета (в пиджаках) и курсанты одной известной академии (в форме). А ещё там ехали совершенно обычные пассажиры. Четвероклассник стал всех считать. Оказалось, что лицеистов и студентов вместе ехало втрое больше, чем обычных пассажиров и курсантов вместе. При этом количество обычных пассажиров составляло половину от числа курсантов и треть от числа студентов. Сколько людей было в каждой из ехавших групп, если студентов было на 14 больше, чем обычных пассажиров?\\