67. Пусть на преодоление одного этажа тратится Э секунд, а на остановку --- О секунд. Тогда для поездок Пети и Тани верны следующие равенства: $10\text{Э}+3\text{О}=54\text{ с},\ 5\text{Э}+1\text{О}=23\text{ с}.$ Повторим поездку Тани два раза, получим равенство $10\text{Э}+2\text{О}=46\text{ с}.$ Его левая часть отличается от левой части первого равенства на одну остановку, а правая часть --- на 8 секунд. Значит, одна остановка занимает 8 секунд и
$5\text{Э}+8\text{ с}=23\text{ с},$ поэтому проезд одного этажа занимает 3 секунды. Тогда Коля будет ехать $8\cdot3+2\cdot8=40$ секунд.\\
