\section{Раздел 8: примеры без приёмов рациональных вычислений решения.}
Все вычисления этого раздела выполняются в столбик и не требуют дополнительных указаний, поэтому приведём к ним только ответы.
$\begin{array}{lllllllllll}
1.\ 3000109& 2.\ 6000083& 3.\ 274003& 4.\ 685002& 5.\ \text{Первое}& 6.\ \text{Второе}& 7.\ 2339& 8.\ 2339\\ 9.\ 10307000&
10.\ 30107000& 11.\ 60410& 12.\ 21460& 13.\ 243425& 14.\ 335344& 15.\ 2071&
16.\ 3051\\ 17.\ 117& 18.\ 128&
19.\ 8057& 20.\ 7506& 21.\ 105070& 22.\ 283& 23.\ 37341& 24.\ 28889\\
25.\ 68869& 26.\ 24849& 27.\ 12677& 28.\ 14139& 29.\ 15896& 30.\ 8028&
31.\ 15070& 32.\ 123000\\
33.\ 41750& 34.\ 2032& 35.\ 665150& 36.\ 29965& 37.\ 614000& 38.\ 10000000&
39.\ 122654& 40.\ 965817\\ 41.\ 31255& 42.\ 632896& 43.\ 3040& 44.\ 597254&
45.\ 51219& 46.\ 1049964& 47.\ 809& 48.\ 14357\\
49.\ 254633& 50.\ 408& 51.\ 280388& 52.\ 570900& 53.\ 1713686& 54.\ 84000&
55.\ 111777& 56.\ 45793\\ 57. 867904& 58.\ 2057& 59.\ 2014& 60.\ 20&
61.\ 112020& 62.\ 7003& 63.\ 146& 64.\ 480080\\
65.\ 730800& 66.\ 45766& 67.\ 61728& 68.\ 100508& 69.\ 100805& 70.\ 200309&
71.\ 300208& 72.\ 123 \end{array}$\\
73. 4\qquad
74. Сто тысяч двести семьдесят восемь.\qquad
75. Девяносто девять тысяч семьсот двадцать три.\\
$\begin{array}{lllllllllllll}
76.\ 11& 77.\ \cfrac{1}{2}& 78.\ 9 &79.\ 362& 80.\ 5& 81.\ 68908&
82.\ 653& 83.\ 14211&
84.\ 36& 85.\ 2640
\end{array}$\\
86. $(698\stackrel{3}{+}56\stackrel{1}{:}8\stackrel{2}{\cdot}9\stackrel{4}{-}486)\stackrel{5}{:}2\stackrel{6}{-}178.$
87. Первое.\quad 88. 683 \quad 89. 1057 \quad 90. 77 \quad 91. 2222 \quad 92. 60 \quad 93. 90 \quad 94. 90 \quad 95. 401104 \quad 96. 301103 \quad 97. 10201 \quad 98. 20101 \quad 99. 20 \quad 100. $C=822,\ A=815,\ B=758.$ \\ 101. 303 \quad 102. 149 \quad 103. 1002304 \quad 104. 1034002 \quad 105. 415 \quad 106. 376 \quad 107. 9 \quad 108. 25 \quad 109. 2017 \\ 110. 2018 \quad 111. 270068 \quad 112. 1754 \quad 113. 2003106 \quad 114. 2032006 \quad 115. 2483 \quad 116. 83829537\\
117. Найдём $A=36\cdot17-12\cdot14\cdot3+72:2=612-504+36=144,\ B=70\cdot16\cdot19:14:32\cdot2-(19\cdot4-15\cdot2)=95-46=49.$ Первое число может выражать площадь квадрата со стороной 12 см, а второе --- со стороной 7 см (так как $12\cdot12=144,\ 7\cdot7=49$).\\
$\begin{array}{lllllllllllll}
118.\ 15429& 119.\ 21904& 120.\ 98& 121.\ 97& 122.\ 254& 123.\ 363 & 124.\ 2550 & 125.\ 3240 & 126.\ 2533 \end{array}$\\
127. 4304 \ 128. 30573 \ 129. 32336 \ 130. 221 \ 131. 221 \ 132. 1005203 \ 133. 2003105
\newpage
