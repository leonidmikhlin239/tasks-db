\section{Раздел 1: примеры без приёмов рациональных вычислений}
Во всех примерах задача вычислить, если не написано иное.\\
1. Сто одиннадцать миллионов четыре тысячи тридцать три разделите на 37. Ответ запишите цифрами.\\
2. Двести двадцать два миллиона три тысячи семьдесят один разделите на 37. Ответ запишите цифрами.\\
3. Двадцать миллионов две тысячи двести девятнадцать разделите на 73. Ответ запишите цифрами.\\
4. Пятьдесят миллионов пять тысяч сто сорок шесть разделите на 73. Ответ запишите цифрами.\\
5. Что больше $36155457:351$ или $45338764:452?$\\
6. Что больше $43276279:431$ или $36819186:354?$\\
$\begin{array}{lll}
7.\ 47633:19+(70+30\cdot 411):50-416.&
8.\  39219:17+(40+60\cdot 511):50-582.\\
9.\  2461914383:239+6103.&
10.\  7194135893:239+6013.\\
11.\  380+2281140:38.&
12.\  890+1830730:89.\\
13.\  302010-58585.&
14.\  403020-67676.\\
15.\  109763:53.&
16.\  173907:57.\\
17.\  83+17\cdot 2.&
18.\ 86+14\cdot 3.\\
19.\ 273938:34.&
20.\ 322758:43.\\
21.\ (205\cdot104-74601:243)\cdot5+5.&
22.\ 183+(1677:43-888:12:2)\cdot50.\\
23.\ 453\cdot87-(756232+48998):389.&
24.\ 35620+26910:(150070-306\cdot490)-6938.\\
25.\ 74340+64680:(138250-406\cdot340)-5779.&
26.\ 27\cdot(30405-29496)+28764:94.\\
27.\ 43457-57\cdot(432+3456:32).&
28.\ 16179-720\cdot(15+19):12.\\
29.\ 8602:17+38\cdot(534-129).&
30.\ 37\cdot206+9338:(212-189).\\
31.\ 38\cdot402-(4127+5349):46.&
32.\ (701\cdot352-6148:58):2-323.\\
33.\ 167\cdot(5047:49+8232:56).&
34.\ (510:17+24)\cdot38-80:4.\\
35.\ 506384+3\cdot(53007-52275:615).&
36.\ 165165-37232:179\cdot650.\\
37.\ 2018\cdot131-18\cdot307+176\cdot2018.&
38.\ 2385388973:239+19293.\\
39.\ 465\cdot264-8904:(22\cdot308-6692).&
40.\  3456789-1243\cdot(74152:92+1198).\\
41.\  4283+1332\cdot(3069-2973):4-4996.&
42.\  700000-225\cdot307+315:35\cdot219.\\
43.\  (298230-305\cdot310):67.&
44.\ 562987+34267.\\
45.\ 56008-4789.&
46.\ 1483\cdot708.\\
47.\ 29933:37.&
48.\ (208896:68-2864)\cdot35+7077.\\
49.\  18081:9+627\cdot407-10260:4.&
50.\  (5080\cdot604-432\cdot7002+200):107.\end{array}$\\$\begin{array}{lll}
51.\ (112436-(468\cdot309-32543))\cdot(280388:367).\end{array}$\\
$\begin{array}{lll}
52.\ 5709000:10.&
53.\ 780087+933599.&
54.\ 31080000:370.\\
55.\ 79158+32619.&
56.\ 93756-47963.&
57.\ 284\cdot3056.\\
58.\ 748748:364.&
59.\  98404-126\cdot(397+86480:235).&
60.\ 7200:90-10\cdot6.\\
61.\ 8400:3\cdot40+20.&
62.\ 1386594:198.&
63.\ 2700:(150-90:6)+7\cdot18.\\
64.\ 24004:34\cdot680.&
65.\ 15428:38\cdot1800.&
66.\ (23199:57-22557:73)\cdot467.\end{array}$\\
$67.\ (60501:67-68595:85)\cdot643.$\\
68. Запишите цифрами число сто тысяч пятьсот восемь.\\
69. Запишите цифрами число сто тысяч восемьсот пять.\\
70. Запишите цифрами число двести тысяч триста девять.\\
71. Запишите цифрами число триста тысяч двести восемь.\\
72. Запишите и решите пример: <<Найдите произведение разности чисел 32 и 29 и суммы чисел 17 и 24>>.\\
73. Запишите и решите пример: <<Найдите частное суммы чисел 15 и 37 и разности чисел 29 и 16>>.\\
74. Витя сложил девяносто восемь тысяч двадцать восемь и две тысячи двести пятьдесят. Запишите цифрами выражение, которое составил Витя. Запишите словами, сколько получилось у Вити в итоге.\\
75. Маша вычла две тысячи триста девять из ста двух тысяч тридцати двух. Запишите цифрами выражение, которое составила Маша. Запишите словами, сколько получилось у Маши в итоге.\\
76. Какой остаток при делении на 19 даёт число 505?\\
77. $3:(3:2):4.$\qquad  \qquad                           78. $6:(1:6):4.$\\
79. Найдите делимое, если делитель равен 36, частное 10, остаток 2.\\
80. Найдите частное от деления утроенной суммы 483 и 2017 на произведение 5 и 300.\\
81. Найдите сумму удвоенного произведения суммы и разности чисел 150 и 86 и утроенного произведения этих же чисел.\\
82. Найдите сумму удвоенного произведения чисел 27 и 12 и частного от деления разности этих же чисел на 3.\\
83. Запишите число, состоящее из суммы 13 тысяч, 12 сотен и 11 единиц.\\
84. Чему равна утроенная половина четверти числа 96?\\
85. $205\cdot409+156738:519-81507.$\\
86. Расставьте порядок действий над примером. Вычислять НЕ надо.\\
$(698+56:8\cdot9-486):2-178.$\\
87. Что больше: миллиард десятков или тысяча миллионов?\\
88. $42\cdot13+137.$ 89. $28539:27.$\\ 90. Найдите число, седьмая часть которого равна трети числа 33. \\  91. $2020:(2+0-1+9) \times (20-1 \times 9).$\qquad
92. $2019:(700-30+3) \times (2020:101).$\\
93. Найдите сумму цифр у произведения $24024024024024\times33.$\\
94. Найдите сумму цифр у произведения $36 036 036 036 036 \times 22.$\\
95. $95863856:239.$ \quad 96. $71963617:239.$ \quad 97. $2438039 : 239$ \quad 98. $4804139 : 239$\\
99. Что получится, если 10 сотен разделить на 5 десятков?\\
100. Вычислите числа $A,\ B,\ C$ и расположите их в порядке убывания.\\
$A=4075:25\cdot5; \qquad B=3059-59\cdot39; \quad C=1860:(60-(37+18))+15\cdot30.$\\
101. $2200:(3\cdot37-11)+2\cdot141-1.$ \quad 102. $(((17+18)\cdot4+533)\cdot2+4):9-1.$ \quad 103. $23052992:23.$ \\ 104. $33088064:32.$
\quad 105. $93+7\cdot46.$ \quad 106. $94+6\cdot47.$\\
107. Вычислите три седьмых числа 21.\\
108. Вычислите пять девятых числа 45.\\
109. $68578:34$ \ 110. $86774:43$ \quad 111. $821 \cdot 340 - 9567 + 576 - 60993 : 753$ \\ 112. $1000 - 32032 : 208 + 551156 : 607$ \quad
113. $34052802 : 17$ \quad 114. $38608114 : 19$ \\ 115. $6040 \cdot 706 : 80 - (302000 - 39430) : (4960 : 80) \cdot 12$\\
116. $30000\cdot 3000 - 5908539 : (98 \cdot 89 - 132370 : 62) \cdot 6879$\\
117. Вычислите значения двух числовых выражений
$$\begin{array}{l} A=36\cdot17-12\cdot14\cdot3+72:2;\\
B=70\cdot16\cdot19:14:32\cdot2-(19\cdot4-15\cdot2).\end{array}$$
Может ли какое-то из этих чисел выражать в квадратных сантиметрах площадь квадрата, сторона которого выражена целым числом сантиметров? Чему в этом случае равна сторона квадрата?\\
$\begin{array}{lll}
118.\ 953 + 47 \cdot 308,& 119.\ 948 + 52 \cdot 403,& 120.\ 100 - (28 - 12) : (14 - 6),\\
121.\ 100 - (27 - 15) : (9 - 5),
\end{array}$\\
122. Из удвоенной суммы тридцати пяти и ста восьми вычли две трети разности пятидесяти и двух. Запишите полученное в результате число.\\
123. К утроенной разности ста сорока трёх и двадцати восьми прибавили три четверти суммы девятнадцати и пяти. Запишите полученное в результате число.\\
$\begin{array}{lll}
124.\ 51 \cdot 23 + 102 \cdot 31 : 2 - 3 \cdot 4 \cdot 17,& 125.\ 54 \cdot 21 - 18 \cdot 5 \cdot 3 + 108 \cdot 44 : 2,\\
126.\ 7462-4929,& 127.\ 8563-4259,\\
128.\ 387\cdot79,& 129.\ 376\cdot86,\\
130.\ 84422:382,& 131.\ 91715:415,\\
132.\ 57296571 : 57, & 133.\ 114176985 : 57.
\end{array}$
\newpage
