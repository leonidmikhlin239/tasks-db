140. Раз расстояние между колобками через 30 секунд после столкновения стало равно 26 метров, скорость их удаления (равная сумме их скоростей) равна 52м/мин. Если скорость более медленного колобка равна $x,$ то скорость более быстрого --- $x+4$ и $x+x+4=52,\ 2x=48,\ x=24$м/мин. Значит, более быстрый колобок катится со скоростью $24+4=28$м/мин. Он удалялся от своей избушки 4 минуты и приближался к ней 30 секунд, то есть можно считать, что он удалялся 3 минуты и 30 секунд. Тогда он окажется на расстоянии $3\cdot28+28:2=98$м.\\