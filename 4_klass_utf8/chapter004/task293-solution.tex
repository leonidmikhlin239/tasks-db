293. После проведения вертикального разреза сумма периметров дух полученных прямоугольников будет больше периметра исходного прямоугольника на удвоенную вертикальную сторону. Значит, длина удвоенной вертикальной стороны равна $45-38=7$см. Так как периметр всего прямоугольника равен 38 см, длина удвоенной горизонтальной стороны равна $38-7=31$см. Периметр прямоугольника, полученного после горизонтального разреза, равен длине удвоенной горизонтальной стороны, сложенной с длиной одной вертикальной стороны, то есть $31\text{ см}+7\text{ см}:2=345$мм.\\