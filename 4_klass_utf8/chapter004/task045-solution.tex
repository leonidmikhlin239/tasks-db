45. Посчитаем, сколько времени поезд тратит на поездку из Ярославля в Лучегорск и обратно. С $04-45$ до $19-05$ проходит 14 часов 20 минут, а с $8-25$ до $8-45$ проходит 20 минут. На самом деле, конечно, в обоих случаях проходит ещё некоторое количество полных суток, для нас важно то, что это количество одинаковое в обоих случаях (поезд едет по одному маршруту с одинаковой скоростью). Поэтому разница в потраченном на эти поездки времени составляет $14:20-0:20=14$ часов. За счёт чего появилась эта разница, если поезд едет по тому же маршруту с той же скоростью? Всё дело в том, что Лучегорск находится восточнее Ярославля, соответственно лежит в другом часовом поясе и когда поезд приезжает из Ярославля в Лучегорск, разница во времени прибавляется к текущему времени, а на пути назад, наоборот, вычитается. Поэтому и получилось, что на поездку из Ярославля в Лучегорск поезд тратит на 14 часов больше, чем на обратную дорогу. Если поезд проводит в дороге $t$ часов, а разница во времени составляет $x$ часов, то верно соотношение: $(t+x)-(t-x)=14,\ 2x=14, x=7$ч. Таким образом, разница во времени между Ярославлем и Лучегорском составляет 7 часов.\\
