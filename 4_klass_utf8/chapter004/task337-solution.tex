337. Нам подходят кубики, у которых окрашено 0 или 2 грани (больше быть не может). Первые кубики --- это кубики внутри параллелепипеда, их $2\cdot3\cdot3=18$ штук. Вторые кубики --- это кубики, расположенные вдоль рёбер, кроме первого и последнего (у них покрашено по 3 грани). Таких кубиков $(2+3+3)\cdot4=32.$ Значит, всего подходящих кубиков $18+32=50$ штук.\\
