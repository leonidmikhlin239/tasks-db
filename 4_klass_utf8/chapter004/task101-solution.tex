101. Легковой автомобиль догоняет грузовик со скоростью $90-60=30$км/ч. Тогда первая их встреча произойдёт через $120:30=4$ч. За это время легковой автомобиль проедет $4\cdot90=360$км, а значит расстояние от города до деревни равно $360+90=450$км. Легковой автомобиль доедет до деревни ещё через $90:90=1$ч, за это время грузовик проедет $60\cdot1=60$км. Тогда к тому моменту, как легковой автомобиль в деревне развернётся и поедет назад, расстояние между ними будет равно $90-60=30$км. Скорость их сближения будет равна $90+60=150$км/ч. Расстояние между ними равно одной пятой от скорости их сближения, а значит до встречи пройдёт одна пятая часть часа, то есть 12 минут. Таким образом, расстояние между городом и деревней равно 450 км, а до второй встречи пройдёт $4\text{ч}+1\text{ч}+12\text{мин}=5\text{ч}12\text{мин}.$\\
