212. Так как $1764:54=32$ (ост. 36), Суслик сделает $32\cdot60+60:3\cdot2=1960$ шагов, из которых $1960:2=980$ будут сделаны правой лапой. В свою очередь Хома сделает сначала $32\cdot80=2560$ шагов, после чего ему останется пройти ещё 36 метров. На 54 метра он делает 80 шагов, 36 составляет две трети от 54, а  $80\cdot2:3=53$ (ост. 1). Значит, всего он сделает $2560+53=2613$ полных шагов, из которых 1306 будут сделаны правой лапой, так как он начал с левой. Значит, Хома сделает на $1306-980=326$ шагов больше.\\
