76. Раз аквариум заполнен наполовину, объём воды в нём равен $120:2=60\text{л}=60\text{дм}^3.$ Вода, разлитая по комнате, примет форму прямоугольного параллелепипеда, длина и ширина которого равны длине и ширине комнаты, а высота равна искомой высоте слоя воды. Так как объём прямоугольного параллелепипеда вычисляется по формуле $V=a\cdot b\cdot c,$ для нахождения его высоты необходимо поделить объём на произведение длины и ширины: $c=60\text{дм}^3:(3\text{м}\cdot4\text{м})=60\text{дм}^3:12\text{м}^2=60\text{дм}^3:1200\text{дм}^2=60000\text{см}^3:120000\text{см}^2=60000000\text{мм}^3:12000000\text{мм}^2=
5\text{мм}.$ Перевод в более мелкие единицы необходимо было осуществлять до тех пор, пока делимое не станет больше делителя. Для упрощения записи одинаковое количество нулей в делимом и делителе можно в процессе зачёркивать.\\