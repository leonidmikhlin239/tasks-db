241. Раз 27 мая пришлось на один и тот же день недели, количество прошедших дней должно было делиться на 7. В обычном году 365 дней, число 365 даёт при делении на 7 остаток 1. В високосном году 366 дней, что даёт остаток 2. Необходимо сложить наименьшее количество остатков так, чтобы получить в результате 7, и при этом остатки 2 можно использовать только через три (високосным год бывает раз в четыре года). Этого можно добиться, сложив $2+1+1+1+2=7.$ Значит, наименьшая разница в годах равна 5 лет. \\
