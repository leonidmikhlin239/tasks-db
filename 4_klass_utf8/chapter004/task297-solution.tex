297. За 6 дней Васины часы уйдут вперёд на 1 час, а за 4 дня Петины часы отстанут на 1 час. Значит, за 12 дней Васины часы уходят вперёд на 2 часа, а Петины --- отстают на 3 часа, то есть разница между ними увеличивается на 5 часов. Изначально разница между часами составляет 4 часа, а чтобы они показывали одинаковое время, она должна составлять 24 часа, то есть должна 4 раза увеличиться на 5 часов, что произойдёт по прошествии $4\cdot12=48$ дней. Если к 10 мая прибавить 48 дней, получится <<58 мая>>, то есть $58-31=27$ июня. Значит, одинаковое время часы покажут в полдень 27 июня.\\
