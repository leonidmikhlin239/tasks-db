144. Мастер складывал паркет из дощечек трёх видов: квадратной формы площадью $64\text{см}^2,$ прямоугольной и треугольной формы (при этом у треугольных дощечек две стороны равны между собой). Мастер приложил к квадратной дощечке прямоугольную стороной, равной стороне квадратной дощечки, и получил прямоугольник площадью $96\text{см}^2.$ Большая сторона полученного прямоугольника оказалось равной одной из сторон треугольной дощечки. Найдите стороны треугольной дощечки, если её периметр равен периметру полученного прямоугольника. Рассмотрите разные случаи.\\