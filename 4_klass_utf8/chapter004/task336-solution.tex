336. Нам подходят кубики, у которых окрашено 0 или 2 грани (больше быть не может). Первые кубики --- это кубики внутри параллелепипеда, их $2\cdot2\cdot3=12$ штук. Вторые кубики --- это кубики, расположенные вдоль рёбер, кроме первого и последнего (у них покрашено по 3 грани). Таких кубиков $(2+2+3)\cdot4=28.$ Значит, всего подходящих кубиков $12+28=40$ штук.\\
