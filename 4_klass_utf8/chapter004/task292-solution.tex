292. После проведения вертикального разреза сумма периметров дух полученных прямоугольников будет больше периметра исходного прямоугольника на удвоенную вертикальную сторону. Значит, длина удвоенной вертикальной стороны равна $45-36=9$см. Так как периметр всего прямоугольника равен 36 см, длина удвоенной горизонтальной стороны равна $36-9=27$см. Периметр прямоугольника, полученного после горизонтального разреза, равен длине удвоенной горизонтальной стороны, сложенной с длиной одной вертикальной стороны, то есть $27\text{ см}+9\text{ см}:2=315$мм.\\
