44. Посчитаем, сколько времени поезд тратит на поездку из Москвы в Талдан и обратно. С $00-35$ до $18-15$ проходит 17 часов 40 минут, а с $8-23$ до $14-03$ проходит 5 часов 40 минут. На самом деле, конечно, в обоих случаях проходит ещё некоторое количество полных суток, для нас важно то, что это количество одинаковое в обоих случаях (поезд едет по одному маршруту с одинаковой скоростью). Поэтому разница в потраченном на эти поездки времени составляет $17:40-5:40=12$ часов. За счёт чего появилась эта разница, если поезд едет по тому же маршруту с той же скоростью? Всё дело в том, что Талдан находится восточнее Москвы, соответственно лежит в другом часовом поясе и когда поезд приезжает из Москвы в Талдан, разница во времени прибавляется к текущему времени, а на пути назад, наоборот, вычитается. Поэтому и получилось, что на поездку из Москвы в Талдан поезд тратит на 12 часов больше, чем на обратную дорогу. Если поезд проводит в дороге $t$ часов, а разница во времени составляет $x$ часов, то верно соотношение: $(t+x)-(t-x)=12,\ 2x=12, x=6$ч. Таким образом, разница во времени между Москвой и Талданом составляет 6 часов.\\
