333. Май содержит 4 полных недели и ещё три дня. Раз пятниц было больше, чем сред, пятница в эти три дня попала, а среда --- нет. Это могло произойти только если пятница была 29.05 или 30.05. Тогда в первом случае 1.05 было также пятницей, а во втором --- четвергом (прошло 28 и 29 дней соответственно). Построим цепочку для невисокосного и високосного года: $1.05\stackrel{-30}{\rightarrow}1.04\stackrel{-31}{\rightarrow}1.03
\stackrel{-28(-29)}{\rightarrow}1.02\stackrel{-2}{\rightarrow}30.01.$
В невисокосном году надо "отмотать назад" $30+31+28+2=91$ день, а в високосном --- $30+31+29+2=92.$ Так как 91 делится на 7, а 92 даёт остаток 1, в первом случае 30.01 могло быть пятницей или четвергом, а во втором --- четвергом или средой. Значит, все возможные ответы --- это среда, четверг и пятница.\\