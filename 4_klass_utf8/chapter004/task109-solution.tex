109. Валера потратил 10г краски на $10\cdot10\cdot6=600\text{см}^2.$ Вася распилил большой куб на $10\cdot10\cdot10=1000$ маленьких кубиков, общая площадь поверхности которых равна $1\cdot1\cdot6\cdot1000=6000\text{см}^2,$ из которых $600\text{см}^2$ уже покрашены, а значит окрасить надо ещё $6000-600=5400\text{см}^2.$ Эта площадь больше в $5400:600=9$ раз, а значит краски потребуется больше в 9 раз: $10\cdot9=90$г.\\
