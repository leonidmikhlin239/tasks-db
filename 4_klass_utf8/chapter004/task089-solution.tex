89. Второй человек вышел на 2 часа позже, первый за это время успел уйти на $6\cdot2=12$км. Всадник вёз письмо 30 минут, второй за это время пройдёт $6:2=3$км. Значит, всадник проехал оставшиеся $12-3=9$км за 30 минут, то есть он проезжает $9\cdot2=18$км/ч. Расстояние между первым и вторым человеком за это время не поменялось, так как они идут с одинаковой скоростью. За те 40 минут, что он ждал ответа, первый человек увеличит расстояние между ними на $2\cdot2=4$км (первый идёт со скоростью 6км/ч, то есть 2км/20мин) и оно будет составлять $12+4=16$км. Всадник будет догонять первого человека со скоростью $18\text{км/ч}-6\text{км/ч}=12\text{км/ч}$ или 2км/10 мин. Поэтому на доставку ответа ему понадобится $16\text{км}:2\text{км/10мин}=8\cdot10\text{мин}=80\text{мин}=$1ч 20 мин.\\
