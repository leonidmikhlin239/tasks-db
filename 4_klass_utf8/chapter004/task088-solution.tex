88. Второй человек вышел на 3 часа позже, первый за это время успел уйти на $4\cdot3=12$км. Всадник вёз письмо 45 минут, второй за это время пройдёт $3\cdot1=3$км (второй идёт со скоростью 4км/ч, то есть 1км/15мин). Значит, всадник проехал оставшиеся $12-3=9$км за 45 минут, то есть он проезжает 3км за 15мин или 12км за 1ч. Расстояние между первым и вторым человеком за это время не поменялось, так как они идут с одинаковой скоростью. За те полчаса, что он ждал ответа, первый человек увеличит расстояние между ними на $4:2=2$км и оно будет составлять $12+2=14$км. Всадник будет догонять первого человека со скоростью $12\text{км/ч}-4\text{км/ч}=8\text{км/ч}$ или 2км/15 мин. Поэтому на доставку ответа ему понадобится $14\text{км}:2\text{км/15мин}=7\cdot15\text{мин}=105\text{мин}=$1ч 45 мин.\\
