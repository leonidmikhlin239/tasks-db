217. Если все измерения куска мыла уменьшились в 2 раза, его объём, равный произведению трёх его измерений, уменьшился в $2\cdot2\cdot2=8$ раз. Значит, после семи стирок от куска мыла осталась одна восьмая его часть, то есть израсходовалось семь восьмых и на одну стирку тратится одна восьмая часть. Поэтому оставшегося куска хватит на одну последнюю стирку.\\