121. Всего до веломагазина пешеход шёл $2:30+1:30=4$ часа. Так как его скорость на велосипеде в 3 раза больше, на обратный путь он потратит в 3 раза меньше времени: $4\text{ч}:3=240\text{мин}:3=80\text{мин}=1:20.$ Тогда домой он приедет в $10:00+2:30+1:00+1:30+0:30+1:20=16:50.$\\