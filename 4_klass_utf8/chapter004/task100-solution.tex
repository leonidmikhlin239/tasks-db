100. Автобус догоняет самосвал со скоростью $100-80=20$км/ч. Тогда первая их встреча произойдёт через $60:20=3$ч. За это время автобус проедет $3\cdot100=300$км, а значит расстояние от санатория до порта равно $300+300=600$км. Автобус доедет до порта ещё через $300:100=3$ч, за это время самосвал проедет $80\cdot3=240$км. Тогда к тому моменту, как автобус в порту развернётся и поедет назад, расстояние между ними будет равно $300-240=60$км. Скорость их сближения будет равна $100+80=180$км/ч. Расстояние между ними равно трети от скорости их сближения, а значит до встречи пройдёт треть часа, то есть 20 минут. Таким образом, расстояние между санаторием и портом равно 600 км, а до второй встречи пройдёт $3\text{ч}+3\text{ч}+20\text{мин}=6\text{ч}20\text{мин}.$\\