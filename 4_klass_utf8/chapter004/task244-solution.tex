244. Собака удаляется от хозяина со скоростью $30-6=24$км/ч, а сближается с ним со скоростью $12+6=18$км/ч. На какое расстояние собака от хозяина убежала, на такое же расстояние она с ним в итоге и сблизилась. Пусть убегала она $x$ минут, а сближалась $y$ минут, тогда $24x=18y,\ 4x=3y,$ при этом $x+y=7.$ Число 7 в таком соотношении можно поделить единственным способом: $x=3,\ y=4.$ Тогда всего она пробежала $30\text{км/ч}\cdot3\text{мин}+12\text{км/ч}\cdot4\text{мин}=500\text{м/мин}\cdot3\text{мин}+
200\text{м/мин}\cdot4\text{мин}=2300$м.\\
