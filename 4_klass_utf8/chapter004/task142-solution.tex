142. Пусть за первый промежуток времени Баба Яга пролетела $x$км, тогда за второй она пролетела $x+10$км, а за третий --- $x+10+20=x+30$км, и верно равенство $x+x+10+x+30=70,\ 3x=30,\ x=10$км. Так как все промежутки времени были равными, они составляли $45:3=15$мин. Таким образом, на втором промежутке Баба Яга пролетела $10+10=20$км за 15 минут, поэтому её скорость равна $20\cdot4=80$км/ч.\\
