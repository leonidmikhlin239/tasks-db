59. Посчитаем количество игр девочек друг с другом. Первая сыграла со всеми остальными 3 игры, вторая --- ещё 2 (игра с первой девочкой уже посчитана), аналогично третья ещё 1, а все игры четвёртой девочки уже окажутся посчитаны. Значит, всего девочки сыграли друг с другом $3+2+1=6$ игр, в которых они независимо от результатов этих игр набрали на всех $6\cdot2=12$ очков (если одна девочка выигрывает у другой, они получают суммарно $2+0=2$ очка, а если они играют вничью, то также получают суммарно $1+1=2$ очка). Каждый из 6 мальчиков сыграл со всеми 4 девочками, значит всего мальчики с девочками провели $6\cdot4=24$ партии. В них было разыграно $24\cdot2=48$ очков, из которых девочки получили $40-12=28$ очков, а мальчики --- оставшиеся $48-28=20.$ Когда мальчик с девочкой играют вничью, разница между количеством очков, набранных всеми девочками, и количеством очков, набранных всеми мальчиками, не меняется. Победа девочки увеличивает эту разницу на 2 очка, а победа мальчика на те же 2 очка уменьшает. Так как итоговая разница оказалась равна $28-20=8$ очкам, девочки одержали на $8:2=4$ победы больше.\\
