308. Пусть в 2020 году Кирилл потратил $2x$ минут на решение задач сидя, $2y$ минут на решение задач бегом и $2z$ минут на решение задач на велосипеде. Тогда
$2x+2y+2z=3$ч, откуда $x+y+z=180:2=90$мин. В 2021 году он потратил $x+2y+2z=140$мин, значит $y+z=140-90=50$мин, $x=90-50=40$мин. В 2022 году он потратил $2x+y+2z=155$мин, значит $z=155-90-40=25$мин, а $y=50-25=25$мин. Тогда в 2023 году он потратит $2x+2y+z=2\cdot40+2\cdot25+25=155$мин или 2 часа 35 минут.\\
