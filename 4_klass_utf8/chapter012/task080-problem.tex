80. Пусть верблюмот весит $x$кг, тогда слонопотам с одной стороны весит $5x$кг, а с другой стороны --- $x+100$кг. Значит, $5x=x+100,\ 4x=100,\ x=25$кг. Значит, верблюмот весит 25 кг, а слонопотам --- $5\cdot25=125$ кг, вместе они весят $25+125=150$кг. Пусть кошкалот весит $y$кг, тогда антиконда весит $y+16$кг и верно равенство $y+y+16=150,\ 2y+16=150,\ 2y=134,\ y=67$кг. Таким образом, антиконда весит $67+16=83$кг и в порядке убывания животные располагаются следующим образом: слонопотам 125 кг, антиконда 83 кг, кошкалот 67 кг, верблюмот 25 кг.\\
