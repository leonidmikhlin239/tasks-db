60. Посчитаем количество игр мальчиков друг с другом. Первый сыграл со всеми остальными 5 игр, второй --- ещё 4 (игра с первым мальчиком уже посчитана), аналогично третий ещё 3 и так далее. Значит, всего мальчики сыграли друг с другом $5+4+3+2+1=15$ игр, в которых они независимо от результатов этих игр набрали на всех $15\cdot2=30$ очков (если один мальчик выигрывает у другого, они получают суммарно $2+0=2$ очка, а если они играют вничью, то также получают суммарно $1+1=2$ очка). Каждый из 6 мальчиков сыграл со всеми 4 девочками, значит всего мальчики с девочками провели $6\cdot4=24$ партии. В них было разыграно $24\cdot2=48$ очков, из которых мальчики получили $40-30=10$ очков, а девочки --- оставшиеся $48-10=38.$ Когда мальчик с девочкой играют вничью, разница между количеством очков, набранных всеми девочками, и количеством очков, набранных всеми мальчиками, не меняется. Победа девочки увеличивает эту разницу на 2 очка, а победа мальчика на те же 2 очка уменьшает. Так как итоговая разница оказалась равна $38-10=28$ очкам, девочки одержали на $28:2=14$ побед больше.\\
