119. Нарисуем круги Эйлера для множеств марок Дениски, Мишки и Алёнки. У Дениски и Мишки нет одинаковых марок, значит в пересечении их множеств и в пересечении всех трёх множеств марок 0, как и уникальных марок у Алёнки. Пусть у Дениски и Алёнки одинаковых марок $x,$ тогда столько же уникальных марок у Мишки. Пусть у Дениски уникальных марок $y,$ тогда у Алёнки с Мишкой одинаковых марок тоже должно быть $y$ (так как у Дениски и Мишки марок поровну). Таким образом, у всех детей по $x+y$ марок, значит у Алёнки столько же марок, сколько у Дениски.\\
