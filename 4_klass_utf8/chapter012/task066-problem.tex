66. Пусть на преодоление одного этажа тратится Э секунд, а на остановку --- О секунд. Тогда для поездок Пети и Тани верны следующие равенства: $12\text{Э}+3\text{О}=57\text{ с},\ 6\text{Э}+1\text{О}=25\text{ с}.$ Повторим поездку Тани два раза, получим равенство $12\text{Э}+2\text{О}=50\text{ с}.$ Его левая часть отличается от левой части первого равенства на одну остановку, а правая часть --- на 7 секунд. Значит, одна остановка занимает 7 секунд и
$6\text{Э}+7\text{ с}=25\text{ с},$ поэтому проезд одного этажа занимает 3 секунды. Тогда Коля будет ехать $10\cdot3+2\cdot7=44$ секунды.\\
