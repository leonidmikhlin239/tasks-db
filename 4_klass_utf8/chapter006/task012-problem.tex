12. На прямой отмечено 180 точек так, что расстояние между любыми соседними точками равно 5 см. Чему равно расстояние между крайними точками?\\
