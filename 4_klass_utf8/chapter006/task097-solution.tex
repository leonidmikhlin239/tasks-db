97. Между первым и последним поездом находится $12-1=11$ интервалов. Если интервал равен 6 минут или меньше, то между первым и последним поездом пройдёт не более, чем $6\cdot11=66$ минут и за 90 минут Марк точно должен увидеть ещё хотя бы один поезд. Если интервал равен 9 минут, то между первым и последним поездом пройдёт хотя бы $9\cdot11=99$ минут и Марк не успеет за 90 минут увидеть 8 поездов. Покажем, как интервал может быть 7 или 8 минут. Пусть интервал равен 7 минут, тогда разобьём 90 минут на три интервала: $6,5+77+6,5=90.$ Если первый поезд проехал мимо Марка через 6,5 минут после начала наблюдения, он увидит ровно 8 поездов за $7\cdot11=77$ минут. Пусть интервал равен 8 минут, тогда разобьём 90 минут на три интервала: $1+88+1=90.$ Если первый поезд проехал мимо Марка через 1 минуту после начала наблюдения, он увидит ровно 8 поездов за $8\cdot11=88$ минут.\\
