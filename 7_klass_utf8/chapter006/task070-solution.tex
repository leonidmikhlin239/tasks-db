71. а) $\cfrac{x}{x^2+2x+4}+\cfrac{x^2+8}{x^3-8}-\cfrac{1}{x-2}=\cfrac{x}{x^2+2x+4}+\cfrac{x^2+8}{(x-2)(x^2+2x+4)}-\cfrac{1}{x-2}=$\\$
\cfrac{x^2-2x+x^2+8-x^2-2x-4}{(x-2)(x^2+2x+4)}=\cfrac{x^2-4x+4}{(x-2)(x^2+2x+4)}=\cfrac{(x-2)^2}{(x-2)(x^2+2x+4)}=\cfrac{x-2}{x^2+2x+4}.$\\
$\cfrac{x^2}{x^2-4}-\cfrac{2}{2-x}=\cfrac{x^2}{(x-2)(x+2)}+\cfrac{2}{x-2}=\cfrac{x^2+2x+4}{(x-2)(x+2)}.$\\
$\cfrac{x-2}{x^2+2x+4}\cdot\cfrac{x^2+2x+4}{(x-2)(x+2)}=\cfrac{1}{x+2}.$\\
б) $y=\cfrac{1}{f(x)}=x+2,$ при этом нельзя брать $x=\pm2,$ так как эти значения обнуляли знаменатель.
$$\begin{tikzpicture}[scale=0.2]
\tikzset {line01/.style={line width =0.5pt}}
\tikzset{line02/.style={line width =1pt}}
\tikzset{line03/.style={dashed,line width =0.5pt}}
%\filldraw [black] (0,0) circle (1pt);
\draw [->] (-10,0) -- (10,0);
\draw [->] (0,-10) -- (0,10);
\draw[line01] (-6,-4) -- (8,10);
\draw (10.2,0.7) node {\scriptsize $x$};
\draw (-1.9,-1.5) node {\scriptsize $-2$};
\draw (2,-0.7) node {\scriptsize $2$};
\draw (-0.7,4) node {\scriptsize $4$};
%\draw (0.7,6) node {\scriptsize $6$};
%\draw (-5.5,-2) node {\scriptsize $-6$};
\draw[line03] (2,0) -- (2,4);
\draw[line03] (0,4) -- (2,4);
\draw (0.7,10.2) node {\scriptsize $y$};
\draw (2,4) circle (10pt);
\draw (-2,0) circle (10pt);
\end{tikzpicture}$$
