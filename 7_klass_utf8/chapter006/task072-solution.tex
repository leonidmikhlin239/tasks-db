73. а) Найдём прямую $AC:\ \begin{cases} b=4,\\ -3k+b=-2.\end{cases}\Leftrightarrow\begin{cases} b=4,\\ k=2.\end{cases}\Rightarrow y=2x+4.$
Найдём прямую $BD:\ \begin{cases} k+b=-4,\\ -20k+b=59.\end{cases}\Leftrightarrow\begin{cases} 21k=-63,\\ -20k+b=59.\end{cases}
\Leftrightarrow\begin{cases} k=-3,\\ b=-1.\end{cases}\Rightarrow y=-3x-1.$\\
б) Найдём точку пересечения прямых $AC$ и $BD: 2x+4=-3x-1,\ 5x=-5,\ x=-1,\ y=2\cdot(-1)+4=2.$ Найдём прямую $BC:\ \begin{cases} k+b=-4,\\ -3k+b=-2.\end{cases}\Leftrightarrow\begin{cases} 4k=-2,\\ -3k+b=-2.\end{cases}\Leftrightarrow\begin{cases} k=-\cfrac{1}{2},\\ b=-\cfrac{7}{2}.\end{cases}\Rightarrow y=-\cfrac{1}{2}x-\cfrac{7}{2}.$ Точка этой прямой, лежащая на оси абсцисс, имеет ординату $y=0,$ а значит $-\cfrac{1}{2}x-\cfrac{7}{2}=0,\ x=-7.$ Таким образом, необходимо найти уравнение прямой, проходящей через точки $(-1;2)$ и $(-7;0): \ \begin{cases} -k+b=2,\\ -7k+b=0.\end{cases}\Leftrightarrow\begin{cases} 6k=2,\\ -7k+b=0.\end{cases}\Leftrightarrow\begin{cases} k=\cfrac{1}{3},\\ b=\cfrac{7}{3}.\end{cases}\Rightarrow y=\cfrac{1}{3}x+\cfrac{7}{3}.$\\
