21. Так как прямые идут <<вправо--вверх>>, значения $k_1$ и $k_2$ положительны, при этом наклон прямой $l_2$ круче, значит $k_2>k_1.$ Так как точки пересечения прямых с осью ординат расположены выше оси абсцисс, значения $b_1$ и $b_2$ положительны, при этом точка пересечения прямой $l_2$ находится выше, значит $b_2>b_1.$ Таким образом, $k_2b_2>k_1b_1.$\\
