2. $\begin{tikzpicture}[scale=0.6]
\tikzset {line01/.style={line width =0.5pt}}
\tikzset{line02/.style={line width =1pt}}
\tikzset{line03/.style={dashed,line width =0.9pt}}
\filldraw [black] (0,0) circle (1pt);
\draw [->] (-2,0) -- (2,0);
\draw [->] (0,-2) -- (0,2);
\draw[line01] (-1.5,2) -- (2,-0.5);
\draw (2,0.25) node {\scriptsize $x$};
\draw (0.25,2) node {\scriptsize $y$};
\end{tikzpicture}$
На рисунке изображён график функции вида $y=kx+b.$ Для этого графика ответьте на вопросы: а) Каков знак коэффициента $k?$ б) Проходит ли график через точку $(-1;b)?$\\ в) Пусть $k=2; b=1.$ Проходит ли график функции через точку $(1,73;3,47)?$\\
