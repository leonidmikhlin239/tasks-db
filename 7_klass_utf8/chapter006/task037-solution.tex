37. Найдём точку пересечения прямых $y=2x+3$ и $y=8x+7:\ 2x+3=8x+7,\ -6x=4,\ x=-\cfrac{2}{3},\ y=2\cdot\left(-\cfrac{2}{3}\right)+3=\cfrac{5}{3}.$ Тогда необходимо построить график прямой $y=-6\cdot\left(-\cfrac{2}{3}\right)x+3\cdot\cfrac{5}{3}-9=4x-4.$ Построим его по двум точкам $(0;-4)$ и $(1;0).$
$$\begin{tikzpicture}[scale=0.2]
\tikzset {line01/.style={line width =0.5pt}}
\tikzset{line02/.style={line width =1pt}}
\tikzset{line03/.style={dashed,line width =0.5pt}}
%\filldraw [black] (0,0) circle (1pt);
\draw [->] (-10,0) -- (10,0);
\draw [->] (0,-10) -- (0,10);
\draw[line01] (3,8) -- (-1.5,-10);
%\draw[line03] (-1,1) -- (0,1);
%\draw[line03] (-1,0) -- (-1,1);
%\draw[line01] (0,-3) -- (-2,5);
%\draw (0.6,-4) node {\tiny $-4$};
%\draw (-1.6,-0.7) node {\tiny $-1$};
\draw (10.2,0.7) node {\scriptsize $x$};
\draw (-1,-4) node {\tiny $-4$};
\draw (1.1,-1) node {\tiny $1$};
\draw (0.7,10.2) node {\scriptsize $y$};
\end{tikzpicture}$$
