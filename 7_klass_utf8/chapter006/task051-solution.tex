51. Найдём точку пересечения прямых, заданных уравнениями $y=-2x+4$ и $y=\cfrac{1}{2}x+1:\ -2x+4=\cfrac{1}{2}x+1,\ -\cfrac{5}{2}x=-3,\ x=\cfrac{6}{5},\ y=-\cfrac{12}{5}+4=\cfrac{8}{5}.$ Раз прямая $y=kx+b$ параллельна прямой $y=-\cfrac{1}{3}x-21,$ её коэффициент $k=-\cfrac{1}{3}.$ Подставим в её уравнение координаты найденной точки: $\cfrac{8}{5}=-\cfrac{1}{3}\cdot\cfrac{6}{5}+b,\ b=2.$
Проведём прямую через две точки $\left(0;2\right)$ и $(6;0).$
$$\begin{tikzpicture}[scale=0.2]
\tikzset {line01/.style={line width =0.5pt}}
\tikzset{line02/.style={line width =1pt}}
\tikzset{line03/.style={dashed,line width =0.5pt}}
%\filldraw [black] (0,0) circle (1pt);
\draw [->] (-10,0) -- (10,0);
\draw [->] (0,-10) -- (0,10);
\draw[line01] (9,-1) -- (-6,4);
%\draw[line03] (-1,1) -- (0,1);
%\draw[line03] (-1,0) -- (-1,1);
%\draw[line01] (0,-3) -- (-2,5);
%\draw (0.6,-4) node {\tiny $-4$};
%\draw (-1.6,-0.7) node {\tiny $-1$};
\draw (10.2,0.7) node {\scriptsize $x$};
\draw (0.5,2.3) node {\tiny $2$};
\draw (6,-0.5) node {\tiny $6$};
\draw (6,-4) node {\scriptsize $y=-\frac{1}{3}x+2$};
\draw (0.7,10.2) node {\scriptsize $y$};
\end{tikzpicture}$$
