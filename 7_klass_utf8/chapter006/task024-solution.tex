24. Построим график по двум точкам $(-1;1)$ и $(0;-4).$
$$\begin{tikzpicture}[scale=0.2]
\tikzset {line01/.style={line width =0.5pt}}
\tikzset{line02/.style={line width =1pt}}
\tikzset{line03/.style={dashed,line width =0.5pt}}
%\filldraw [black] (0,0) circle (1pt);
\draw [->] (-10,0) -- (10,0);
\draw [->] (0,-10) -- (0,10);
\draw[line01] (1,-9) -- (-2,6);
\draw[line03] (-1,1) -- (0,1);
\draw[line03] (-1,0) -- (-1,1);
%\draw[line01] (0,-3) -- (-2,5);
\draw (0.6,-4) node {\tiny $-4$};
\draw (-1.6,-0.7) node {\tiny $-1$};
\draw (10.2,0.7) node {\scriptsize $x$};
\draw (0.7,1) node {\tiny $1$};
\draw (0.7,10.2) node {\scriptsize $y$};
\end{tikzpicture}$$
Значение $y=16$ достигается при $x=-4,$ а так как функция убывает, неравенство $y\geqslant16$ выполняется при $x\leqslant-4.$\\
