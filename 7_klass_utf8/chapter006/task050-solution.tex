50. Найдём точку пересечения прямых, заданных уравнениями $y=2x-5$ и $y=-x+4:\ 2x-5=-x+4,\ 3x=9,\ x=3,\ y=-3+4=1.$ Раз прямая $y=kx+b$ параллельна прямой $y=-\cfrac{1}{2}x+11,$ её коэффициент $k=-\cfrac{1}{2}.$ Подставим в её уравнение координаты найденной точки: $1=-\cfrac{1}{2}\cdot3+b,\ b=\cfrac{5}{2}.$
Проведём прямую через две точки $\left(0;\cfrac{5}{2}\right)$ и $(5;0).$
$$\begin{tikzpicture}[scale=0.2]
\tikzset {line01/.style={line width =0.5pt}}
\tikzset{line02/.style={line width =1pt}}
\tikzset{line03/.style={dashed,line width =0.5pt}}
%\filldraw [black] (0,0) circle (1pt);
\draw [->] (-10,0) -- (10,0);
\draw [->] (0,-10) -- (0,10);
\draw[line01] (7,-1) -- (-7,6);
%\draw[line03] (-1,1) -- (0,1);
%\draw[line03] (-1,0) -- (-1,1);
%\draw[line01] (0,-3) -- (-2,5);
%\draw (0.6,-4) node {\tiny $-4$};
%\draw (-1.6,-0.7) node {\tiny $-1$};
\draw (10.2,0.7) node {\scriptsize $x$};
\draw (1,3) node {\tiny $\frac{5}{2}$};
\draw (5,1) node {\tiny $5$};
\draw (5,-4) node {\scriptsize $y=-\frac{1}{2}x+\frac{5}{2}$};
\draw (0.7,10.2) node {\scriptsize $y$};
\end{tikzpicture}$$
