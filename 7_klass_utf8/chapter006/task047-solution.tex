47. Найдём прямую, проходящую через данные точки: $\begin{cases} 4k+b=-5,\\ k^2+b=0.\end{cases}\Leftrightarrow
\begin{cases} 4k+b=-5,\\ k^2-4k-5=0.\end{cases}\Leftrightarrow\begin{cases} 4k+b=-5,\\ (k-5)(k+1)=0.\end{cases}\Leftrightarrow
\left[\begin{array}{l}\begin{cases} b=-25,\\ k=5.\end{cases}\\ \begin{cases} b=-1,\\ k=-1.\end{cases}\end{array}\right.$ Для нахождения точки пересечения необходимо решить систему уравнений: $\left[\begin{array}{l}\begin{cases} y=5x-25,\\ 5x-2y+17=0.\end{cases}\\ \begin{cases} y=-x-1,\\ 5x-2y+17=0.\end{cases}\end{array}\right.
\Leftrightarrow\left[\begin{array}{l}\begin{cases} y=5x-25,\\ 5x-10x+50+17=0.\end{cases}\\ \begin{cases} y=-x-1,\\ 5x+2x+2+17=0.\end{cases}\end{array}\right.
\Leftrightarrow\left[\begin{array}{l}\begin{cases} y=42,\\ x=\frac{67}{5}.\end{cases}\\ \begin{cases} y=\frac{12}{7},\\ x=-\frac{19}{7}.\end{cases}\end{array}\right.$ Первую прямую проведём через точки $(5;0)$ и $(0;-25),$ а вторую --- через точки $(-1;0)$ и $(0;-1).$
$$\begin{array}{lr}
\begin{tikzpicture}[scale=0.2]
\tikzset {line01/.style={line width =0.5pt}}
\tikzset{line02/.style={line width =1pt}}
\tikzset{line03/.style={dashed,line width =0.5pt}}
%\filldraw [black] (0,0) circle (1pt);
\draw [->] (-5,0) -- (10,0);
\draw [->] (0,-30) -- (0,10);
\draw[line01] (6,5) -- (-1,-30);
%\draw[line03] (-1,1) -- (0,1);
%\draw[line03] (-1,0) -- (-1,1);
%\draw[line01] (0,-3) -- (-2,5);
%\draw (0.6,-4) node {\tiny $-4$};
%\draw (-1.6,-0.7) node {\tiny $-1$};
\draw (10.2,0.7) node {\scriptsize $x$};
\draw (1,-25) node {\tiny $-25$};
\draw (5.3,-1) node {\tiny $5$};
\draw (0.7,10.2) node {\scriptsize $y$};
\end{tikzpicture}&
\begin{tikzpicture}[scale=0.2]
\tikzset {line01/.style={line width =0.5pt}}
\tikzset{line02/.style={line width =1pt}}
\tikzset{line03/.style={dashed,line width =0.5pt}}
%\filldraw [black] (0,0) circle (1pt);
\draw [->] (-10,0) -- (10,0);
\draw [->] (0,-10) -- (0,10);
\draw[line01] (6,-7) -- (-8,7);
%\draw[line03] (-1,1) -- (0,1);
%\draw[line03] (-1,0) -- (-1,1);
%\draw[line01] (0,-3) -- (-2,5);
%\draw (0.6,-4) node {\tiny $-4$};
%\draw (-1.6,-0.7) node {\tiny $-1$};
\draw (10.2,0.7) node {\scriptsize $x$};
\draw (1,-1) node {\tiny $-1$};
\draw (-1,1) node {\tiny $-1$};
\draw (0.7,10.2) node {\scriptsize $y$};
\end{tikzpicture}\end{array}$$
