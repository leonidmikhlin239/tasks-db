31. Посчитаем массу всех глыб: $50\cdot800+60\cdot1000+60\cdot1500=190000\text{ кг}=190\text{ т}.$ Будем сразу пытаться решить пункт б): $38\cdot5=190,$ значит если эти глыбы и можно погрузить на 38 грузовиков, то пустого пространства ни в одном грузовике остаться не должно. Если пустого пространства не остаётся, глыб по 800кг в грузовике может быть только 5, тогда оставшиеся 1000кг занимает одна глыба по 1000кг. Понадобится $50:5=10$ грузовиков, на которые мы погрузим все глыбы по 800кг и 10 глыб по 1000кг. Остаётся ещё 60 глыб по 1500кг и $60-10=50$ глыб по 1000кг. Если пустого пространства не остаётся, глыб по 1500кг в грузовике может быть только 2, тогда оставшиеся 2000кг занимают две глыбы по 1000кг. Понадобится $50:2=25$ грузовиков, на которые мы погрузим все глыбы по 1000кг и $2\cdot25=50$ глыб по 1500кг. Таким образом, останется ещё $60-50=10$ глыб по 1500кг. Уже понятно, что в 38 грузовиков все глыбы погрузить не получится, так как без пустого пространства погрузить оставшиеся глыбы невозможно. В 39 грузовиков из можно погрузить следующим образом: в 3 грузовика поместить по 3 глыбы, а в последний --- одну. Тогда всего понадобится как раз $10+25+3+1=39$ грузовиков.\\
