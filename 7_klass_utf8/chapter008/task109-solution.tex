109. а) Попробуем подобрать пример, в котором в каждой группе по одному числу $x,\ y$ и $z.$ Они поменяются на $10x+3,\ 10y+7$ и $z.$ Тогда должно выполняться равенство $10x+3+10y+7+z=8x+8y+8z,\ 2x+2y+10=7z.$ Такие числа нетрудно подобрать, например $x=2,\ y=7, z=4.$\\
б) Пусть в первой группе $m$ чисел с суммой $x,$ во второй группе $n$ чисел с суммой $y,$ а в третьей группе числа с суммой $z.$ Тогда
$10x+3m+10y+7n+z=17x+17y+17z,\ 7x+7y+16z=3m+7n.$ Но так как все числа натуральны, $x\geqslant m$ и $y\geqslant n,$ поэтому левая часть всегда больше правой. Значит, в 17 раз сумма всех чисел увеличиться не могла.\\
