38. Пусть расстояние до института равно $s,$ а скорость Агаты на обратном пути равна $v.$ Тогда средняя скорость вычисляется по формуле $\cfrac{2s}{\cfrac{s}{100}+\cfrac{s}{v}}=\cfrac{2s}{\cfrac{sv+100s}{100v}}=\cfrac{200v}{v+100}.$\\
а) Приравняем это выражение к 90: $\cfrac{200v}{v+100}=90,\ 200v=90v+9000,\ 110v=9000,\ v=\cfrac{900}{11}.$ Получившийся ответ целым числом не является, значит средняя скорость за эти две поездки не может составить 90 км/ч.\\
б) Возьмём $v=60,$ тогда $\cfrac{200\cdot60}{60+100}=75.$ Значит, средняя скорость за эти две поездки может оказаться равной целому числу километров в час.\\
