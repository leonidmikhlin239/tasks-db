41. Раз имеет смысл говорить о трети и о четверти учеников, количество учеников в школе делится на 3 и на 4, а значит делится и на 12. Пусть в школе учится $12x$ человек. Тогда двойки получили $12x:3+20=4x+20$ человек, а тройки --- $12x:4+30=3x+30$ человек. Раз некоторые ещё получили четвёрки, верно неравенство $4x+20+3x+30<12x,\ 7x+50<12x,\ 50<5x,\ x>10.$ Вычтем из количества двоечников количество троечников: $4x+20-(3x+30)=4x+20-3x-30=x-10>0.$ Раз получилось положительное число, двоечников больше, чем троечников.\\