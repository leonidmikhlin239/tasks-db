34. Пусть изначально в каждом пакетике было $a$ шариков, а коробок было $b.$ Тогда приравняем общее количество шариков в двух случаях: $2ab=3(a-3)(b-1),\ 2ab=3ab-3a-9b+9,\ 0=ab-3a-9b+9,\ a(b-3)-9(b-3)-18=0,\ (a-9)(b-3)=18.$ Теперь необходимо разобрать все способы разложения числа 18 на два множителя, для каждого из них вычисляя общее количество шариков. Если $a-9=18,\ b-3=1,$ то $2ab=216.$ Если $a-9=9,\ b-3=2,$ то $2ab=180.$ Если $a-9=6,\ b-3=3,$ то $2ab=180.$ Если $a-9=3,\ b-3=6,$ то $2ab=216.$ Если $a-9=2,\ b-3=9,$ то $2ab=264.$ Если $a-9=1,\ b-3=18,$ то $2ab=420.$ Таким образом, наименьшее возможное количество шариков равно 180, а наибольшее --- 420.\\