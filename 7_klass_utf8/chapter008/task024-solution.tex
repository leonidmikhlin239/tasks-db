24. а) Пусть в первом штабеле $x$ коробок по 19кг. Тогда в нём $30-x$ коробок по 49 кг, во втором штабеле $33-x$ коробок по 19кг и $30-(33-x)=x-3$ коробки по 49 кг.
Поэтому $A=|S_1-S_2|=|19x+49(30-x)-19(33-x)-49(x-3)|=|990-60x|=30|33-2x|\geqslant30$кг, так как $|33-2x|\geqslant1.$\\
б) Пусть в первом штабеле $x$ коробок по 19кг и $y$ коробок по 49кг. Тогда во втором штабеле $33-x$ коробок по 19кг и $27-y$ коробок по 49кг. Поэтому
$A=|S_1-S_2|=|19x+49y-19(33-x)-49(27-y)|=2|19x+49y-975|.$ Если это выражение равно нулю, то $19x+49y=975,\ 19(x+y)=15(65-2y).$ Правая часть нечётна и делится на 15, числа 19 и 15 взаимно просты, значит $x+y$ может быть равно только 15 или 45 $(x+y\leqslant33+27=60<75).$ В первом случае $2y=65-19=46,\ y=23,\ x=15-23=-8,$ что невозможно. Во втором случае $2y=65-57=8,\ y=4,\ x=45-4=41>30,$ что также невозможно. Значит, $A$ не может быть равно нулю.\\