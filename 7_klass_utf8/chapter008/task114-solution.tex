114. Заметим, что при натуральных $x$ и $y$ множители не обязательно должны быть натуральными. Рассмотрим все возможные разложения 6 на два целых множителя. Если $x-3=-6$ или $x-3=-3,$ то $x$ не является натуральным числом, что невозможно. Если $x-3=-2,$ то $x=1,$ а $y=(1-6:(-2)):2=2.$ Если $x-3=-1,$ то $x=2,$ а $y=(2-6:(-1)):2=4.$ Если $x-3=1,$ то $x=4,$ а $y=(4-6:1):2=-1,$ что невозможно. Если $x-3=2,$ то $x=5,$ а $y=(5-6:2):2=1.$ Если $x-3=3,$ то $x=6,$ а $y=(6-6:3):2=2.$ Если $x-3=6,$ то $x=9,$ а $y=(9-6:6):2=4.$ Итого $(x;y)\in\{(1;2),\ (2;4),\ (5;1),\ (6;2),\ (9;4)\}.$\\
