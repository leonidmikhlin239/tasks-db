121. Если $2n-1>n-5>0,$ то дробь точно не может быть целой, так как является правильной. Значит, достаточно разобрать случаи $n\in\{1; 2; 3; 4; 5\}.$ При $n=1$ дробь равна $\cfrac{1-5}{2\cdot1-1}=-4,$ это значение подходит. При $n=2$ дробь равна $\cfrac{2-5}{2\cdot2-1}=-1,$ это значение тоже подходит.
При $n=3$ дробь равна $\cfrac{3-5}{2\cdot3-1}=-0,4,$ это значение не подходит. При $n=4$ дробь равна $\cfrac{4-5}{2\cdot4-1}=-\cfrac{1}{7},$ это значение тоже не подходит. При $n=5$ дробь равна $\cfrac{5-5}{2\cdot5-1}=0,$ это значение подходит. Таким образом, дробь может быть равна $-4,\ -1$ и 0.\\
