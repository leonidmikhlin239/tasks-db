25. а) Пусть в первом штабеле $x$ коробок по 13кг. Тогда в нём $22-x$ коробок по 29 кг, во втором штабеле $25-x$ коробок по 13кг и $22-(25-x)=x-3$ коробки по 29 кг.
Поэтому $A=|S_1-S_2|=|13x+29(22-x)-13(25-x)-29(x-3)|=|400-32x|=16|25-2x|\geqslant16$кг, так как $|25-2x|\geqslant1.$\\
б) Пусть в первом штабеле $x$ коробок по 13кг и $y$ коробок по 29кг. Тогда во втором штабеле $25-x$ коробок по 13кг и $19-y$ коробок по 29кг. Поэтому
$A=|S_1-S_2|=|13x+29y-13(25-x)-29(19-y)|=2|13x+29y-438|.$ Если это выражение равно нулю, то $13x+29y=438,\ 13(x+y)=2(219-8y).$ Правая часть делится на 13, числа 2 и 13 взаимно просты, значит $219-8y$ делится на 13. Перебором остатков от деления на 13 найдём, что $y$ (при условии $0\leqslant y\leqslant19)$ может быть равно только 3 или 16. Тогда в первом случае $x=30-3=27>25,$ что невозможно. Во втором случае $x=14-16=-2,$ что также невозможно. Значит, $A$ не может быть равно нулю.\\