28. Число кратно трём тогда и только тогда, когда его сумма цифр делится на три. Если первые две цифры числа отличаются на 1, их сумма нечётна. Поэтому если последняя цифра числа равна 7, то сумма двух первых может быть равна 5, 11 или 17. Таких чисел 6: 237, 327, 567, 657, 897, 987. Если же последняя цифра числа равна 8, то сумма первых двух может быть равна 1, 7 или 13. Таких чисел 5: 108, 348, 438, 678, 768. Таким образом, всего {\it хороших} чисел $6+5=11.$\\