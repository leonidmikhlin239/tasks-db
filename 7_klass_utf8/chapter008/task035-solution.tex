35. Пусть изначально в каждом пакетике было $a$ шариков, а коробок было $b.$ Тогда приравняем общее количество шариков в двух случаях: $3ab=2(a+3)(b+1),\ 3ab=2ab+2a+6b+6,\ ab-2a-6b-6=0,\ a(b-2)-6(b-2)-18=0,\ (a-6)(b-2)=18.$ Теперь необходимо разобрать все способы разложения числа 18 на два множителя, для каждого из них вычисляя общее количество шариков. Если $a-6=18,\ b-2=1,$ то $3ab=216.$ Если $a-6=9,\ b-2=2,$ то $3ab=180.$ Если $a-6=6,\ b-2=3,$ то $3ab=180.$ Если $a-6=3,\ b-2=6,$ то $3ab=216.$ Если $a-6=2,\ b-2=9,$ то $3ab=264.$ Если $a-6=1,\ b-2=18,$ то $3ab=420.$ Таким образом, наименьшее возможное количество шариков равно 180, а наибольшее --- 420.\\
