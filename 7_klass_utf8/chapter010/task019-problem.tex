19. $A=3\cfrac{23}{38}+2\cfrac{35}{38}:\left(24,175-28\cfrac{4}{5}
ight)=3\cfrac{23}{38}+2\cfrac{35}{38}:\left(-\left(28\cfrac{4}{5}-24\cfrac{175}{1000}
ight)
ight)=
3\cfrac{23}{38}+\cfrac{111}{38}\cdot\left(-\cfrac{8}{37}
ight)=$\\$=3\cfrac{23}{38}-\cfrac{24}{38}=2\cfrac{37}{38}.$ Для вычисления значения $P$ обозначим $x=857,$ тогда $P=(x-1)(x+1)-$\\$-(x+2)(x-2)=(x^2-1)-(x^2-4)=3.$ $K=2+\cfrac{81}{243}+\cfrac{27}{243}+\cfrac{18}{243}+\cfrac{6}{243}+\cfrac{1}{243}=2\cfrac{133}{243}.$ Значит, наибольшим является число $P.$\\
