51. Пусть скорости первого и второго поездов равны $x$ и $y$км/ч соответственно, расстояние между B и C равно $S,$ а затрачиваемое до встречи время равно $t.$ Тогда имеем систему уравнений
$\begin{cases} tx=60+S,\\ ty=S,\\ (t-2)(x+25)=60+S,\\ (t-2)(y+20)=S\end{cases}$. Поделив первое уравнение на второе, а третье на четвёртое, получим соотношение $\cfrac{x}{y}=\cfrac{60+S}{S}=\cfrac{x+25}{y+20},$ откуда $xy+20x=xy+25y,\ 20x=25y,\ x=1,25y.$ Поэтому $\cfrac{60+S}{S}=1,25,\ 60+S=1,25S,\ 0,25S=60,\ S=240$км. Перепишем первое и третье уравнения: $\begin{cases}tx=300,\\(t-2)(x+25)=300\end{cases}\Leftrightarrow\begin{cases}tx=300,\\tx+25t-2x-50=300\end{cases}\Leftrightarrow
\begin{cases}tx=300,\\25t=2x+50\end{cases}\Rightarrow$\\$ x\cdot\cfrac{2x+50}{25}=300,\ 2x^2+50x=7500,\ 4x^2+100x=15000,\ 4x^2+100x+625=15625,\ (2x+25)^2=125^2,\ x=50$км/ч, тогда $y=50:1,25=40$км/ч.\\
