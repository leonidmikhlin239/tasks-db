70. Скорость движения лодки по течению реки равна $9+1=10$км/ч, а против течения --- $9-1=8$км/ч. Пусть она отплыла на $x$ километров, тогда должно выполняться неравенство $\cfrac{x}{10}+\cfrac{x}{8}\leqslant9,\ \cfrac{9x}{40}\leqslant9,\ x\leqslant40$км. Значит, максимум лодка может отплыть на 40 километров.\\
