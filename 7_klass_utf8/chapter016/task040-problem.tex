40. Пусть скорость мотоцикла равна $x$км/ч, тогда скорость автомобиля равна $x+30$км/ч. За время, прошедшее до встречи, вместе они проедут два расстояния от станции до посёлка, значит $(x+x+30)\cdot1\cfrac{3}{5}=2\cdot104,\ \cfrac{16}{5}x+48=208,\ \cfrac{16}{5}x=160,\ x=50$км/ч. Значит, встреча произошла на расстоянии $50\cdot\cfrac{8}{5}=80$км от станции. \\
