39. Пусть скорости работы девочек равны $t,\ l$ и $k$ соответственно. Тогда верны равенства\\ $\begin{cases}t+l=\cfrac{1}{12},\\ t+k=\cfrac{1}{20},\\ l+k=\cfrac{1}{15}.\end{cases}$ Отсюда $t+l+k=\cfrac{\cfrac{1}{12}+\cfrac{1}{20}+\cfrac{1}{15}}{2}=\cfrac{1}{10},$ поэтому $t=\cfrac{1}{10}-\cfrac{1}{15}=\cfrac{1}{30},\ k=\cfrac{1}{10}-\cfrac{1}{12}=\cfrac{1}{60},\ l=\cfrac{1}{10}-\cfrac{1}{20}=\cfrac{1}{20}.$ Значит, их зарплаты должны соотноситься как $\cfrac{1}{30}:\cfrac{1}{60}:\cfrac{1}{20}=2:1:3.$ Поэтому Таня должна получить $\cfrac{2}{6}\cdot1800=600$р, Катя $\cfrac{1}{6}\cdot1800=300$р и Люба $\cfrac{3}{6}\cdot1800=900$р.\\
