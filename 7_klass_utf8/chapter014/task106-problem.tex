106. $(|x-3|-a)(x-1)=0\Leftrightarrow \left[\begin{array}{l}|x-3|-a=0,\\ x-1=0.\end{array}
ight.\Leftrightarrow
\left[\begin{array}{l}
\begin{cases}
\left[\begin{array}{l}
x-3=a,\\
x-3=-a.
\end{array}
ight.\\
a\geqslant0.
\end{cases}\\
x=1.
\end{array}
ight.\Leftrightarrow
\left[\begin{array}{l}
\begin{cases}
\left[\begin{array}{l}
x=a+3,\\
x=3-a.
\end{array}
ight.\\
a\geqslant0.
\end{cases}\\
x=1.
\end{array}
ight.$
Ровно два различных корня у этого уравнения может быть только в случае, когда какие-то два корня совпадают. Если $a+3=3-a,$ то $a=0.$ Если $a+3=1,$ то $a=-2,$ что противоречит условию $a\geqslant0.$ Если $3-a=1,$ то $a=2.$ Таким образом, $a\in\{0;2\}.$\\
