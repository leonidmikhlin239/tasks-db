\section{Стандартные задачи}
1. В овощной магазин завезли картофель и морковь. В первый день продали $40\%$ картофеля и $\cfrac{2}{3}$ моркови, что составило 20т. Во второй день продали $\cfrac{2}{3}$ оставшегося картофеля и всю оставшуюся морковь --- всего 15т. Сколько картофеля и сколько моркови было завезено в магазин?\\
2. В город отправляли арбузы и дыни. В первый день отправили $\cfrac{1}{3}$ всех арбузов и $60\%$ всех дынь, что составило 44т. Во второй день отправили 46т, которые составились из $75\%$ оставшихся арбузов и всех оставшихся дынь. Сколько арбузов и сколько дынь было выделено для отправки в город?\\
3. Велосипедист собирался проехать 210 км с постоянной скоростью. Из-за дождя первую половину пути он ехал со скоростью на $40\%$ меньше намеченной. Чтобы наверстать упущенное, вторую половину пути он ехал со скоростью на $40\%$ больше намеченной. В результате он опоздал к намеченному сроку на 2 часа. С какой скоростью он предполагал ехать?\\
4. Участник авторалли рассчитывал проехать расстояние 360 км с постоянной скоростью. Из-за тумана первую половину дистанции он ехал со скоростью на $20\%$ меньше намеченной. На второй половине пути, чтобы наверстать упущенное время, он увеличил скорость на $20\%$ по сравнению с намеченной. В результате он затратил на весь путь на 15 минут больше, чем предполагал. С какой скоростью он предполагал ехать?\\
5. За 3 часа Люба на мотоцикле проезжает то же расстояние,  что Вадик на велосипеде за 5ч. Скорость мотоцикла на 12 км/ч больше скорости велосипедиста. Определить скорость каждого.\\
6. Костя на <<Жигулях>> за 2 часа проезжает на 50 км больше, чем Ира на BMW за 1 час. Скорость BMW в 1,5 раза больше скорости <<Жигулей>>. Определить скорость каждого.\\
7. Расстояние между городами A и B машина прошла за 1ч 15 мин. Обратный путь машина прошла за 1ч 30 мин. Найдите скорость машины, если известно, что на обратном пути скорость машины была на 10 км/ч меньше.\\
8. Теплоход прошёл расстояние между пунктами A и B по течению за 4ч 30 мин, а из B в A против течения он прошёл за 6ч 18мин. Какова скорость теплохода в стоячей воде, если скорость течения 2,4 км/ч?\\
9. Из посёлка в город выехал автобус со скоростью 60 км/ч. Через час после выезда автобуса из посёлка выехал мотоцикл и догнал автобус через 4 часа после выезда автобуса. С какой скоростью ехал мотоциклист?\\
10. Расстояние между посёлками A и B равно 300 км. Из посёлка A в посёлок B выехал автобус, движущийся с постоянной скоростью 60 км/ч. Через час после выезда автобуса из посёлка B в посёлок A с постоянной скоростью выехал мотоциклист, который встретился с автобусом через 1,5 часа. С какой скоростью ехал мотоциклист?\\
11. Два поезда вышли в разное время навстречу друг другу из двух пунктов, расстояние между которыми 1231 км. Скорость первого поезда 50 км/ч, а второго 59 км/ч. Пройдя расстояние 700 км, первый поезд встретился со вторым. На сколько часов один из них вышел раньше другого?\\
12. Два автомобиля вышли в разное время навстречу друг другу из двух пунктов, расстояние между которыми 910 км. Скорость первого автомобиля 80 км/ч, а второго 90 км/ч. Пройдя расстояние 640 км, первый автомобиль встретился со вторым. На сколько часов один из них вышел позже другого?\\
13. Болельщик хочет успеть на стадион к началу матча. Если он пойдёт из дома пешком со скоростью 5 км/ч, то опоздает на 1 ч, а если поедет на велосипеде со скоростью 10 км/ч, то приедет за 30 мин до начала матча. Чему равно расстояние от дома до стадиона?\\
14. Турист, находящийся в спортивном лагере, должен успеть на железнодорожную станцию. Если он поедет на велосипеде со скоростью 15 км/ч, то опоздает на 30 мин, а если на мопеде со скоростью 40 км/ч, то приедет за 2 ч до отхода поезда. Чему равно расстояние от лагеря до станции?\\
15. Скорость катера по течению реки равна 45,2 км/ч, а против --- 36,2 км/ч. Найти скорость течения реки.\\
16. Скорость катера по течению реки равна 40,6 км/ч, а против --- 32,6 км/ч. Найти скорость течения реки.\\
17. Половину пути мотоциклист ехал со скоростью 45 км/ч, а затем задержался на 10 мин, а поэтому, чтобы наверстать потерянное время, он увеличил скорость на 15 км/ч. Каков весь путь мотоциклиста?\\
18. Автобус прошёл $\cfrac{5}{6}$ пути со скоростью 50 км/ч, а затем задержался на 3 мин. Чтобы прибыть в конечный пункт вовремя, оставшуюся часть пути он шёл со скоростью 60 км/ч. Найдите путь, пройденный автобусом.\\
19. Автомобилист преодолел расстояние от города до посёлка за 1 ч 12 мин, двигаясь с постоянной скоростью. Когда он поехал обратно, пошёл дождь, поэтому автомобилист снизил скорость на 20 км/ч и ехал на 24 мин дольше. Найдите расстояние между городом и посёлком.\\
20. Автомобилист в дождливую погоду преодолел расстояние от города до посёлка за 1 ч 48 мин, двигаясь с постоянной скоростью. Когда он поехал обратно, выглянуло солнце, поэтому автомобилист увеличил скорость на 20 км/ч и доехал на 24 мин быстрее. Найдите расстояние между городом и посёлком.\\
21. Через первую трубу бассейн наполняется за 35 минут. За сколько минут наполняет бассейн вторая труба, если вместе они наполняют его за 10 минут?\\
22. Вася съедает торт за 28 минут. За сколько этот торт съедает Петя, если вместе они съедят его за 12 минут?\\
23. Скорость течения реки составляет $5\%$ от скорости катера. Двигаясь против течения, катер за 3 часа проходит на 40 км меньше, чем за 3 часа 40 минут по течению. Найдите скорость катера против течения.\\
24. Скорость течения реки составляет $10\%$ от скорости лодки. Двигаясь против течения реки, лодка за 3 часа 20 минут проходит на 28 км меньше, чем за 4 часа движения по течению. Найдите скорость лодки по течению.\\
25. Маше задано выучить английские глаголы и существительные. Утром она выучила $\cfrac{1}{12}$ всех глаголов и $\cfrac{1}{16}$ всех существительных, всего 5 слов. Вечером она выучила ещё $\cfrac{1}{4}$ всех оставшихся глаголов и $\cfrac{1}{5}$ всех оставшихся существительных. Оказалось, что вечером Маша выучила на 8 глаголов больше, чем существительных. Сколько существительных и сколько глаголов было задано Маше?\\
26. Васе задано решить задачи по алгебре и геометрии. В первый день он решил $\cfrac{1}{15}$ всех задач по алгебре и $\cfrac{1}{25}$ всех задач по геометрии, получилось 5 задач. Во второй день он решил $\cfrac{1}{7}$ остатка задач по алгебре и $\cfrac{1}{6}$ оставшихся задач по геометрии. Оказалось, что во второй день задач по геометрии Вася решил на 2 больше, чем по алгебре. Сколько задач по алгебре и сколько задач по геометрии было задано?\\
27. Два зайца и пять кроликов съедают одну тарелку моркови за восемь секунд, а семь зайцев и четыре кролика съедают такую же тарелку моркови за четыре секунды. Определите, за сколько секунд с этим же количеством моркови справятся заяц и два кролика (все зайцы едят одинаково быстро, все кролики --- тоже).\\
28. Три зайца и два кролика съедают одну тарелку моркови за двенадцать секунд, а пять зайцев и восемь кроликов съедают такую же тарелку моркови за четыре секунды. Определите, за сколько секунд с этим же количеством моркови справятся два зайца и кролик (все зайцы едят одинаково быстро, все кролики --- тоже).\\
29. Если Вася идёт в спортшколу пешком, а возвращается на трамвае, то всего он затрачивает на дорогу 1,5 часа. Если же он в качестве дополнительной тренировки в обе стороны идёт пешком, то на всю дорогу в спортшколу и обратно домой у него уходит 2,5 часа. Какое время Вася затратит на дорогу в спортшколу и обратно домой, если он весь путь проедет на трамвае?\\
30. Двигаясь с определённой скоростью, пешеход пройдёт намеченный путь за 2,5 ч. Но если через 2 часа от начала пути он уменьшит свою скорость на 4 км/ч, то пройдёт весь путь за 3 часа. Найдите длину пути.\\
31. Двадцать семь карандашей и тридцать три ручки стоят 246 рублей. Сколько стоит карандаш, если он на 2 рубля дешевле ручки?\\
32. Пешеход идёт вдоль дороги. Мимо него проезжают попутные автобусы с интервалом 12 минут. С каким интервалом в минутах автобусы проезжают мимо остановки, если скорость автобуса в шесть раз больше скорости пешехода?\\
33. Грузовик проезжает некоторое расстояние за 10 часов. Если бы он проезжал в час на 10 км больше, то ему потребовалось бы на тот же путь 8 часов. Каким было расстояние и скорость движения грузовика?\\
34. Из двух городов A и B, расстояние между которыми равно 600 км, одновременно навстречу друг другу выехали два поезда. Через 2 ч 24 мин расстояние между ними впервые стало равным 240 км. С какой скоростью идут поезда, если скорость первого поезда на 14 км/ч больше скорости второго?\\
35. Пешеход половину пути шёл со скоростью 3 км/ч, а другую половину пути со скоростью 5 км/ч. Найти длину всего пути, пройденного пешеходом, если всего он находился в пути 8 часов.\\
36. Оксана делает некоторую работу за 7 часов, Марина за 6 часов, а Борис Викторович за 3 часа. После того, как Оксана сделала половину всей работы, к ней присоединились Марина и Борис Викторович. За какое время была сделана вся работа и какую её часть сделала Марина?\\
37. Путь от города до посёлка автомобиль проезжает за 2,5 ч. Если он увеличит скорость на 20 км/ч, то за 2 часа он проедет путь на 15 км больший, чем расстояние от города до посёлка. Найдите это расстояние.\\
38. Из A в B выехали два велосипедиста. Первый половину времени, затраченного на весь путь, ехал со скоростью 25 км/ч, а остальное время --- со скоростью 20 км/ч. Второй первую половину пути ехал со скоростью 20 км/ч, а вторую со скоростью 25 км/ч. Кто из них раньше приехал в B?\\
39. Таня и Люба красят забор за 12 часов, Таня и Катя выкрасят этот же забор за 20 часов, а Люба и Катя --- за 15 часов. За работу всем трём девочкам заплатили 1800 рублей. Сколько денег должна получить каждая девочка?\\
40. От станции к посёлку, удалённому на 104 км, отправились одновременно мотоциклист и автомобилист. Скорость автомобиля на 30 км/ч больше скорости мотоцикла. Прибыв в посёлок, автомобиль сразу повернул обратно и встретил мотоциклиста через 1 ч 36 мин после его выезда со станции. На каком расстоянии от станции произошла встреча?\\
41. Бассейн заполняется водой, поступающей из двух труб. Первая труба может наполнить бассейн за 12 часов, а вторая --- за 20ч. После двух часов работы одной первой трубы была включена вторая труба. Сколько времени ушло на заполнение всего бассейна, и какую часть бассейна заполнила первая труба?\\
42. Лена и Наташа живут в одном доме и учатся в одной школе. Лена доходит от дома до школы за 20 минут, а Наташа --- за 30 минут. Через сколько минут Лена догонит Наташу, если Наташа выйдет из дома на 5 минут раньше Лены?\\
43. Надо застелить ковром пол в комнате, ширина которой на 1 м меньше длины. Если купить ковёр, длина и ширина которого на 50 см меньше длины и ширины комнаты, то он будет на 2550 р дешевле, чем ковёр, покрывающий весь пол. Найдите длину и ширину комнаты, если известно, что 1 $\text{м}^2$ ковра стоит 600р.\\
44. Катер за 3 часа по течению и 5 часов против течения проходит 76 км. Найдите скорость течения и собственную скорость катера, если за 6 часов по течению катер проходит столько же, сколько за 9 часов против течения.\\
45. Петя вышел из школы и пошёл по направлению к дому со скоростью 4 км/ч. Одновременно с ним от дома к школе выехал на мопеде его брат Серёжа со скоростью 42 км/ч. Встретив по дороге Петю, Серёжа доехал до школы, мгновенно развернулся и поехал к дому. Таким образом Серёжа ездил между домом и школой до тех пор, пока Петя не пришёл домой. Сколько раз братья встретятся, пока Петя идёт от школы до дома, если расстояние между зданиями 2,8 км?\\
46. Ребёнок Эрвин за полгода обучения в школе научился доезжать до неё за одно и то же время. Каждое утро он тратит на поездку в метро вдвое больше времени, чем на поездку на троллейбусе, при этом 8 минут, что составляет $\cfrac{2}{9}$ от всей поездки в метро, он тратит на ожидание поездов и подъём на эскалаторе. Сколько времени Эрвин добирается до школы, если на весь пеший путь он тратит на 29 минут меньше, чем проводит в метро?\\
47. Ребёнок Эрвин за полгода обучения в школе научился доезжать из неё до дома за одно и то же время. Каждый вечер он тратит на поездку в метро втрое больше времени, чем на поездку на троллейбусе, при этом 9 минут, что составляет $\cfrac{3}{13}$ от всей поездки в метро, он тратит на ожидание поезда и подъём на эскалаторе. Сколько времени Эрвин добирается до дома, если на весь пеший путь он тратит на 31 минуту меньше, чем проводит в метро.\\
48. Бригада из 5 садовников за 3 часа посадила 30 деревьев. Сколько деревьев посадят 4 садовника за 4 часа?\\
49. Стог сена корова съедает за 6 дней, а коза --- за 12 дней. За сколько дней они съедят стог сена вместе?\\
50. Три бригады вспахали два поля общей площадью 96 га. Первое поле было вспахано за 3 дня, причём работали все вместе. Второе поле вспахали за 6 дней вторая и третья бригады. Если бы все три бригады проработали на втором поле 1 день, то оставшаяся часть второго поля первая бригада могла бы вспахать за 8 дней. Сколько гектаров в день может вспахать первая бригада?\\
51. Два поезда выехали одновременно в одном направлении из городов A и B, которые расположены на расстоянии 60 км друг от друга и одновременно прибыли на станцию C. Если бы один из поездов увеличил скорость на 25 км/ч, а другой на 20 км/ч, то они прибыли бы в C также одновременно, но на два часа раньше. Найдите скорости поездов.\\
52. Карлсон съедает банку варенья за 10 минут, Фрекен Бок --- за 12 минут, а Малыш --- за 15 минут. За сколько минут они съедят банку варенья втроём?\\
53. Петя и Вася вскапывают грядку за 10 минут, а один Петя --- за 15 минут. На сколько минут Вася дольше Пети вскапывает грядку, работая один?\\
54. Два пешехода вышли одновременно из своих сёл А и В навстречу друг другу. После встречи первый шёл 50 минут до села В, а второй шёл 18 минут до села А. Сколько минут они шли до встречи?\\
55. Два пешехода вышли одновременно из своих сёл А и В навстречу друг другу. После встречи первый шёл 45 минут до села В, а второй шёл 20 минут до села А. Сколько минут они шли до встречи?\\
56. Мастер и ученик должны были каждый день вместе делать некоторое число деталей. В первый день ученик работал три часа, а мастер --- два, в результате они сделали 0,9 нужного числа деталей. Во второй день наоборот --- мастер работал три часа, а ученик два и они перевыполнили план на $15\%.$ За какое время справился бы с заданием ученик в одиночку?\\
57. Мастер и ученик должны были каждый день вместе делать некоторое число деталей. В первый день ученик работал три часа, а мастер --- два, в результате они сделали $\frac{4}{5}$ нужного числа деталей. Во второй день наоборот --- мастер работал три часа, а ученик два и они перевыполнили план на $5\%.$ За какое время справился бы с заданием ученик в одиночку?\\
58. Первые 60 {\it км} машина ехала со скоростью 40 {\it км/ч,} а следующие 30 {\it км} --- со скоростью 60 {\it км/ч.} Какова средняя скорость $V$ {\it км/ч} на всём пути? В ответе напишите число $V$ без единиц измерения.\\
59. Первые 80 {\it км} поезд шёл со скоростью 60 {\it км/ч,} а следующие 20 {\it км} --- со скоростью $V$ {\it км/ч.} Известно, что средняя скорость на всём пути равна 50 {\it км/ч.} Найдите $V.$ В ответе напишите число $V$ без единиц измерения.\\
60. Первые 80 {\it км} машина ехала со скоростью 60 {\it км/ч,} а следующие 24 {\it км} --- со скоростью 34 {\it км/ч.} Какова средняя скорость $V$ {\it км/ч} на всём пути? В ответе напишите число $V$ без единиц измерения.\\
61. Первые 100 {\it км} поезд шёл со скоростью 80 {\it км/ч,} а следующие 50 {\it км} --- со скоростью $V$ {\it км/ч.} Известно, что средняя скорость на всём пути равна 72 {\it км/ч.} Найдите $V.$ В ответе напишите число $V$ без единиц измерения.\\
62. Имеется сплав золота и серебра. Золото в сплаве составляет $30\%.$ Если бы в сплаве золота было на 2 {\it г} меньше, а серебра --- на 12 {\it г} больше, золото составляло бы $25\%$ от общей массы этого сплава. Найдите массу первоначального сплава. В ответе запишите количество граммов.\\
63. Имеется сплав меди и олова массой 40 {\it кг.} При добавлении 5 {\it кг} меди процентное содержание меди увеличилось на 10 процентных пунктов. Найдите процентное содержание олова в первоначальном сплаве.\\
64. Имеется сплав золота и серебра. Серебро в сплаве составляет $70\%.$ Если бы в сплаве золота было на 10 {\it г} больше, а серебра --- на 4 {\it г} меньше, серебро составляло бы $50\%$ от общей массы этого сплава. Найдите массу первоначального сплава. В ответе запишите количество граммов.\\
65. Имеется сплав меди и олова массой 60 {\it кг.} При добавлении 12 {\it кг} олова процентное содержание меди уменьшилось на 10 процентных пунктов. Найдите процентное содержание меди в первоначальном сплаве.\\
66. Расстояние между двумя пунктами поезд проходит по расписанию за 2,5 {\it ч.} Однако по прошествии 1 часа 40 минут поезд по техническим причинам снизил скорость на 10 {\it км/ч,} в результате чего он опоздал на 10 минут. Найдите первоначальную скорость поезда.\\
67. Расстояние между пунктами $A$ и $B$ составляет 840 {\it км.} Автомобиль выехал из пункта $A$ в направлении к $B$ со скоростью 60 {\it км/ч.} Спустя некоторое время из пункта $B$ в направлении к $A$ выехал автомобиль со скоростью 65 {\it км/ч.} К моменту их встречи один из них проехал на 60 {\it км} больше другого. На сколько часов позже выехал автомобиль из пункта $B$ по сравнению с автомобилем, выехавшим из пункта $A?$\\
68. Расстояние между двумя пунктами поезд проходит по расписанию за 3,5 {\it ч.} Однако по прошествии 1,5 часа поезд по техническим причинам снизил скорость на 5 {\it км/ч,} в результате чего он опоздал на 8 минут. Найдите первоначальную скорость поезда.\\
69. Расстояние между пунктами $A$ и $B$ составляет 680 {\it км.} Автомобиль выехал из пункта $B$ в направлении к $A$ со скоростью 80 {\it км/ч.} Спустя некоторое время из пункта $A$ в направлении к $B$ выехал автомобиль со скоростью 100 {\it км/ч.} К моменту их встречи один из них проехал на 40 {\it км} больше другого. На сколько часов позже выехал автомобиль из пункта $A$ по сравнению с автомобилем, выехавшим из пункта $B?$\\
70. На какое наибольшее число километров может отплыть лодка от пристани против течения реки, если собственная скорость лодки 9 км/ч, скорость течения реки 1 км/ч, чтобы успеть вернуться через 9 часов?\\
71. На какое наибольшее число километров может отплыть лодка от пристани против течения реки, если собственная скорость лодки 8 км/ч, скорость течения реки 2 км/ч, чтобы успеть вернуться через 4 часа?\\
72. Игорь и Паша могут покрасить забор за 4 часа, Паша и Володя могут покрасить этот же забор за 12 часов, а Володя и Игорь --- за 9 часов. За какое время мальчики покрасят забор, работая втроём?\\
73. Лена и Ира могут обработать грядку клубники за 9 часов, Ира и Оля могут обработать эту же грядку за 4 часа, а Оля и Лена --- за 12 часов. За какое время девочки обработают грядку, работая втроём?\\
74. На практику в издательство пришли две студентки. Оказалось, что одна из них набирает текст в два раза медленнее, чем куратор практики, а вторая --- в три раза медленнее, чем куратор. Им выдали перенабрать рукопись. Работая вдвоём, они выполнили работу за 6 часов. За сколько часов выполнил бы эту работу куратор?\\
75. Два мотоциклиста ехали навстречу друг другу. Скорость каждого мотоциклиста 100 км/ч. Параллельно шоссе проходит железная дорога, по которой ехал длинный товарный поезд. Один из мотоциклистов проехал мимо этого поезда в три раза быстрее, чем другой. Какова скорость поезда?\\
76. Петя доехал на своём велосипеде от дома до школы, двигаясь с постоянной скоростью. Если бы он увеличил скорость на 3 м/с, то он доехал бы до школы в три раза быстрее. Во сколько раз быстрее он доехал бы до школы, увеличив скорость на 6 м/c?\\
77. По плану Василий в $16:00$ должен был выйти из дома и отправиться на дачу. К этому времени с дачи за ним должна была приехать машина. Но Василий от нетерпения уже в $15:00$ отправился пешком машине навстречу. Встретив по пути машину, он сел в нее и приехал на дачу на 10 минут раньше запланированного времени. С какой скоростью ехал автомобиль, если Василий шел со скоростью 6 км/ч?\\
78. На овощебазе хранят капусту, морковь и картофель. Картофель при этом занимает в 2 раза больше ящиков, чем морковь, а на капусту приходится в 3 раза больше ящиков, чем на картофель и морковь вместе взятых. Какой процент всех использованных ящиков отведён под хранение капусты?\\
79. На овощебазе хранят капусту, морковь и картофель. Морковь при этом занимает в 3 раза меньше ящиков, чем картошка, а на капусту приходится в 9 раз больше ящиков, чем на картофель и морковь вместе взятых. Какой процент всех использованных ящиков отведён под хранение капусты?\\
80. Автомобиль ехал по дороге сначала со скоростью 90 км/ч. Когда ему осталось проехать на
20 км больше, чем он уже проехал, автомобиль увеличил скорость на $20\%.$ В результате средняя
скорость на всём пути составила 100 км/ч. Каков был путь?\\
81.  Расстояние между городами равно 200 км. Автомобиль сначала ехал по дороге со скоростью
135 км/ч, затем увеличил скорость на $20\%$ и с такой скоростью доехал до конечной цели. Оказалось,
что средняя скорость на всём пути 150 км/ч. На каком расстоянии от старта автомобиль увеличил
скорость?
\newpage
