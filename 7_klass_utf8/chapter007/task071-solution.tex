71. Скорость движения лодки по течению реки равна $8+2=10$км/ч, а против течения --- $8-2=6$км/ч. Пусть она отплыла на $x$ километров, тогда должно выполняться неравенство $\cfrac{x}{10}+\cfrac{x}{6}\leqslant4,\ \cfrac{8x}{30}\leqslant4,\ x\leqslant15$км. Значит, максимум лодка может отплыть на 15 километров.\\
