\section{стандартные задачи решения}
1. Пусть картофеля привезли $x$т, а моркови --- $y.$ Тогда составим систему уравнений: \\$\begin{cases}0,4x+\cfrac{2}{3}y=20,\\ \cfrac{2}{3}\cdot0,6x+\cfrac{1}{3}y=15.\end{cases}\Leftrightarrow
\begin{cases}0,4x+\cfrac{2}{3}y=20,\\ 0,4x+\cfrac{1}{3}y=15.\end{cases}\Leftrightarrow
\begin{cases}\cfrac{1}{3}y=5,\\ 0,4x+\cfrac{1}{3}y=15.\end{cases}\Leftrightarrow
\begin{cases}y=15\text{ т},\\ x=25\text{ т}.\end{cases}$\\
2. Пусть арбузов привезли $x$т, а дынь --- $y.$ Тогда составим систему уравнений: \\$\begin{cases}\cfrac{1}{3}x+0,6y=44,\\ 0,75\cdot\cfrac{2}{3}x+0,4y=46.\end{cases}\Leftrightarrow
\begin{cases}\cfrac{1}{3}x+0,6y=44,\\ \cfrac{1}{2}x+0,4y=46.\end{cases}\Leftrightarrow
\begin{cases}x+1,8y=132,\\ x+0,8y=92.\end{cases}\Leftrightarrow
\begin{cases}y=40,\\ 0,4x+\cfrac{1}{3}y=15.\end{cases}\Leftrightarrow
\begin{cases}y=40\text{ т},\\ x=60\text{ т}.\end{cases}$\\
3. Пусть велосипедист предполагал ехать со скоростью $x$км/ч. Тогда она собирался потратить $\cfrac{210}{x}$ч, а потратил $\cfrac{105}{0,6x}+\cfrac{105}{1,4x}.$ Раз он опоздал на 2 часа, верно соотношение $\cfrac{210}{x}+2=\cfrac{105}{0,6x}+\cfrac{105}{1,4x},\ \cfrac{210}{x}+2=\cfrac{175}{x}+\cfrac{75}{x},\ \cfrac{40}{x}=2,\ x=20$км/ч.\\
4. Пусть участник авторалли предполагал ехать со скоростью $x$км/ч. Тогда она собирался потратить $\cfrac{360}{x}$ч, а потратил $\cfrac{180}{0,8x}+\cfrac{180}{1,2x}.$ Раз он затратил на 15 минут больше, верно соотношение $\cfrac{360}{x}+\cfrac{1}{4}=\cfrac{180}{0,8x}+\cfrac{180}{1,2x},\ \cfrac{360}{x}+\cfrac{1}{4}=\cfrac{225}{x}+\cfrac{150}{x},\ \cfrac{15}{x}=\cfrac{1}{4},\ x=60$км/ч.\\
5. Пусть скорость Вадика равна $x$км/ч, тогда скорость Любы $x+12$км/ч и верно соотношение $3(x+12)=5x,\ 3x+36=5x,\ 2x=36,\ x=18\text{ км/ч},\ x+12=30$км/ч.\\
6. Пусть скорость <<Жигулей>> равна $x$км/ч, тогда скорость BMW $1,5x$км/ч и верно соотношение $2x=1,5x+50,\ 0,5x=50,\ x=100\text{ км/ч},\ 1,5x=150$км/ч.\\
7. Пусть скорость машины на пути из A в B равна $x$км/ч, тогда скорость на обратном пути равна $x-10$км/ч и верно соотношение $1\cfrac{1}{4}x=1\cfrac{1}{2}(x-10),\
\cfrac{5}{4}x=\cfrac{3}{2}x-\cfrac{3}{2}\cdot10,\ \cfrac{1}{4}x=15,\ x=60$км/ч.\\
8. Пусть скорость теплохода в стоячей воде равна $x$км/ч, тогда верно соотношение $4\cfrac{1}{2}(x+2,4)=6\cfrac{3}{10}(x-2,4),\ 4,5x+10,8=6,3x-15,12,\
1,8x=25,92=14,4$км/ч.\\
9. За час автобус успеет отъехать от города на 60 км. Раз мотоциклист догнал его за $4-1=3$ часа, он догонял автобус со скоростью $60:3=20$км/ч, поэтому его скорость была равна $60+20=80$км/ч.\\
10. Через час после выезда расстояние между автобусом и пунктом B будет равно $300-60=240$км. Раз они с мотоциклистом встретились через 1,5 часа, скорость их сближения была равна $240:1,5=160$км/ч, значит скорость мотоциклиста равна $160-60=100$км/ч.\\
11. Первый поезд ехал $700:50=14$ч, а второй --- $(1231-700):59=9$ч, то есть первый выехал на $14-9=5$ часов раньше.\\
12. Первый автомобиль ехал $640:80=8$ч, а второй --- $(910-640):90=3$ч, то есть первый выехал на $8-3=5$ часов раньше.\\
13. Пусть расстояние от дома до стадиона равно $x$км, тогда так как пешком болельщик тратит на 1ч$+$30мин$=\cfrac{3}{2}$ч больше на дорогу, чем при поездке на велосипеде, верно соотношение $\cfrac{x}{5}=\cfrac{x}{10}+\cfrac{3}{2},\ \cfrac{x}{10}=\cfrac{3}{2},\ x=15$км.\\
14. Пусть расстояние от лагеря до станции равно $x$км, тогда так на велосипеде турист тратит на 30мин$+2$ч$=\cfrac{5}{2}$ч больше на дорогу, чем при поездке на мопеде, верно соотношение $\cfrac{x}{15}=\cfrac{x}{40}+\cfrac{5}{2},\ \cfrac{x}{24}=\cfrac{5}{2},\ x=60$км.\\
15. Пусть скорость катера в стоячей воде равна $x$км/ч, а скорость течения равна $y$км/ч, тогда верны равенства $\begin{cases} x+y=45,2,\\ x-y=36,2.\end{cases}
\Rightarrow 2y=9,\ y=4,5$км/ч.\\
16. Пусть скорость катера в стоячей воде равна $x$км/ч, а скорость течения равна $y$км/ч, тогда верны равенства $\begin{cases} x+y=40,6,\\ x-y=32,6.\end{cases}
\Rightarrow 2y=8,\ y=4$км/ч.\\
17. Пусть весь путь мотоциклиста равен $2x,$ тогда верно равенство $\cfrac{x}{45}+\cfrac{1}{6}+\cfrac{x}{60}=\cfrac{2x}{45},\
\cfrac{x+10}{60}=\cfrac{x}{45},\ 45x+450=60x,\ 15x=450,\ x=30,\ 2x=60$км.\\
18. Пусть весь путь автобуса равен $6x,$ тогда верно равенство $\cfrac{5x}{50}+\cfrac{1}{20}+\cfrac{x}{60}=\cfrac{6x}{50},\
\cfrac{x+3}{60}=\cfrac{x}{50},\ 50x+150=60x,\ 10x=150,\ x=15,\ 6x=90$км.\\
19. Пусть изначальная скорость автомобилиста была равна $x$км/ч, тогда верно соотношение $1\cfrac{1}{5}x=1\cfrac{3}{5}(x-20),\ \cfrac{6}{5}x=\cfrac{8}{5}x-32,\
\cfrac{2}{5}x=32,\ x=80$км/ч. Тогда расстояние между городом и посёлком равно $\cfrac{6}{5}\cdot80=96$км.\\
20. Пусть изначальная скорость автомобилиста была равна $x$км/ч, тогда верно соотношение $1\cfrac{4}{5}x=1\cfrac{2}{5}(x+20),\ \cfrac{9}{5}x=\cfrac{7}{5}x+28,\
\cfrac{2}{5}x=28,\ x=70$км/ч. Тогда расстояние между городом и посёлком равно $\cfrac{9}{5}\cdot70=126$км.\\
21. Скорость наполнения бассейна второй трубой равна $\cfrac{1}{10}-\cfrac{1}{35}=\cfrac{1}{14}$б/мин, значит она наполняет бассейн за 14 минут.\\
22. Скорость поедания торта Петей равна $\cfrac{1}{12}-\cfrac{1}{28}=\cfrac{1}{21}$т/мин, значит Петя съедает торт за 21 минуту.\\
23. Пусть скорость катера в стоячей воде равна $x$км/ч, тогда верно соотношение $3\cdot0,95x+40=3\cfrac{2}{3}\cdot1,05x,\
2,85x+40=3,85x,\ x=40,\ 0,95\cdot40=38$км/ч.\\
24. Пусть скорость лодки в стоячей воде равна $x$км/ч, тогда верно соотношение $3\cfrac{1}{3}\cdot0,9x+28=4\cdot1,1x,\
3x+28=4,4x,\ 1,4x=28,\ x=20,\ 1,1\cdot20=22$км/ч.\\
25. Пусть было задано выучить $x$ глаголов и $y$ существительных, тогда верны равенства\\
$\begin{cases}\cfrac{1}{12}x+\cfrac{1}{16}y=5,\\ \cfrac{1}{4}\cdot\cfrac{11}{12}x-\cfrac{1}{5}\cdot\cfrac{15}{16}y=8.\end{cases}\Leftrightarrow
\begin{cases}\cfrac{1}{4}x+\cfrac{3}{16}y=15,\\ \cfrac{11}{48}x-\cfrac{3}{16}y=8.\end{cases}\Leftrightarrow
\begin{cases}\cfrac{23}{48}x=23,\\ \cfrac{11}{48}x-\cfrac{3}{16}y=8.\end{cases}\Leftrightarrow
\begin{cases} x=48,\\ y=16.\end{cases}$\\
26. Пусть было задано решить $x$ задач по алгебре и $y$ по геометрии, тогда верны равенства\\
$\begin{cases}\cfrac{1}{15}x+\cfrac{1}{25}y=5,\\ \cfrac{1}{7}\cdot\cfrac{14}{15}x-\cfrac{1}{6}\cdot\cfrac{24}{25}y=-2.\end{cases}\Leftrightarrow
\begin{cases}\cfrac{4}{15}x+\cfrac{4}{25}y=20,\\ \cfrac{2}{15}x-\cfrac{4}{25}y=-2.\end{cases}\Leftrightarrow
\begin{cases}\cfrac{6}{15}x=18,\\ \cfrac{2}{15}x-\cfrac{4}{25}y=-2.\end{cases}\Leftrightarrow
\begin{cases} x=45,\\ y=50.\end{cases}$\\
27. Пусть зайцы едят со скоростью $x,$ а кролики $y$ тарелок моркови в секунду. Тогда верны равенства\\
$\begin{cases}2x+5y=\cfrac{1}{8},\\ 7x+4y=\cfrac{1}{4}.\end{cases}\Leftrightarrow
\begin{cases}8x+20y=\cfrac{1}{2},\\ 35x+20y=\cfrac{5}{4}.\end{cases}\Leftrightarrow
\begin{cases}8x+20y=\cfrac{1}{2},\\ 27x=\cfrac{3}{4}.\end{cases}\Leftrightarrow
\begin{cases} y=\cfrac{1}{72},\\ x=\cfrac{1}{36}.\end{cases}$
Тогда 1 заяц и 2 кролика едят со скоростью $\cfrac{1}{36}+2\cdot\cfrac{1}{72}=\cfrac{1}{18}$ тарелок в секунду и съедят одну тарелку моркови за 18 секунд.\\
28. Пусть зайцы едят со скоростью $x,$ а кролики $y$ тарелок в секунду. Тогда верны равенства\\
$\begin{cases}3x+2y=\cfrac{1}{12},\\ 5x+8y=\cfrac{1}{4}.\end{cases}\Leftrightarrow
\begin{cases}12x+8y=\cfrac{1}{3},\\ 5x+8y=\cfrac{1}{4}.\end{cases}\Leftrightarrow
\begin{cases}7x=\cfrac{1}{12},\\ 5x+8y=\cfrac{1}{4}.\end{cases}\Leftrightarrow
\begin{cases} x=\cfrac{1}{84},\\ y=\cfrac{1}{42}.\end{cases}$
Тогда 2 зайца и 1 кролик едят со скоростью $2\cdot\cfrac{1}{84}+\cfrac{1}{42}=\cfrac{1}{21}$ тарелок в секунду и съедят одну тарелку за 21 секунду.\\
29. На дорогу в одну сторону пешком Вася тратит $2,5:2=1,25$ч, значит на дорогу в одну сторону на трамвае он тратит $1,5-1,25=0,25$ч. Поэтому на дорогу в обе стороны на трамвае он потратит $2\cdot0,25=0,5$ч$=30$мин.\\
30. Пусть изначальная скорость пешехода равна $x$км/ч, тогда верно равенство $2,5x=2x+1\cdot(x-4),\ 0,5x=4,\ x=8$км/ч. Поэтому весь путь равен $2,5\cdot8=20$км.\\
31. Пусть карандаш стоит $x$ рублей, тогда ручка стоит $x+2$ рубля и верно равенство $27x+33(x+2)=246,\ 27x+33x+66=246,\ 60x=180,\ x=3$рубля.\\
32. Пусть скорость пешехода равна $x$м/мин, тогда скорость автобусов равна $6x$м/мин. Интервал, с которым автобусы проезжают мимо пешехода --- это то время, которое необходимо следующему автобусу, чтобы догнать пешехода в тот момент, когда он поравнялся с некоторым автобусом. Так как автобус догоняет пешехода за 12 минут, а догоняет его со скоростью $6x-x=5x$м/мин, расстояние между автобусами равно $5x\cdot12=60x$м. Интервал, за который автобусы проезжают мимо неподвижной остановки --- это как раз то время, которое необходимо автобусу на преодоление этого расстояния, то есть $60x:6x=10$минут.\\
33. Пусть скорость грузовика равна $x$км/ч, тогда верно равенство $10x=8(x+10),\ 10x=8x+80,\ 2x=80,\ x=40$км/ч, а расстояние $10\cdot40=400$км.\\
34. Пусть скорость второго поезда равна $x$км/ч, тогда скорость первого поезда равна $x+14$ и $(x+x+14)\cdot2\cfrac{2}{5}=600-240,\ \cfrac{24}{5}x+\cfrac{168}{5}=360,\ 24x+168=1800,\ 24x=1632,\ x=68$км/ч, $x+14=82$км/ч.\\
35. Пусть весь путь равен $2x$км, тогда верно равенство $\cfrac{x}{3}+\cfrac{x}{5}=8,\ 5x+3x=120,\ 8x=120,\ x=15,\ 2x=30$км.\\
36. Скорость Оксаны $\cfrac{1}{7}$ работы в час, Марины $\cfrac{1}{6},$ а Бориса Викторовича --- $\cfrac{1}{3}.$ Тогда всю работу они сделают за $\cfrac{\cfrac{1}{2}}{\cfrac{1}{7}}+\cfrac{\cfrac{1}{2}}{\cfrac{1}{7}+\cfrac{1}{6}+\cfrac{1}{3}}=\cfrac{7}{2}+\cfrac{7}{9}=4\cfrac{5}{18}$ч$=4$ч 16мин 40с. Марина выполнила $\cfrac{7}{9}\cdot\cfrac{1}{6}=\cfrac{7}{54}$ частей работы.\\
37. Пусть скорость автомобиля равна $x$км/ч, тогда верно равенство $2,5x+15=2(x+20),\ 2,5x+15=2x+40,\ 0,5x=25,\ x=50$км/ч. Тогда расстояние от города до посёлка равно $2,5\cdot50=125$км.\\
38. Пусть первый велосипедист потратил $2x$ч времени, тогда расстояние равно $25x+20x=45x.$ Второй велосипедист потратит на этот же путь $\cfrac{22,5x}{20}+
\cfrac{22,5x}{25}=2,025x>2x.$ Значит, первый велосипедист приедет раньше.\\
39. Пусть скорости работы девочек равны $t,\ l$ и $k$ соответственно. Тогда верны равенства\\ $\begin{cases}t+l=\cfrac{1}{12},\\ t+k=\cfrac{1}{20},\\ l+k=\cfrac{1}{15}.\end{cases}$ Отсюда $t+l+k=\cfrac{\cfrac{1}{12}+\cfrac{1}{20}+\cfrac{1}{15}}{2}=\cfrac{1}{10},$ поэтому $t=\cfrac{1}{10}-\cfrac{1}{15}=\cfrac{1}{30},\ k=\cfrac{1}{10}-\cfrac{1}{12}=\cfrac{1}{60},\ l=\cfrac{1}{10}-\cfrac{1}{20}=\cfrac{1}{20}.$ Значит, их зарплаты должны соотноситься как $\cfrac{1}{30}:\cfrac{1}{60}:\cfrac{1}{20}=2:1:3.$ Поэтому Таня должна получить $\cfrac{2}{6}\cdot1800=600$р, Катя $\cfrac{1}{6}\cdot1800=300$р и Люба $\cfrac{3}{6}\cdot1800=900$р.\\
40. Пусть скорость мотоцикла равна $x$км/ч, тогда скорость автомобиля равна $x+30$км/ч. За время, прошедшее до встречи, вместе они проедут два расстояния от станции до посёлка, значит $(x+x+30)\cdot1\cfrac{3}{5}=2\cdot104,\ \cfrac{16}{5}x+48=208,\ \cfrac{16}{5}x=160,\ x=50$км/ч. Значит, встреча произошла на расстоянии $50\cdot\cfrac{8}{5}=80$км от станции. \\
41. За 2 часа первая труба заполнит бассейн на $\cfrac{1}{12}\cdot2=\cfrac{1}{6}$ часть. Значит, на то, чтобы заполнить остаток бассейна до конца, уйдёт $\cfrac{\cfrac{5}{6}}{\cfrac{1}{12}+\cfrac{1}{20}}=6\cfrac{1}{4}$ч, поэтому всего бассейн был заполнен за $2+6\cfrac{1}{4}=8\cfrac{1}{4}$ч или 8 ч 15 мин. Первая труба работала всё время, значит она заполнила $\cfrac{1}{12}\cdot\cfrac{33}{4}=\cfrac{11}{16}$ бассейна.\\
42. Наташа успеет пройти $5\cdot\cfrac{1}{30}=\cfrac{1}{6}$ расстояния до школы, значит Лена догонит её через $\cfrac{\cfrac{1}{6}}{\cfrac{1}{20}-\cfrac{1}{30}}=10$минут.\\
43. Пусть ширина комнаты равна $x$м, тогда длина равна $x+1$м и площадь $x(x+1)=x^2+x\text{м}^2.$ Площадь ковра, закрывающего только часть комнаты, равна $(x-0,5)(x+0,5)=x^2-0,25,$ что меньше на $x^2+x-(x^2-0,25)=x+0,25\text{м}^2.$ Исходя из стоимости ковра, найдём эту площадь: $x+0,25=\cfrac{2550}{600}=4,25\Rightarrow x=4$м, $x+1=5$м.\\
44. Пусть собственная скорость катера равна $x$км/ч, а скорость течения --- $y$км/ч. Тогда получаем систему уравнений: $\begin{cases}3(x+y)+5(x-y)=76,\\
6(x+y)=9(x-y).\end{cases}\Leftrightarrow\begin{cases}8x-2y=76,\\
15y=3x.\end{cases}\Leftrightarrow\begin{cases}40y-2y=76,\\
5y=x.\end{cases}\Leftrightarrow\begin{cases}38y=76,\\
5y=x.\end{cases}\Leftrightarrow\begin{cases} y=2\text{км/ч},\\
x=10\text{км/ч}.\end{cases}$\\
45. Петя придёт домой через $2,8:4=0,7$ч. За это время Серёжа проедет $42\cdot0,7=29,4$км, что составляет $29,4:2,8=10,5$ расстояний от дома до школы. Значит, он встретит Петю 11 раз (он встречает его каждый раз, когда проезжает полное расстояние, и встретит 11-ый раз, когда будет проезжать последнюю половину расстояния по направлению от дома к школе).\\
46. В метро Эрвин едет $8:\cfrac{2}{9}=36$мин, тогда пешком он идёт $36-29=7$мин, на троллейбусе едет $36:2=18$мин. Таким образом, всего он тратит $36+7+18+8=69$минут или 1 час 9 минут.\\
47. В метро Эрвин едет $9:\cfrac{3}{13}=39$мин, тогда пешком он идёт $39-31=8$мин, на троллейбусе едет $39:3=13$мин. Таким образом, всего он тратит $39+8+13+9=69$минут или 1 час 9 минут.\\
48. Один садовник за один час посадит $30:5:3=2$ дерева, значит 4 садовника за 4 часа посадят $2\cdot4\cdot4=32$ дерева.\\
49. Им потребуется $\cfrac{1}{\cfrac{1}{6}+\cfrac{1}{12}}=4$дня.\\
50. Пусть бригады пашут со скоростями $x,\ y$ и $z$ гектаров в день соответственно, а первое поле имеет площадь $S$ гектаров. Тогда имеем систему уравнений
$\begin{cases}3(x+y+z)=S,\\ 6(y+z)=96-S,\\ x+y+z+8x=96-S.\end{cases}\Leftrightarrow
\begin{cases}3x+3y+3z+6y+6z=96,\\ 6y+6z=96-S,\\ 9x+y+z=96-S.\end{cases}\Leftrightarrow
\begin{cases}3x+9y+9z=96,\\ 6y+6z=9x+y+z,\\ 9x+y+z=96-S.\end{cases}\Leftrightarrow
\begin{cases}x+3(y+z)=32,\\ 9x=5(y+z),\\ 9x+y+z=96-S.\end{cases}\Rightarrow$\\$
x+3\cdot\cfrac{9x}{5}=32,\ 32x=180,\ x=5$га/д.\\
51. Пусть скорости первого и второго поездов равны $x$ и $y$км/ч соответственно, расстояние между B и C равно $S,$ а затрачиваемое до встречи время равно $t.$ Тогда имеем систему уравнений
$\begin{cases} tx=60+S,\\ ty=S,\\ (t-2)(x+25)=60+S,\\ (t-2)(y+20)=S\end{cases}$. Поделив первое уравнение на второе, а третье на четвёртое, получим соотношение $\cfrac{x}{y}=\cfrac{60+S}{S}=\cfrac{x+25}{y+20},$ откуда $xy+20x=xy+25y,\ 20x=25y,\ x=1,25y.$ Поэтому $\cfrac{60+S}{S}=1,25,\ 60+S=1,25S,\ 0,25S=60,\ S=240$км. Перепишем первое и третье уравнения: $\begin{cases}tx=300,\\(t-2)(x+25)=300\end{cases}\Leftrightarrow\begin{cases}tx=300,\\tx+25t-2x-50=300\end{cases}\Leftrightarrow
\begin{cases}tx=300,\\25t=2x+50\end{cases}\Rightarrow$\\$ x\cdot\cfrac{2x+50}{25}=300,\ 2x^2+50x=7500,\ 4x^2+100x=15000,\ 4x^2+100x+625=15625,\ (2x+25)^2=125^2,\ x=50$км/ч, тогда $y=50:1,25=40$км/ч.\\
52. Они съедят банку за $\cfrac{1}{\cfrac{1}{10}+\cfrac{1}{12}+\cfrac{1}{15}}=4$минуты.\\
53. Вася вскапывает грядку за $\cfrac{1}{\cfrac{1}{10}-\cfrac{1}{15}}=30$ минут, что на $30-15=15$ минут дольше, чем Петя.\\
54. Пусть скорость первого пешехода равна $x$м/мин, а второго --- $y$м/мин. Тогда расстояние от места встречи до пункта B равно $50x,$ а расстояние от места встречи до пункта A равно $18y.$ До места встречи пешеходы добрались одновременно, значит $\cfrac{50x}{y}=\cfrac{18y}{x},\ 50x^2=18y^2,\ 25x^2=18y^2,\ 5x=3y\Rightarrow50x=30y$ и до места встречи пешеходы шли $30y:y=30$ минут.\\
55. Пусть скорость первого пешехода равна $x$м/мин, а второго --- $y$м/мин. Тогда расстояние от места встречи до пункта B равно $45x,$ а расстояние от места встречи до пункта A равно $20y.$ До места встречи пешеходы добрались одновременно, значит $\cfrac{45x}{y}=\cfrac{20y}{x},\ 45x^2=20y^2,\ 9x^2=4y^2,\ 3x=2y\Rightarrow45x=30y$ и до места встречи пешеходы шли $30y:y=30$ минут.\\
56. Пусть мастер работает со скоростью $x$ плана в час, а ученик --- $y.$ Тогда имеем систему уравнений
$\begin{cases}2x+3y=0,9,\\ 3x+2y=1,15.\end{cases}\Leftrightarrow
\begin{cases}6x+9y=2,7,\\ 6x+4y=2,3.\end{cases}\Rightarrow 5y=0,4,\ y=0,08.$ Таким образом, ученику на выполнение плана понадобится $1:0,08=12,5$ часов.\\
57. Пусть мастер работает со скоростью $x$ плана в час, а ученик --- $y.$ Тогда имеем систему уравнений
$\begin{cases}2x+3y=\cfrac{4}{5},\\ 3x+2y=1,05.\end{cases}\Leftrightarrow
\begin{cases}6x+9y=2,4,\\ 6x+4y=2,1.\end{cases}\Rightarrow 5y=0,3,\ y=0,06.$ Таким образом, ученику на выполнение плана понадобится $1:0,06=16\cfrac{2}{3}$ часов или 16 часов 40 минут.\\
58. Средняя скорость $V$ равна $\cfrac{60+30}{\cfrac{60}{40}+\cfrac{30}{60}}=45.$\\
59. Составим уравнение нахождения средней скорости: $50=\cfrac{80+20}{\cfrac{80}{60}+\cfrac{20}{V}}=\cfrac{75V}{V+15},\ 50V+750=75V,\ 25V=750,\ V=30.$\\
60. Средняя скорость $V$ равна $\cfrac{80+24}{\cfrac{80}{60}+\cfrac{24}{34}}=51.$\\
61. Составим уравнение нахождения средней скорости: $72=\cfrac{100+50}{\cfrac{100}{80}+\cfrac{50}{V}}=\cfrac{120V}{V+40},\ 72V+2880=120V,\ 48V=2880,\ V=60.$\\
62. Пусть масса первоначального сплава в граммах равна $x.$ Тогда золота в нём $0,3x$ и $0,3x-2=0,25(x-2+12),\ 0,3x-2=0,25x+2,5,\ 0,05x=4,5,\ x=90$г.\\
63. Пусть процентное содержание меди в первоначальном сплаве в процентах равно $x.$ Тогда меди в этом сплаве $\cfrac{40x}{100}$ и $\cfrac{40x}{100}+5=\cfrac{x+10}{100}\cdot45,\ 40x+500=45x+450,\ 5x=50,\ x=10\%,$ а олова --- $100-10=90\%.$\\
64. Пусть масса первоначального сплава в граммах равна $x.$ Тогда серебра в нём $0,7x$ и $0,7x-4=0,5(x-4+10),\ 0,7x-4=0,5x+3,\ 0,2x=7,\ x=35$г.\\
65. Пусть процентное содержание меди в первоначальном сплаве в процентах равно $x.$ Тогда меди в этом сплаве $\cfrac{60x}{100}$ и $\cfrac{60x}{100}=\cfrac{x-10}{100}\cdot72,\ 60x=72x-720,\ 12x=720,\ x=60\%.$\\
66. Поезд проехал 1 час 40 минут, то есть $\cfrac{5}{3}$ч с первоначальной скоростью и 1 час со сниженной скоростью (так как опоздал на 10 минут). Пусть первоначальная скорость поезда равна $x$км/ч, тогда расстояние между двумя пунктами равно $2,5x$ с одной стороны и  $\cfrac{5}{3}x+x-10=\cfrac{8}{3}x-10$ с другой, поэтому $2,5x=\cfrac{8}{3}x-10,\ \cfrac{1}{6}x=10,\ x=60$км/ч.\\
67. Пусть один автомобиль проехал $x$км, тогда другой проехал $x+60$ и $x+x+60=840,\ 2x=780,\ x=390$км. Значит, один автомобиль проехал 390 км, а другой --- $840-390=450$км. Так как $\cfrac{390}{60}<\cfrac{450}{65},$ 390 км должен был проехать второй автомобиль, а 450 км --- первый (так как первый ехал дольше). Тогда первый ехал $\cfrac{450}{60}=7,5$ч, а второй --- $\cfrac{390}{65}=6$ч, поэтому он выехал позже на $7,5-6=1,5$ часа.\\
68. Поезд проехал 1,5 ч с первоначальной скоростью и $2+\cfrac{8}{60}=\cfrac{32}{15}$ часа со сниженной скоростью (так как опоздал на 8 минут). Пусть первоначальная скорость поезда равна $x$км/ч, тогда расстояние между двумя пунктами равно $3,5x$ с одной стороны и  $1,5x+\cfrac{32}{15}(x-5)=\cfrac{109}{30}x-\cfrac{32}{3}$ с другой, поэтому $3,5x=\cfrac{109}{30}x-\cfrac{32}{3},\ \cfrac{2}{15}x=\cfrac{32}{3},\ x=80$км/ч.\\
69. Пусть один автомобиль проехал $x$км, тогда другой проехал $x+40$ и $x+x+40=680,\ 2x=640,\ x=320$км. Значит, один автомобиль проехал 320 км, а другой --- $680-320=360$км. Если первый ехал $\cfrac{360}{80}=4,5$ч, а второй --- $\cfrac{320}{100}=3,2$ч, то второй выехал позже на $4,5-3,2=1,3$ часа. Если первый ехал $\cfrac{320}{80}=4$ч, а второй --- $\cfrac{360}{100}=3,6$ч, то второй выехал позже на $4-3,6=0,4$ часа. Возможны оба случая.\\
70. Скорость движения лодки по течению реки равна $9+1=10$км/ч, а против течения --- $9-1=8$км/ч. Пусть она отплыла на $x$ километров, тогда должно выполняться неравенство $\cfrac{x}{10}+\cfrac{x}{8}\leqslant9,\ \cfrac{9x}{40}\leqslant9,\ x\leqslant40$км. Значит, максимум лодка может отплыть на 40 километров.\\
71. Скорость движения лодки по течению реки равна $8+2=10$км/ч, а против течения --- $8-2=6$км/ч. Пусть она отплыла на $x$ километров, тогда должно выполняться неравенство $\cfrac{x}{10}+\cfrac{x}{6}\leqslant4,\ \cfrac{8x}{30}\leqslant4,\ x\leqslant15$км. Значит, максимум лодка может отплыть на 15 километров.\\
72. Пусть скорости покраски мальчиков равны $x,\ y$ и $z$ (забора в час). Тогда имеет место система $\begin{cases} x+y=\cfrac{1}{4},\\ y+z=\cfrac{1}{12},\\ x+z=\cfrac{1}{9}.\end{cases}\Rightarrow x+y+z=\cfrac{1}{2}\cdot\left(\cfrac{1}{4}+\cfrac{1}{12}+\cfrac{1}{9}\right)=\cfrac{2}{9}.$ Значит, втроём мальчики покрасят забор за $1:\cfrac{2}{9}=4,5$ часа.\\
73. Пусть скорости обработки грядки девочек равны $x,\ y$ и $z$ (грядки в час). Тогда имеет место система $\begin{cases} x+y=\cfrac{1}{9},\\ y+z=\cfrac{1}{4},\\ x+z=\cfrac{1}{12}.\end{cases}\Rightarrow x+y+z=\cfrac{1}{2}\cdot\left(\cfrac{1}{9}+\cfrac{1}{4}+\cfrac{1}{12}\right)=\cfrac{2}{9}.$ Значит, втроём девочки обработают грядку за $1:\cfrac{2}{9}=4,5$ часа.\\
74. Пусть скорость набора рукописи куратором равна $x$ (рукописи в час), тогда у студенток она $\cfrac{x}{2}$ и $\cfrac{x}{3}$ и $\cfrac{x}{2}+\cfrac{x}{3}=\cfrac{1}{6},\ \cfrac{5x}{6}=\cfrac{1}{6},\ x=\cfrac{1}{5}.$ Значит, куратор наберёт рукопись за $1:\cfrac{1}{5}=5$ часов.\\
75. Расстояние они проехали одинаковое, значит один мотоциклист проезжал его со скоростью в три раза большей скорости другого. Пусть скорость поезда равна $x$ км/ч, тогда мотоциклист, едущий навстречу, проезжает мимо него со скоростью $x+100,$ а едущий в том же направлении --- со скоростью $100-x.$ Таким образом, $x+100=3(100-x),\ x+100=300-3x,\ 4x=200,\ x=50$км/ч.\\
76. Пусть расстояние от дома до школы равно $S$м, а скорость Пети равна $v$ м/с, тогда $3\cdot\cfrac{S}{v+3}=\cfrac{S}{v},$ откуда $3v=v+3,\ 2v=3,\ v=1,5$м/с. Значит, изначально время на путь от дома до школы занимал $\cfrac{S}{1,5}$с, а после увеличения скорости на 6 м/с будет занимать $\cfrac{S}{7,5}$с, что в $\cfrac{S}{1,5}:\cfrac{S}{7,5}=5$ раз меньше.\\
77. Машина сэкономила время на дорогу от места встречи до Васиного дома и обратно, получилось 10 минут. Значит, от места встречи до дома она ехала бы 5 минут, поэтому встретились она на 5 минут раньше назначенного времени, в 15:55. Значит, Вася прошёл расстояние от дома до места встречи за 55 минут, в 11 раз дольше, чем ехала бы машина. Поэтому скорость машины в 11 раз больше Васиной и равна $11\cdot6=66$ км/ч.\\
78. Пусть морковь занимает $x$ ящиков, тогда картофель занимает $2x$ ящиков, а капуста --- $(x+2x)\cdot3=9x$ ящиков. Тогда под хранение капусты отведено $\cfrac{9x}{x+2x+9x}\cdot100\%=75\%$ ящиков.\\
79. Пусть морковь занимает $x$ ящиков, тогда картофель занимает $3x$ ящиков, а капуста --- $(x+3x)\cdot9=36x$ ящиков. Тогда под хранение капусты отведено $\cfrac{36x}{x+3x+36x}\cdot100\%=90\%$ ящиков.\\
80. Пусть автомобиль проехал $x$км, тогда ему осталось проехать $x+20$км, и выполнено соотношение $\cfrac{x}{90}+\cfrac{x+20}{90\cdot1,2}=\cfrac{x+x+20}{100},\
\cfrac{x}{90}+\cfrac{x+20}{108}=\cfrac{x+10}{50}\Big|\cdot2700,\
30x+25(x+20)=54(x+10),\ 30x+25x+500=54x+540,\ x=40$км. Таким образом, весь путь составлял $40+40+20=100$км.\\
81. Пусть автомобиль увеличил скорость через $x$км от старта, тогда ему оставалось проехать ещё $200-x$км, и выполнено соотношение
$\cfrac{x}{135}+\cfrac{200-x}{135\cdot1,2}=\cfrac{200}{150},\
\cfrac{x}{135}+\cfrac{200-x}{162}=\cfrac{4}{3}\Big|\cdot 810,\
6x+5(200-x)=1080,\ 6x+1000-5x=1080,\ x=80$км.
\newpage
