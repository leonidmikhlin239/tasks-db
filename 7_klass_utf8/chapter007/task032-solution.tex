32. Пусть скорость пешехода равна $x$м/мин, тогда скорость автобусов равна $6x$м/мин. Интервал, с которым автобусы проезжают мимо пешехода --- это то время, которое необходимо следующему автобусу, чтобы догнать пешехода в тот момент, когда он поравнялся с некоторым автобусом. Так как автобус догоняет пешехода за 12 минут, а догоняет его со скоростью $6x-x=5x$м/мин, расстояние между автобусами равно $5x\cdot12=60x$м. Интервал, за который автобусы проезжают мимо неподвижной остановки --- это как раз то время, которое необходимо автобусу на преодоление этого расстояния, то есть $60x:6x=10$минут.\\
