76. Пусть расстояние от дома до школы равно $S$м, а скорость Пети равна $v$ м/с, тогда $3\cdot\cfrac{S}{v+3}=\cfrac{S}{v},$ откуда $3v=v+3,\ 2v=3,\ v=1,5$м/с. Значит, изначально время на путь от дома до школы занимал $\cfrac{S}{1,5}$с, а после увеличения скорости на 6 м/с будет занимать $\cfrac{S}{7,5}$с, что в $\cfrac{S}{1,5}:\cfrac{S}{7,5}=5$ раз меньше.\\