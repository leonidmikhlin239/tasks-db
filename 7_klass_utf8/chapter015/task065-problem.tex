66. а) Проведём прямую через две точки $(-1;1)$ и $(0;3).$
$$\begin{tikzpicture}[scale=0.2]
\tikzset {line01/.style={line width =0.5pt}}
\tikzset{line02/.style={line width =1pt}}
\tikzset{line03/.style={dashed,line width =0.5pt}}
%\filldraw [black] (0,0) circle (1pt);
\draw [->] (-6.5,0) -- (10,0);
\draw [->] (0,-10) -- (0,10);
\draw[line01] (-4,-5) -- (3,9);
\draw[line03] (-1,1) -- (0,1);
\draw[line03] (-1,0) -- (-1,1);
%\draw[line01] (0,-3) -- (-2,5);
%\draw (0.6,-4) node {\tiny $-4$};
%\draw (-1.6,-0.7) node {\tiny $-1$};
\draw (10.2,0.7) node {\scriptsize $x$};
\draw (0.5,3) node {\tiny $3$};
\draw (0.5,1) node {\tiny $1$};
\draw (-1.5,-1) node {\tiny $-1$};
\draw (0.7,10.2) node {\scriptsize $y$};
\end{tikzpicture}$$
б) Подставим координаты точки $A$ в уравнение прямой: $1=k\cdot(-1)+3,\ k=2.$\\
в) Ось абсцисс эта прямая пересекает в точке $\left(-\cfrac{3}{2};0
ight)$ и образует прямоугольный треугольник с катетами $\cfrac{3}{2}$ и $3.$ Его площадь равна $\cfrac{1}{2}\cdot\cfrac{3}{2}\cdot3=\cfrac{9}{4}.$\\
г) Так как угловые коэффициенты у этих прямых разные, они пересекаются.\\
