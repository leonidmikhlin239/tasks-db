75. а) Построим график по двум точкам $(0;6)$ и $(3;0).$
$$\begin{tikzpicture}[scale=0.2]
\tikzset {line01/.style={line width =0.5pt}}
\tikzset{line02/.style={line width =1pt}}
\tikzset{line03/.style={dashed,line width =0.5pt}}
%\filldraw [black] (0,0) circle (1pt);
\draw [->] (-10,0) -- (10,0);
\draw [->] (0,-10) -- (0,10);
\draw[line01] (-2,10) -- (5,-4);
%\draw[line03] (-1,1) -- (0,1);
%\draw[line03] (-1,0) -- (-1,1);
%\draw[line01] (0,-3) -- (-2,5);
%\draw (0.6,-4) node {\tiny $-4$};
%\draw (-1.6,-0.7) node {\tiny $-1$};
\draw (10.2,0.7) node {\scriptsize $x$};
\draw (0.7,6) node {\tiny $6$};
\draw (3,0.9) node {\tiny $3$};
\draw (0.7,10.2) node {\scriptsize $y$};
\end{tikzpicture}$$
б) Найдём точку пересечения прямых $y=x+3$ и $x-2y+9=0:\ \begin{cases}y=x+3,\\ x-2y+9=0. \end{cases}\Leftrightarrow
\begin{cases}y=x+3,\\ x-2x-6+9=0. \end{cases}\Leftrightarrow
\begin{cases}y=6,\\ x=3. \end{cases}$ Расстояние от точки $(3;0)$ до точки $(3;6)$ равно 6.\\
