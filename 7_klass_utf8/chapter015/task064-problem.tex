65. Найдём точку пересечения прямых $y=\cfrac{1}{3}x-2$ и $y=6-x:\ \cfrac{1}{3}x-2=6-x,\ \cfrac{4}{3}x=8,\ x=6,\ y=6-x=0.$ Прямая $y=-4x-3$ пересекает ось $Oy$ в точке с абсциссой $x=0$ и ординатой $y=0-3=-3.$ Найдём прямую, проходящую через полученные точки: $\begin{cases} 6k+b=0,\\ 0+b=-3.\end{cases}\Leftrightarrow
\begin{cases} k=\cfrac{1}{2},\\ b=-3.\end{cases}$ Проведём прямую через две точки $(0;-3)$ и $(6;0).$
$$\begin{tikzpicture}[scale=0.2]
\tikzset {line01/.style={line width =0.5pt}}
\tikzset{line02/.style={line width =1pt}}
\tikzset{line03/.style={dashed,line width =0.5pt}}
%\filldraw [black] (0,0) circle (1pt);
\draw [->] (-6.5,0) -- (10,0);
\draw [->] (0,-10) -- (0,10);
\draw[line01] (8,1) -- (-6,-6);
%\draw[line03] (-1,1) -- (0,1);
%\draw[line03] (-1,0) -- (-1,1);
%\draw[line01] (0,-3) -- (-2,5);
%\draw (0.6,-4) node {\tiny $-4$};
%\draw (-1.6,-0.7) node {\tiny $-1$};
\draw (10.2,0.7) node {\scriptsize $x$};
\draw (1,-3) node {\tiny $-3$};
\draw (6,-0.5) node {\tiny $6$};
\draw (0.7,10.2) node {\scriptsize $y$};
\end{tikzpicture}$$
