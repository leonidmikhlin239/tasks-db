57. Найдём точку, в которой прямая $y=-0,5x+4$ пересекает ось $Ox$ (в ней $y=0):\ 0=-0,5x+4,\ x=8$ и ось $Oy$ (в ней $x=0):\ y=-0,5\cdot0+4=4.$ Если $BA$ является медианой, точка $A$ лежит посередине между $O$ и $C,$ а значит точка $C$ имеет координаты $(0;8).$ Проведём прямую через неё и точку $B(8;0):\
\begin{cases} 8=0+b,\\ 0=8k+b.\end{cases}\Rightarrow y=-x+8.$
$$\begin{tikzpicture}[scale=0.2]
\tikzset {line01/.style={line width =0.5pt}}
\tikzset{line02/.style={line width =1pt}}
\tikzset{line03/.style={dashed,line width =0.5pt}}
%\filldraw [black] (0,0) circle (1pt);
\draw [->] (-10,0) -- (10,0);
\draw [->] (0,-10) -- (0,10);
\draw[line01] (9,-1) -- (-1,9);
%\draw[line03] (-1,1) -- (0,1);
%\draw[line03] (-1,0) -- (-1,1);
%\draw[line01] (0,-3) -- (-2,5);
%\draw (0.6,-4) node {\tiny $-4$};
%\draw (-1.6,-0.7) node {\tiny $-1$};
\draw (10.2,0.7) node {\scriptsize $x$};
\draw (0.5,8) node {\tiny $8$};
\draw (8,0.5) node {\tiny $8$};
\draw (0.7,10.2) node {\scriptsize $y$};
\end{tikzpicture}$$
