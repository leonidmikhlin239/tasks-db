72. а) Найдём прямую $AB:\ \begin{cases} b=-7,\\ 3k+b=2.\end{cases}\Leftrightarrow\begin{cases} b=-7,\\ k=3.\end{cases}\Rightarrow y=3x-7.$
Найдём прямую $CD:\ \begin{cases} k+b=1,\\ -30k+b=63.\end{cases}\Leftrightarrow\begin{cases} 31k=-62,\\ -30k+b=63.\end{cases}
\Leftrightarrow\begin{cases} k=-2,\\ b=3.\end{cases}\Rightarrow y=-2x+3.$\\
б) Найдём точку пересечения прямых $AB$ и $CD: 3x-7=-2x+3,\ 5x=10,\ x=2,\ y=3\cdot2-7=-1.$ Найдём прямую $BC:\ \begin{cases} 3k+b=2,\\ k+b=1.\end{cases}\Leftrightarrow\begin{cases} 2k=1,\\ k+b=1.\end{cases}\Leftrightarrow\begin{cases} k=\cfrac{1}{2},\\ b=\cfrac{1}{2}.\end{cases}\Rightarrow y=\cfrac{1}{2}x+\cfrac{1}{2}.$ Точка этой прямой, лежащая на оси абсцисс, имеет ординату $y=0,$ а значит $\cfrac{1}{2}x+\cfrac{1}{2}=0,\ x=-1.$ Таким образом, необходимо найти уравнение прямой, проходящей через точки $(2;-1)$ и $(-1;0): \ \begin{cases} 2k+b=-1,\\ -k+b=0.\end{cases}\Leftrightarrow\begin{cases} 3k=-1,\\ -k+b=0.\end{cases}\Leftrightarrow\begin{cases} k=-\cfrac{1}{3},\\ b=-\cfrac{1}{3}.\end{cases}\Rightarrow y=-\cfrac{1}{3}x-\cfrac{1}{3}.$\\
