39. Пусть расстояние до института равно $s,$ а скорость Насти на обратном пути равна $v.$ Тогда средняя скорость вычисляется по формуле $\cfrac{2s}{\cfrac{s}{80}+\cfrac{s}{v}}=\cfrac{2s}{\cfrac{sv+80s}{80v}}=\cfrac{160v}{v+80}.$\\
а) Приравняем это выражение к 70: $\cfrac{160v}{v+80}=70,\ 160v=70v+5600,\ 90v=5600,\ v=\cfrac{560}{9}.$ Получившийся ответ целым числом не является, значит средняя скорость за эти две поездки не может составить 70 км/ч.\\
б) Возьмём $v=20,$ тогда $\cfrac{160\cdot20}{20+80}=32.$ Значит, средняя скорость за эти две поездки может оказаться равной целому числу километров в час.\\
