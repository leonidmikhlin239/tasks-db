122. Если $2n-5>n-1>0,$ то дробь точно не может быть целой, так как является правильной. Значит, достаточно разобрать случаи $n\in\{1; 2; 3; 4\}.$ При $n=1$ дробь равна $\cfrac{1-1}{2\cdot1-5}=0,$ это значение подходит. При $n=2$ дробь равна $\cfrac{2-1}{2\cdot2-5}=-1,$ это значение тоже подходит.
При $n=3$ дробь равна $\cfrac{3-1}{2\cdot3-5}=2,$ это значение также подходит. При $n=4$ дробь равна $\cfrac{4-1}{2\cdot4-5}=1,$ это значение тоже подходит. Таким образом, дробь может быть равна $-2,\ -1,\ 0$ и 1.\\
