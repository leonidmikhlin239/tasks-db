116. Пусть число имеет вид $\overline{abc}=100a+10b+c.$ Тогда $\overline{abc}-\overline{cba}=100a+10b+c-100c-10b-a=99(a-c)=198,\ a-c=2,\ a=c+2.$ Так как сумма квадратов цифр равна 101, получим равенство $b^2=101-a^2-c^2.$ Переберём все возможные значения буквы $c$ от 1 до 7. Если $c=1,$ то $a=3$ и $b^2=101-1-9=91,$ чего быть не может. Если $c=2,$ то $a=4$ и $b^2=101-4-16=81,\ b=9,$ то есть подходит число 492. Если $c=3,$ то $a=5$ и $b^2=101-9-25=67,$ чего быть не может. Если $c=4,$ то $a=6$ и $b^2=101-16-36=49,\ b=7$ то есть подходит число 674. Если $c=5,$ то $a=7$ и $b^2=101-25-49=27,$ чего быть не может. Если $c=6,$ то $a=8$ и $b^2=101-36-64=1,\ b=1,$ то есть подходит число 816. Если $c=7,$ то $a=9$ и $b^2=101-81-49=-29,$ чего быть не может. Таким образом, подходят только числа 492, 674, 816.\\
