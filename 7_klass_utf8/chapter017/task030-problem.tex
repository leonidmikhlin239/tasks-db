30. Посчитаем массу всех глыб: $50\cdot700+60\cdot1000+80\cdot1500=215000\text{ кг}=215\text{ т}.$ Будем сразу пытаться решить пункт б): $43\cdot5=215,$ значит если эти глыбы и можно погрузить на 43 грузовика, то пустого пространства ни в одном грузовике остаться не должно. Если пустого пространства не остаётся, глыб по 700кг в грузовике может быть только 5, тогда оставшиеся 1500кг занимает одна глыба по 1500кг. Понадобится $50:5=10$ грузовиков, на которые мы погрузим все глыбы по 700кг и 10 глыб по 1500кг. Остаётся ещё 60 глыб по 1000кг и $80-10=70$ глыб по 1500кг. Если пустого пространства не остаётся, глыб по 1500кг в грузовике может быть только 2, тогда оставшиеся 2000кг занимают две глыбы по 1000кг. Понадобится $60:2=30$ грузовиков, на которые мы погрузим все глыбы по 1000кг и $2\cdot30=60$ глыб по 1500кг. Таким образом, останется ещё $70-60=10$ глыб по 1500кг. Уже понятно, что в 43 грузовика все глыбы погрузить не получится, так как без пустого пространства погрузить оставшиеся глыбы невозможно. В 44 грузовика их можно погрузить следующим образом: в 3 грузовика поместить по 3 глыбы, а в последний --- одну. Тогда всего понадобится как раз $10+30+3+1=44$ грузовика.\\
