79. а) Самое большое простое число в этом ряду равно 157.\\
б) На 5 делятся числа от $95=5\cdot19$ до $160=5\cdot32,$ то есть $32-19+1=14$ чисел. На 25 делятся числа 100, 125 и 150, что даёт дополнительные три пятёрки. Также есть число 125, в множителях которого есть ещё одна дополнительная пятёрка. Значит, всего это произведение делится на $5^{18}.$\\
в) Двоек в этом произведении ещё больше (каждое второе число чётно), значит последние 18 цифр равны 0.\\
