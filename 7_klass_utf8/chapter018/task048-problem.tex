52. Если меньшей стороной является основание, то стороны треугольника равны $x,\ x+8,\ x+8.$ Возможны два случая: $x+x+8=20,\ 2x=12,\ x=6$см или $x+8+x+8=20,\ 2x=4,\ x=2$см. Тогда стороны равны 6см, 14см и 14см или 2см, 10см и 10см. Неравенство треугольника выполняется для обоих случаев. Если меньшей стороной является боковая сторона, то стороны треугольника равны $x,\ x-8,\ x-8.$ Возможны два случая: $x+x-8=20,\ 2x=28,\ x=14$см или $x-8+x-8=20,\ 2x=36,\ x=18$см. Тогда стороны равны 14см, 6см и 6см или 18см, 10см и 10см. Для первого из этих случаев не выполняется неравенство треугольника: $6+6<14.$ Таким образом, длина основания может быть равна 2см, 6см или 18см.\\
