122. Возможны два случая: стороны треугольника в дециметрах равны $x,\ x$ и $x+1$ или $x,\ x$ и $x-1.$ В первом случае $x+x+x+1=4,\ 3x=3,\ x=1$дм, $x+1=2$дм, во втором случае $x+x+x-1=4,\ 3x=5,\ x=\cfrac{5}{3}$дм, $x-1=\cfrac{2}{3}$дм. В первом случае не выполняется неравенство треугольника: $1+1=2,$ а во втором выполняется, значит стороны треугольника могут быть равны только $\cfrac{5}{3}$дм, $\cfrac{5}{3}$дм, $\cfrac{2}{3}$дм.\\
