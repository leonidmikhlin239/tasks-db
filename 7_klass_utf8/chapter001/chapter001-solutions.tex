\section{Числовые выражения решения}
1. $\left(\cfrac{2}{3}-1,3+\cfrac{3}{4}\right):1,4+\cfrac{1}{6}=\left(\cfrac{17}{12}-\cfrac{13}{10}\right):\cfrac{14}{10}+\cfrac{1}{6}=\cfrac{7}{60}\cdot\cfrac{10}{14}+\cfrac{1}{6}=
\cfrac{1}{12}+\cfrac{1}{6}=\cfrac{1}{4}.$\\
2. $4,4-0,28:\left(\cfrac{1}{9}-\cfrac{7}{15}+0,375\right)=4,4-0,28:\left(-\cfrac{16}{45}+\cfrac{3}{8}\right)=4,4-\cfrac{28}{100}:\cfrac{7}{360}=4,4-\cfrac{7}{25}\cdot\cfrac{360}{7}=
\\=4,4-14,4=-10.$\\
3. $\left(1\cfrac{11}{24}+\cfrac{13}{36}\right)\cdot1,44-\cfrac{8}{15}\cdot0,5625=\cfrac{131}{72}\cdot\cfrac{144}{100}-\cfrac{8}{15}\cdot\cfrac{5625}{10000}=
2,62-0,3=2,32.$\\
4. $\left(\cfrac{31}{66}+1\cfrac{10}{33}\right)\cdot1,32-\cfrac{8}{13}\cdot0,1625=\cfrac{117}{66}\cdot\cfrac{132}{100}-\cfrac{8}{13}\cdot\cfrac{1625}{10000}=
2,34-0,1=2,24.$\\
5. $\cfrac{15\cfrac{13}{29}\cdot3,625+28:\cfrac{7}{15}}{\cfrac{20}{49}\cdot9,8+0,625:0,75}=\cfrac{\cfrac{448}{29}\cdot\cfrac{3625}{1000}+28\cdot\cfrac{15}{7}}{\cfrac{20}{49}\cdot\cfrac{98}{10}+\cfrac{625}{1000}:\cfrac{75}{100}}=
\cfrac{56+60}{4+\cfrac{5}{6}}=\cfrac{116}{\cfrac{29}{6}}=24.$ Значит, искомое число равно \\$24:0,048=500.$\\
6. $\cfrac{12\cdot(3,4-1,275)\cdot\cfrac{16}{17}}{\cfrac{5}{18}\cdot\left(1\cfrac{7}{85}
+6\cfrac{2}{17}\right):\cfrac{1}{2}}=\cfrac{12\cdot2,125\cdot\cfrac{16}{17}}{\cfrac{5}{18}\cdot\cfrac{612}{85}\cdot\cfrac{2}{1}}=
\cfrac{12\cdot2,125\cdot\cfrac{16}{17}}{\cfrac{5}{18}\cdot\cfrac{612}{85}\cdot\cfrac{2}{1}}=\cfrac{24}{4}=6.$ Значит, искомое число равно $6:0,024=250.$\\
7. $\left(-5,17:1\cfrac{3}{4}+1,67\cdot\cfrac{4}{7}\right)\cdot\left(-1\cfrac{1}{11}\right)=\left(-5,17\cdot\cfrac{4}{7}+1,67\cdot\cfrac{4}{7}\right)\cdot\left(-1\cfrac{1}{11}\right)=$\\
$=\cfrac{4}{7}\cdot\left(-5,17+1,67\right)\cdot\left(-\cfrac{12}{11}\right)=\cfrac{4}{7}\cdot\left(-\cfrac{7}{2}\right)\cdot\left(-\cfrac{12}{11}\right)=\cfrac{24}{11}.$\\
8. $\left(7,42\cdot\cfrac{5}{9}-(-11,48):1\cfrac{4}{5}\right):0,35=\left(7,42\cdot\cfrac{5}{9}+11,48\cdot\cfrac{5}{9}\right):\cfrac{35}{100}=
\cfrac{5}{9}\cdot\left(7,42+11,48\right)\cdot\cfrac{100}{35}=\cfrac{5}{9}\cdot\cfrac{189}{10}\cdot$\\$\cdot\cfrac{20}{7}=30.$\\
9. $\left(3,14\cdot\cfrac{15}{4}-2,72:\left(-6\cfrac{2}{5}\right)\right):\left(23,9-\cfrac{351}{30}\right)=
\left(\cfrac{314}{100}\cdot\cfrac{15}{4}-\cfrac{272}{100}\cdot\left(-\cfrac{5}{32}\right)\right):\left(\cfrac{239}{10}-\cfrac{117}{10}\right)=$\\
$=\left(\cfrac{471}{40}+\cfrac{17}{40}\right):\cfrac{122}{10}=\cfrac{488}{40}\cdot\cfrac{10}{122}=1.$\\
10. $\left(27,2:5\cfrac{3}{5}-0,314\cdot\left(-\cfrac{50}{7}\right)\right):\left(\cfrac{942}{200}+2,39\right)=
\left(\cfrac{272}{10}\cdot\cfrac{5}{28}-\cfrac{314}{1000}\cdot\left(-\cfrac{50}{7}\right)\right):\left(\cfrac{942}{200}+\cfrac{239}{100}\right)=$\\
$=\left(\cfrac{34}{7}+\cfrac{157}{70}\right):\cfrac{710}{100}=\cfrac{497}{70}\cdot\cfrac{100}{710}=1.$\\
11. $\left(1\cfrac{3}{4}:1,125-1,75:\cfrac{2}{3}\right)\cdot1\cfrac{5}{7}=
\left(\cfrac{7}{4}:\cfrac{1125}{1000}-\cfrac{175}{100}:\cfrac{2}{3}\right)\cdot\cfrac{12}{7}=
\left(\cfrac{7}{4}\cdot\cfrac{8}{9}-\cfrac{7}{4}\cdot\cfrac{3}{2}\right)\cdot\cfrac{12}{7}=
\left(\cfrac{14}{9}-\cfrac{21}{8}\right)\cdot\cfrac{12}{7}=$\\$=-\cfrac{77}{72}\cdot\cfrac{12}{7}=-\cfrac{11}{6}.$\\
12. $19\cfrac{1}{3}:4,75-\left(5\cfrac{1}{3}-3,5\cdot\left(-\cfrac{4}{19}\right)\right)=
\cfrac{58}{3}:\cdot{475}{100}-\left(\cfrac{16}{3}+\cfrac{7}{2}\cdot\cfrac{4}{19}\right)=
\cfrac{58}{3}\cdot\cfrac{4}{19}-\left(\cfrac{16}{3}+\cfrac{14}{19}\right)=$\\$=\cfrac{232}{57}-\cfrac{346}{57}=-\cfrac{114}{57}=-2.$\\
13. $17,5\cdot\cfrac{4}{17}-\left(6\cfrac{1}{3}-3\cfrac{1}{3}:(-4,25)\right)=
\cfrac{175}{10}\cdot\cfrac{4}{17}-\left(\cfrac{19}{3}+\cfrac{10}{3}\cdot\cfrac{100}{425}\right)=
\cfrac{70}{17}-\left(\cfrac{19}{3}+\cfrac{40}{51}\right)=\cfrac{70}{17}-\cfrac{121}{17}=-3.$\\
14. $\left(1\cfrac{13}{14}-3\cfrac{4}{7}:1,25\right)\cdot\left(2015\cfrac{3}{7}-2016\cfrac{46}{91}\right)-
(-0,7)^2=\left(\cfrac{27}{14}-\cfrac{25}{7}\cdot\cfrac{100}{125}\right)\cdot\left(-\left(2016\cfrac{46}{91}-2015\cfrac{39}{91}\right)\right)-$\\
$-0,49=\left(\cfrac{27}{14}-\cfrac{20}{7}\right)\cdot\left(-\left(1\cfrac{1}{13}\right)\right)-
0,49=\left(-\cfrac{13}{14}\right)\cdot\left(-\cfrac{14}{13}\right)-0,49=0,51.$ Искомое число равно $0,51:0,17=3.$\\
15. $\left(1\cfrac{5}{14}-3\cfrac{3}{7}:1,6\right)\cdot\left(2015\cfrac{5}{7}-2016\cfrac{27}{77}\right)-
(-0,2)^2=\left(\cfrac{19}{14}-\cfrac{24}{7}\cdot\cfrac{10}{16}\right)\cdot\left(-\left(2016\cfrac{27}{77}-2015\cfrac{55}{77}\right)\right)-$\\
$-0,04=\left(\cfrac{19}{14}-\cfrac{15}{7}\right)\cdot\left(-\cfrac{49}{77}\right)-0,04=\left(-\cfrac{11}{14}\right)\cdot\left(-\cfrac{7}{11}\right)-0,04=
0,46.$ Искомое число равно $0,46:0,23=2.$\\
16. $\cfrac{3\cdot\left(\cfrac{7}{30}-0,125:1\cfrac{1}{8}\right)}{\left(5:1,8-2\cfrac{1}{3}:1,5\right):2\cfrac{2}{3}}=
\cfrac{3\cdot\left(\cfrac{7}{30}-\cfrac{125}{1000}\cdot\cfrac{8}{9}\right)}{\left(5\cdot\cfrac{10}{18}-\cfrac{7}{3}\cdot{2}{3}\right)\cdot\cfrac{3}{8}}=
\cfrac{3\cdot\left(\cfrac{7}{30}-\cfrac{1}{9}\right)}{\left(\cfrac{25}{9}-\cfrac{14}{9}\right)\cdot\cfrac{3}{8}}=
\cfrac{3\cdot\cfrac{11}{90}}{\cfrac{11}{9}\cdot\cfrac{3}{8}}=\cfrac{\cfrac{11}{30}}{\cfrac{11}{24}}=\cfrac{4}{5}.$\\
17. $\cfrac{5\cdot\left(\cfrac{26}{35}-3,75:8\cfrac{3}{4}\right)}{\left(13:1\cfrac{2}{3}-0,3:\cfrac{1}{18}\right)\cdot3\cfrac{4}{7}}=
\cfrac{5\cdot\left(\cfrac{26}{35}-\cfrac{375}{100}\cdot\cfrac{4}{35}\right)}{\left(13\cdot\cfrac{3}{5}-\cfrac{3}{10}\cdot\cfrac{18}{1}\right)\cdot\cfrac{25}{7}}=
\cfrac{5\cdot\left(\cfrac{26}{35}-\cfrac{3}{7}\right)}{\left(\cfrac{39}{5}-\cfrac{27}{5}\right)\cdot\cfrac{25}{7}}=
\cfrac{5\cdot\cfrac{11}{35}}{\cfrac{12}{5}\cdot\cfrac{25}{7}}=\cfrac{\cfrac{11}{7}}{\cfrac{60}{7}}=\cfrac{11}{60}.$\\
18. $\cfrac{0,725+0,6+\cfrac{7}{40}+\cfrac{11}{20}}{0,128\cdot6\cfrac{1}{4}-0,0345:\cfrac{3}{25}}\cdot0,25=
\cfrac{1,325+\cfrac{29}{40}}{\cfrac{128}{1000}\cdot\cfrac{25}{4}-\cfrac{345}{10000}\cdot\cfrac{25}{3}}\cdot\cfrac{1}{4}=
\cfrac{\cfrac{1325}{1000}+\cfrac{29}{40}}{\cfrac{4}{5}-\cfrac{23}{80}}\cdot\cfrac{1}{4}=
\cfrac{\cfrac{41}{20}}{\cfrac{41}{80}}\cdot\cfrac{1}{4}=4\cdot\cfrac{1}{4}=1.$\\
19. $A=3\cfrac{23}{38}+2\cfrac{35}{38}:\left(24,175-28\cfrac{4}{5}\right)=3\cfrac{23}{38}+2\cfrac{35}{38}:\left(-\left(28\cfrac{4}{5}-24\cfrac{175}{1000}\right)\right)=
3\cfrac{23}{38}+\cfrac{111}{38}\cdot\left(-\cfrac{8}{37}\right)=$\\$=3\cfrac{23}{38}-\cfrac{24}{38}=2\cfrac{37}{38}.$ Для вычисления значения $P$ обозначим $x=857,$ тогда $P=(x-1)(x+1)-$\\$-(x+2)(x-2)=(x^2-1)-(x^2-4)=3.$ $K=2+\cfrac{81}{243}+\cfrac{27}{243}+\cfrac{18}{243}+\cfrac{6}{243}+\cfrac{1}{243}=2\cfrac{133}{243}.$ Значит, наибольшим является число $P.$\\
20. $B=5\cfrac{23}{34}+3\cfrac{27}{34}:\left(23,225-28\cfrac{3}{5}\right)=5\cfrac{23}{34}+3\cfrac{27}{34}:\left(-\left(28\cfrac{3}{5}-23\cfrac{225}{1000}\right)\right)=
5\cfrac{23}{34}+\cfrac{129}{34}\cdot\left(-\cfrac{8}{43}\right)=5\cfrac{23}{34}-$\\$-\cfrac{24}{34}=4\cfrac{33}{34}.$ Для вычисления значения $Q$ обозначим $x=766,$ тогда $Q=(x+2)(x-2)-(x+3)(x-3)=$\\$=(x^2-4)-(x^2-9)=5.$ $M=4\cfrac{81}{243}+\cfrac{27}{243}+\cfrac{18}{243}+\cfrac{3}{243}+\cfrac{2}{243}=4\cfrac{131}{243}.$
Значит, наибольшим является\\ число $Q.$\\
21. $\cfrac{1,7\cdot9,6+3,5\cdot1,7-1,7\cdot 3,1}{12\cfrac{3}{4}:\left(1\cfrac{8}{15}+0,25-3\cfrac{1}{30}-1\cfrac{3}{4}\right)}=
\cfrac{1,7\cdot(9,6+3,5-3,1)}{\cfrac{51}{4}:\left(\cfrac{23}{15}+\cfrac{1}{4}-\cfrac{91}{30}-\cfrac{7}{4}\right)}=
\cfrac{1,7\cdot10}{\cfrac{51}{4}:\left(-\cfrac{6}{4}-\cfrac{45}{30}\right)}=\cfrac{17}{\cfrac{51}{4}:\left(-3\right)}=$\\$=\cfrac{17}{-\cfrac{17}{4}}=-4.$\\
22. $\cfrac{6,5\cdot1,5-7,6\cdot6,5-13,9\cdot6,5}{\left(3\cfrac{4}{5}+1\cfrac{5}{12}-0,8-3\cfrac{1}{3}\right):\cfrac{1}{4}}=
\cfrac{6,5\cdot(1,5-7,6-13,9)}{\left(\cfrac{19}{5}+\cfrac{17}{12}-\cfrac{4}{5}-\cfrac{10}{3}\right)\cdot4}=
\cfrac{6,5\cdot(-20)}{\left(3-\cfrac{23}{12}\right)\cdot4}=\cfrac{-130}{\cfrac{13}{12}\cdot4}=\cfrac{-130}{\cfrac{13}{3}}=$\\$=-30.$\\
23. $\left(0,315\cdot0,725-0,75:\cfrac{3}{20}\cdot0,01+0,315\cdot0,275\right):\left(\cfrac{1}{4}-\cfrac{7}{20}\right)=$\\$=
\left(0,315\cdot(0,725+0,275)-\cfrac{3}{4}\cdot\cfrac{20}{3}\cdot\cfrac{1}{100}\right):\left(\cfrac{5}{20}-\cfrac{7}{20}\right)=
\left(0,315-\cfrac{1}{20}\right):\left(-\cfrac{1}{10}\right)=\left(0,315-0,05\right)\cdot$\\$\cdot\left(-10\right)=-2,65.$\\
24. $\left(1,48-\left(46:27-(0,52-16:54)\right)\right)^2=\left(1,48-\cfrac{46}{27}+0,52-\cfrac{8}{27}\right)^2=(2-2)^2=0.$\\
25. $\left(20\cfrac{4}{5}+1\cfrac{5}{12}-0,8-3\cfrac{1}{3}\right)\cdot\cfrac{1}{6}=
\left(20+1\cfrac{5}{12}-3\cfrac{4}{12}\right)\cdot\cfrac{1}{6}=18\cfrac{1}{12}\cdot\cfrac{1}{6}=3\cfrac{1}{72}.$\\
26. $\left(1\cfrac{1}{4}-14,05\right):0,04+13,8:\cfrac{1}{13}=-\left(14,05-1,25\right):0,04+13,8\cdot13=
-12,8:0,04+13,8\cdot13=$\\$=-320+179,4=-140,6.$\\
27. $\cfrac{0,03+0,07:\left(1\cfrac{7}{24}+\cfrac{7}{30}-2\cfrac{9}{40}\right)+\cfrac{7}{36}\cdot\cfrac{18}{35}}{-0,1^2}=
\left(0,03+0,07:\left(1\cfrac{63}{120}-2\cfrac{9}{40}\right)+\cfrac{1}{10}\right)\cdot(-100)=$\\$=
(0,03+0,07:\left(-\cfrac{7}{10}\right)+0,1)\cdot(-100)=(0,03-0,1+0,1)\cdot(-100)=-3.$\\
28. $2379\cdot23782378-2378\cdot23792379=2379\cdot2378\cdot10001-2378\cdot2379\cdot10001=0.$\\
29. Пусть $x=21567,$ тогда $\cfrac{((x-10)(x+10)+100)((x-20)(x+20)+400)}{x^4}=$\\$=\cfrac{(x^2-100+100)(x^2-400+400)}{x^4}=
\cfrac{x^2\cdot x^2 }{x^4}=1.$\\
30. Пусть $x=32012,$ тогда $\cfrac{((x-10)(x+10)+100)((x-20)(x+20)+400)}{x^4}=$\\$=\cfrac{(x^2-100+100)(x^2-400+400)}{x^4}=
\cfrac{x^2\cdot x^2 }{x^4}=1.$\\
31. Пусть $x=41812,$ тогда $\cfrac{((x-10)(x+10)+100)((x-20)(x+20)+400)}{x^4}=$\\$=\cfrac{(x^2-100+100)(x^2-400+400)}{x^4}=
\cfrac{x^2\cdot x^2 }{x^4}=1.$\\
32. Пусть $x=51917,$ тогда $\cfrac{((x-10)(x+10)+100)((x-20)(x+20)+400)}{x^4}=$\\$=\cfrac{(x^2-100+100)(x^2-400+400)}{x^4}=
\cfrac{x^2\cdot x^2 }{x^4}=1.$\\
33. $\cfrac{3\cfrac{1}{3}\cdot1,9+19,5:4\cfrac{1}{2}}{\cfrac{62}{75}-0,16}:\cfrac{3,5+4\cfrac{2}{3}+2\cfrac{2}{15}}{0,5\left(1\cfrac{1}{20}+4,1\right)}=
\cfrac{\cfrac{10}{3}\cdot\cfrac{19}{10}+\cfrac{195}{10}\cdot\cfrac{2}{9}}{\cfrac{62}{75}-\cfrac{4}{25}}:\cfrac{3,5+6,8}{0,5\left(1,05+4,1\right)}=
\cfrac{\cfrac{19}{3}+\cfrac{13}{3}}{\cfrac{2}{3}}:\cfrac{10,3}{2,575}=$\\$=
\cfrac{32}{3}\cdot\cfrac{3}{2}:4=4.$\\
34. $\cfrac{\left(\cfrac{3}{5}+0,425-0,005\right):0,1}{30,5+\cfrac{1}{6}+3\cfrac{1}{3}}+\cfrac{6\cfrac{3}{4}+5\cfrac{1}{2}}
{26:3\cfrac{5}{7}}-0,05=\cfrac{\left(0,6+0,42\right)\cdot10}{30,5+3,5}+\cfrac{12\cfrac{1}{4}}
{26:\cfrac{26}{7}}-0,05=\cfrac{1,02\cdot10}{34}+$\\$+\cfrac{\cfrac{49}{4}}
{7}-0,05=0,3+1,75-0,05=2.$\\
35. $\cfrac{278\cfrac{1}{6}+529\cfrac{1}{21}+129\cfrac{5}{7}+54\cfrac{3}{14}}{\left(1\cfrac{3}{8}-4,2638+7\cfrac{3}{7}-2,7362+1\cfrac{11}{56}\right)\cdot(2,39\cdot73-236+23,9\cdot2,7)}=$\\$=
\cfrac{278+529+129+54+\cfrac{7}{42}+\cfrac{2}{42}+\cfrac{30}{42}+\cfrac{9}{42}}{\left(1\cfrac{21}{56}+7\cfrac{24}{56}+1\cfrac{11}{56}-(2,7362+4,2638)\right)\cdot(23,9\cdot7,3-236+23,9\cdot2,7)}=$\\$=
\cfrac{990+\cfrac{8}{7}}{\left(10-7\right)\cdot(23,9\cdot(7,3+2,7)-236)}=\cfrac{990+\cfrac{8}{7}}{3\cdot3}=110\cfrac{8}{63}.$\\
36. $\cfrac{119\cfrac{1}{42}+289\cfrac{23}{42}+108\cfrac{1}{6}+144\cfrac{3}{7}}
{\left(3\cfrac{2}{9}-4,4561+5\cfrac{3}{7}-5,5439+3\cfrac{22}{63}\right)\cdot
(3,66\cdot49-363+36,6\cdot5,1)}=$\\$=\cfrac{119+289+108+144+\cfrac{1}{42}+\cfrac{23}{42}+\cfrac{1}{6}+\cfrac{3}{7}}
{\left(3\cfrac{2}{9}+5\cfrac{3}{7}+3\cfrac{22}{63}-(4,4561+5,5439)\right)\cdot
(36,6\cdot4,9-363+36,6\cdot5,1)}=$\\$=\cfrac{660+\cfrac{7}{6}}
{\left(12-10\right)\cdot(36,6\cdot(4,9+5,1)-363)}=\cfrac{660+\cfrac{7}{6}}{2\cdot3}=110\cfrac{7}{36}.$\\
37. $\cfrac{\left(-1\cfrac{4}{5}-6\cfrac{1}{3}+8,75\right):1\cfrac{2}{3}-\cfrac{-2,08}{16}}
{-1\cfrac{7}{25}:1,92+\cfrac{7}{22}\cdot\left(-3\cfrac{2}{3}\right)}=
\cfrac{\left(1-\cfrac{4}{5}-\cfrac{1}{3}+\cfrac{3}{4}\right):\cfrac{5}{3}+\cfrac{208}{1600}}
{-\cfrac{32}{25}:\cfrac{192}{100}-\cfrac{7}{22}\cdot\cfrac{11}{3}}=
\cfrac{\cfrac{37}{60}\cdot\cfrac{3}{5}+\cfrac{208}{1600}}
{-\cfrac{32}{25}\cdot\cfrac{100}{192}-\cfrac{7}{6}}=$\\$=
\cfrac{\cfrac{37}{100}+\cfrac{13}{100}}
{-\cfrac{2}{3}-\cfrac{7}{6}}=\cfrac{\cfrac{1}{2}}{-\cfrac{11}{6}}=-\cfrac{3}{11}.$\\
38. $\cfrac{0,27\cdot\left(4\cfrac{2}{5}-0,9\right)-\cfrac{4.2}{4\cfrac{4}{9}}}
{\cfrac{6,2}{0,31}-\cfrac{5}{6}\cdot0,9}=\cfrac{0,27\cdot\left(4,4-0,9\right)-\cfrac{\cfrac{42}{10}}{\cfrac{40}{9}}}
{20-\cfrac{5}{6}\cdot\cfrac{9}{10}}=\cfrac{0,27\cdot3.5-\cfrac{189}{200}}
{20-\cfrac{25}{27}}=\cfrac{0,27\cdot3.5-\cfrac{189}{200}}
{20-\cfrac{25}{27}}=$\\$=\cfrac{\cfrac{189}{200}-\cfrac{189}{200}}
{20-\cfrac{25}{27}}=0.$\\
39. $\cfrac{180\cdot3,91-168+859\cdot1,8-768}{239\cfrac{5}{6}-237\cfrac{2}{3}}=
\cfrac{1,8\cdot391-168+859\cdot1,8-768}{\cfrac{13}{6}}=\cfrac{1,8\cdot(391+859)-936}{\cfrac{13}{6}}=$\\$=\cfrac{1,8\cdot1250-936}{\cfrac{13}{6}}=
\cfrac{2250-936}{\cfrac{13}{6}}=\cfrac{1313+1}{\cfrac{13}{6}}=606\cfrac{6}{13}.$\\
40. $\cfrac{1,7\cdot229-1155+7,91\cdot170+937}{366\cfrac{5}{6}-364\cfrac{29}{42}}=\cfrac{1,7\cdot229-1155+791\cdot1,70+937}{\cfrac{15}{7}}=$\\$=
\cfrac{1,7\cdot(229+791)-1155+937}{\cfrac{15}{7}}=\cfrac{1,7\cdot1020-218}{\cfrac{15}{7}}=\cfrac{1734-218}{\cfrac{15}{7}}=
\cfrac{1515+1}{\cfrac{15}{7}}=707\cfrac{7}{15}.$\\
41. $\left(3\cfrac{7}{12}+4\cfrac{7}{12}: \left(2\cfrac{1}{3}-5\cfrac{1}{12}\right)\right):
\left(3,25:5\cfrac{7}{22}-8\cfrac{5}{18}\right)=$\\$=\left(3\cfrac{7}{12}-\cfrac{55}{12}: \left(5\cfrac{1}{12}-2\cfrac{4}{12}\right)\right):
\left(\cfrac{325}{100}:\cfrac{117}{22}-8\cfrac{5}{18}\right)=\left(3\cfrac{7}{12}-\cfrac{55}{12}:\cfrac{11}{4}\right):
\left(\cfrac{325}{100}\cdot\cfrac{22}{117}-8\cfrac{5}{18}\right)=$\\$=\left(\cfrac{43}{1 2}-\cfrac{20}{12}\right):
\left(\cfrac{11}{18}-\cfrac{149}{18}\right)=\cfrac{23}{12}:
\left(-\cfrac{138}{18}\right)=\cfrac{23}{12}\cdot\left(-\cfrac{18}{138}\right)=-\cfrac{1}{4}.$\\
42. Пусть $x=1567,$ тогда $1566\cdot1568+1=(x-1)(x+1)+1=x^2-1+1=x^2.$\\
43. Пусть $x=28\cfrac{13}{17},$ тогда $29\cfrac{13}{17}\cdot 30\cfrac{13}{17}-28\cfrac{13}{17}\cdot31\cfrac{13}{17}=
(x+1)(x+2)-x(x+3)=(x^2+2x+x+2)-$\\$-(x^2+3x)=x^2+3x+2-x^2-3x=2.$ Значит, сравниваемые числа равны.\\
44. $\cfrac{2\cfrac{3}{4}:1,1+3\cfrac{1}{3}}{2,5-0,4\cdot3\cfrac{1}{3}}:\cfrac{5}{7}-
\cfrac{\left(2\cfrac{1}{6}+4,5\right)\cdot0,375}{2,75-1\cfrac{1}{2}}=
\cfrac{\cfrac{5}{2}+\cfrac{10}{3}}{\cfrac{5}{2}-\cfrac{4}{3}}\cdot\cfrac{7}{5}-
\cfrac{\cfrac{20}{3}\cdot\cfrac{3}{8}}{1,25}=\cfrac{\cfrac{35}{6}}{\cfrac{7}{6}}\cdot\cfrac{7}{5}-
\cfrac{\cfrac{5}{2}}{\cfrac{5}{4}}=5\cdot\cfrac{7}{5}-2=7-2=5.$\\
45. $\cfrac{3\cfrac{1}{3}:10+0,175:0,35}{1,75-1\cfrac{11}{17}\cdot\cfrac{51}{56}}-
\cfrac{\left(\cfrac{11}{18}-\cfrac{1}{15}\right):1,4}{\left(0,5-\cfrac{1}{9}\right)\cdot3}=
\cfrac{\cfrac{10}{3}:10+0,5}{1,75-\cfrac{28}{17}\cdot\cfrac{51}{56}}-
\cfrac{\cfrac{49}{90}:\cfrac{14}{10}}{\left(\cfrac{1}{2}-\cfrac{1}{9}\right)\cdot3}=
\cfrac{\cfrac{1}{3}+\cfrac{1}{2}}{1,75-1,5}-
\cfrac{\cfrac{49}{90}\cdot\cfrac{5}{7}}{\cfrac{7}{18}\cdot3}=$\\$=
\cfrac{\cfrac{5}{6}}{\cfrac{1}{4}}-
\cfrac{\cfrac{7}{18}}{\cfrac{7}{6}}=\cfrac{10}{3}-\cfrac{1}{3}=3.$\\
46. $\cfrac{\left(85\cfrac{7}{30}-83\cfrac{5}{18}\right):2\cfrac{2}{3}}{0,04}\cdot\cfrac{3}{11}=
\cfrac{\cfrac{88}{45}:\cfrac{8}{3}}{\cfrac{1}{25}}\cdot\cfrac{3}{11}=\cfrac{\cfrac{11}{15}}{\cfrac{1}{25}}\cdot\cfrac{3}{11}=
\cfrac{55}{3}\cdot\cfrac{3}{11}=5.$\\
47. $\cfrac{11}{14}=\cfrac{55}{70}>\cfrac{55}{71}.$\\
48. $A=8,67:0,017-239\cdot2\frac{8}{9}=510-239\cdot\frac{26}{9}=-180\cfrac{4}{9}.$ $B=15\text{НОД}(70, 175)-\text{НОК}(70,175)=15\cdot35-$\\$-350=175.$$C=(-2,(3))^2=\left(\cfrac{7}{3}\right)^2=\cfrac{49}{9}=5\cfrac{4}{9}.$ Значит, $A+B+C=0$ и никакое число сумме двух других не равно.\\
49. $A=7,28:0,013-239\cdot2\frac{5}{9}=560-239\cdot\cfrac{23}{9}=-50\cfrac{7}{9}.$ $B=25\text{НОД}(84, 36)-\text{НОК}(84,36)=25\cdot12-$\\$-252=48.$
$C=(-1,(6))^2=\left(\cfrac{5}{3}\right)^2=\cfrac{25}{9}=2\cfrac{7}{9}.$ Значит, $A+B+C=0$ и никакое число сумме двух других не равно.\\
50. $\left(\left(3,8\cdot1\cfrac{4}{7}-2,5\cdot3,8\right)\cdot4\cfrac{3}{13}-\cfrac{1}{14}\right):2,5=
\left(\cfrac{38}{10}\cdot\left(\cfrac{11}{7}-\cfrac{5}{2}\right)\cdot\cfrac{55}{13}-\cfrac{1}{14}\right):\cfrac{5}{2}=$\\$
\left(\cfrac{19}{5}\cdot\left(-\cfrac{13}{14}\right)\cdot\cfrac{55}{13}-\cfrac{1}{14}\right)\cdot\cfrac{2}{5}=
\left(-\cfrac{209}{14}-\cfrac{1}{14}\right)\cdot\cfrac{2}{5}=(-15)\cdot\cfrac{2}{5}=-6.$\\
51. $\left(2,3-5\cfrac{2}{3}+1,4\right):29,5\cdot3-1,8=
\left(\cfrac{37}{10}-\cfrac{17}{3}\right):\cfrac{59}{2}\cdot3-1,8=
-\cfrac{59}{30}\cdot\cfrac{2}{59}\cdot3-1,8=-0,2-1,8=-2.$\\
52. $\left(2\cfrac{4}{9}\left(2\cfrac{3}{11}\cdot2,7-5,7\cdot2\cfrac{3}{11}\right)-\cfrac{1}{3}\right):5\cfrac{2}{3}=
\left(\cfrac{22}{9}\cdot\cfrac{25}{11}\cdot\left(2,7-5,7\right)-\cfrac{1}{3}\right):\cfrac{17}{3}=$\\$
\left(\cfrac{50}{9}\cdot\left(-3\right)-\cfrac{1}{3}\right)\cdot\cfrac{3}{17}=
\left(-\cfrac{50}{3}-\cfrac{1}{3}\right)\cdot\cfrac{3}{17}=
(-17)\cdot\cfrac{3}{17}=-3.$\\
53. $\left(3,2-\left(4\cfrac{1}{3}-1,5\right)\right):5,5\cdot7-\cfrac{2}{15}=
\left(\cfrac{47}{10}-\cfrac{13}{3}\right):\cfrac{11}{2}\cdot7-\cfrac{2}{15}=
\cfrac{11}{30}\cdot\cfrac{2}{11}\cdot7-\cfrac{2}{15}=
\cfrac{7}{15}-\cfrac{2}{15}=\cfrac{1}{3}.$\\
54. $\cfrac{\left(5\cfrac{4}{45}-4\cfrac{1}{6}\right):5\cfrac{8}{15}}{\left(4\cfrac{2}{3}+0,75\right)\cdot3\cfrac{9}{13}}\cdot34\cfrac{2}{7}+\cfrac{0,3:0,01}{70}+\cfrac{2}{7}=
\cfrac{\left(1\cfrac{8}{90}-\cfrac{15}{90}\right):\cfrac{83}{15}}{\left(4\cfrac{8}{12}+\cfrac{9}{12}\right)\cdot\cfrac{48}{13}}\cdot\cfrac{240}{7}+\cfrac{30}{70}+\cfrac{2}{7}=
\cfrac{\cfrac{83}{90}\cdot\cfrac{15}{83}}{\cfrac{65}{12}\cdot\cfrac{48}{13}}\cdot\cfrac{240}{7}+\cfrac{5}{7}=
\cfrac{\cfrac{1}{6}}{20}\cdot\cfrac{240}{7}+\cfrac{5}{7}=\cfrac{2}{7}+\cfrac{5}{7}=1.$\\
55. $\cfrac{\left(\cfrac{3}{5}+0,425-0,005\right):0,1}{\cfrac{1}{6}+3\cfrac{1}{3}+30,5}+\cfrac{6\cfrac{3}{4}+5\cfrac{1}{2}}{26:3\cfrac{5}{7}}-0,05=
\cfrac{\left(\cfrac{3}{5}+\cfrac{21}{50}\right)\cdot10}{34}+\cfrac{11\cfrac{3}{4}+\cfrac{2}{4}}{26:\cfrac{26}{7}}-0,05=
\cfrac{\cfrac{51}{50}\cdot10}{34}+\cfrac{\cfrac{49}{4}}{7}-0,05=
\cfrac{3}{10}+\cfrac{7}{4}-0,05=2.$\\
56. Пусть $20202021=x,$ тогда $(x-1)(x+1)-(x+2)(x-2)=(x^2+x-x-1)-(x^2-2x+2x-4)=x^2-1-x^2+4=3.$\\
57. $193\cfrac{1}{12}+207\cfrac{1}{3}-225\cfrac{13}{52}=(193+207-225)+\left(\cfrac{1}{12}+\cfrac{1}{3}-\cfrac{1}{4}\right)=175+\cfrac{1}{6}=175\cfrac{1}{6}.$
\newpage
