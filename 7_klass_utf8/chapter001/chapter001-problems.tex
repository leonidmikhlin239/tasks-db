\documentclass[12pt]{article}
%\usepackage{newlistok}
\usepackage[T1,T2A]{fontenc} %
%\usepackage[cp866]{inputenc} %
\usepackage[cp1251]{inputenc} %
\usepackage[russian]{babel} %
\usepackage{amsmath} %
\usepackage{amssymb}
\usepackage{graphicx}
\usepackage{pgfplots}
\graphicspath{{pictures/}}
\DeclareGraphicsExtensions{.pdf,.png,.jpg}
\pgfplotsset{compat=1.9}
\usepackage{tikz}
\textheight=257mm %
\textwidth =180mm
%\usepackage[active]{srcltx} %
\mathsurround=2pt %
%\textheight=185mm \textwidth =120mm %
\textheight=257mm %
\textwidth =190mm %
\hoffset=-18mm %
\voffset=-41mm
\author{Л.С.\;Михлин}
\title{Пособие для поступающих в 5 класс физико--математических школ}
\date{2020 --- ...}
\begin{document}
\tableofcontents
\newpage
\section {Числовые выражения задачи}
$\begin{array}{ll}
1.\ \left(\cfrac{2}{3}-1,3+\cfrac{3}{4}\right):1,4+\cfrac{1}{6},&
2.\ 4,4-0,28:\left(\cfrac{1}{9}-\cfrac{7}{15}+0,375\right),
\\
3.\ \left(1\cfrac{11}{24}+\cfrac{13}{36}\right)\cdot1,44-\cfrac{8}{15}\cdot0,5625,&
4.\ \left(\cfrac{31}{66}+1\cfrac{10}{33}\right)\cdot1,32-\cfrac{8}{13}\cdot0,1625,
\end{array}$
\\
$\begin{array}{ll}
5.\ \text{Найти число, 4,8\% которого равно }\cfrac{15\cfrac{13}{29}\cdot3,625+28:\cfrac{7}{15}}{\cfrac{20}{49}\cdot9,8+0,625:0,75},\\
6.\ \text{Найти число, 2,4\% которого равно }\cfrac{12\cdot(3,4-1,275)\cdot\cfrac{16}{17}}{\cfrac{5}{18}\cdot\left(1\cfrac{7}{85}
+6\cfrac{2}{17}\right):\cfrac{1}{2}},\\
\end{array}$\\
$\begin{array}{ll}
7.\ \left(-5,17:1\cfrac{3}{4}+1,67\cdot\cfrac{4}{7}\right)\cdot\left(-1\cfrac{1}{11}\right),&
8.\ \left(7,42\cdot\cfrac{5}{9}-(-11,48):1\cfrac{4}{5}\right):0,35,\end{array}$ \\ $\begin{array}{ll}
9.\ \left(3,14\cdot\cfrac{15}{4}-2,72:\left(-6\cfrac{2}{5}\right)\right):\left(23,9-\cfrac{351}{30}\right),\\
10.\ \left(27,2:5\cfrac{3}{5}-0,314\cdot\left(-\cfrac{50}{7}\right)\right):\left(\cfrac{942}{200}+2,39\right),&
11.\ \left(1\cfrac{3}{4}:1,125-1,75:\cfrac{2}{3}\right)\cdot1\cfrac{5}{7},\\
12.\ 19\cfrac{1}{3}:4,75-\left(5\cfrac{1}{3}-3,5\cdot\left(-\cfrac{4}{19}\right)\right),&
13.\ 17,5\cdot\cfrac{4}{17}-\left(6\cfrac{1}{3}-3\cfrac{1}{3}:(-4,25)\right),\end{array}$ \\
$\begin{array}{ll}
14.\ \text{Найдите число, 17\% которого равно } \left(1\cfrac{13}{14}-3\cfrac{4}{7}:1,25\right)\cdot\left(2015\cfrac{3}{7}-2016\cfrac{46}{91}\right)-
(-0,7)^2\\
15. \ \text{Найдите число, 23\% которого равно }
\left(1\cfrac{5}{14}-3\cfrac{3}{7}:1,6\right)\cdot\left(2015\cfrac{5}{7}-2016\cfrac{27}{77}\right)-
(-0,2)^2
\end{array}$\\
$\begin{array}{ll}
16.\ \cfrac{3\cdot\left(\cfrac{7}{30}-0,125:1\cfrac{1}{8}\right)}{\left(5:1,8-2\cfrac{1}{3}:1,5\right):2\cfrac{2}{3}}&
17.\ \cfrac{5\cdot\left(\cfrac{26}{35}-3,75:8\cfrac{3}{4}\right)}{\left(13:1\cfrac{2}{3}-0,3:\cfrac{1}{18}\right)\cdot3\cfrac{4}{7}}\\
18.\ \cfrac{0,725+0,6+\cfrac{7}{40}+\cfrac{11}{20}}{0,128\cdot6\cfrac{1}{4}-0,0345:\cfrac{3}{25}}\cdot0,25,&
 \end{array}$\\
$\begin{array}{ll}
19.\ \text{Найдите наибольшее из чисел } A=3\cfrac{23}{38}+2\cfrac{35}{38}:\left(24,175-28\cfrac{4}{5}\right),\\ P=856\cdot858-859\cdot855,\ K=2\cfrac{1}{3}+\cfrac{1}{9}+\cfrac{2}{27}+\cfrac{2}{81}+\cfrac{1}{243},\\
20.\ \text{Найдите наибольшее из чисел }
B=5\cfrac{23}{34}+3\cfrac{27}{34}:\left(23,225-28\cfrac{3}{5}\right),\\
Q=768\cdot764-769\cdot763,\ M=4\cfrac{1}{3}+\cfrac{1}{9}+\cfrac{2}{27}+\cfrac{1}{81}+\cfrac{2}{243},\end{array}$\\
$\begin{array}{ll}
21.\ \cfrac{1,7\cdot9,6+3,5\cdot1,7-1,7\cdot 3,1}{12\cfrac{3}{4}:\left(1\cfrac{8}{15}+0,25-3\cfrac{1}{30}-1\cfrac{3}{4}\right)},&
22.\ \cfrac{6,5\cdot1,5-7,6\cdot6,5-13,9\cdot6,5}{\left(3\cfrac{4}{5}+1\cfrac{5}{12}-0,8-3\cfrac{1}{3}\right):\cfrac{1}{4}},\end{array}$
\\
$\begin{array}{ll}
23.\ \left(0,315\cdot0,725-0,75:\cfrac{3}{20}\cdot0,01+0,315\cdot0,275\right):
\left(\cfrac{1}{4}-\cfrac{7}{20}\right),
\end{array}$\\
$\begin{array}{ll}
24.\ \left(1,48-\left(46:27-(0,52-16:54)\right)\right)^2,&
25.\ \left(20\cfrac{4}{5}+1\cfrac{5}{12}-0,8-3\cfrac{1}{3}\right)\cdot\cfrac{1}{6},\\
26.\ \left(1\cfrac{1}{4}-14,05\right):0,04+13,8:\cfrac{1}{13},&
27.\ \cfrac{0,03+0,07:\left(1\cfrac{7}{24}+\cfrac{7}{30}-2\cfrac{9}{40}\right)+
\cfrac{7}{36}\cdot\cfrac{18}{35}}{-0,1^2},\end{array}$\\
$\begin{array}{ll}
28.\ 2379\cdot23782378-2378\cdot23792379,\end{array}$\\
$\begin{array}{ll}
29.\ \cfrac{(21557\cdot21577+100)(21547\cdot21587+400)}{21567^4},&
30.\ \cfrac{(32002\cdot32022+100)(31992\cdot32032+400)}{32012^4},\\
31.\ \cfrac{(41802\cdot41822+100)(41792\cdot41832+400)}{41812^4},&
32.\ \cfrac{(51907\cdot51927+100)(51897\cdot51937+400)}{51917^4},
\end{array}$\\
$\begin{array}{ll}
33.\ \cfrac{3\cfrac{1}{3}\cdot1,9+19,5:4\cfrac{1}{2}}{\cfrac{62}{75}-0,16}:\cfrac{3,5+4\cfrac{2}{3}+2\cfrac{2}{15}}{0,5\left(1\cfrac{1}{20}+4,1\right)},
34.\
\cfrac{\left(\cfrac{3}{5}+0,425-0,005\right):0,1}{30,5+\cfrac{1}{6}+3\cfrac{1}{3}}+\cfrac{6\cfrac{3}{4}+5\cfrac{1}{2}}
{26:3\cfrac{5}{7}}-0,05,
\end{array}$\\
$\begin{array}{ll}
35.\ \cfrac{278\cfrac{1}{6}+529\cfrac{1}{21}+129\cfrac{5}{7}+54\cfrac{3}{14}}{\left(1\cfrac{3}{8}
-4,2638+7\cfrac{3}{7}-2,7362+1\cfrac{11}{56}\right)\cdot(2,39\cdot73-236+23,9\cdot2,7)},\\
36.\
\cfrac{119\cfrac{1}{42}+289\cfrac{23}{42}+108\cfrac{1}{6}+144\cfrac{3}{7}}
{\left(3\cfrac{2}{9}-4,4561+5\cfrac{3}{7}-5,5439+3\cfrac{22}{63}\right)\cdot
(3,66\cdot49-363+36,6\cdot5,1)},\end{array}$\\
$\begin{array}{ll}
37.\ \cfrac{\left(-1\cfrac{4}{5}-6\cfrac{1}{3}+8,75\right):1\cfrac{2}{3}-\cfrac{-2,08}{16}}
{-1\cfrac{7}{25}:1,92+\cfrac{7}{22}\cdot\left(-3\cfrac{2}{3}\right)},&
38.\ \cfrac{0,27\cdot\left(4\cfrac{2}{5}-0,9\right)-\cfrac{4.2}{4\cfrac{4}{9}}}
{\cfrac{6,2}{0,31}-\cfrac{5}{6}\cdot0,9},\\
39.\ \cfrac{180\cdot3,91-168+859\cdot1,8-768}{239\cfrac{5}{6}-237\cfrac{2}{3}},&
40.\ \cfrac{1,7\cdot229-1155+7,91\cdot170+937}{366\cfrac{5}{6}-364\cfrac{29}{42}},\end{array}\\
\begin{array}{l}
41.\ \left(3\cfrac{7}{12}+4\cfrac{7}{12}: \left(2\cfrac{1}{3}-5\cfrac{1}{12}\right)\right):
\left(3,25:5\cfrac{7}{22}-8\cfrac{5}{18}\right),\\
42.\ \text{Найдите положительное число, если его квадрат равен } 1566\cdot1568+1,\\
43.\ \text{Сравните } 2\text{ и } 29\cfrac{13}{17}\cdot 30\cfrac{13}{17}-28\cfrac{13}{17}\cdot31\cfrac{13}{17}.
\end{array}$\\
$\begin{array}{ll}
44.\ \cfrac{2\cfrac{3}{4}:1,1+3\cfrac{1}{3}}{2,5-0,4\cdot3\cfrac{1}{3}}:\cfrac{5}{7}-
\cfrac{\left(2\cfrac{1}{6}+4,5\right)\cdot0,375}{2,75-1\cfrac{1}{2}},&
45.\ \cfrac{3\cfrac{1}{3}:10+0,175:0,35}{1,75-1\cfrac{11}{17}\cdot\cfrac{51}{56}}-
\cfrac{\left(\cfrac{11}{18}-\cfrac{1}{15}\right):1,4}{\left(0,5-\cfrac{1}{9}\right)\cdot3},\\
46.\ \cfrac{\left(85\cfrac{7}{30}-83\cfrac{5}{18}\right):2\cfrac{2}{3}}{0,04}\cdot\cfrac{3}{11},&
47.\ \text{Сравните } \cfrac{11}{14} \text{ и } \cfrac{55}{71}.
\end{array}$\\
48. Выясните, равно ли одно из чисел сумме двух других\\
$A=8,67:0,017-239\cdot2\frac{8}{9},\ B=15\text{НОД}(70, 175)-\text{НОК}(70,175),\ C=(-2,(3))^2.$\\
49. Выясните, равно ли одно из чисел сумме двух других\\
$A=7,28:0,013-239\cdot2\frac{5}{9},\ B=25\text{НОД}(84, 36)-\text{НОК}(84,36),\ C=(-1,(6))^2.$\\
50. $\left(\left(3,8\cdot1\cfrac{4}{7}-2,5\cdot3,8\right)\cdot4\cfrac{3}{13}-\cfrac{1}{14}\right):2,5.$\quad
51. $\left(2,3-5\cfrac{2}{3}+1,4\right):29,5\cdot3-1,8.$\\
52. $\left(2\cfrac{4}{9}\left(2\cfrac{3}{11}\cdot2,7-5,7\cdot2\cfrac{3}{11}\right)-\cfrac{1}{3}\right):5\cfrac{2}{3}.$\quad
53. $\left(3,2-\left(4\cfrac{1}{3}-1,5\right)\right):5,5\cdot7-\cfrac{2}{15},$\\
54. $\cfrac{\left(5\cfrac{4}{45}-4\cfrac{1}{6}\right):5\cfrac{8}{15}}{\left(4\cfrac{2}{3}+0,75\right)\cdot3\cfrac{9}{13}}\cdot34\cfrac{2}{7}+\cfrac{0,3:0,01}{70}+\cfrac{2}{7},$\quad
55. $\cfrac{\left(\cfrac{3}{5}+0,425-0,005\right):0,1}{\cfrac{1}{6}+3\cfrac{1}{3}+30,5}+\cfrac{6\cfrac{3}{4}+5\cfrac{1}{2}}{26:3\cfrac{5}{7}}-0,05.$\\
56. $20202020\cdot20202022-20202023\cdot20202019.$\quad
57. $193\cfrac{1}{12}+207\cfrac{1}{3}-225\cfrac{13}{52}.$
\newpage
