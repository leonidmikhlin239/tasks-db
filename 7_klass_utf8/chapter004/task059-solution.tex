59. Пусть масса третьего слитка равна $m,$ а доля серебра в нём равна $x.$ Тогда имеем систему уравнений $\begin{cases}0,56(5+m)=5\cdot0,3+xm,\\0,6(3+m)=3\cdot0,3+xm.\end{cases}\Leftrightarrow
\begin{cases}2,8+0,56m=1,5+xm,\\ 1,8+0,6m=0,9+xm.\end{cases}\Leftrightarrow
\begin{cases}1,3+0,56m=xm,\\ 0,9+0,6m=xm.\end{cases}\Leftrightarrow
\begin{cases}0,04m-0,4=0,\\ x=\cfrac{0,9+0,6m}{m}.\end{cases}\Leftrightarrow
\begin{cases}m=10,\\ x=0,69.\end{cases}$ Значит, масса третьего слитка равна 10 кг, а процентное содержание серебра в нём --- $69\%.$\\
