29. Пусть у Васи $x$ фантиков, а у Пети $y.$ Тогда сначала у Пети стало $0,8y$ фантиков, а у Васи $x+0,2y.$ Затем у Пети станет $0,8y+0,2(x+0,2y)=0,2x+0,84y,$ что равно изначальному количеству, а значит $0,2x+0,84y=y,\ 0,2x=0,16y,\ x=0,8y.$ Тогда если Петя отдаст Васе $48\%$ своих фантиков, у Васи станет $0,8y+0,48y=1,28x$ фантиков, что больше в $1,28y:(0,8y)=1,6$ раза.\\
