61. Пусть изначально в колбе было $x$л кислоты, тогда её доля составляла $\cfrac{x}{10}.$ После первого отливания кислоты останется $x-7\cdot\cfrac{x}{10}=\cfrac{3x}{10},$ а её доля станет равна $\cfrac{3x}{100}.$ После второго отливания кислоты останется $\cfrac{3x}{10}-7\cdot\cfrac{3x}{100}=\cfrac{9x}{100},$ а её доля будет равна $\cfrac{9x}{1000}.$ Значит, верно равенство $\cfrac{9x}{1000}=0,081,\ 9x=81,\ x=9$л. Таким образом, изначально в растворе было $\cfrac{9}{10}\cdot100\%=90\%$ кислоты.\\
