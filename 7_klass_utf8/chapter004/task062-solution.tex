62. Пусть изначально в колбе было $x$л кислоты, тогда её доля составляла $\cfrac{x}{10}.$ После первого отливания кислоты останется $x-6\cdot\cfrac{x}{10}=\cfrac{4x}{10},$ а её доля станет равна $\cfrac{4x}{100}.$ После второго отливания кислоты останется $\cfrac{4x}{10}-6\cdot\cfrac{4x}{100}=\cfrac{16x}{100},$ а её доля будет равна $\cfrac{16x}{1000}.$ Значит, верно равенство $\cfrac{16x}{1000}=0,064,\ 16x=64,\ x=4$л. Таким образом, изначально в растворе было $\cfrac{4}{10}\cdot100\%=40\%$ кислоты.
\newpage
