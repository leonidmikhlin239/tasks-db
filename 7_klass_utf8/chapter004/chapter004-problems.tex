\section{Проценты задачи}
1. На весеннем турслёте ФМЛ №239 было $60\%$ учащихся лицея, а на уборке листьев в Летнем саду было $80\%$ лицеистов. При этом каждый ученик лицея был на слёте или в Летнем саду. Сколько процентов учащихся лицея были и на слёте, и на уборке листьев?\\
2. На новогоднем вечере в ФМЛ №239 $80\%$ учащихся лицея, пришедших на вечер, были на представлении, а на дискотеке --- $90\%$. Сколько процентов лицеистов, пришедших на новогодний вечер, были и на представлении, и на дискотеке?\\
3. Найдите положительное число, если $45\%$ от него составляют столько же, сколько составляют $20\%$ от числа, ему обратного.\\
4. Найти положительное число, если $27\%$ от него равны $90\%$ от его квадрата.\\
5. Число 8 составляет $50\%$ от числа $2N+6.$ Найдите $N+1.$\\
6. Найдите число $a,$ если $50\%$ от числа $a+1$ равно $40\%$ от числа $a+3.$\\
7. Свежий виноград содержит $75\%$ влаги, а сушёный виноград (изюм) --- $6\%.$
Сколько потребуется свежего винограда для приготовления 4 кг изюма?\\
8. Свежий виноград содержит $80\%$ влаги, а сушёный виноград (изюм) --- $5\%.$
Сколько потребуется свежего винограда для приготовления 1 кг изюма?\\
9. Груши, содержащие $65\%$ воды, при сушке потеряли $50\%$ своей массы. Сколько процентов воды содержат сушёные груши?\\
10. Яблоки, содержащие, $70\%$ воды, при сушке потеряли $60\%$ своей массы. Сколько
процентов воды содержат сушёные яблоки?\\
11. Одна сторона прямоугольника равна 90 см, а другая составляет $70\%$ длины первой. Найдите периметр и площадь этого прямоугольника.\\
12. Одна сторона прямоугольника равна 80 см, а другая составляет $65\%$ длины первой. Найдите периметр и площадь этого прямоугольника.\\
13. Банковский вклад в мае увеличился на $20\%,$ а в июне уменьшился на $20\%,$ после чего на счёте оказалось 6720 рублей. Найдите сумму вклада на конец апреля.\\
14. Банковский вклад в мае увеличился на $10\%,$ а в июне уменьшился на $10\%,$ после чего на счёте оказалось 10890 рублей. Найдите сумму вклада на конец апреля.\\
15. Лёша на $20\%$ умнее Вадика, а Костя на $10\%$ умнее Лёши. На сколько процентов Костя умнее Вадика?\\
16. Ира разговаривает по телефону на $25\%$ больше чем Люба, а Яна разговаривает по телефону на $10\%$ больше чем Ира. На сколько процентов Яна разговаривает по телефону больше чем Люба?\\
17. Избирательная комиссия после выборов недосчиталась $20\%$ бюллетеней от числа всех
проголосовавших. Спустя некоторое время нашли $70\%$ пропавших бюллетеней, а затем
ещё $5\%$ от числа всех голосовавших. Все ли пропавшие бюллетени нашли?\\
18. Избирательная комиссия после выборов недосчиталась $30\%$ бюллетеней от числа всех
проголосовавших. Спустя некоторое время нашли $80\%$ пропавших бюллетеней, а затем
ещё $5\%$ от числа всех голосовавших. Все ли пропавшие бюллетени нашли? \\
19. В классе число отсутствующих составляет $25\%$ от числа присутствующих. После того, как пришёл один опоздавший, число присутствующих стало в пять раз больше числа отсутствующих. Сколько всего человек в классе?\\
20. В классе число отсутствующих составляет $20\%$ от числа присутствующих. После того, как один ученик ушёл, число присутствующих стало в четыре раза больше числа отсутствующих. Сколько всего человек в классе?\\
21. Длину прямоугольного участка земли увеличили на $30\%,$ а ширину --- на $20\%,$ в результате чего его площадь увеличилась на 28 $\text{м}^2.$ Определите площадь исходного участка.\\
22. Длину прямоугольного участка земли увеличили на $40\%,$ а ширину --- на $10\%,$ в результате чего его площадь увеличилась на 27 $\text{м}^2.$ Определите площадь исходного участка.\\
23. Население города $N$ ежегодно увеличивается на $6\%$ За последние два года в нём стало на 86520 человек больше. Сколько сейчас жителей в $N?$\\
24. Население города $N$ ежегодно увеличивается на $7\%$ За последние два года в нём стало на 86940 человек больше. Сколько сейчас жителей в $N?$\\
25. Если брат отдаст сестре 300 рублей, то денег у них станет поровну. Если сестра отдаст брату $40\%$ своих денег, то у неё станет в три раза меньше денег, чем у брата.
Определите, сколько денег у брата и сколько у сестры.\\
26. Если сестра отдаст брату 400 рублей, то денег у них станет поровну. Если брат отдаст сестре $20\%$ своих денег, то у неё станет в два раза больше денег, чем у брата.
Определите, сколько денег у брата и сколько у сестры.\\
27. Есть два куска разных сплавов серебра, весят они одинаково, но доля серебра в них различна. Если сплавить первый кусок с половиной второго, то в полученном сплаве будет
$40\%$ серебра, а если наоборот --- второй с половиной первого, то будет $50\%$ серебра. Найти, сколько процентов серебра в каждом куске.\\
28. У Васи есть своя коллекция фантиков, у Пети --- своя. Как-то раз они решили поменяться фантиками. Сначала Вася отдал Пете $10\%$ своих фантиков. Затем Петя перемешал свою новую коллекцию, выбрал $10\%$ фантиков и отдал их Васе, который с изумлением обнаружил, что теперь у него стало столько же фантиков, сколько было сначала. Наконец, Вася передал Пете $36\%$ фантиков. Определите, во сколько раз увеличилось число фантиков в коллекции Пети после всех обменов.\\
29. У Васи есть своя коллекция фантиков, у Пети --- своя. Как-то раз они решили поменяться фантиками. Сначала Петя отдал Васе $20\%$ своих фантиков. Затем Вася перемешал свою новую коллекцию, выбрал $20\%$ фантиков и отдал их Пете, который с изумлением обнаружил, что теперь у него стало столько же фантиков, сколько было сначала. Наконец, Петя передал Васе $48\%$ фантиков. Определите, во сколько раз увеличилось число фантиков в коллекции Васи после всех обменов.\\
30. Сколько воды нужно добавить к 750г $15\%$ раствора сахара, чтобы процентное содержание сахара стало $5\%?$\\
31. Сколько воды нужно выпарить из 1,5кг $5\%$ раствора соли, чтобы концентрация соли составила $12\%?$\\
32. Первое число составляет $80\%$ от третьего числа, а второе --- $30\%$ от третьего числа. Найдите эти числа, если их среднее арифметическое равно 21,21.\\
33. Мама оставила Васе деньги на завтрак. После того, как Вася купил сок за 20 рублей, у него осталось ещё $90\%$ от исходной суммы. Сколько рублей оставила мама Васе?\\
34. Сколько получится сухой ромашки из 80 кг свежей, если она при сушке теряет $65\%$ своего веса?\\
35. В январе килограмм винограда стоил 200 рублей. В марте цена на виноград выросла на $4\%,$ а в июне снизилась на $4\%.$ Сколько стоил виноград в июне?\\
36. Смешали 7 литров $16\%$-го раствора некоторого вещества с 3 литрами $6\%$-го раствора этого же вещества. Найдите концентрацию полученного раствора.\\
37. Разделите число 80 на две части так, чтобы одна часть составляла $60\%$ от другой.\\
38. Из двух положительных чисел одно увеличили на $1\%,$ другое на $4\%,$ при этом их сумма увеличилась на $3\%.$ Найти большее из этих чисел, если меньшее равно $8.$\\
39. Имеются два сплава с разным содержанием золота. В первом сплаве содержится $30\%,$ а во втором --- $50\%$ золота. В каком отношении надо взять первый и второй сплавы, чтобы получить из них новый сплав, содержащий $35\%$ золота?\\
40. Из данных четырёх чисел первые три относятся между собой как $\cfrac{1}{15}:0,1: \cfrac{1}{3},$ а четвёртое составляет $80\%$ третьего. Найдите эти числа, если известно, что разность между суммой третьего и четвёртого числа и суммой первого и второго числа равна 26.\\
41. Имеются два сплава, массы которых отличаются на 54 килограмма. Первый сплав содержит $10\%$ олова, второй --- $30\%$ олова. Из этих двух сплавов получили третий сплав, который содержит $18,2\%$ олова. Найдите массу более лёгкого сплава.\\
42. В двух магазинах были одинаковые цены на некоторый товар. В первом магазине цены на этот товар уменьшили на $20\%,$ а потом ещё на $20\%,$ а во втором магазине цену снизили сразу на  $40\%.$ Найдите отношение цены товара в первом магазине к цене товара во втором магазине после всех снижений.\\
43. Цена одной упаковки вареников с вишней в течение года менялась три раза. Сначала она увеличилась на $20\%,$ затем уменьшилась на $5\%$ и, наконец, возросла на $20\%.$ Определите первоначальную цену упаковки вареников, если в конце года она была 171 рубль.\\
44. Даны числа $a$ и $b,$ такие что $a>b>1.$ Расположите в порядке возрастания следующие числа: $\cfrac{b}{a}; a\cdot b; (90\%\text{ от }a)\cdot(120\% \text{ от } b); \cfrac{b^2}{a^2}; (a-b)(b-a).$\\
45. Даны числа $a$ и $b,$ такие что $a>b>1.$ Расположите в порядке возрастания следующие числа: $\cfrac{b}{a}; b\cdot a; (170\%\text{ от }a)\cdot(90\% \text{ от } b); \cfrac{b^2}{a^2}; (a-b)(b-a).$\\
46. Сплав массой 600г содержит $10\%$ меди. Сколько меди нужно добавить к этому количеству сплава, чтобы в новом сплаве содержалось $20\%$ меди?\\
47. Представьте число 200 в виде двух слагаемых, таких, что $25\%$ одного равны $37,5\%$ другого.\\
48. Представьте 200 в виде разности так, что $30\%$ уменьшаемого равны $70\%$ вычитаемого.\\
49. Смешали 30 г 20$\%$-го раствора соли с 10 г другого раствора, и получили раствор с концентрацией соли 25$\%$. Определить концентрацию соли во втором растворе.\\
50. Сплав серебра с золотом содержит $40\%$ золота. Сколько нужно добавить золота к слитку сплава весом 10 кг, чтобы в образовавшемся новом сплаве золота стало $80\%?$\\
51. Кусок сплава меди с оловом имеет массу 60 кг и содержит $60\%$ олова. Сколько чистой меди следует добавить к этому сплаву, чтобы содержание олова в нём составило $40\%?$\\
52. Разделите 90 на две части так, чтобы $40\%$ одной части были на 15 больше, чем $30\%$ другой части.\\
53. В свежих грибах $80\%$ воды. При сушке грибов испарилось $75\%$ имевшейся в них воды. Сколько процентов составляет масса воды в сухих грибах от общей массы сухих грибов?\\
54. При сушке грибов испарилось $80\%$ имевшейся в них воды. Доля воды в сушёных грибах составляет $25\%$ общей массы сушёных грибов. Найдите, сколько процентов массы свежих грибов составляла вода.\\
55. В туристическом клубе девочки составляли $25\%.$ После того, как в клуб приняли ещё десять девочек, девочки стали составлять $30\%.$ Сколько мальчиков в клубе?\\
56. В хоре мальчики составляли $25\%.$ После того, как в хор приняли ещё трёх мальчиков, мальчики стали составлять $28\%.$ Сколько девочек в хоре?\\
57. Процентное содержание соли в растворе сначала снизилось на $20\%,$ а затем повысилось на $20\%.$ На сколько процентов изменилось первоначальное содержание соли?\\
58. Числитель дроби увеличили на $20\%.$ На сколько процентов надо уменьшить знаменатель, чтобы получить дробь, которая в три раза больше исходной?\\
59. Имеется три слитка. Первый слиток имеет массу 5 кг, а второй --- 3 кг, и каждый из этих двух слитков содержит $30\%$ серебра. Если первый слиток сплавить с третьим, то получится слиток, содержащий $56\%$ серебра, а если второй слиток сплавить с третьим, то получится слиток, содержащий $60\%$ серебра. Найдите массу третьего слитка и процентное содержание серебра в нём.\\
60. Федя начал играть в шахматы и следить за своим рейтингом. За первый день его рейтинг вырос на $25\%.$ За второй день --- упал на $50\%.$ На сколько процентов и в какую сторону должен измениться рейтинг за третий день, чтобы вернуться к первоначальному уровню?\\
61.  В колбе находилось неизвестное количество процентов раствора объёмом 10 л. Оказалось, что
если отлить 7 литров раствора и влить 7 литров воды, тщательно размешать, а потом повторить
операцию, то получится $8,1\%$ раствор кислоты. Сколько процентов кислоты было в изначальном растворе?\\
62. В колбе находилось неизвестное количество процентов раствора объёмом 10 л. Оказалось, что
если отлить 6 литров раствора и влить 6 литров воды, тщательно размешать, а потом повторить
операцию, то получится $6,4\%$ раствор кислоты. Сколько процентов кислоты было в изначальном
растворе?

ewpage
