27. Пусть оба куска весят $m$кг, а доля серебра в них равна $x$ и $y$ соответственно. Тогда весь сплав весит $m+0,5m=1,5m$кг, а количество серебра в нём в первом случае равно $mx+0,5my=m(x+0,5y),$ а во втором --- $0,5mx+my=m(0,5x+y).$ Выразив долю серебра в получившихся сплавах, получим и решим систему из двух линейных уравнений:\\
$\begin{cases}
\cfrac{m(x+0,5y)}{1,5m}=0,4\\
\cfrac{m(0,5x+y)}{1,5m}=0,5
\end{cases}
\Leftrightarrow
\begin{cases}
x+0,5y=0,6\\
0,5x+y=0,75
\end{cases}
\Leftrightarrow
\begin{cases}
x+0,5y=0,6\\
x+2y=1,5
\end{cases}
\Leftrightarrow
\begin{cases}
1,5y=0,9\\
x+2y=1,5
\end{cases}
\Leftrightarrow$\\$\Leftrightarrow
\begin{cases}
y=0,6\\
x=0,3
\end{cases}
$\\
Значит, первый кусок содержит $30\%$ серебра, а второй --- $60\%.$\\
