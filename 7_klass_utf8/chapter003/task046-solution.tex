46. $(x+1)(x^2-x+1)(x^3-1)+1-x^6+x^4=(x^3+1)(x^3-1)+1-x^6+x^4=x^6-1+1-x^6+x^4=x^4.$ Если $|x|=3,$ то $x^4=(x^2)^2=(|x|^2)^2=9^2=81.$\\
