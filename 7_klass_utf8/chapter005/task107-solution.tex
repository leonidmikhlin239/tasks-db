107. $(|x+2|-a)(x+1)=0\Leftrightarrow \left[\begin{array}{l}|x+2|-a=0,\\ x+1=0.\end{array}\right.\Leftrightarrow
\left[\begin{array}{l}
\begin{cases}
\left[\begin{array}{l}
x+2=a,\\
x+2=-a.
\end{array}\right.\\
a\geqslant0.
\end{cases}\\
x=-1.
\end{array}\right.\Leftrightarrow
\left[\begin{array}{l}
\begin{cases}
\left[\begin{array}{l}
x=a-2,\\
x=-2-a.
\end{array}\right.\\
a\geqslant0.
\end{cases}\\
x=-1.
\end{array}\right.$
Ровно два различных корня у этого уравнения может быть только в случае, когда какие-то два корня совпадают. Если $a-2=2-a,$ то $a=0.$ Если $-2-a=-1,$ то $a=-1,$ что противоречит условию $a\geqslant0.$ Если $a-2=-1,$ то $a=1.$ Таким образом, $a\in\{0;1\}.$\\
