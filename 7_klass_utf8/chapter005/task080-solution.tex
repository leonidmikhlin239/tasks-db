80. $4a^2(x-1)+9(x+1)=12ax\Leftrightarrow 4a^2x-4a^2+9x+9=12ax\Leftrightarrow
x(4a^2-12a+9)=4a^2-9\Leftrightarrow x(2a-3)^2=(2a-3)(2a+3)\Leftrightarrow
\left[\begin{array}{l}\begin{cases} x\in\varnothing,\\ a=-\cfrac{3}{2}.\end{cases}\\
\begin{cases} x\in\mathbb{R},\\ a=\cfrac{3}{2}.\end{cases}\\\begin{cases} x=\cfrac{2a+3}{2a-3},\\ a\neq\pm\cfrac{3}{2}.\end{cases}\end{array}\right.$\\
Единица точно является корнем уравнения (как и все остальные числа) при $a=\cfrac{3}{2}.$ В другом случае она корнем являться не может, так как если $1=\cfrac{2a+3}{2a-3},$ то $2a-3=2a+3\Leftrightarrow -3=3.$ При $a=-2$ найдём $x=\cfrac{2\cdot(-2)+3}{2\cdot(-2)-3}=\cfrac{1}{7}.$ Более двух корней (бесконечно много) это уравнение имеет при $a=\cfrac{3}{2}.$\\
