\section{Уравнения и неравенства задачи}
$\begin{array}{ll}
1.\ \cfrac{x}{2}+\cfrac{x-1}{3}=\cfrac{1}{30}-\left(\cfrac{x}{4}+\cfrac{x}{5}\right),&
2.\ \cfrac{y}{4}+\cfrac{y-1}{5}=\cfrac{1}{30}-\cfrac{y}{2}-\cfrac{y}{3},\\
3.\ \begin{cases} x^2-2xy=1025,\\x-2y=25.\end{cases},&
4.\ \begin{cases} x^2+2xy=1080,\\x+2y=40.\end{cases},\end{array}$\\
$\begin{array}{l}
5.\ \text{Существуют ли такие значения }x,\text{ при подстановке которых значение}\\
\text{выражения }4+(x+1)^2-4(x+1)\text{ будет отрицательно?}\\
6.\ \text{Существуют ли такие значения }x,\text{ при подстановке которых значение}\\
\text{выражения }(4-x)^2+25-10(4-x)\text{ будет отрицательно?}
\end{array}$\\
$\begin{array}{ll}
7.\ \cfrac{2x+7}{3}-\cfrac{x-3}{2}=4x,&
8.\ \cfrac{3x+11}{2}-\cfrac{2x+7}{3}=4x,\\
9.\ (x+2)(x^2-2x+4)-x(x+2)(x-2)=12,\\
10.\ (x+1)(x^2-x+1)-x(x+3)(x-3)=10,\end{array}$\\
$\begin{array}{ll}11.\ \cfrac{2x-3}{5}-\cfrac{1-x}{4}+\cfrac{5x+1}{20}=3-x,&
12.\ \cfrac{x-2}{5}-\cfrac{5-2x}{4}+\cfrac{4x-1}{20}=4-x,
\end{array}$\\
$\begin{array}{l}
13.\ \text{Найдите все пары чисел }x,\ y, \text{ для каждой из которых значение}\\
\text{выражения }(x+y)^2-10x+4y-2xy+29\text{ равно нулю,}\\
14.\ \text{Найдите все пары чисел }x,\ y, \text{ для каждой из которых значение}\\
\text{выражения }(x-y)^2+2x+4y+2xy+5\text{ равно нулю,}
\end{array}$\\
$\begin{array}{ll}
15.\ x+\cfrac{2x-7}{2}-\cfrac{3x+1}{5}=5-\cfrac{x+6}{2},&
16.\ \cfrac{2x-5}{6}+\cfrac{x+2}{4}=\cfrac{5-2x}{3}-\cfrac{6-7x}{4}-x,
\end{array}$\\
$\begin{array}{l}
17.\ \text{Докажите, что выражение }2x(3-x)-(x+1)(x+5)+4
\text{ принимает лишь}\\ \text{отрицательные значения,}\\
18.\ \text{Докажите, что выражение }3x(1-2x)-(x+2)(x+1)+1
\text{ принимает лишь}\\ \text{отрицательные значения,}\\
19.\ \text{Найдите все числа }x,\ y,\text{ удовлетворяющие условию } 9x^2+y^2-12x+2y+5=0,\\
20.\ \text{Найдите все числа }x,\ y,\text{ удовлетворяющие условию } 4x^2+y^2-4x+4y+5=0,\\
21.\ \text{Существуют ли такие значения чисел }x,\ y \text{ при которых многочлены }\\
2x^2+5xy-8\text{ и } 3y^2-5xy+10\text{ одновременно принимали бы отрицательные значения?}\\
22.\ \text{Существуют ли такие значения чисел }x,\ y \text{ при которых многочлены }\\
3x^2-7xy+5\text{ и } 2y^2+7xy-4\text{ одновременно принимали бы отрицательные значения?}
\end{array}$\\
$\begin{array}{ll}
23.\ x-\cfrac{20x-(10-3x)}{156}=\cfrac{26x-51}{52}-\cfrac{2(1-3x)}{13},\\
24.\ \cfrac{3(1,2-x)}{10}-\cfrac{5+7x}{4}=x+\cfrac{9x+0,2}{20}-\cfrac{4(13x-0,6)}{5},\end{array}$\\
$\begin{array}{ll}25.\ \cfrac{2x-3}{9}-\cfrac{3x-9}{2}+2x=3-\cfrac{2-x}{3},&
26.\ \cfrac{3x-2}{9}-\cfrac{2x-3}{2}+3x=2-\cfrac{9-x}{3},
\end{array}$\\
$\begin{array}{l}
27.\ \text{Докажите, что выражение }9x^2+8y-6xy+y^2+18-24x  \text{ принимает}\\
\text{положительные значения при любых значениях переменных,}\\
28.\ \text{Докажите, что выражение }16y^2+6x-8xy+x^2+12-24y  \text{ принимает}\\
\text{положительные значения при любых значениях переменных,}\\
29.\ \text{Найдите } |k-3-5k^2|,\text{ где }k\text{ --- корень уравнения }\\
(2x-1)(4x^2+2x+1)-2x(2x-3)(2x+3)=38x+3,\\
30.\ \text{Найдите } |m-1-10m^2|,\text{ где }m\text{ --- корень уравнения }\\
(2x+1)(4x^2-2x+1)-2x(2x+5)(2x-5)=30x-1,\end{array}$\\
$\begin{array}{ll}
31.\ x^2y^2+17+x^2-8xy+2x=0,&
32.\ x^2y^2+10+y^2+6xy-2y=0,\\
33.\ \cfrac{2x-3}{5}+\cfrac{5x+1}{20}=3-x-\cfrac{x-1}{4},&
34.\ \cfrac{x-2}{5}+\cfrac{2x-5}{4}=4-x-\cfrac{4x-1}{20},
\end{array}$\\
$\begin{array}{l}
35.\ 5(x+2)-(2-3x)^3=(3x-1)(9x^2+3x+1)-(2x-1)(27x-1),\\
36.\ 9(4-x)-(3-2x)^3=(2x-1)(4x^2+2x+1)-(3x-2)(12x-1),
\end{array}$\\
$\begin{array}{ll}
37.\ 2x^2+4y^2-4xy-6x+9=0,&
38.\ 5x^2+y^2+4xy-2x+1=0,\\
39.\ \cfrac{x^2}{x-1}=\cfrac{1}{x-1},&
40.\ \cfrac{2x-3}{5}-\cfrac{1-x}{4}+\cfrac{5x+1}{20}=3-x,\\
41.\ \begin{cases}\cfrac{x+3y}{4}+\cfrac{4x-2y}{3}=-\cfrac{7}{6},\\ \cfrac{x+3y}{6}+\cfrac{2x-y}{4}=\cfrac{7}{12}. \end{cases},&
42.\ \cfrac{44}{4-x^2}+\cfrac{2x+7}{x-2}=\cfrac{3-x}{x+2},\\
43.\ \cfrac{3x-1}{7}-\cfrac{2x+1}{2}=\cfrac{x}{14}-1,&
44.\ \cfrac{7}{x-2}=3+\cfrac{x^3+27}{(x+3)(x-2)},\end{array}$\\
$\begin{array}{ll}
45.\ \cfrac{x^2-4x-8}{5x-x^2}=\cfrac{x^2-3x-7}{x(x-5)},&
46.\ \begin{cases} \cfrac{2}{x}+\cfrac{9}{y-1}=-2,\\ \cfrac{4}{x}- \cfrac{3}{y-1}=3. \end{cases},\\
47.\ \cfrac{9-4a^2-4ab-b^2}{4a^2+2ab+3b-9}=\cfrac{3+2a+b}{x},&
48.\ (x-7): \cfrac{1}{3}\cdot 5=7: 1,4\cdot x+9x,\end{array}\\
\begin{array}{ll}
49.\ \cfrac{x+3}{8}=\cfrac{x-7}{3}+1,&
50.\ \begin{cases} \cfrac{1}{x+y}+\cfrac{1}{2x-y}=\cfrac{7}{12},\\
\cfrac{1}{2x-y}-\cfrac{1}{x+y}=\cfrac{5}{12}.\end{cases},\\
51.\ 3(x+1)(x+2)=12+(3x-4)(x+2),&
52.\ \begin{cases} 4x^2-49y^2=10(2x-7y),\\ x+y=45.\end{cases},\\
53.\ (x-3)(x^2+3x+9)-x(x+5)(x-5)=23,&
54.\ \cfrac{2x-1}{6}-\cfrac{3-x}{4}=6-x,\\
55.\ 49(x-1)^2+14(x-1)+1=0,\\
56.\ |x+y-z|+(y-3)^2+(2x-4)^4=0,\end{array}$\\
$57.\ (x-2)^3-x(1-2x)^2+(3x+1)(9x^2-3x+1)=24x^3-2x^2,$\\
$\begin{array}{ll}58.\ \cfrac{(x-1)^2-5(x-1)+4}{x-2}=0,&
59.\ (2x^5-242x^3)^4+\left |33-3|x|\right|=0,\end{array}$\\$\begin{array}{l}
60.\ \left(4|x|-3,4\right)\left(2|x|+7\cfrac{1}{3}\right)^3=\left(2|x|+7\cfrac{1}{3}\right)\left(4|x|-3,4\right)^3,\\
\end{array}$\\
$\begin{array}{ll}
61.\ x-(2x+(3x-(4x+(5x-7))))=11,\\
62.\ \cfrac{(x-2)^3+(x+2)^3}{x+2,25}=\cfrac{2(x-3)(x^2+3x+9)}{x+2,25},\end{array}$\\
$\begin{array}{l}
63.\ \left(\cfrac{97^3-53^3}{44}+97\cdot 53\right): (152,5^2-27,5^2)=x:(19,25^2-18,25\cdot20,25),
\end{array}$\\
$\begin{array}{ll}
64.\ \left(\cfrac{4x-7}{0,2}+\cfrac{6x-3}{0,4}\right)^2=\left(70-\cfrac{4x+1}{0,3}\right)^2,&
65.\ \cfrac{x^2+2x-3}{x^3+2x^2+3x-6}=0,\\
66.\ (x^2+1)(18x-17)(29-30x)=0,&
67.\ (x^2+6x+5)^2+\left|1-|x|\right|=0,\\
68.\ |2y-x|+(7-|3y+1|)^6=0,&
69.\ \left((2-7x)^2-0,26\right)^2-0,01=0,\\
70.\ \cfrac{36-16x^2}{2x-3}=\cfrac{24-12x}{x-2}-12,&
71.\ \cfrac{64-4x^2}{x-4}=\cfrac{11-22x}{2x-1}-7,\\
72.\ \cfrac{x+1}{3}-\cfrac{x+2}{2}=1,&
73.\ (1-3x)(x+1)=(3x-1)(2x+1),\\
74.\ (x+4)^2=x+4,&
75.\ \cfrac{x^3-9x}{x^2-6x+9}=0,\\
76.\ \cfrac{5x-1}{4}-\cfrac{x-2}{3}=10-x,&
77.\ (x+1)(x^2-x+1)-x(x+2)(x-2)=3,
\end{array}$\\
78. Верно ли утверждение <<Если $x\geqslant-5,5,$ то $x>-6$>>?\\
79. Дано уравнение $a^2(x-1)=9(9x+9-2a).$ Найдите все те значения параметра $a,$ при каждом из которых данное уравнение: а) не имеет корней; б) имеет ровно один корень; в) имеет более одного корня.\\
80. Дано уравнение $4a^2(x-1)+9(x+1)=12ax.$ а) При каких $a$ уравнение имеет корень, равный 1? б) Решите уравнение при $a=-2;$ в) При каких $a$ уравнение имеет более двух корней?\\
81. При каком $a$ уравнение $(a^2-4)x=a^2+5a+6$ имеет бесконечно много решений?\\
82. При каких значениях параметра $a$ уравнение $a^3-a^2x=5ax+25a$ имеет бесконечно много корней?\\
83. При каких значениях параметра $p$ уравнение $p^3x+6p^2=9px+p^3+9p$ не имеет корней?\\
84. Докажите, что многочлен $x^2+2x+y^2-4y+6$ при любых значениях входящих в него переменных принимает положительные значения,\\
85. Докажите, что многочлен $x^2-4x+y^2-6y+15$ при любых значениях входящих в него переменных принимает положительные значения,\\
86. Для каждого значения параметра $a$ решите уравнение: $a^2\left(1-\cfrac{1}{x}\right)-a\left(1+\cfrac{1}{x}\right)=\cfrac{a-3}{x}.$\\
87. Найдите все значения параметра $a,$ при которых уравнения $6x+1=0$ и $2x-a=0$ имеют общие корни.\\
88. $\cfrac{9+7x}{2}-1+\cfrac{1-2x}{7}=3x.$\qquad\qquad
89. $\cfrac{7y}{12}+\cfrac{2-y}{4}-1=\cfrac{5y-6}{9}-\cfrac{1}{2}.$\\
90. Решите систему уравнений относительно $x$ и $y:\ \begin{cases}2x-3y=5b-a,\\ 3x-2y=a+5b. \end{cases}$\\
91. Решите систему уравнений относительно $x$ и $y:\ \begin{cases}5x-2y=3a+7b,\\ 2x-5y=7b-3a. \end{cases}$\\
$92.\ \cfrac{3x-7}{4}-\cfrac{9x+11}{8}=\cfrac{3-x}{2}, \quad 93.\ \cfrac{4x-3}{2}-\cfrac{5-2x}{3}=\cfrac{3x-4}{3},$\\
$94.\ (3x-1)^2-8(x+1)^2=(x+2)(x-2), \quad 95.\ (2x+1)^2-3(x-5)^2=(x+3)(x-3),$\\
$96.\ x^2+4y^2+z^2=12y-2x-4z-14, \quad 97.\ 9x^2+y^2+z^2=6y-12x+4z-17.$\\
$98.\ \cfrac{2x-5}{3}-\cfrac{x}{4}=\cfrac{4-x}{2}+x, \quad 99.\ \cfrac{3-2x}{6}-\left(1,5+\cfrac{4x-7}{3}\right)=\cfrac{x}{2}+2\cfrac{5}{12},$\\
$100.\ \cfrac{x+3}{2}-\cfrac{x-5}{3}=\cfrac{6-x}{5}\cdot1,5+3, \quad 101.\ \cfrac{2-3x}{4}-\left(\cfrac{x-4}{3}-x\right)=\cfrac{x}{3}+1\cfrac{5}{24},$\\
102. $\cfrac{3x-1}{4}-\cfrac{4x+1}{3}=x+1,$ \quad 103. $\cfrac{2x-3}{3}-\cfrac{x+4}{4}=x-2,$\\
104. Не вычисляя, сравните $a=2021\cdot2022\cdot2027$ и $b=2024^3.$\\
105. Не вычисляя, сравните $a=2023\cdot2024\cdot2028$ и $b=2026^3.$\\
106. При каком значении параметра $a$ уравнение $(|x-3|-a)(x-1)=0$ имеет ровно два различных корня?\\
107. При каком значении параметра $a$ уравнение $(|x+2|-a)(x+1)=0$ имеет ровно два различных корня?\\
108. $(x+1)(x-5)=(x+1)(3-x),$ \quad 109. $\left(\cfrac{6x-1}{x^2+6x}+\cfrac{6x+1}{x^2-6x}\right):\cfrac{x^2+1}{x^2-36}-\cfrac{12}{x-1}=\cfrac{12}{x-x^2},$\\
110 $(x+4)(x-1)=(2x+1)(x-1),$ \quad 111. $\cfrac{12}{x-x^2}+\cfrac{12}{x-1}=\left(\cfrac{6x-1}{x^2+6x}+\cfrac{6x+1}{x^2-6x}\right):\cfrac{x^2+1}{x^2-36},$\\
112. $(x^2-3x+1)^2-(x^2-2x-1)^2=0,$ \quad 113. $(3x-1)(2x+5)=1-9x^2.$\\
114. Найдите наименьшее значение выражения $(2a-1)(2a+1)+3b(3b-4a).$\\
115. $\cfrac{x-3}{x-1}=\cfrac{x^2-5}{2x-2}.$\\
116. При каких значениях параметра $a$ уравнения $5ax - 7a + 3 = 7x$ и $\cfrac{x+7}{4}-\cfrac{7x-1}{6}=1$ будут иметь одинаковые корни?\\
117. При каких значениях параметра $a$ уравнения $3a + 9ax + 13 = 5x$ и $\cfrac{5-x}{3}-\cfrac{7x+11}{4}=1$ будут иметь одинаковые корни?
\newpage
