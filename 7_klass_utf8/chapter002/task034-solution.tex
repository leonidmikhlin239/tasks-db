34. $\cfrac{3^{10}-5\cdot3^8-3^7}{2\cdot9^4+3^7-10\cdot9^3}=\cfrac{3^7\cdot(3^3-5\cdot3-1)}{9^3\cdot(2\cdot9+3-10)}=
\cfrac{3^7\cdot11}{3^6\cdot11}=3.$\\
