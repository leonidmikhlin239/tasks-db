\section{Степенные функции задачи}
$\begin{array}{ll}
1.\ \cfrac{(-1)^{2n+1}\cdot(-a)^{3n+4}\cdot(3an)^2}{(-a)^{3n-1}(2a^2n)^3},&
2.\ \cfrac{(-1)^{4k-3}\cdot(-b)^{5k+2}\cdot(5bk^2)^2}{(-b)^{5k-3}(4bk)^3},\end{array}$\\
$\begin{array}{ll}
3.\ \text{При каком наименьшем натуральном }n \text{ число } 45^n \text{делится нацело на } 75^{10}?\\
4.\ \text{При каком наименьшем натуральном }n \text{ число } 75^n \text{делится нацело на } 45^{10}?
\end{array}$\\
$\begin{array}{llll}
5.\ \cfrac{3^{32}-3\cdot 9^{14}}{26\cdot27^{10}},&
6.\ \cfrac{2^{50}-2\cdot4^{22}}{31\cdot8^{15}},&
7.\ \cfrac{4^{2n+3}\cdot 2^{2n-1}}{(-8)^{2n}},&
8.\ \cfrac{9^{2n+3}\cdot3^{2n-2}}{(-27)^{2n}},\\
9.\ \cfrac{14^5}{2^6\cdot 7^4},&
10.\ \cfrac{6^6}{2^7\cdot 3^5},&
11.\ \cfrac{(6^4)^2}{4^4\cdot9^5},&
12.\ \cfrac{(14^3)^3}{7^8\cdot8^3},\\
13.\ \left(\cfrac{3x^2y^5}{5z^6}\right)^5\left(\cfrac{25z^5}{9x^2y^6}\right)^3,&
14.\ \left(\cfrac{4xy^4}{5z^5}\right)^5\left(\cfrac{25z^4}{16xy^5}\right)^3,&
15.\ \cfrac{4\cdot36^n}{3^{2n-3}\cdot2^{2n+2}},\end{array}\\
\begin{array}{llll}
16.\ \cfrac{3^{15}-3\cdot27^4}{3^9\cdot6^4},&
17.\ \cfrac{2^{16}-2\cdot 8^4}{2^{11}\cdot 14^2},&
18.\ \cfrac{(4\cdot3^{17}-3^{16})\cdot242}{(11\cdot 3^5)^3\cdot(-2)^3},\\
19.\ \cfrac{(3\cdot 2^{20}+7\cdot 2^{19})\cdot 52}{(-1)^7\cdot(13\cdot8^4)^2},&
20.\ \cfrac{8\cdot100^n}{2^{2n+2}\cdot5^{2n-2}},
\end{array}$\\
$\begin{array}{ll}
21.\ \text{Найдите число, восьмая степень которого равна }\cfrac{21^9\cdot(6^2\cdot16)^3}{12^9\cdot3^4\cdot63},
\end{array}$\\
$\begin{array}{ll}
22.\ \cfrac{45^{2n+1}}{(-15)^{2n}\cdot 9^{n-1}\cdot25},&
23.\ \cfrac{28^{2k+1}}{(-14)^{2k}\cdot4^{k-1}\cdot49},
\end{array}$\\
$\begin{array}{ll}
24.\ \text{Какое число и на сколько больше: } A=2\cfrac{54}{55}+1\cfrac{65}{66},\ B=\cfrac{28^7\cdot49^2}{16\cdot14^{10}},\\
25.\ \text{Какое число и на сколько больше: } A=1\cfrac{32}{33}+1\cfrac{87}{88},\ B=\cfrac{45^8\cdot125}{27^2\cdot15^{10}},
\end{array}$\\
$\begin{array}{ll}
26.\ \cfrac{(5\cdot3^{18}+(-3)^{19})\cdot2^{34}}{12^{18}},&
27.\ \cfrac{(7\cdot4^8+(-4)^9)\cdot3^{14}}{36^8},
\end{array}$\\
$\begin{array}{l}
28.\ \text{На сколько процентов А больше В? }
A=\cfrac{(7^{10}-7^9-7^8)^2}{41\cdot49^8},\ B=5379^2-5378\cdot5380,\\
29.\ \text{На сколько процентов А больше В? }
A=\cfrac{(4^{10}-4^9-4^7)^2}{47\cdot16^7},\ B=9551^2-9552\cdot9550,
\end{array}$\\
$\begin{array}{l}
30.\ \text{При каких }n\text{ число }\cfrac{3^{2n-6}\cdot2^{n+7}}{4^{n-3}\cdot3^{n+7}}\text{ будет целым?}
\end{array}$\\
$\begin{array}{ll}
31.\ (3^{3n+1}-4\cdot3^{3n+2})(7\cdot2^n+2^{n+1}):54^{n+1},&
32.\ (2^{3n+1}-11\cdot2^{3n+2})(3\cdot7^n+7^{n+1}):56^{n+1},\end{array}$\\
$\begin{array}{llll}33.\ \cfrac{4^{10}-4^9-4^7}{2^{20}+2^{17}+11\cdot2^{15}},&
34.\ \cfrac{3^{10}-5\cdot3^8-3^7}{2\cdot9^4+3^7-10\cdot9^3},&
35.\ \cfrac{50^3}{(2^2)^3\cdot5^6},&
36.\ \cfrac{5^{3n+2}\cdot5^{1-n}}{(5^{n+1})^2},\\
37.\ \cfrac{(8^{2020}+8^{2019})^2}{(4^{2019}-4^{2018})^3},&
38.\ \cfrac{(4^{3021}-4^{3020})^3}{(8^{3020}+8^{3019})^2},&
39.\ \cfrac{8^{20}\cdot20^5}{4^{35}\cdot25^2},&
40.\ \cfrac{28^3\cdot5^6}{35^2\cdot10^4},\\
41.\ \cfrac{12^6}{3^5\cdot2^{11}},&
42.\ \cfrac{20^{10}}{5^{10}\cdot2^{19}},
\end{array}$\\
$\begin{array}{ll}
43.\ \text{Найдите расстояние между числом, противоположным А, и числом, обратным В,}\\
\text{где } A=\cfrac{36^3\cdot15^2}{18^4\cdot10^3}, \ B=\cfrac{3^{48}-3^{47}+17\cdot3^{46}}{27^{15}\cdot23},\\
44.\ \text{На координатной прямой найдите расстояние между точками }A(a)\text{ и } B(b),\\
\text{если } a=\cfrac{-14^2\cdot25^3}{49\cdot(-10)^6}, \ b=\cfrac{7^{40}+7^{38}-2\cdot7^{39}}{6^2\cdot49^{19}},\end{array}\\
\begin{array}{ll}
45.\ \cfrac{3^7\cdot15^5\cdot4^9}{8^4\cdot9^4\cdot30^4},&
46.\ \cfrac{4^4\cdot\left((3^3)^2: 3^2\right)}{27^3:3^5},\\
47.\ \text{Найдите число, восьмая степень которого равна }
\cfrac{(12^3\cdot9^3)^2\cdot14^9}{6^{18}\cdot56},
\end{array}$\\
$\begin{array}{ll}
48.\ \cfrac{(-12a^2b^4)^3(-a)^4}{(-3ab)^3(-2b^2)^3},&
49.\ \cfrac{(-2a^2bc^3)^4(3b)^3}{24a^5(-b^2c^4)^3},\\
50.\
\cfrac{42^9}{(6^2)^3\cdot7^9}+\cfrac{2^{50}-2\cdot4^{22}}{31\cdot8^{15}},&
51.\ \cfrac{50^3}{(2^2)^3\cdot5^6}-\cfrac{3^{32}-3\cdot9^{14}}{26\cdot27^{10}},\\
52.\ \cfrac{5^2\cdot54^{27}}{(3^{80}+3^{81}+3^{82})(32^5+8^8+4^{12})},&
53.\ \cfrac{5^2\cdot24^{27}}{(2^{83}+2^{84}+2^{82})(9^{12}+27^8+81^{6})},\\
54.\ \cfrac{7\cdot6^5-7^2\cdot3^5}{35\cdot3^4-3^5\cdot10},&
55.\ \cfrac{45\cdot2^7-5^3\cdot6^4}{70\cdot6^2-3^2\cdot2^5},\\
56.\ \cfrac{162\cdot2^7-8\cdot3^5+2^3\cdot3^4}{3^5\cdot2^7-3^7\cdot2^4},&
57.\ \cfrac{15\cdot3^8-5^3\cdot3^5}{5\cdot3^6-50\cdot3^5}.
\end{array}
$\\
58. Запишите выражение $27^6\cdot169^9:(5^9)^2$ в виде куба степени с натуральным показателем.\\
59. Запишите выражение $16^6\cdot121^{12}:(5^{12})^2$ в виде куба степени с натуральным показателем.\\
60. $\cfrac{4^6\cdot 8^2}{32^5:64^3},$\quad
61. $\cfrac{\left(\cfrac{2}{3}\right)^{30}\cdot(5^3)^4\cdot6^{30}}{5^9\cdot(15+1)^{15}}.$\\
62. Найдите число, $9\%$ от которого составляют $\cfrac{1}{\left(\cfrac{6^{13}\cdot12^7\cdot0,25^{12}\cdot3^7}{9^{14}}+\cfrac{2}{3}\right)^2}+\cfrac{1}{\left(1\cfrac{1}{9}\right)^2}.$\\
63. Найдите $99\%$ от числа $\cfrac{1}{3\cfrac{2}{3}}-\cfrac{1}{\left(\cfrac{1}{0,8}\right)^2-\cfrac{6^{13}\cdot18^9\cdot0,25^{13}}{27^{10}}}.$
\newpage
