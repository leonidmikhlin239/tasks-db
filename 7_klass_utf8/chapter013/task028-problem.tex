28. Пусть у Васи $x$ фантиков, а у Пети $y.$ Тогда сначала у Васи стало $0,9x$ фантиков, а у Пети $y+0,1x.$ Затем у Васи станет $0,9x+0,1(y+0,1x)=0,91x+0,1y,$ что равно изначальному количеству, а значит $0,91x+0,1y=x,\ 0,1y=0,09x,\ y=0,9x.$ Тогда если Вася отдаст Пете $36\%$ своих фантиков, у Пети станет $0,9x+0,36x=1,26x$ фантиков, что больше в $1,26x:(0,9x)=1,4$ раза.\\
