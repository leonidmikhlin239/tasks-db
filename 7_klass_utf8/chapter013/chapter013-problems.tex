\section{проценты решения}
1. Посчитаем общее количество учащихся: $60\%+80\%=140\%.$ Получилось больше $100\%$ за счёт тех учащихся, которые побывали и на турслёте, и в Летнем саду, а значит их было $140-100=40\%.$\\
2. Посчитаем общее количество учащихся: $80\%+90\%=170\%.$ Получилось больше $100\%$ за счёт тех учащихся, которые побывали и на представлении, и на дискотеке, а значит их было $170-100=70\%.$\\
3. $0,45x=0,2\cdot\cfrac{1}{x},\ 0,45x^2=0,2,\ x^2=\cfrac{4}{9},\ x=\cfrac{2}{3}.$\\
4. $0,27x=0,9x^2,\ x=0,27:0,9,\ x=0,3.$\\
5. $8=0,5(2N+6),\ 2N+6=16,\ N=5,\ N+1=6.$\\
6. $0,5(a+1)=0,4(a+3),\ 0,5a+0,5=0,4a+1,2,\ 0,1a=0,7,\ a=7.$\\
7. В 4кг изюма содержится $4\cdot(1-0,06)=3,76$кг сухого вещества. В свежем винограде оно составляет $100-75=25\%,$ значит винограда необходимо взять $3,76:0,25=15,04$кг.\\
8. В 1кг изюма содержится $1\cdot(1-0,05)=0,95$кг сухого вещества. В свежем винограде оно составляет $100-80=20\%,$ значит винограда необходимо взять $0,95:0,2=4,75$кг.\\
9. Если изначальная масса груш равна $m,$ они содержат $(100-65):100\cdot m=0,35m$ сухого вещества. В сушёных грушах оно составляет $0,35m:0,5m\cdot100\%=70\%,$ а значит воды в них $100-70=30\%.$\\
10. Если изначальная масса яблок равна $m,$ они содержат $(100-70):100\cdot m=0,3m$ сухого вещества. В сушёных яблоках оно составляет $0,3m:0,4m\cdot100\%=75\%,$ а значит воды в них $100-75=25\%.$\\
11. Вторая сторона равна $90\cdot0,7=63$см, а значит периметр и площадь равны $P=(90+63)\cdot2=306$см и $S=90\cdot63=5670\text{ см}^2.$\\
12. Вторая сторона равна $80\cdot0,65=52$см, а значит периметр и площадь равны $P=(80+52)\cdot2=$\\$=264$см и $S=80\cdot52=4160\text{ см}^2.$\\
13. $x\cdot1,2\cdot0,8=6720,\ x\cdot0,96=6720,\ x=7000$ рублей.\\
14. $x\cdot1,1\cdot0,9=10890,\ x\cdot0,99=10890,\ x=11000$ рублей.\\
15. $\text{Л}=1,2\cdot\text{В},\ \text{К}=1,1\cdot\text{Л}=1,1\cdot1,2\cdot\text{В}=1,32\cdot\text{В},$ значит Костя умнее Вадика на $32\%.$\\
16. $\text{И}=1,25\cdot\text{Л},\ \text{Я}=1,1\cdot\text{Л}=1,1\cdot1,25\cdot\text{Л}=1,375\cdot\text{Л},$ значит Яна разговаривает по телефону на $37,5\%$ больше, чем Люба.\\
17. Если всего бюллетеней было $x,$ то нашли $0,7\cdot0,2x+0,05x=0,19x<0,2x,$ а значит не все пропавшие бюллетени.\\
18. Если всего бюллетеней было $x,$ то нашли $0,8\cdot0,3x+0,05x=0,29x<0,3x,$ а значит не все пропавшие бюллетени.\\
19. Если отсутствующих было $x,$ то присутствующих было $x:0,25=4x.$ После того, как пришёл один опоздавший, присутствующих стало $4x+1,$ а отсутствующих --- $x-1,$ а значит $5(x-1)=4x+1,\ 5x-5=4x+1,\ x=6.$ Таким образом, всего в классе $6+4\cdot6=30$ человек.\\
20. Если отсутствующих было $x,$ то присутствующих было $x:0,2=5x.$ После того, как один человек ушёл, присутствующих стало $5x-1,$ а отсутствующих --- $x+1,$ а значит $4(x+1)=5x-1,\ 4x+4=5x-1,\ x=5.$ Таким образом, всего в классе $5+5\cdot5=30$ человек.\\
21. Пусть длина и ширина участка были равны $a$ и $b,$ тогда $1,3a\cdot1,2b=ab+28,\ 1,56ab=ab+28,\ 0,56ab=28,\ ab=50\text{м}^2.$\\
22. Пусть длина и ширина участка были равны $a$ и $b,$ тогда $1,4a\cdot1,1b=ab+27,\ 1,54ab=ab+27,\ 0,54ab=27,\ ab=50\text{м}^2.$\\
23. Пусть изначально в городе $N$ проживало $x$ человек, тогда $x\cdot1,06\cdot1,06=x+86520,\ 1,1236x=x+86520,\ 0,1236x=86520,\ x=700000.$ Значит, сейчас в городе $N$ проживает $700000+86520=786520$ человек.\\
24. Пусть изначально в городе $N$ проживало $x$ человек, тогда $x\cdot1,07\cdot1,07=x+86940,\ 1,1449x=x+86940,\ 0,1449x=86940,\ x=600000.$ Значит, сейчас в городе $N$ проживает $600000+86940=686940$ человек.\\
25. Пусть у сестры $x$ рублей, тогда у брата $x+600$ и $0,6x\cdot3=x+600+0,4x,\ 1,8x=1,4x+600,\ 0,4x=600,\ x=1500.$ Значит, у сестры 1500 рублей, а у брата $1500+600=2100$ рублей.\\
26. Пусть у брата $x$ рублей, тогда у сестры $x+800$ и $0,8x\cdot2=x+800+0,2x,\ 1,6x=1,2x+800,\ 0,4x=800,\ x=2000.$ Значит, у брата 2000 рублей, а у сестры $2000+800=2800$ рублей.\\
27. Пусть оба куска весят $m$кг, а доля серебра в них равна $x$ и $y$ соответственно. Тогда весь сплав весит $m+0,5m=1,5m$кг, а количество серебра в нём в первом случае равно $mx+0,5my=m(x+0,5y),$ а во втором --- $0,5mx+my=m(0,5x+y).$ Выразив долю серебра в получившихся сплавах, получим и решим систему из двух линейных уравнений:\\
$\begin{cases}
\cfrac{m(x+0,5y)}{1,5m}=0,4\\
\cfrac{m(0,5x+y)}{1,5m}=0,5
\end{cases}
\Leftrightarrow
\begin{cases}
x+0,5y=0,6\\
0,5x+y=0,75
\end{cases}
\Leftrightarrow
\begin{cases}
x+0,5y=0,6\\
x+2y=1,5
\end{cases}
\Leftrightarrow
\begin{cases}
1,5y=0,9\\
x+2y=1,5
\end{cases}
\Leftrightarrow$\\$\Leftrightarrow
\begin{cases}
y=0,6\\
x=0,3
\end{cases}
$\\
Значит, первый кусок содержит $30\%$ серебра, а второй --- $60\%.$\\
28. Пусть у Васи $x$ фантиков, а у Пети $y.$ Тогда сначала у Васи стало $0,9x$ фантиков, а у Пети $y+0,1x.$ Затем у Васи станет $0,9x+0,1(y+0,1x)=0,91x+0,1y,$ что равно изначальному количеству, а значит $0,91x+0,1y=x,\ 0,1y=0,09x,\ y=0,9x.$ Тогда если Вася отдаст Пете $36\%$ своих фантиков, у Пети станет $0,9x+0,36x=1,26x$ фантиков, что больше в $1,26x:(0,9x)=1,4$ раза.\\
29. Пусть у Васи $x$ фантиков, а у Пети $y.$ Тогда сначала у Пети стало $0,8y$ фантиков, а у Васи $x+0,2y.$ Затем у Пети станет $0,8y+0,2(x+0,2y)=0,2x+0,84y,$ что равно изначальному количеству, а значит $0,2x+0,84y=y,\ 0,2x=0,16y,\ x=0,8y.$ Тогда если Петя отдаст Васе $48\%$ своих фантиков, у Васи станет $0,8y+0,48y=1,28x$ фантиков, что больше в $1,28y:(0,8y)=1,6$ раза.\\
30. В 750г $15\%$ раствора сахара содержится $750\cdot0,15=112,5$г сахара. В новом растворе это количество должно составлять $5\%,$ значит всего в нём $112,5:0,05=2250$г, что на $2250-750=1500$г больше, чем в изначальном.\\
31. В 1500г $5\%$ раствора соли содержится $1500\cdot0,05=75$г соли. В новом растворе это количество должно составлять $12\%,$ а значит всего в нём $75:0,12=625$г, что на $1500-625=875$г меньше, чем в изначальном.\\
32. Пусть третье число равно $x,$ тогда первое и второе числа равны $0,8x$ и $0,3x,$ поэтому $(x+0,8x+0,3x):3=21,21,\ 0,7x=21,21,\ x=30,3.$ Значит, эти числа равны $30,3\cdot0,8=24,24,\ 30,3\cdot0,3=9,09$ и $30,3.$\\
33. Мама оставила Васе $20:(1-0,9)=200$ рублей.\\
34. Получится $80\cdot(1-0,65)=28$кг сухой ромашки.\\
35. В июне виноград стоил $200\cdot1,04\cdot0,96=199,68$р, то есть 199 рублей 68 копеек.\\
36. Концентрация полученного раствора равна $(7\cdot0,16+3\cdot0,06):(7+3)\cdot100\%=13\%.$\\
37. Пусть одна часть равна $x,$ тогда вторая равна $80-x$ и $x=0,6(80-x),\ x=48-0,6x,\ 1,6x=48,\ x=30.$ Значит, эти части равны 30 и $80-30=50.$\\
38. Так как сумма увеличилась более, чем на $(1+4):2=2,5\%,$ на $4\%$ было увеличено большее из чисел. Тогда $(x+8)\cdot1,03=8\cdot1,01+x\cdot1,04,\
1,03x+8,24=8,08+1,04x,\ 0,01x=0,16,\ x=16.$\\
39. Возьмём $x$кг первого сплава и $y$кг второго сплава, тогда с одной стороны в полученном сплаве будет $0,3x+0,5y$ золота, а с другой --- $0,35(x+y).$ Значит,
$0,3x+0,5y=0,35(x+y),\ 0,3x+0,5y=0,35x+0,35y,\ 0,15y=0,05x,\ x=3y.$ То есть первого сплава необходимо взять в 3 раза больше и отношение равно $3:1.$\\
40. Пусть эти числа равны $\cfrac{1}{15}x,\ 0,1x,\ \cfrac{1}{3}x$ и $0,8\cdot\cfrac{1}{3}x=\cdot{4}{15}x.$ Тогда $\left(\cfrac{1}{3}x+\cdot{4}{15}x
ight)-
\left(\cfrac{1}{15}x+0,1x
ight)=26,\ \cfrac{13}{30}x=26,\ x=60.$ Значит, эти числа равны $\cfrac{1}{15}\cdot60=4,\ 0,1\cdot60=6,\ \cfrac{1}{3}\cdot60=20$ и $\cfrac{4}{15}\cdot60=16.$\\
41. Так как содержание увеличилось менее, чем на $(10+30):2=20\%,$ больше на 54 килограмма было сплава с 10-процентным содержанием олова. Тогда с одной стороны олова в полученном сплаве было $(x+54)\cdot0,1+x\cdot0,3=0,4x+5,4$кг а с другой --- $(x+x+54)\cdot0,182=0,364x+9,828$кг. Значит, $0,4x+5,4=0,364x+9,828,\
0,036x=4,428,\ x=123$кг.\\
42. Пусть изначальная цена товара была равна $x,$ тогда это отношение равно $\cfrac{0,8\cdot0,8x}{0,6x}=\cfrac{16}{15}.$\\
43. Пусть изначально вареники стоили $x$ рублей, тогда $x\cdot1,2\cdot0,95\cdot1,2=171,\ 1,368x=171,\ x=125$ рублей.\\
44. $(90\%\text{ от }a)\cdot(120\% \text{ от } b)=0,9a\cdot1,2b=1,08ab>ab>1.$ Последнее число является отрицательным, значит самым маленьким. $\cfrac{b^2}{a^2}=\left(\cfrac{b}{a}
ight)^2<\cfrac{b}{a}$ так как $\cfrac{b}{a}<1.$ Значит, числа располагаются в следующем порядке:
$(a-b)(b-a),\ \cfrac{b^2}{a^2},\ \cfrac{b}{a},\ a\cdot b,\ (90\%\text{ от }a)\cdot(120\% \text{ от } b).$\\
45. $(170\%\text{ от }a)\cdot(90\% \text{ от } b)=1,7a\cdot0,9b=1,53ab>ba>1.$ Последнее число является отрицательным, значит самым маленьким. $\cfrac{b^2}{a^2}=\left(\cfrac{b}{a}
ight)^2<\cfrac{b}{a}$ так как $\cfrac{b}{a}<1.$ Значит, числа располагаются в следующем порядке:
$(a-b)(b-a),\ \cfrac{b^2}{a^2},\ \cfrac{b}{a},\ b\cdot a,\ (170\%\text{ от }a)\cdot(90\% \text{ от } b)$\\
46. Всего меди в сплаве $600\cdot0,1=60$г. Пусть добавили $x$г меди, тогда $\cfrac{x+60}{x+600}=0,2,\ x+60=0,2x+120,\ 0,8x=60,\ x=75$г.\\
47. Пусть первое слагаемое равно $x,$ тогда второе слагаемое равно $200-x$ и верно соотношение $0,25x=0,375(200-x),\ 0,25x=75-0,375x,\ 0,625x=75,\ x=120.$ Значит, эти слагаемые равны 120 и $200-120=80.$\\
48. Пусть вычитаемое равно $x,$ тогда уменьшаемое равно $x+200$ и верное соотношение $0,3(x+200)=0,7x,\ 0,3x+60=0,7x,\ 0,4x=60,\ x=150.$ Значит, искомое представление $350-150=200.$\\
49. Пусть доля соли во втором растворе равна $x,$ тогда $30\cdot0,2+10x=(30+10)\cdot0,25,\ 6+10x=10,\ 10x=4,\ x=0,4.$ Значит, концентрация соли во втором растворе равна $40\%.$\\
50. Пусть добавили $x$кг золота, тогда в образовавшемся сплаве с одной стороны будет $10\cdot0,4+x=x+4$кг золота, а с другой стороны --- $(x+10)\cdot0,8=0,8x+8$кг, а значит $x+4=0,8x+8,\ 0,2x=4,\ x=20$кг.\\
51. Пусть добавили $x$кг меди, тогда в образовавшемся сплаве с одной стороны будет $60\cdot0,6=36$кг олова, а с другой стороны --- $(x+60)\cdot0,4=0,4x+24$кг, а значит $36=0,4x+24,\ 0,4x=12,\ x=30$кг.\\
52. Пусть первая часть равна $x,$ тогда вторая часть равна $90-x$ и $0,4x-0,3(90-x)=15,\ 0,4x-27+0,3x=15,\ 0,7x=42,\ x=60.$ Значит, эти части равны 60 и $90-60=30.$\\
53. Пусть изначальная масса грибов равна $x,$ тогда в них было $0,8x$ воды и $x-0,8x=0,2x$ сухого вещества. После сушки в них осталось $0,8x\cdot(1-0,75)=0,2x$ воды, а общая масса стала равна $0,2x+0,2x=0,4x.$ Значит, масса воды в сушёных грибах составляет $0,2x:(0,4x)\cdot100\%=50\%.$\\
54. Пусть масса сушёных грибов равна $x,$ тогда в них содержится $0,25x$ воды и $0,75x$ сухого вещества. В свежих грибах было $0,25x:(1-0,8)=1,25x$ воды, а общая масса была равна $1,25x+0,75x=2x.$ Значит, масса воды в свежих грибах составляла $1,25x:(2x)\cdot100\%=62,5\%.$\\
55. Пусть девочек в клубе было $x,$ тогда всего детей было $x:0,25=4x$ и верно соотношение $(4x+10)\cdot0,3=x+10,\ 1,2x+3=x+10,\ 0,2x=7,\ x=35.$ Значит, мальчиков в клубе было $4\cdot35-35=105.$\\
56. Пусть мальчиков в хоре было $x,$ тогда всего детей было $x:0,25=4x$ и верно соотношение $(4x+3)\cdot0,28=x+3,\ 1,12x+0,84=x+3,\ 0,12x=2,16,\ x=18.$ Значит, девочек в хоре было $4\cdot18-18=54.$\\
57. Пусть изначальное содержание соли было $x\%,$ тогда оно сначала стало $0,8x\%,$ а потом --- $1,2\cdot0,8x=0,96x\%,$ то есть изменилось на $4\%.$\\
58. Пусть изначальная дробь была $\cfrac{a}{b},$ тогда $\cfrac{1,2a}{xb}=3\cdot \cfrac{a}{b},\ x=1,2:3=0,4.$ Значит, новый знаменатель составляет $60\%$ от изначального, то есть изначальный уменьшили на $40\%.$\\
59. Пусть масса третьего слитка равна $m,$ а доля серебра в нём равна $x.$ Тогда имеем систему уравнений $\begin{cases}0,56(5+m)=5\cdot0,3+xm,\\0,6(3+m)=3\cdot0,3+xm.\end{cases}\Leftrightarrow
\begin{cases}2,8+0,56m=1,5+xm,\\ 1,8+0,6m=0,9+xm.\end{cases}\Leftrightarrow
\begin{cases}1,3+0,56m=xm,\\ 0,9+0,6m=xm.\end{cases}\Leftrightarrow
\begin{cases}0,04m-0,4=0,\\ x=\cfrac{0,9+0,6m}{m}.\end{cases}\Leftrightarrow
\begin{cases}m=10,\\ x=0,69.\end{cases}$ Значит, масса третьего слитка равна 10 кг, а процентное содержание серебра в нём --- $69\%.$\\
60. Пусть Федин изначальный рейтинг равен $x\%.$ Тогда в первый день он станет равен $1,25x\%,$ а во второй --- $0,5\cdot1,25x\%=0,625x\%.$ Чтобы в третий день он опять стал равен $x\%,$ его необходимо умножить на $1:0,625=1,6,$ то есть увеличить на $60\%.$\\
61. Пусть изначально в колбе было $x$л кислоты, тогда её доля составляла $\cfrac{x}{10}.$ После первого отливания кислоты останется $x-7\cdot\cfrac{x}{10}=\cfrac{3x}{10},$ а её доля станет равна $\cfrac{3x}{100}.$ После второго отливания кислоты останется $\cfrac{3x}{10}-7\cdot\cfrac{3x}{100}=\cfrac{9x}{100},$ а её доля будет равна $\cfrac{9x}{1000}.$ Значит, верно равенство $\cfrac{9x}{1000}=0,081,\ 9x=81,\ x=9$л. Таким образом, изначально в растворе было $\cfrac{9}{10}\cdot100\%=90\%$ кислоты.\\
62. Пусть изначально в колбе было $x$л кислоты, тогда её доля составляла $\cfrac{x}{10}.$ После первого отливания кислоты останется $x-6\cdot\cfrac{x}{10}=\cfrac{4x}{10},$ а её доля станет равна $\cfrac{4x}{100}.$ После второго отливания кислоты останется $\cfrac{4x}{10}-6\cdot\cfrac{4x}{100}=\cfrac{16x}{100},$ а её доля будет равна $\cfrac{16x}{1000}.$ Значит, верно равенство $\cfrac{16x}{1000}=0,064,\ 16x=64,\ x=4$л. Таким образом, изначально в растворе было $\cfrac{4}{10}\cdot100\%=40\%$ кислоты.

ewpage
