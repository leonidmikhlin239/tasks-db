\section{Геометрия задачи}
1. Доказать, что если медиана треугольника является высотой, то треугольник равнобедренный.\\
2. Доказать, что если высота треугольника является биссектрисой, то треугольник равнобедренный.\\
3. Точка $D$ лежит на стороне $AB$ треугольника $ABC,$ причём $AD=DB=DC,\ DE\parallel BC.$ Доказать, что $DE\perp AC.$\\
4. На сторонах $AB$ и $BC$ равнобедренного треугольника $ABC\ (AB=AC)$ отмечены точки $D$ и $E$ так, что $AD=DE$ и $DE\parallel AC.$ Доказать, что $AE\perp BC.$\\
5. В четырёхугольнике $PRNL\ (PR\parallel NL)$ на стороне $PR$ взята точка $M,$ а на стороне $NL$ --- точка $K.$ Оказалось, что $MN$ --- биссектриса угла $RMK,$ а $ML$ --- биссектриса угла $PMK.$ Найдите длину $LN,$ если $MK=5.$\\
6. В четырёхугольнике $STGQ\ (ST\parallel GQ)$ на стороне $ST$ взята точка $M,$ а на стороне $GQ$ --- точка $K.$ Оказалось, что $MG$ --- биссектриса угла $TMK,$ а $MQ$ --- биссектриса угла $SMK.$ Найти длину $MK,$ если $GQ=12.$\\
7. Докажите, что сумма длин диагоналей выпуклого четырёхугольника больше, чем половина периметра этого четырёхугольника.\\
8. В треугольнике из двух различных вершин проведены отрезки, соединяющие эти вершины с точками на противоположных сторонах. Доказать, что сумма длин этих отрезков меньше периметра треугольника.\\
9. Доказать, что если треугольники $ABC$ и $A_1B_1C_1$ равны, то $AE=A_1E_1,$ где $E$ и $E_1$ --- середины медиан $BD$ и $B_1D_1$ соответственно.\\
10. Доказать, что если треугольники $MNP$ и $M_1N_1P_1$ равны, то $ME=M_1E_1,$ где $E$ и $E_1$ --- середины высот $PH$ и $P_1H_1$ соответственно.\\
11. В прямоугольном треугольнике $ABC$ на гипотенузе $AB$ выбрана точка $D$ так, что
$\Delta ACD$ равносторонний. Докажите, что $\Delta BCD$ равнобедренный.\\
12. В прямоугольном треугольнике $ABC$ на гипотенузе $AB$ выбрана точка $D$ так, что
$\Delta BCD$ равнобедренный с углом $120^\circ.$ Докажите, что $\Delta ACD$ равносторонний.\\
13. Треугольник $ABC$ --- равнобедренный $(AB=BC),\ \angle B=24^\circ.\ CP$ ---
биссектриса треугольника, $PK\parallel BC$ (точка $K$ лежит на стороне $AC$). Найдите угол $\angle KPC.$\\
14. Треугольник $ABC$ --- равнобедренный $(AB=BC),\ \angle C=72^\circ.\ AP$ ---
биссектриса треугольника, $PK\parallel AB$ (точка $K$ лежит на стороне $AC$). Найдите угол $\angle KPA.$\\
15. Существует ли равнобедренный треугольник, в котором биссектриса одного из углов равна одной из сторон треугольника? (Не забудьте доказать полученный Вами ответ).\\
16. В равнобедренном треугольнике биссектриса одного из углов равна одной из сторон треугольника. Верно ли, что этот треугольник --- прямоугольный? (Не забудьте доказать полученный Вами ответ).\\
17. В равнобедренном треугольнике один из углов равен $120^\circ,$ а высота, проведённая к боковой стороне, равна 17 см. Найдите основание треугольника.\\
18. В равнобедренном треугольнике один из углов равен $120^\circ,$ а основание треугольника равно 10 см. Найдите высоту, проведённую к боковой стороне.\\
19. Можно ли два равнобедренных треугольника с равными боковыми сторонами расположить так, чтобы один лежал внутри другого?\\
20. Можно ли треугольник, две стороны которого равны 566 и 566, поместить в треугольник, две стороны которого равны 239 и 566?\\
21. В прямоугольном треугольнике $ABC$ с гипотенузой $BC$ и углом $B,$ равным $60^\circ,$ проведена высота $AD.$ Найдите $DC,$ если $DB=2\text{см}.$\\
22. В прямоугольном треугольнике $ABC$ с гипотенузой $AC,$ равной 12 см, проведена высота $BD.$ Найдите $CD$ и $DA,$ если угол $\angle A=30^\circ.$\\
23. Сколько существует неравных между собой равнобедренных треугольников со стороной 5см и углом $30^\circ ?$\\
24. Сколько существует неравных между собой равнобедренных треугольников со стороной 5см и углом $60^\circ ?$\\
25. В прямоугольном треугольнике $ABC$ с гипотенузой $AC$ угол $A$ равен $60^\circ,\ BC=6\text{см}.\ AL$ --- биссектриса треугольника $ABC.$ Найдите высоту $LH$ треугольника $ALC.$\\
26. В прямоугольном треугольнике $ABC$ с гипотенузой $AC$ угол $A$ равен $60^\circ.$ Через середину $M$ отрезка $AC$ проведён перпендикуляр к нему, пересекающий прямую $BA$ в точке $T.$ $BC=3\text{см}.$ Найдите$MT.$\\
27. В треугольнике $ABC$ углы $A$ и $B$ равны соответственно $48^\circ$ и $76^\circ.$ Найдите угол между биссектрисой и высотой, проведёнными из вершины $C.$\\
28. В треугольнике $ABC$ углы $A$ и $B$ равны соответственно $64^\circ$ и $24^\circ.$ Найдите угол между биссектрисой и высотой, проведёнными из вершины $A.$\\
29. В треугольнике $ABC$ медиана $AM$ перпендикулярна стороне $AC.$ Найти угол $BAC,$ если $AB=2AC.$\\
30. В треугольнике $ABC$ медиана $AM$ составляет со стороной $AB$ угол $30^\circ.$ Найти угол $BAC,$ если $AB=2AC.$\\
31. В четырёхугольнике $ABCD\ AB=BC.$ Лучи $BA$ и $CD$ пересекаются в точке $E,$ а лучи $AD$ и $BC$ --- в точке $F.$ Известно, что $BE=BF$ и $\angle DEF=25^\circ.$ Найдите $\angle EFD.$\\
32. В четырёхугольнике $KLMN\ NK=LK.$ Лучи $KL$ и $NM$ пересекаются в точке $P,$ а лучи $LM$ и $KN$ --- в точке $Q.$ Известно, что $KP=KQ$ и $\angle MPQ=28^\circ.$ Найдите $\angle PQM.$\\
33. В прямоугольном треугольнике один из углов равен $30^\circ.$ Докажите, что в этом треугольнике отрезок серединного перпендикуляра, проведённого к гипотенузе до пересечения с катетом, втрое меньше большего катета.\\
34. В прямоугольном треугольнике один из углов равен $60^\circ.$ Через середину гипотенузы проведён перпендикуляр до пересечения с катетом. Докажите, что больший катет втрое больше длины построенного перпендикуляра.\\
35. Найдите углы равнобедренного треугольника, если один из его внешних углов равен $130^\circ.$\\
36. Найдите периметр равнобедренного треугольника, если две его стороны равны 6см и 10см.\\
37. В треугольнике $ABC$ угол $A$ равен $40^\circ,$ угол $B$ равен $20^\circ,$ а $AB-BC=4.$  Найдите длину биссектрисы угла $C.$\\
38. В треугольнике $MNP$ угол $M$ равен $40^\circ,$ угол $N$ равен $20^\circ,$ а $MN-NP=8.$  Найдите длину биссектрисы угла $P.$\\
39. В остроугольном треугольнике $ABC$ угол $B$ равен $60^\circ.$ Найдите, в каком отношении биссектриса $BL$ делит высоту $AH.$\\
40. В прямоугольном треугольнике $ABC$ с гипотенузой $AC$ угол $A$ равен $60^\circ,\
BC=6\text{см}.\ AL$ --- биссектриса треугольника $ABC.$ Найдите высоту $LH$ треугольника $ALC.$\\
41. Можно ли какой-либо прямоугольный треугольник разрезать на два треугольника, один из которых равносторонний, а другой равнобедренный?\\
42. Может ли одна из биссектрис треугольника делить другую биссектрису пополам?\\
43. В прямоугольном треугольнике гипотенуза равна 4 см, а острый угол --- $30^\circ.$ Высота, проведённая из вершины прямого угла, делит гипотенузу на два отрезка. Найдите длины этих отрезков.\\
44. В прямоугольном треугольнике гипотенуза равна 6 см, а острый угол --- $30^\circ.$ Высота, проведённая из вершины прямого угла, делит гипотенузу на два отрезка. Найдите длины этих отрезков.\\
45. Два угла равнобедренного треугольника пропорциональны числам 5 и 2. Найдите угол между биссектрисами неравных углов.\\
46. В прямоугольнике $ABCD$ сторона $BC$ в 2 раза больше стороны $AB.$ на продолжении стороны $AD$ за точку $D$ выбрана точка $F.$ Пусть $E$ --- середина стороны $AD,\ \angle DFC=30^\circ.$ Найдите $\angle EBF.$\\
47. В прямоугольнике $ABCD\ AD=2AB.$ На стороне $BC$ отмечена точка $M$ так, что $MA$ --- биссектриса $\angle BMD.$ Найдите $\angle BMA.$\\
48. Внешний угол при вершине $B$ прямоугольного треугольника $ABC$ равен $120^\circ,$ биссектриса угла $\angle ABC$ равна 2 см. Найдите длину стороны $AC,$ если известно, что $\angle C=90^\circ.$\\
49. Внешний угол при вершине $B$ прямоугольного треугольника $ABC$ равен $150^\circ,$ биссектриса острого угла $A$ равна 3 см. Найдите длину стороны $CB.$\\
50. На стороне $CB$ прямоугольного треугольника $ABC$ взята точка $P,$ а на гипотенузе $AB$ взята точка $S.$ При этом $\angle B=35^\circ,\ \angle SCB=20^\circ,\ \angle BAP=10^\circ.$ Докажите, что треугольники $ACP$ и $ACS$ равнобедренные.\\
51. На стороне $AB$ четырёхугольника $ABCD$ взята точка $E.$ При этом $CB\perp BA,$ $DE\perp BA.$ Диагональ $AC$ пересекает отрезок $DE$ в точке $M.$ Известно, что
$\angle BCE=40^\circ,$ $\angle EMA=65^\circ,\ \angle EDA=45^\circ.$ Докажите, что треугольники $CEA$ и $CED$ равнобедренные.\\
52. Какие-то две стороны равнобедренного треугольника отличаются на 8 см, а какие-то две составляют в сумме 20 см. Определите все значения, которые может принимать длина основания такого треугольника.\\
53. Какие-то две стороны равнобедренного треугольника составляют в сумме 16 см, а какие-то две отличаются на 6 см. Определите все значения, которые может принимать длина основания такого треугольника.\\
54. В прямоугольном треугольнике $ABC$ с углом $A,$ равным $30^\circ,$ к гипотенузе $AC$ проведена высота $BH.$ На стороне $BC$ выбрана точка $K$ так, что $KC=HC.$ Лучи $AB$ и $HK$ пересекаются в точке $N.$ Найдите отношение отрезков $AH$ и $KN.$\\
55.  В прямоугольном треугольнике $ABC$ с углом $B,$ равным $30^\circ,$ к гипотенузе $AB$ проведена высота $CH.$ На продолжении стороны $BC$ за точку $C$ выбрана точка $K$ так, что $KC=HC.$ Отрезки $AC$ и $HK$ пересекаются в точке $M.$ Найдите отношение отрезков $BH$ и $KM.$\\
56. Биссектриса внешнего угла $ABD$ треугольника $ABC$ пересекает биссектрису угла $ACB$ в точке $K,\ \angle CKB=19^\circ.$ Найдите $\angle BAC.$\\
57. Биссектриса внешнего угла $ACD$ треугольника $ABC$ пересекает биссектрису угла $ABC$ в точке $M,\ \angle BAC=52^\circ.$ Найдите $\angle BMC.$\\
58. Равные отрезки $AB$ и $CD$ пересекаются в их общей середине $E,\ AD=CE.$ прямая, проходящая через точку $E$ и перпендикулярная к $DE,$ пересекает отрезок $BD$ в точке $M.$ Докажите, что расстояние от точки $M$ до прямой $BC$ в два раза меньше длины отрезка $MD.$\\
59. Равные отрезки $AD$ и $BC$ пересекаются в их общей середине $E,\ AB=DE.$ прямая, проходящая через точку $E$ и перпендикулярная к $BE,$ пересекает луч $CD$ в точке $K.$ Докажите, что расстояние от точки $D$ до прямой $KE$ в четыре раза меньше длины отрезка $KC.$\\
60. Точки $B$ и $O$ расположены по разные стороны от прямой $AC,$ при этом  $OA=OB=OC$ и $\angle AOB = 52^\circ.$ Найдите $\angle ACB.$\\
61. Точки $A$ и $O$ расположены по разные стороны от прямой $BC,$ при этом  $OA=OB=OC$ и $\angle ACB= 17  ^\circ.$ Найдите $\angle AOB.$\\
62. Через вершину $B$ треугольника $ABC$ провели прямую $l,$ параллельную $AC.$ Биссектриса угла $\angle BCA$ пересекает прямую $l$ в точке $D.$ Точка $K$ такова, что $B$ --- середина $DK.$ Докажите, что $\Delta CDK$ --- прямоугольный.\\
63. В треугольнике $ABC$ провели медиану $BD.$ Нашлась такая точка $K,$ что $BK\parallel AC$ и $\angle KBA=\angle ABD.$ Докажите, что $\Delta ABC$ --- прямоугольный.\\
64. На стороне $BC$ треугольника $ABC$ расположены точки $P$ и $K$ так, что $AP=BP$ и $KC=AK.$ При этом оказалось, что величина угла $PAK$ равна $30^\circ.$ Найдите угол $BAC.$\\
65. На стороне $PC$ треугольника $PKC$ расположены точки $A$ и $B$ так, что $AP=AK$ и $KB=BC.$ При этом оказалось, что величина угла $AKB$ равна $40^\circ.$ Найдите угол $PKC.$\\
66. В треугольнике $ABC$ угол $B$ равен $30^\circ,$ угол $A$ равен $120^\circ.$ Из вершины $B$ проведена высота $BH,$ при этом оказалось, что $HC=$1дм 2 см. Найдите расстояние от точки $A$ до прямой $BC.$\\
67. Из вершины $K$ треугольника $PTK$ проведена высота $KH,$ при этом оказалось, что $HP=$2дм 1см. Найдите расстояние от точки $T$ до прямой $PK,$ если известно, что угол $K$ равен $30^\circ,$ угол $T$ равен $120^\circ.$\\
68. Из отрезков длиной 3, 5, 6, 11 составили четырёхугольник, одна из диагоналей которого имеет целую длину. Чему может быть равна эта длина?\\
69. Точка $K$ лежит на отрезке $AB,$ а точка $M$ --- на отрезке $BC.$ Отрезки $AM$ и $CK$ пересекаются в точке $P.$ Оказалось, что $\angle ABC=37^\circ,\ \angle BAM : \angle CAM=4:7,$ $\angle ACK: \angle BCK=7:4.$ Найдите величину угла $APC.$\\
70. Точка $M$ лежит на отрезке $AC,$ а точка $P$ --- на отрезке $BC.$ Отрезки $AP$ и $BM$ пересекаются в точке $K.$ Оказалось, что $\angle ACB=26^\circ,\ \angle CAP : \angle BAP=5:6,$ $\angle ABM: \angle CBM=6:5.$ Найдите величину угла $AKB.$\\
71. Точка $M$ --- середина стороны $AC$ треугольника $ABC,$ в котором $\angle B=90^\circ,\ \angle A=30^{\circ}.$ На стороне $AB$ отмечена такая точка $D,$ что $AD=BC.$ Пусть $DP$ --- высота треугольника $MBD.$ Докажите, что удвоенный периметр треугольника $MDP$ больше периметра треугольника $MDB.$\\
72. Точка $D$ --- середина стороны $AB$ треугольника $ABC,$ в котором $\angle C=90^{\circ},\ \angle A=60^{\circ}.$ На стороне $BC$ отмечена такая точка $M,$ что $BM=AC.$ Пусть $MK$ --- высота треугольника $MCD.$ Докажите, что удвоенный периметр треугольника $MDK$ больше периметра треугольника $MDC.$\\
73. На стороне $AB$ квадрата $ABCD$ построен равносторонний треугольник $MAB,$ причём точка $M$ лежит вне квадрата. Найдите углы треугольника $DMC.$\\
74. На стороне $AB$ квадрата $ABCD$ построен равносторонний треугольник $NAB,$ причём точка $N$ лежит внутри квадрата. Найдите углы треугольника $DNC.$\\
75. Острый угол прямоугольного треугольника равен $30^\circ,$ гипотенуза равна 8. Найдите отрезки, на которые делит гипотенузу высота, проведённая из вершины прямого угла.\\
76. Острый угол прямоугольного треугольника равен $60^\circ.$ Высота к гипотенузе делит её на два отрезка, длина большего из которых равна 12. Найдите длину гипотенузы.\\
77. Четырёхугольник $ABCD$ называется дельтоидом, если в нём $BA=AD$ и $BC=CD.$ Докажите, что его диагонали $AC$ и $BD$ перпендикулярны.\\
78. В треугольнике $ABC$ угол $B$ равен $90^\circ,\ CC_1$ --- биссектриса, $CC_1=16$см, $BC_1=8$см. Найдите внешний угол при вершине $A.$\\
79. В треугольнике $ABC\ \angle C=90^\circ,$ проведена высота $CD,\ BC=2BD.$ Найдите $AD,$ если $BC=4.$\\
80. В треугольнике $KLM$ известно, что $KM<ML<LK.$ Укажите наибольший угол треугольника.\\
81. Найдите градусные меры трёх углов треугольника, если два его внешних угла равны соответственно $80^\circ$ и $115^\circ.$\\
82. На сколько частей делят плоскость четыре прямые, являющиеся продолжениями сторон четырёхугольника, никакие две стороны которого не параллельны?\\
83. Две стороны треугольника равны 2 см и 6 см. Третья сторона выражается целым числом сантиметров. Какой может быть её длина? Выпишите все возможные варианты.\\
84. В прямоугольном треугольнике $KLM$ с углами $\angle M=90^\circ$ и $\angle K=60^\circ$ проведена высота $MH,$ причём $KH=6.$ Найдите длину $LH.$\\
85. В треугольнике $ABC$ угол $B$ --- прямой, $BD$ --- высота,  $BC$ в два раза больше $DC.$ Найти отношение длин отрезков $DC$ и $AD.$\\
86. Построить треугольник по двум сторонам и медиане, проведённой к меньшей из них.\\
87. В треугольнике $ABC\ AB=BC,\ \angle C=72^\circ,\ AP$ --- биссектриса, $PK\parallel AB,\ PK$ пересекает сторону $AC$ в точке $K.$ Найдите $\angle KPA.$\\
88. В треугольнике $ABC$ биссектрисы $AA_1$ и $BB_1$ пересекаются в точке $M,$ при этом $\angle AMB=120^\circ.$ Найдите $\angle C.$\\
89. В треугольнике $ABC\ AB=BC,\ AC=8,$ точка $E$ лежит на стороне $BC,$ причём $BE=EC.$ Точка $E$ делит периметр треугольника $ABC$ (считая от вершины $A$) на две части, из которых одна больше другой на 2. Найдите $AB.$\\
90. Как с помощью циркуля и линейки разделить угол в $54^\circ$ на три равные части?\\
91. В треугольнике $ABC$ угол $A$ равен $70^\circ.$ Биссектрисы углов $A$ и $C$ пересекаются в точке $O.$ Угол $AOC$ равен $115^\circ.$ Найдите углы $B$ и $C$ треугольника $ABC,$ а также углы $AOB$ и $BOC.$\\
92. Биссектрисы внешних углов при вершинах $A$ и $B$ равнобедренного треугольника $ABC\ (AB=BC)$ пересекаются в точке $O,$ угол $AOB=70^\circ.$ Найдите углы треугольника $ABC.$\\
93. Длина отрезка $BC$ равна 8. Точка $A$ лежит на прямой $BC,$ но не принадлежит отрезку $BC,$ причём $5AB=AC.$ Точка $D$ принадлежит отрезку $BC$ и $4DC=BC.$ Найдите длину отрезка $AD.$\\
94. На сторонах угла $A,$ равного $127^\circ,$ отмечены точки $B$ и $C,$ а внутри угла --- точка $D$ так, что $\angle ABD=25^\circ,\ \angle ACD=19^\circ.$ На луче $BD$ отмечена точка $P$ так, что точка $D$ лежит между точками $B$ и $P.$ Найдите угол $PDC.$\\
95. В треугольнике $ABC$ высоты $AH$ и $BP$ равны между собой, угол $ABP$ равен углу $CAH.$ Найдите углы треугольника.\\
96. В треугольнике $ABC\ \angle A: \angle B: \angle C=3:5:2.$ На прямой, содержащей медиану $CM,$ отложен отрезок $MK,$ равный $CM.$ Найдите $\angle ABK.$\\
97. В равнобедренном треугольнике $ABC$ с основанием $AB$ проведены биссектрисы $AA_1$ и $BB_1,$ которые пересекаются в точке $O.$ Найти углы треугольника $AOB_1,$ если один из углов треугольника $ABC$ на $30^\circ$ больше другого. Рассмотреть не менее двух случаев.\\
98. В равнобедренном треугольнике биссектриса угла при основании делит медиану, проведённую из другого угла при основании, пополам. Найдите стороны треугольника, если его периметр равен 20 см.\\
99. В треугольнике $ABC\ \angle A: \angle B=2:5,\ \angle B: \angle C=5:11.$ Найдите угол между высотой и биссектрисой, проведёнными из вершины меньшего угла.\\
100. В треугольнике $ABC$ угол $B$ равен $100^\circ.$ На луче $CA$ отмечена точка $M$ так, что $MA=AB,$ и точка $A$ находится между точками $M$ и $C.$ На луче $AC$ отмечена точка $N$ так, что $CN=BC,$ и точка $C$ находится между точками $A$ и $N.$ Найдите градусную меру угла $MBN.$\\
101. В треугольнике $ABC\ AB=BC,$ точка $T$ --- середина стороны $AB,$ точка $H$ --- середина стороны $BC,$ отрезок $TP$ перпендикулярен к стороне $AB,$ отрезок $KH$ перпендикулярен к стороне $BC$ (точки $P$ и $K$ лежат на стороне $AC),\ \angle ABC=120^\circ,\ AC=21$ см. Найдите длину отрезка $PK.$\\
102. Через вершины $A$ и $C$ треугольника $ABC$ проведены прямые, перпендикулярные биссектрисе угла $ABC$ и пересекающие прямые $CB$ и $BA$ в точках $K$ и $M$ соответственно. Найдите $AB,$ если $BM=8,\ KC=1.$\\
103. Дан треугольник $ABC$ с углами $30^\circ,\ 70^\circ$ и $80^\circ$ соответственно. Внутри треугольника взята точка $O,$ такая, что треугольники $AOB,\ AOC$ и $BOC$ являются равнобедренными с общей вершиной $O.$ Найдите углы этих равнобедренных треугольников.\\
104. Треугольник $PQR$ --- прямоугольный $(\angle R=90^\circ),\ \angle PQR=60^\circ,\ QR=6,5.$ Через точку $Q$ проведена прямая, параллельная прямой $PR$ и на ней взята точка $S$ так, что $PR=QS.$\\
а) Между какими целыми числами лежит длина отрезка $QS?$\\
б) Найдите углы треугольника $ROS,$ если $RO$ --- биссектриса треугольника $QRS.$\\
105. Треугольник $KLM$ --- прямоугольный $(\angle M=90^\circ),\ \angle KLM=60^\circ,\ LM=8,5.$ Через точку $L$ проведена прямая, параллельная прямой $KM$ и на ней взята точка $N$ так, что $KM=LN.$\\
а) Между какими целыми числами лежит длина отрезка $LN?$\\
б) Найдите углы треугольника $MON,$ если $MO$ --- биссектриса треугольника $LMN.$\\
106. $BM$ --- медиана в треугольнике $ABC,\ MD$ --- биссектриса угла $AMB,\ ME$ --- биссектриса угла $BMC.$ Найдите угол $DME.$\\
107. Биссектриса угла при вершине треугольника пересекает основание под углом $73^\circ,$ а биссектрису одного из углов при основании под углом $58^\circ.$ Найдите углы треугольника.\\
108. Биссектриса угла при вершине треугольника пересекает основание под углом $71^\circ,$ а биссектрису одного из углов при основании под углом $57^\circ.$ Найдите углы треугольника.\\
109. В $\Delta KHM\ KH=12,\ HM=9,\ MK=18.$ Через точку $A,$ лежащую на стороне $HM,$ проведён перпендикуляр к биссектрисе $\angle M,$ пересекающий сторону $KM$ в точке $C,$ и перпендикуляр к биссектрисе $\angle H,$ пересекающий сторону $KH$ в точке $B.$ В каком отношении точка $A$ делит сторону $HM,$ если $KC=2KB?$\\
110. В $\Delta KHM\ KH=12,\ HM=7,\ MK=17.$ Через точку $A,$ лежащую на стороне $HK,$ проведён перпендикуляр к биссектрисе $\angle K,$ пересекающий сторону $KM$ в точке $B,$ и перпендикуляр к биссектрисе $\angle H,$ пересекающий сторону $MH$ в точке $C.$ В каком отношении точка $A$ делит сторону $HK,$ если $MC=0,5MB?$\\
111. Четырёхугольник $ABCD$ называется дельтоидом, если в нём $AB=BC$ и $AD=DC.$ Докажите, что точка $O$ пересечения диагоналей дельтоида является серединой одной из его диагоналей. \\
112. Точки $A,\ B$ и $C$ лежат на одной прямой, причём точка $C$ расположена вдвое дальше от одной из точек $A$ и $B,$ чем от другой. Найдите $AB,$ если $AC=18.$\\
113. В треугольнике $ABC\ \angle A=40^\circ, \angle C=90^\circ.$ Определите угол между высотой и медианой, проведёнными из вершины угла $C.$\\
114. Постройте с помощью циркуля и линейки треугольник $ABC$ по его стороне $a,$ сумме сторон $b+c$ и высоте, проведённой к стороне $c.$\\
115. В $\Delta SPM:$ мера угла $P$ равна $90^\circ,$ мера угла $M$ равна $60^\circ,$ $ST$ --- биссектриса $\Delta SPM,\ |PT|=26,\ TF$ --- высота $\Delta TSM.$ Найдите $|TF|.$\\
116. В $\Delta MNK:\ |MN|=|NK|=|MK|,\ |MN|=13,\ P$ --- середина $|MK|,\ R\in[NK],\ (PR)\perp(NK).$ Найдите $|KR|.$\\
117. В треугольнике $ABC$ проведена биссектриса $AD,$ а в треугольнике $ADC$ --- биссектриса $DE.$ Оказалось, что $\angle ABD=43^\circ,$ а $DE=CD.$ Найдите $\angle BAC.$\\
118. В треугольнике $ABC$ проведена биссектриса $BE,$ а в треугольнике $BAE$ --- биссектриса $ED.$ Оказалось, что $\angle ECB=22^\circ,$ а $ED=AE.$ Найдите $\angle ABC.$\\
119. Точка $M$ --- середина стороны $AC$ треугольника $ABC.$ Точка $D$ на стороне $BC$ такова, что $\angle BMA=\angle DMC.$ Оказалось, что $CD+DM=BM.$ Докажите, что $\angle ACB+\angle ABM=\angle BAC.$\\
120. В выпуклом пятиугольнике $ABCDE$ известно, что $AE=AD,\ AC=AB$ и $\angle DAC=\angle AEB+\angle ABE.$ Докажите, что сторона $DC$ в два раза больше медианы $AK$ треугольника $ABE.$\\
121. Возможно ли, чтобы медианы острых углов прямоугольного треугольника были перпендикулярны? Приведите пример такого треугольника или докажите, что его не существует.\\
122. Периметр равнобедренного треугольника равен 4{\it дм.} Известно, что разность между длинами двух сторон равна 1{\it дм.} Найдите длины сторон этого треугольника. (В ответе следует указать длины трёх сторон с единицами длины).\\
123. Найдите периметр равнобедренного треугольника, в котором длины двух сторон равны 8{\it см} и 12{\it см.}\\
124. Периметр равнобедренного треугольника равен 25{\it см.} Известно, что разность между длинами двух сторон равна 5{\it см.} Найдите длины сторон этого треугольника. (В ответе следует указать длины трёх сторон с единицами длины).\\
125. Найдите периметр равнобедренного треугольника, в котором длины двух сторон равны 6{\it см} и 14{\it см.}\\
126. Точки $A,\ B,\ C$ расположены на одной прямой. Известно, что $AB=10$ и $AC=4.$ Найдите все возможные значения $BC.$\\
127. В треугольнике $ABC$ проведена биссектриса $BK.$ Известно, что $BK=KC,\ \angle AKB=80^\circ.$ Найдите $\angle BAC.$\\
128. Точки $C,\ D,\ K$ расположены на одной прямой. Известно, что $CD=8$ и $CK=3\cdot DK.$ Найдите все возможные значения $DK.$\\
129. В остроугольном треугольнике $KLM$ проведены высота $KA$ и биссектриса $LB.$ Известно, что они пересекаются в точке $H$ и $\angle AHB=100^\circ.$ Найдите $\angle KLM.$\\
130. В треугольнике $ABC\ \angle B=30^\circ,\ \angle C=60^\circ.$ Через середину стороны $BC$ проведена прямая $p,$ перпендикулярная $BC.$ Прямая $p$ пересекает сторону $AB$ в точке $L.$ Известно, что один из отрезков $BL$ и $AL$ больше другого на 4 {\it см.} Найдите длину отрезка прямой $p,$ находящегося внутри данного треугольника.\\
131. В равнобедренном треугольнике $ABC\ \angle A=30^\circ,\ \angle B=120^\circ,\ BC=8${\it см.} Проведены высота $AK$ данного треугольника и высота $KL$ треугольника $AKB.$ Найдите длину $BL.$\\
132. В треугольнике $KLM\ \angle L=30^\circ,\ \angle K=60^\circ.$ Через середину стороны $KL$ проведена прямая $p,$ перпендикулярная $KL.$ Прямая $p$ пересекает сторону $LM$ в точке $A$ и продолжение стороны $KM$ в точке $B.$ Известно, что $AB=3${\it см.} Найдите длину $AL.$\\
133. В равнобедренном треугольнике $KLM\ \angle K=30^\circ,\ \angle L=120^\circ,\ KL=6${\it см.} Проведены высота $MA$ данного треугольника и высота $AB$ треугольника $LAM.$ Найдите длину $BM.$\\
134. В равнобедренном треугольнике $ABC$ с вершиной $B$ на стороне $BC$ взята точка $K$ такая, что $CA=AK=KB.$ Периметр треугольника $CAK$ равен 4, периметр треугольника $AKB$ равен 5. Вычислите периметр данного треугольника.\\
135. В равнобедренном треугольнике $ABC$ с вершиной $A$ на стороне $AC$ взята точка $K$ такая, что $CB=BK=KA.$ Периметр треугольника $CBK$ равен 6, периметр треугольника $AKB$ равен 7. Вычислите периметр данного треугольника.\\
136. Верно ли, что треугольники $ABC$ и $MKP$ равны, если $AB=3,\ BC=4,\ \angle C=30^\circ;\ MK=3,\ KP=4,\ \angle P=30^\circ?$ Ответ обоснуйте.\\
137. Верно ли, что треугольники $ABC$ и $MKP$ равны, если $AB=5,\ BC=8,\ \angle C=30^\circ;\ MK=5,\ KP=8,\ \angle P=30^\circ?$ Ответ обоснуйте.\\
138. В равнобедренном треугольнике один из углов в два раза меньше другого. Найдите величину наименьшего угла этого треугольника.\\
139. Медиана $AM$ треугольника $ABC$ перпендикулярна его биссектрисе $BK.$ Найдите $AB,$ если $BC=12.$\\
140. В равнобедренном треугольнике один из углов в два раза больше другого. Найдите величину наименьшего угла этого треугольника.\\
141. Биссектриса $AP$ треугольника $ABC$ перпендикулярна его медиане $CK.$ Найдите $AB,$ если $AC=8.$\\
142. Длины сторон треугольника --- последовательные натуральные числа. Найдите периметр треугольника, если известно, что одна из его медиан перпендикулярна одной из его биссектрис.\\
143. Постройте с помощью циркуля и линейки треугольник по двум его углам и периметру.\\
144. Медиана разбивает треугольник на два треугольника. Докажите, что высоты в этих треугольниках, проведённые к этой медиане, равны.\\
145. Биссектриса $BE$ угла $B$ треугольника $ABC$ равна биссектрисе $BF$ его внешнего угла при той же вершине. Найдите разность двух других углов треугольника $ABC.$\\
146. Две стороны четырёхугольника равны 1 и 7. Одна из диагоналей, длина которой равна 3, делит его на два равнобедренных треугольника. Найдите все возможные значения периметра этого четырёхугольника.\\
147. В прямоугольном треугольнике на гипотенузе $AB$ взята точка $K$ такая, что $AK=CK=6.$ Найдите $BK.$\\
148. Биссектриса внешнего угла треугольника параллельна его стороне. К этой же стороне в треугольнике проведена медиана. Найдите угол между этими биссектрисой и медианой.\\
149. В треугольнике две высоты не меньше сторон, на которые они опущены. Найдите углы треугольника.\\
150. В равнобедренном треугольнике $ABC\ \angle B = 100^\circ.$ Определите угол между прямой, содержащей высоту $AA_1,$ и прямой, содержащей биссектрису $BB_1.$\\
151. В равнобедренном треугольнике $ABC\ \angle B = 130^\circ.$ Определите угол между прямой, содержащей высоту $AA_1,$ и прямой, содержащей биссектрису $BB_1.$\\
152. В треугольнике $ABC\ \angle B = 84^\circ,\ BH$ --- высота. Найдите $\angle BAC,$ если $AH=BC+CH.$\\
153. В треугольнике $ABC\ \angle B = 81^\circ,\ BH$ --- высота. Найдите $\angle BAC,$ если $AH=BC+CH.$\\
154. Известно, что в треугольнике один угол на $18^\circ$ больше другого, а также есть два угла, в сумме дающие $144^\circ.$ Какие могут быть углы у треугольника?\\
155. Известно, что в треугольнике один угол на $24^\circ$ больше другого, а также есть два угла, в сумме дающие $132^\circ.$ Какие могут быть углы у треугольника?\\
156. В прямоугольнике $ABCD$ на стороне $BC$ нашлась такая точка $X,$ что $\angle XD = 90^\circ$ и $\angle XAD =2\angle CXD.$ Найдите $BX : XC.$\\
157. В прямоугольнике $ABCD$ на стороне $BC$ нашлась такая точка $X,$ что $\angle XAD = \angle XDC = 60^\circ.$ Известно, что $XC = 30.$ Найдите $XM,$ где $M$ — середина $AD.$\\
158. Точка $M$ --- середина стороны $BC$ треугольника $ABC.$ На отрезке $AC$ нашлась такая точка $D,$ что $DM$ и $BC$ перпендикулярны. Отрезки $AM$ и $BD$ пересекаются в точке $X.$ Оказалось, что
$AC = 2BX.$ Докажите, что $X$ --- середина отрезка $AM.$\\
159. Дан равнобедренный треугольник $ABC.$ На продолжении боковой стороны $BC$ за вершину $B$ выбрана точка $D,\ M$ --- середина основания $AC.$
Отрезок $DM$ пересекает сторону $AB$ в точке $X.$ Оказалось, что $X$ --- середина $DM.$ Докажите, что $2AX = CD.$

ewpage
