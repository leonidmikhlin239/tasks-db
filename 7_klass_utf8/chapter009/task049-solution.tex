53. Если меньшей стороной является основание, то стороны треугольника равны $x,\ x+6,\ x+6.$ Возможны два случая: $x+x+6=16,\ 2x=10,\ x=5$см или $x+6+x+6=16,\ 2x=4,\ x=2$см. Тогда стороны равны 5см, 11см и 11см или 2см, 8см и 8см. Неравенство треугольника выполняется для обоих случаев. Если меньшей стороной является боковая сторона, то стороны треугольника равны $x,\ x-6,\ x-6.$ Возможны два случая: $x+x-6=16,\ 2x=22,\ x=11$см или $x-6+x-6=16,\ 2x=28,\ x=14$см. Тогда стороны равны 11см, 5см и 5см или 14см, 8см и 8см. Для первого из этих случаев не выполняется неравенство треугольника: $5+5<11.$ Таким образом, длина основания может быть равна 2см, 5см или 14см.\\
