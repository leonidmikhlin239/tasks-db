155. Раз два угла в сумме дают $132^\circ,$ оставшийся угол равен $180^\circ-132^\circ=48^\circ.$ Если есть угол, больший его на $24^\circ,$ то он равен $48^\circ+24^\circ=72^\circ,$ а третий угол равен $180^\circ-48^\circ-72^\circ=60^\circ.$ Если есть угол, меньший его на $24^\circ,$ то он равен $48^\circ-24^\circ=24^\circ,$ а третий угол равен
$180^\circ-48^\circ-24^\circ=108^\circ.$ Если из оставшихся углов один больше другого на $24^\circ,$ то пусть меньший угол равен $x^\circ,$ тогда
$x+x+24^\circ=132^\circ,\ x=54^\circ,\ x+24^\circ=78^\circ.$\\
