24. Пусть $x_1=2x_2,$ тогда $x_1+x_2=3x_2=k+5$ по теореме Виета. Значит, $x_2=\cfrac{k+5}{3}$ является корнем уравнения, подставим его:
$\left(\cfrac{k+5}{3}\right)^2-\cfrac{(k+5)^2}{3}+2k+6=0,\ \cfrac{k^2+10k+25}{9}-\cfrac{(k+5)^2}{3}+2k+6=0,\ k^2+10k+25-3k^2-30k-75+18k+54=0,\
2k^2+2k-4=0,\ k^2+k-2=0,\ (k+2)(k-1)=0,\ k=-2$ или $k=1.$ Рассмотренное нами условие являлось необходимым, но не достаточным, так что найденные значения $k$ необходимо проверить. При $k=-2:\ x^2-3x+2=0,$ корни $x=2$ и $x=1,$ значит $k=-2$ подходит. При $k=1:\ x^2-6x+8=0,$ корни $x=2$ и $x=4,$ значит $k=1$ также подходит.\\
