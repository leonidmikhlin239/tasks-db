\section{Нестандартные задачи}
1. {\it Логарифмом} числа $a>0$ по основанию 2 называется показатель степени, в которую нужно возвести 2, чтобы получить $a.$ Обозначение: $\log_2 a.$\\
Вычислить: а)$\log_2 8;$ б)$2^{\log_2 3};$ в) $4^{\log_2 5};$ г) Подумайте о том, что означает символ $\log_5 a$ и вычислите $2^{\log_5 3\cdot\log_25}.$\\
2. {\it Логарифмом} числа $a>0$ по основанию 3 называется показатель степени, в которую нужно возвести 3, чтобы получить $a.$ Обозначение: $\log_3 a.$\\
Вычислить: а)$\log_3 27;$ б)$3^{\log_3 2};$ в) $9^{\log_3 7};$ г) Подумайте о том, что означает символ $\log_5 a$ и вычислите $3^{\log_5 3\cdot\log_3 5}.$\\
3. Тринадцать различных натуральных чисел в сумме дают 92. Найдите эти числа.\\
4. Одиннадцать различных натуральных чисел в сумме дают 67. Найдите эти числа.\\
5. Три ученика Саша, Дима и Лёша прогуляли информатику. Когда их спросили, кому пришла в голову эта бессмысленная идея, они ответили следующее:\\
Саша: Это не я, это была идея Димы.\\
Дима: Это не я, во всём виноват Лёша.\\
Лёша: Это не я, это Дима.\\
Учитель почувствовал, что среди шести утверждений только половина правда. Кто из учеников оказался инициатором прогула?\\
6. Три ученика Саша, Дима и Лёша прогуляли информатику. Когда их спросили, кому пришла в голову эта бессмысленная идея, они ответили следующее:\\
Саша: Это не я, это была идея Димы.\\
Дима: Это не я, во всём виноват Лёша.\\
Лёша: Это не я, это Дима.\\
Учитель почувствовал, что двое учеников говорят правду только наполовину, а один лжёт. Кто из учеников оказался инициатором прогула?\\
7. Операция $*$ каждым двум числам $x,\ y$ ставит в соответствие число, обозначаемое $x*y.$ При этом для всех чисел $x,\ y,\ z$ выполняется: а) $x*x=0;$ б) $(x+y)*z=x+(y*z).$ Найти $6*14.$\\
8. Операция $*$ каждым двум числам $x,\ y$ ставит в соответствие число, обозначаемое $x*y.$ При этом для всех чисел $x,\ y,\ z$ выполняется: а) $x*x=0;$ б) $(x+y)*z=x+(y*z).$ Найти $8*12.$\\
9. Найдите наименьшее трёхзначное число, сумма цифр которого равна 22.\\
10. Найдите наименьшее трёхзначное число, сумма цифр которого равна 23.\\
11. Сколько нулей стоит в конце числа 100! ($n!$ --- произведение натуральных чисел от 1 до $n$)?\\
12. Сколько нулей стоит в конце числа 101! ($n!$ --- произведение натуральных чисел от 1 до $n$)?\\
13. Не возводя в куб, сравните: $0,123^3+0,124^3+0,125^3$ и 0,002856.\\
14. Не возводя в куб, сравните: $0,131^3+0,132^3+0,133^3$ и 0,002976.\\
15. Даны две параллельные прямые, на одной из которых отмечено 6 точек, а на другой 3 точки. Сколько существует различных треугольников с вершинами в этих точках?\\
16. Даны две параллельные прямые, на одной из которых отмечено 5 точек, а на другой 4 точки. Сколько существует различных треугольников с вершинами в этих точках?\\
17. Дано $(a+1)(b+1)=2ab.$ Найдите числовое значение выражения $\cfrac{(a^2-1)(b^2-1)}{ab}.$\\
18. Дано $(a-1)(b-1)=2ab.$ Найдите числовое значение выражения $\cfrac{(a^2-1)(b^2-1)}{ab}.$\\
19. При каком наименьшем натуральном значении $n$ все дроби $\cfrac{7}{n+9},\ \cfrac{8}{n+10},\ \ldots, \cfrac{31}{n+33}$ одновременно несократимы?\\
20. При каком наименьшем натуральном значении $n$ все дроби $\cfrac{6}{n+8},\ \cfrac{7}{n+9},\ \ldots, \cfrac{29}{n+31}$ одновременно несократимы?\\
21. Про функцию $f$ известно, что $f(a;b;c)+f(d;b;c)=f(a+d;b;c)+2bc.$ Кроме того, $f(a;b;c)=f(b;a;c)=f(c;b;a)$ и $f(1;3;5)=46.$ Найдите: а) $f(3;2;5);$
б) $f(2;6;10).$\\
22. Про функцию $f$ известно, что $f(a;b;c)+f(d;b;c)=f(a+d;b;c)+2bc.$ Кроме того, $f(a;b;c)=f(b;a;c)=f(c;b;a)$ и $f(2;3;5)=62.$ Найдите: а) $f(3;4;5);$
б) $f(4;6;10).$\\
23. Выясните, является ли простым число $2^{10}+5^{12}.$\\
24. Пятнадцать различных натуральных чисел дают в сумме 121. Найдите эти числа.\\
25. Для каких натуральных $n$ число $\sqrt{50-n^2}$ будет целым?\\
26. Для каких натуральных $n$ число $\sqrt{52-n^2}$ будет целым?\\
27. а) Сколько различных чётных пятизначных чисел можно составить из цифр 0,1,2,3,4,5, используя каждую цифру только один раз?\\
б) Какой будет результат, если цифры могут повторяться?\\
28. а) Сколько различных пятизначных чисел, кратных 5, можно составить из цифр 0,1,2,3,4,5, используя каждую цифру только один раз?\\
б) Какой будет результат, если цифры могут повторяться?\\
29. а) Привести пример трёхзначного числа, которое в 37 раз больше суммы своих цифр.\\
б) Найти все такие числа.\\
30. Решите уравнение $(x^2+1)^4+(x-2)^2=1.$\\
31. При делении двузначного числа на сумму его цифр получили частное 7 и остаток 3, при делении этого же числа на число, записанное теми же цифрами в обратном порядке, получили частное 1 и остаток 36. Найдите это число.\\
32. При делении двузначного числа на сумму его цифр получили частное 6 и остаток 11, при делении числа, записанного теми же цифрами в обратном порядке, на сумму его цифр получили частное 4 и остаток 3. Найдите это число.\\
33. Найдите пары натуральных чисел $x$ и $y,$ не превосходящих 30, для которых число $3x^2+5xy-2y^2$ является простым.\\
34. Из множества двузначных натуральных чисел, в которых цифра десятков нечётная, случайным образом выбирают одно число. Найдите вероятность того, что сумма цифр этого числа будет равна 11. Ответ запишите в виде десятичной дроби.\\
35. Из множества двузначных натуральных чисел, в которых цифра десятков чётная, случайным образом выбирают одно число. Найдите вероятность того, что сумма цифр этого числа будет равна 10. Ответ запишите в виде десятичной дроби.\\
36. Имеется четыре полные цистерны с растворами некоторого вещества. Известно, что концентрация вещества в первой цистерне больше, чем в третьей, а во второй --- больше, чем в четвёртой. Эти растворы некоторое время использовались, и затем первые две цистерны слили в один большой сосуд, а третью и четвёртую --- в другой. Верно ли, что концентрация вещества в первом сосуде обязательно больше, чем во втором?\\
37. Разделив двузначное число на сумму его цифр, получили в частном 7 и в остатке 3; разделив это же число на число, записанное теми же цифрами в обратном порядке, получили в частном 1, а в остатке 36. Найдите исходное число.\\
38. В комнате собрались несколько гномов, которые всегда лгут. Все они разного роста и разного веса. Каждый из них сказал: <<Все остальные легче меня, и кто-то из них ниже меня>>. Какое из утверждений А-Г обязательно верно?\\
(А) Самый тяжёлый гном --- самый низкий.\\
(Б) Самый лёгкий гном --- самый низкий.\\
(В) Самый тяжёлый гном --- самый высокий.\\
(Г) Самый лёгкий гном --- самый высокий.\\
(Д) Ни одно из утверждений А-Г не обязано выполняться.\\
39. Найдите наибольшее трёхзначное число, имеющее ровно пять натуральных делителей.\\
40. а) Сколько существует двузначных чисел, все цифры которых нечётные и могут повторяться?\\
б) Сколько существует трёхзначных чисел, все цифры которых чётные и не повторяются?\\
41. а) Сколько существует двузначных чисел, все цифры которых нечётные и не повторяются\\
б) Сколько существует трёхзначных чисел, все цифры которых чётные и могут повторяться?\\
42. Игральный кубик бросают дважды. Найдите вероятность того, что сумма выпавших очков равна 6.\\
43. Игральный кубик бросают дважды. Найдите вероятность того, что сумма выпавших очков окажется равна 5.\\
44. В коробке лежит 30 белых и чёрных шаров. Определите, сколько белых и сколько чёрных шаров в коробке, если среди любых 12 шаров хотя бы 1 белый, а среди любых 20 шаров хотя бы 1 чёрный.\\
45. Для любой пары чисел определена некоторая операция $\text{«$\ast$»,}$ удовлетворяющая следующим свойствам:
$a*(b*c)=(a*b)*c$ и $a*a=1,$ где операция $\text{«$\cdot$»}$ --- операция умножения. Найдите все корни $x$ уравнения
$x*3=2024.$ (Существование такой операции считать известным, доказывать его не надо)
\newpage
