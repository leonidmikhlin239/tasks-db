36. Нет, неверно. Пусть концентрация вещества в первой цистерне равна $20\%,$ во второй $90\%,$ в третьей $10\%,$ а в четвёртой --- $80\%.$ Пусть после использования в первой и четвёртой цистерне осталось по 5 т раствора, а во второй и третьей --- по 1 т. Тогда и в первом, и во втором сосуде будет по $1+5=6$т
раствора, при этом в первом сосуде будет $5\cdot0,2+1\cdot0,9=1,9$т вещества, а во втором --- $1\cdot0,1+5\cdot0,8=4,1$т вещества, поэтому концентрация в нём будет больше.\\
