91. $(x-a)(x-10)+1=x^2-10x-ax+10a+1=x^2-(a+10)x+10a+1.$ У этого квадратного уравнения должны быть целые корни $-b$ и $-c,$ а значит его дискриминант является точным квадратом: $(a+10)^2-4(10a+1)=d^2,\ a^2+20a+100-40a-4-d^2=0,\
a^2-20a+100-d^2=4,\ (a-10)^2-d^2=4,\ (a-10-d)(a-10+d)=4.$ Заметим, что у чисел $a-10-d$ и $a-10+d$ одинаковая чётность, поэтому возможны два случая: $\begin{cases} a-10-d=2,\\ a-10+d=2. \end{cases}\Leftrightarrow \begin{cases} 2a-20=4,\\ a-10+d=2. \end{cases}\Leftrightarrow \begin{cases} a=12,\\ d=0.\end{cases}$ или
$\begin{cases} a-10-d=-2,\\ a-10+d=-2. \end{cases}\Leftrightarrow \begin{cases} 2a-20=-4,\\ a-10+d=-2. \end{cases}\Leftrightarrow \begin{cases} a=8,\\ d=0.\end{cases}$
Найденные значения $a$ надо подставить в исходное выражение и убедиться, что числа $b$ и $c$ получатся не только рациональными, но и целыми:
$(x-12)(x-10)+1=x^2-22x+120+1=(x-11)(x-11),\ (x-8)(x-10)+1=x^2-18x+80+1=(x-9)(x-9).$ Значит, оба полученных значения $a$ подходят.\\
