38. Агата добиралась от дома до института на своём автомобиле с постоянной скоростью 100 км/ч. Обратно она ехала с постоянной скоростью, которая измерялась целым числом километров в час, причём путь до дома занял у неё больше времени, чем путь до института.\\
а) Могла ли её средняя скорость за эти две поездки составить 90 км/ч?\\
б) Могла ли её средняя скорость за эти две поездки оказаться равной целому числу километров в час?\\
