\section{Прогрессии задачи}
1. Туристы поднимаются на перевал высотой 4800 м. В первый день они поднялись на высоту 1800 метров, а в каждый последующий день набирали высоты на 200 м меньше, чем в предыдущий. Через сколько дней туристы достигнут перевала?\\
2. Сплавляясь по реке в первый день, геологи проплыли 40 км, а в каждый последующий день проплывали на 5 км меньше, чем в предыдущий. Сколько дней сплавлялись геологи, если длина реки равна 130 км?\\
3. Найти сумму всех натуральных чисел, не превосходящих 165, которые при делении на 7 дают в остатке 5.\\
4. Найти сумму всех натуральных чисел, не превосходящих 300, которые при делении на 11 дают в остатке 3.\\
5. Найти сумму всех трёхзначных чисел, не кратных 5.\\
6. Найти сумму всех двузначных чисел, не кратных 3.\\
7. В геометрической прогрессии с положительными числами $b_3=32;\ b_7=2.$ Найти $b_5.$\\
8. В геометрической прогрессии с положительными числами $b_2=16;\ b_{10}=4.$ Найти $b_6.$\\
9. При каких $a$ числа $a^2;\ 4a;\ 2a+5$ являются тремя последовательными членами арифметической прогрессии?\\
10. При каких $a$ числа $a^2;\ 3a;\ a+4$ являются тремя последовательными членами арифметической прогрессии?\\
11. Третий член арифметической прогрессии равен 10, а восьмой равен 30. Сколько членов прогрессии нужно взять, чтобы их сумма равнялась 242?\\
12. Третий член арифметической прогрессии равен 21, а девятый равен 51. Сколько членов прогрессии нужно взять, чтобы их сумма равнялась 396?\\
13. Найдите сумму всех трёхзначных чисел, не делящихся на 7 и имеющих последней цифру 3.\\
14. Найдите сумму всех трёхзначных чисел, не делящихся на 11 и имеющих последней цифру 5.\\
15. Найдите сумму первых восьми членов геометрической прогрессии, второй член который равен 6, а четвёртый 24.\\
16. Найдите сумму первых шести членов геометрической прогрессии, третий член который равен 54, а пятый 6.\\
17. Сумма седьмого и тринадцатого членов арифметической прогрессии равна 16. Найдите её десятый член.\\
18. Сумма пятого и семнадцатого членов арифметической прогрессии равна 20. Найдите её одиннадцатый член.\\
19. Найдите сумму первых шести членов арифметической прогрессии, шестой член которой равен $\cfrac{3}{4},$ а десятый $\cfrac{7}{4}.$\\
20. Найдите сумму первых десяти членов арифметической прогрессии, третий член которой равен $(-1),$ а пятый $3.$\\
21. Найдите сумму первых шести членов арифметической прогрессии, первый член которой равен $1,2,$ а четвёртый $1,8.$\\
22. Найдите сумму первых десяти членов арифметической прогрессии, второй член которой равен $(-5),$ а разность шестого и четвёртого $6.$\\
23. Бригада маляров красит забор длиной 240 м, ежедневно увеличивая норму покраски на одно и то же число метров. Известно, что за первый и последний день в сумме бригада покрасила 60 м забора. Определите, сколько дней бригада красила весь забор.\\
24. Улитка ползёт от одного дерева до другого. Каждый день она проползает на одно и то же расстояние больше, чем в предыдущий день. Известно, что за первый и последний день улитка проползла в общей сложности 10 метров. Определите, сколько дней улитка потратила на весь путь, если расстояние между деревьями 150 метров.\\
25. Три числа составляют геометрическую прогрессию. Если первые два из них оставить без изменений, а из последнего вычесть первое, то полученные три числа составят арифметическую прогрессию. Найдите разность арифметической прогрессии, если второе из взятых чисел равно 6.\\
26. Три числа составляют арифметическую прогрессию. Если первые два числа оставить без изменения, а к третьему прибавить сумму двух первых, то полученные числа составят геометрическую прогрессию. Найдите знаменатель этой геометрической прогрессии.\\
27. Сумма первых десяти членов арифметической прогрессии равна 30; четвёртый, седьмой и пятый члены этой прогрессии в указанном порядке составляют геометрическую прогрессию. Найдите разность арифметической прогрессии, если известно, что все её члены различны.\\
28. Сумма первых тринадцати членов арифметической прогрессии равна 130; четвёртый, десятый и седьмой члены этой прогрессии в указанном порядке составляют геометрическую прогрессию. Найдите разность арифметической прогрессии, если известно, что все её члены различны.\\
29. Последовательность $(a_n)$ --- арифметическая прогрессия. Известно, что: $a_5+a_9=40.$ Найдите $a_3+a_7+a_{11}.$\\
30. Последовательность $(a_n)$ --- арифметическая прогрессия. Известно, что: $a_4+a_6=38.$ Найдите $a_2+a_5+a_{8}.$\\
31. Числа $a_1,\ a_2,\ldots,a_{21}$ образуют арифметическую прогрессию. Известно, что сумма членов этой прогрессии с нечётными номерами на 15 больше суммы членов с чётными номерами. Найдите $a_{12},$ если $a_{20}=3a_9.$\\
32. Числа $a_1,\ a_2,\ldots,a_{19}$ образуют арифметическую прогрессию. Известно, что удвоенная сумма членов этой прогрессии с чётными номерами на 10 больше суммы всех членов. Найдите $a_{13},$ если $a_{3}=2a_4.$\\
33. Второй член арифметической прогрессии составляет $88\%$ от первого. Найдите сколько процентов первый член составляет от пятого.\\
34. Сумма первых трёх членов убывающей $(|q|<1)$ геометрической прогрессии равна 21, а сумма их квадратов --- 189. Найдите первый член и знаменатель данной прогрессии.\\
35. Могут ли длины сторон прямоугольного треугольника образовывать геометрическую прогрессию?\\
36. Найдите сумму всех трёхзначных чисел, не делящихся на 11.\\
37. Найдите сумму всех трёхзначных чисел, кратных 7, но не кратных 5.\\
38. Найдите сумму всех трёхзначных чисел, кратных 5, но не кратных 7.\\
39. Найти сумму всех чётных двузначных чисел, делящихся на 3.\\
40. Найти сумму всех двузначных чисел, делящихся на 7.\\
41. В арифметической прогрессии $(a_n)$ $a_1=4.$ При каком значении разности прогрессии произведение $a_5$ и $a_7$ будет наименьшим?\\
42. Найти сумму всех трёхзначных чисел, которые при делении на 11 дают остаток 5.\\
43. Найти сумму всех трёхзначных чисел, которые при делении на 13 дают остаток 7.\\
44. Различные числа $a,\ b,\ c$ являются последовательными членами некоторой геометрической прогрессии. Эти же числа в том же порядке можно рассматривать как первый, второй и четвёртый члены некоторой арифметической прогрессии. Найдите эти числа, если их сумма равна 35.\\
45. Различные числа $a,\ b,\ c$ являются соответственно первым, вторым и шестым членами некоторой арифметической прогрессии. Эти же числа в том же порядке являются последовательными членами некоторой геометрической прогрессии. Найдите эти числа, если $a+c-b=13.$\\
46. Числа $a_1,\ a_2,\ldots$ образуют арифметическую прогрессию. Известно, что $a_{17}+a_{23}=400,\ a_{20}+a_{108}=224.$ Найдите $a_{1}+a_2+\ldots+a_{239}.$\\
47. Произведение седьмого и восьмого членов непостоянной арифметической прогрессии равно произведению пятого и девятого её членов. Найдите одиннадцатый член данной прогрессии.\\
48. Цифры трёхзначного числа образуют арифметическую прогрессию. Если из этого числа вычесть 792, то получится число, записанное теми же цифрами, но в обратном порядке. Если же из цифры десятков вычесть 2, а остальные цифры оставить без изменения, то получится число, цифры которого образуют геометрическую прогрессию. Найдите исходное число.\\
49. Цифры трёхзначного числа образуют геометрическую прогрессию. Если из этого числа вычесть 792, то получится число, записанное теми же цифрами, но в обратном порядке. Если же из цифры сотен вычесть 4, а остальные цифры оставить без изменения, то получится число, цифры которого образуют арифметическую прогрессию. Найдите исходное число.

ewpage
