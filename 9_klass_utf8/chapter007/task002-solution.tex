2. Сплавы геологов представляют из себя арифметическую прогрессию с $a_1=40,\ d=-5,\ S_n=130.$ Тогда $130=40n+\cfrac{n(n-1)}{2}\cdot(-5),\
5n^2-85n+260=0,\ n=4$ (второй корень $n=13$ говорит о том, что геологи, продолжив сплавляться с той же закономерностью, начнут двигаться в обратном направлении и окажутся в конце реки второй раз). Значит, геологи сплавятся за четыре дня.\\
