10. Если они являются последовательными членами арифметической прогрессии в порядке\\ $a^2;\ 3a;\ a+4,$ то $2\cdot3a=a^2+a+4,\ a^2-5a+4=0,\ a=4$ или $a=1.$ Если средним членом является $a^2,$ то $2a^2=3a+a+4,\ 2a^2-4a-4=0,\ a=1\pm\sqrt{3}.$ Если же средним членом является $a+4,$ то $2(a+4)=a^2+3a,\ a^2+a-8=0,\ a=\cfrac{-1\pm\sqrt{33}}{2}.$\\
