9. Если они являются последовательными членами арифметической прогрессии в порядке\\ $a^2;\ 4a;\ 2a+5,$ то $2\cdot4a=a^2+2a+5,\ a^2-6a+5=0,\ a=5$ или $a=1.$ Если средним членом является $a^2,$ то $2a^2=4a+2a+5,\ 2a^2-6a-5=0,\ a=\cfrac{3\pm\sqrt{19}}{2}.$ Если же средним членом является $2a+5,$ то $2(2a+5)=a^2+4a,\ a^2=10,\ a=\pm\sqrt{10}.$\\