25. Пусть первое число равно $x,$ тогда знаменатель геометрической прогрессии равен $\cfrac{6}{x}$ и третье число равно $\cfrac{36}{x}.$ Значит, арифметическую прогрессию составляют числа $x,\ 6,\ \cfrac{36}{x}-x.$ Если средний член равен 6, то $2\cdot6=x+\cfrac{36}{x}-x,\ x=3$ и разность арифметической прогрессии равна $6-3=3.$ Если средний член равен $x,$ то $2x=6+\cfrac{36}{x}-x,\ 3x^2-6x-36=0,\ x^2-2x-12=0,\ x=1\pm\sqrt{13}.$ Тогда разность арифметической прогрессии равна либо $6-(1+\sqrt{13})=5-\sqrt{13},$ либо $6-(1-\sqrt{13})=5+\sqrt{13}.$ Если средний член равен $\cfrac{36}{x}-x,$ то $2\cdot\left(\cfrac{36}{x}-x\right)=x+6,\ 3x^2+6x-72=0,\ x^2+2x-24=0,\ x=4$ или $x=-6.$ Тогда разность арифметической прогрессии равна либо $5-4=1,$ либо $0-(-6)=6.$\\
