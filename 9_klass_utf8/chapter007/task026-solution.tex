26. Пусть первое число равно $x,$ а разность арифметической прогрессии равна $d,$ тогда изначально числа были равны $x,\ x+d,\ x+2d.$ Полученные числа равны $x,\ x+d,\ 3x+3d=3(x+d).$ Если числа $x+d$ и $3(x+d)$ являются последовательными членами этой геометрической прогрессии, то её знаменатель может быть равен 3 или $\cfrac{1}{3}$ в зависимости от того, в каком порядке они расположены. Если же они не последовательные члены, то между ними расположен $x$ и $x^2=(x+d)\cdot3(x+d),\
x^2=3x^2+6xd+3d^2,\ 2x^2+6xd+3d^2=0,\ 2\left(\cfrac{x}{d}\right)^2+6\left(\cfrac{x}{d}\right)+3=0,\ \cfrac{x}{d}=\cfrac{-3\pm\sqrt{3}}{2}.$ В первом случае знаменатель геометрической прогрессии может быть равен $\cfrac{x+d}{x}=1+\cfrac{d}{x}=1+\cfrac{2}{-3+\sqrt{3}}=\cfrac{\sqrt{3}-1}{\sqrt{3}(1-\sqrt{3})}=-\cfrac{1}{\sqrt{3}}$ или $-\sqrt{3},$ если числа расположены в обратном порядке. Во втором случае $\cfrac{x+d}{x}=1+\cfrac{d}{x}=1+\cfrac{2}{-3-\sqrt{3}}=\cfrac{-\sqrt{3}-1}{-3-\sqrt{3}}=\cfrac{1}{\sqrt{3}}$ или
$\sqrt{3},$ если числа расположены в обратном порядке.
Таким образом, знаменатель геометрической прогрессии может быть равен $3,\ \cfrac{1}{3},\ -\sqrt{3},\ -\cfrac{1}{\sqrt{3}},\ \sqrt{3},\ \cfrac{1}{\sqrt{3}}.$\\