34. Составим систему уравнений: $\begin{cases} b_1\cdot\cfrac{q^3-1}{q-1}=21,\\ b_1^2\cdot\cfrac{q^6-1}{q^2-1}=189.\end{cases}\Leftrightarrow
\begin{cases} b_1(q^2+q+1)=21,\\ b_1^2(q^4+q^2+1)=189.\end{cases}\Leftrightarrow$\\$
\begin{cases} b_1^2(q^4+2q^3+3q^2+2q+1)=441,\\ b_1^2(q^4+q^2+1)=189.\end{cases}\Rightarrow
\cfrac{q^4+2q^3+3q^2+2q+1}{q^4+q^2+1}=\cfrac{441}{189}=\cfrac{7}{3}\Leftrightarrow
3q^4+6q^3+9q^2+6q+3=7q^4+7q^2+7\Leftrightarrow
4q^4-6q^3-2q^2-6q+4=0\Leftrightarrow 2q^4-3q^3-q^2-3q+2=0\Big|:q^2 \Leftrightarrow
2q^2-3q-1-\cfrac{3}{q}+\cfrac{2}{q^2}=0 \Leftrightarrow 2\left(q^2+\cfrac{1}{q^2}\right)-3\left(q+\cfrac{1}{q}\right)-1=0.$
Сделаем замену $q+\cfrac{1}{q}=t,$ тогда $t^2=q^2+2+\cfrac{1}{q^2}$ и $2(t^2-2)-3t-1=0,\ 2t^2-3t-5=0,\ t=\cfrac{5}{2}$ или $t=-1.$ В первом случае $q+\cfrac{1}{q}=
\cfrac{5}{2},\ 2q^2-5q+2=0,\ q=\cfrac{1}{2}$ или $q=2.$ Так как прогрессия убывающая, подходит только $q=\cfrac{1}{2},$ тогда $b_1=\cfrac{21}{\cfrac{1}{4}+\cfrac{1}{2}+1}=12.$  Во втором случае $q+\cfrac{1}{q}=-1,\ q^2+q+1=0,\ q\in\varnothing.$\\
