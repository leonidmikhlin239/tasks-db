46. Пусть объём фляги равен $x$л, тогда в ней $0,96x$ кислоты. После первого долива в фляге останется $0,96x-12\cdot0,96=0,96x-11,52$л кислоты и её концентрация будет равна $\cfrac{0,96x-11,52}{x}.$ После второго долива кислоты в фляге будет $0,96x-11,52-\cfrac{0,96x-11,52}{x}\cdot18=0,32x,$ значит $0,96x^2-11,52x-17,28x+207,36=0,32x^2,\ 0,64x^2-28,8x+207,36=0,\ x^2-45x+324=0,\ (x-9)(x-36)=0,\ x=36$л (второй корень не подходит, так как $x>18$л).\\
