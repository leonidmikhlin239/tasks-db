55. Пусть в первом сосуде было $x$ мл раствора, а во втором --- $y,$ тогда имеем систему уравнений $\begin{cases}\cfrac{70}{x}-\cfrac{60}{y}=0,2,\\
x+y=600.\end{cases}\Leftrightarrow\begin{cases}\cfrac{70}{x}-\cfrac{60}{600-x}=0,2,\\
y=600-x.\end{cases}\Leftrightarrow\begin{cases}\cfrac{42000-70x-60x}{x(600-x)}=\cfrac{1}{5},\\
y=600-x.\end{cases}\Leftrightarrow$\\$\begin{cases}210000-650x=600x-x^2,\\
y=600-x.\end{cases}\Leftrightarrow\begin{cases}x^2-1250x+210000=0,\\
y=600-x.\end{cases}\Leftrightarrow\begin{cases}(x-200)(x-1050)=0,\\
y=600-x.\end{cases}\Leftrightarrow\begin{cases}x=200,\\
y=400.\end{cases}$ Значит, концентрация кислоты в первом растворе равна $70:200\cdot100\%=35\%,$ а во втором --- $60:400\cdot100\%=15\%.$\\
