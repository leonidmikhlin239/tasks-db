9. Пусть весь путь равен $2S$км, а скорость первого автомобиля равна $x$км/ч, тогда имеем уравнение $\cfrac{2S}{x}=\cfrac{S}{x-13}+\cfrac{S}{78},\
156(x-13)=78x+x(x-13),\ x^2-91x+2028=0,\ x=39$ или $x=52.$ Так как $x>48,$ получаем, что скорость первого автомобиля была равна 52 км/ч.\\
