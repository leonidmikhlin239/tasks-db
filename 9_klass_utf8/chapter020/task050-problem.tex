50. Пусть скорость автомобиля равна $x\text{ км/ч}.$ Первый велосипедист за час проедет $24\cdot1=24$ километра, а второй --- $18\cdot1=18$ километров. Тогда $\cfrac{24}{x-24}-\cfrac{18}{x-18}=\cfrac{1}{6}\Leftrightarrow\cfrac{24x-24\cdot18-18x+24\cdot18}{x^2-18x-24x+432}=\cfrac{1}{6}\Leftrightarrow 36x=x^2-42x+432\Leftrightarrow
x^2-78x+432=0\Leftrightarrow x=72\text{ км/ч}$ (второй корень равен $6\text{ км/ч},$ но с такой скоростью автомобиль никого бы не догнал).\\
