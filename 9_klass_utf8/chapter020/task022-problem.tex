22. Когда мальчик бежал вверх, он насчитал в $70:30=\cfrac{7}{3}$ раза больше ступенек, а значит потратил в $\cfrac{7}{3}$ раза больше времени. При этом навстречу ему выползло в $\cfrac{7}{3}$ раза больше ступенек, чем <<убежало>> при спуске. Если мальчик сбежит вниз $\cfrac{7}{3}$ раза, то он насчитает 70 ступенек, а убегут от него столько же ступенек, сколько выпозло при подъёме. Поэтому $70+70=140$ --- это количество ступенек, умноженное на $\left(\cfrac{7}{3}+1
ight)=\cfrac{10}{3},$ эскалатора, то есть всего их $140:\cfrac{10}{3}=42.$\\
