32. Скорость плота равна скорости течения, то есть 50м/мин$=$3000м/ч$=$3км/ч. Тогда 9 км он пройдёт за $7:3=\cfrac{7}{3}$ часа. Лодка плыла на 44 минуты меньше, то есть $\cfrac{7}{3}-\cfrac{11}{15}=\cfrac{8}{5}$ч. Пусть скорость лодки в неподвижной воде равна $x$км/ч, тогда $\cfrac{14}{x+3}+\cfrac{14}{x-3}=\cfrac{8}{5},\
14\left(\cfrac{1}{x+3}+\cfrac{1}{x-3}
ight)=\cfrac{8}{5},\ \cfrac{14\cdot2x}{x^2-9}=\cfrac{8}{5},\ 140x=8x^2-72,\ 2x^2-35x-18=0,\ x=18$км/ч.\\
