13. Пусть изначальная скорость велосипедиста равна $x$(км/ч), тогда $\cfrac{70}{x}=\cfrac{70}{x+3}+3,$\\$
70\left(\cfrac{1}{x}-\cfrac{1}{x+3}
ight)=3,\ \cfrac{3\cdot70}{x^2+3x}=3,\ x^2+3x-70=0,\ x=7.$ Значит, скорость велосипедиста на пути из $B$ в $A$ была равна $7+3=10$км/ч.\\
