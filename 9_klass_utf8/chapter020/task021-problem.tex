21. Когда мальчик бежал вверх, он насчитал в $60:20=3$ раза больше ступенек, а значит потратил в 3 раза больше времени. При этом навстречу ему выползло в 3 раза больше ступенек, чем <<убежало>> при спуске. Если мальчик сбежит вниз 3 раза, то он насчитает 60 ступенек, а убегут от него столько же ступенек, сколько выпозло при подъёме. Поэтому $60+60=120$ --- это учетверённое количество ступенек эскалатора, то есть всего их $120:4=30.$\\
