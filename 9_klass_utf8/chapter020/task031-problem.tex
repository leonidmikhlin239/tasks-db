31. Скорость плота равна скорости течения, то есть 50м/мин$=$3000м/ч$=$3км/ч. Тогда 9 км он пройдёт за $9:3=3$ часа. Лодка плыла на 30 минут меньше, то есть $\cfrac{5}{2}$ч. Пусть скорость лодки в неподвижной воде равна $x$км/ч, тогда $\cfrac{18}{x+3}+\cfrac{18}{x-3}=\cfrac{5}{2},\
18\left(\cfrac{1}{x+3}+\cfrac{1}{x-3}
ight)=\cfrac{5}{2},\ \cfrac{18\cdot2x}{x^2-9}=\cfrac{5}{2},\ 72x=5x^2-45,\ 5x^2-72x-45=0,\ x=15$км/ч.\\
