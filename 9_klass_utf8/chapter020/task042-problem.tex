42. Пусть скорость Васи равна $x$м/мин, а скорость Пети --- $y$м/мин. Тогда расстояние от места встречи до пункта Б равно $11x,$ а до пункта А --- $9y.$ При этом Петя потратил на то, чтобы добраться до места встречи, на 2 минуты меньше, то есть $\cfrac{11x}{y}+2=\cfrac{9y}{x}.$ Пусть $\cfrac{x}{y}=t,$ тогда $11t+2=\cfrac{9}{t},\ 11t^2+2t=9,\ 11t^2+2t-9=0,\ t=\cfrac{9}{11}.$ Значит, $11x=9y$ и весь путь равен $11x+9y=18y$ или $11x+9y=22x,$ поэтому Вася находился в пути $22x:x=22$ минуты, а Петя $18y:y=18$ минут.\\
