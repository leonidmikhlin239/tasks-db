74. Пусть боковая сторона равна $x,$ тогда по теореме косинусов имеем равенство $x^2+x^2-2x\cdot x\cdot \cos\left(\cfrac{\pi}{6}
ight)=1^2,$ откуда $x^2(2-\sqrt{3})=1,\ x^2=\cfrac{1}{2-\sqrt{3}}=\cfrac{2+\sqrt{3}}{4-3}=2+\sqrt{3}.$ Тогда сумма квадратов всех сторон равна $2x^2+1=4+2\sqrt{3}+1=5+2\sqrt{3}.$\\
