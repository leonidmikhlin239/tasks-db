59. По теореме Пифагора половина основания равна $\sqrt{13^2-5^2}=12,$ тогда основание равно $12\cdot2=24.$ Тогда площадь этого треугольника равна $S=\cfrac{1}{2}\cdot5\cdot24=60.$ С другой стороны она равна $rp,$ где $p=\cfrac{13+13+24}{2}=25.$ Поэтому радиус вписанной окружности равен $60:25=2,4.$ Также площадь треугольника можно найти по формуле $S=\cfrac{abc}{4R}=\cfrac{13\cdot13\cdot24}{4R}=\cfrac{169\cdot6}{R}.$ Поэтому радиус описанной окружности равен $\cfrac{169\cdot6}{60}=16,9.$
ewpage
oindent
