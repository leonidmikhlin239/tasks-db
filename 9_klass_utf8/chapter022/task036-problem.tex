36. Пусть основание равно $3x,$ а боковая сторона --- $4x.$ Найдём площадь треугольника разными способами. Выразим $p=\cfrac{4x+4x+3x}{2}=\cfrac{11}{2}x,$ тогда по формуле Герона она равна \\$S=\sqrt{\cfrac{11}{2}x\cdot\cfrac{3}{2}x\cdot \cfrac{3}{2}x\cdot \cfrac{5}{2}x}=\cfrac{3\sqrt{55}}{4}x^2.$ С другой стороны, $S=rp=\cfrac{11}{2}xr.$ Тогда $\cfrac{11}{2}xr=\cfrac{3\sqrt{55}}{4}x^2,\ r=\cfrac{3\sqrt{55}}{22}x.$ Также площадь можно найти как половину произведения высоты на сторону и возможно два случая: высота проведена к основанию или к боковой стороне. В первом случае $\cfrac{1}{2}\cdot10\cdot3x=\cfrac{3\sqrt{55}}{4}x^2,\ x=\cfrac{20}{\sqrt{55}},$ тогда $r=\cfrac{3\sqrt{55}}{22}\cdot\cfrac{20}{\sqrt{55}}=\cfrac{30}{11}.$ Во втором случае
$\cfrac{1}{2}\cdot10\cdot4x=\cfrac{3\sqrt{55}}{4}x^2,\ x=\cfrac{80}{3\sqrt{55}},$ тогда $r=\cfrac{3\sqrt{55}}{22}\cdot\cfrac{80}{3\sqrt{55}}=\cfrac{40}{11}.$\\
