35. Пусть основание равно $4x,$ а боковая сторона --- $3x.$ Найдём площадь треугольника разными способами. Выразим $p=\cfrac{3x+3x+4x}{2}=5x,$ тогда по формуле Герона она равна \\$S=\sqrt{5x\cdot2x\cdot 2x\cdot x}=2\sqrt{5}x^2.$ С другой стороны, $S=rp=5xr.$ Тогда $5xr=2\sqrt{5}x^2,\ r=\cfrac{2\sqrt{5}}{5}x.$ Также площадь можно найти как половину произведения высоты на сторону и возможно два случая: высота проведена к основанию или к боковой стороне. В первом случае $\cfrac{1}{2}\cdot20\cdot4x=2\sqrt{5}x^2,\ x=\cfrac{20}{\sqrt{5}},$ тогда $r=\cfrac{2\sqrt{5}}{5}\cdot\cfrac{20}{\sqrt{5}}=8.$ Во втором случае
$\cfrac{1}{2}\cdot20\cdot3x=2\sqrt{5}x^2,\ x=\cfrac{15}{\sqrt{5}},$ тогда $r=\cfrac{2\sqrt{5}}{5}\cdot\cfrac{15}{\sqrt{5}}=6.$\\
