149. Если катеты треугольника равны $x,$ то по теореме Пифагора $x^2+x^2=8,\ x=2.$ Тогда площадь треугольника равна $S=\cfrac{2\cdot2}{2}=2.$ Если прямая делит треугольник на две части, то пусть площадь одной из них равна $t,$ а другой --- $2-t.$ Тогда необходимо найти наибольшее значение выражения $t(2-t)=2t-t^2.$ Это парабола с ветвями, направленными вниз, её наибольшее значение достигается в вершине $t=-\cfrac{2}{-2}=1$ и равно $2-1=1.$\\
