116. По теореме Пифагора второй катет равен $\sqrt{50^2-30^2}=40$см. Площадь этого треугольника равна $S=\cfrac{30\cdot40}{2}=600\text{ см}^2.$ Если проведённая к гипотенузе высота равна $h,$ то $S=\cfrac{1}{2}h\cdot50=600,\ h=24$см. Если радиус вписанной окружности равен $r,$ то $S=pr=\cfrac{30+40+50}{2}r=60r=600,\ r=10$см.\\