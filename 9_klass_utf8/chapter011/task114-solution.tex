115. По теореме Пифагора гипотенуза равна $\sqrt{15^2+20^2}=25$см. Площадь этого треугольника равна $S=\cfrac{15\cdot20}{2}=150\text{ см}^2.$ Если проведённая к гипотенузе высота равна $h,$ то $S=\cfrac{1}{2}h\cdot25=150,\ h=12$см. Если радиус вписанной окружности равен $r,$ то $S=pr=\cfrac{15+20+25}{2}r=30r=150,\ r=5$см.\\