184. Так как $S_{\Delta ABC}=\cfrac{1}{2}\cdot3\sqrt{5}\cdot2\sqrt{5}\sin(\angle A)=5\sqrt{5},$ найдём $\sin(\angle A)=\cfrac{\sqrt{5}}{3}.$ Тогда $\cos(\angle A)=\pm\sqrt{1-\cfrac{5}{9}}=\pm\cfrac{2}{3}.$ В первом случае по теореме косинусов $BC^2=45+20-2\cdot3\sqrt{5}\cdot2\sqrt{5}\cdot\cfrac{2}{3}=25, BC=5$ и
$P=3\sqrt{5}+2\sqrt{5}+5=5\sqrt{5}+5.$ Во втором случае $BC^2=45+20+2\cdot3\sqrt{5}\cdot2\sqrt{5}\cdot\cfrac{2}{3}=105,\ BC=\sqrt{105}$ и $P=3\sqrt{5}+2\sqrt{5}+\sqrt{105}=5\sqrt{5}+\sqrt{105}.$\\