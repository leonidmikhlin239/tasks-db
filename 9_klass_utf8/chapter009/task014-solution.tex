14. Пусть изначальная скорость велосипедиста равна $x$(км/ч), тогда $\cfrac{98}{x}=\cfrac{98}{x+7}+7,$\\$
98\left(\cfrac{1}{x}-\cfrac{1}{x+7}\right)=7,\ \cfrac{7\cdot98}{x^2+7x}=7,\ x^2+7x-98=0,\ x=7.$ Значит, скорость велосипедиста на пути из $B$ в $A$ была равна $7+7=14$км/ч.\\
