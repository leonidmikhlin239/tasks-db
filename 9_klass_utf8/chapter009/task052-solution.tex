52. Пусть скорости работы 1-го, 2-го, 3-го и 4-го станка (в тиражах в час) равны $x,\ y,\ z$ и $t$ соответственно. Тогда имеем систему уравнений
$\begin{cases} x+y+t=\cfrac{1}{1\frac{48}{60}}=\cfrac{5}{9},\\ x+y+z=\cfrac{1}{2\frac{15}{60}}=\cfrac{4}{9},\\ z+t=\cfrac{1}{1,5}=\cfrac{2}{3}.\end{cases}$ Сложив все уравнения, получим соотношение $2x+2y+2z+2t=\cfrac{5}{9}+\cfrac{4}{9}+\cfrac{2}{3},\ 2(x+y+z+t)=\cfrac{5}{3},\ x+y+z+t=\cfrac{5}{6}.$ Значит, при совместной работе всех четырёх станков тираж будет напечатан за $\cfrac{1}{\frac{5}{6}}=\cfrac{6}{5}\text{ ч}=$ 1 ч 12 мин.\\