19. Пусть скорости покраски забора мальчиками равны $x,\ y$ и $z$ (забора в час). Тогда имеем систему уравнений $\begin{cases} x+y=\cfrac{1}{3},\\ y+z=\cfrac{1}{6},\\ x+z=\cfrac{1}{4}.\end{cases}$ откуда $x+y+z=\cfrac{\cfrac{1}{3}+\cfrac{1}{6}+\cfrac{1}{4}}{2}=\cfrac{3}{8}.$ Значит, втроём мальчики покрасят забор за $1:\cfrac{3}{8}=\cfrac{8}{3}$ч$=2$ч 20 мин.\\