\section{Стандартные задачи решения}
1. $6=0,4(N+5),\ N+5=15,\ N=10.$\\
2. $9=0,75(N+3),\ N+3=12,\ N=9.$\\
3. $\cfrac{2n^2+5n-5}{n+1}=\cfrac{2n^2+2n+3n+3-8}{n+1}=2n+3-\cfrac{8}{n+1}.$ Если это выражение является целым, то 8 делится на $n+1,$ то есть $n+1$ может быть равно 2, 4 или 8, поэтому $n\in \{1;3;7\}.$\\
4. $\cfrac{3n^2+4n-3}{n+3}=\cfrac{3n^2+9n-5n-15+12}{n+3}=3n-5+\cfrac{12}{n+3}.$ Если это выражение является целым, то 12 делится на $n+3,$ то есть $n+3$ может быть равно 4, 6 или 12, поэтому $n\in \{1;3;9\}.$\\
5. Пусть доли золота равны $x$ и $y,$ тогда имеем систему уравнений $\begin{cases} 40x+60y=0,62\cdot100,\\ \cfrac{x+y}{2}=0,61.\end{cases}\Leftrightarrow
\begin{cases} 20x+30y=31,\\ x+y=1,22.\end{cases}\Leftrightarrow
\begin{cases} 20x+36,6-30x=31,\\ y=1,22-x.\end{cases}\Leftrightarrow
\begin{cases} x=0,56,\\ y=0,66.\end{cases}$
Значит, сплавы были 56-й и 66-й пробы.\\
6. Пусть стали первого сорта взяли $x$т, тогда стали второго сорта взяли $140-x$т и $0,05x+0,4(140-x)=140\cdot0,3,\ 0,05x+56-0,4x=42,\ 0,35x=14,\ x=40.$ Значит, стали первого сорта надо взять 40 т, а второго --- $140-40=100$т.\\
7. Совместная скорость заполнения бассейна двумя трубами равна $\cfrac{1}{12}+\cfrac{1}{20}=\cfrac{2}{15}.$ Значит, вместе они заполнят бассейн за $1:\cfrac{2}{15}=7,5$ часов.\\
8. Если обе трубы будут открыты, бассейн будет заполняться со скоростью $\cfrac{1}{6}-\cfrac{1}{15}=\cfrac{1}{10}$. Значит, он заполнится за $1:\cfrac{1}{10}=10$ часов.\\
9. Пусть весь путь равен $2S$км, а скорость первого автомобиля равна $x$км/ч, тогда имеем уравнение $\cfrac{2S}{x}=\cfrac{S}{x-13}+\cfrac{S}{78},\
156(x-13)=78x+x(x-13),\ x^2-91x+2028=0,\ x=39$ или $x=52.$ Так как $x>48,$ получаем, что скорость первого автомобиля была равна 52 км/ч.\\
10. Пусть весь путь равен $2S$км, а скорость первого автомобиля равна $x$км/ч, тогда имеем уравнение $\cfrac{2S}{x}=\cfrac{S}{x-16}+\cfrac{S}{96},\
192(x-16)=96x+x(x-16),\ x^2-112x+3072=0,\ x=48$ или $x=64.$ Так как $x>57,$ получаем, что скорость первого автомобиля была равна 64 км/ч.\\
11. Пусть скорость работы первого печника равна $x,$ а скорость работы второго --- $y$ (печей в час). Тогда имеем систему уравнений
$\begin{cases} x+y=\cfrac{1}{12},\\ 2x+3y=\cfrac{1}{5}.\end{cases}
\Leftrightarrow \begin{cases} y=\cfrac{1}{12}-x,\\ 2x+\cfrac{1}{4}-3x=\cfrac{1}{5}.\end{cases}
\Leftrightarrow \begin{cases} y=\cfrac{1}{30},\\ x=\cfrac{1}{20}.\end{cases}$
Значит, первый печник может сложить печь за $1:\cfrac{1}{20}=20$ часов.\\
12. Пусть скорость работы первой бригады равна $x,$ а скорость работы второй --- $y$ (урожая в день). Тогда имеем систему уравнений
$\begin{cases} x+y=\cfrac{1}{8},\\ 3x+12y=\cfrac{3}{4}.\end{cases}
\Leftrightarrow \begin{cases} y=\cfrac{1}{8}-x,\\ 3x+\cfrac{3}{2}-12x=\cfrac{3}{4}.\end{cases}
\Leftrightarrow \begin{cases} y=\cfrac{1}{24},\\ x=\cfrac{1}{12}.\end{cases}$
Значит, вторая бригада может закончить уборку урожая за $1:\cfrac{1}{24}=24$ дня.\\
13. Пусть изначальная скорость велосипедиста равна $x$(км/ч), тогда $\cfrac{70}{x}=\cfrac{70}{x+3}+3,$\\$
70\left(\cfrac{1}{x}-\cfrac{1}{x+3}\right)=3,\ \cfrac{3\cdot70}{x^2+3x}=3,\ x^2+3x-70=0,\ x=7.$ Значит, скорость велосипедиста на пути из $B$ в $A$ была равна $7+3=10$км/ч.\\
14. Пусть изначальная скорость велосипедиста равна $x$(км/ч), тогда $\cfrac{98}{x}=\cfrac{98}{x+7}+7,$\\$
98\left(\cfrac{1}{x}-\cfrac{1}{x+7}\right)=7,\ \cfrac{7\cdot98}{x^2+7x}=7,\ x^2+7x-98=0,\ x=7.$ Значит, скорость велосипедиста на пути из $B$ в $A$ была равна $7+7=14$км/ч.\\
15. Пусть долили $x$л воды, тогда $0,4(x+10)=0,6\cdot10,\ 0,4x+4=6,\ x=5$л.\\
16. Концентрация получившегося раствора составляет $0,12\cdot5:(5+12)=0,05=5\%.$\\
17. Пусть всего автомобиль потратил на дорогу $2t$ч, тогда средняя скорость автомобиля равна $\cfrac{90t+60t}{2t}=75$км/ч.\\
18. Пусть весь путь составляет $2S$км, тогда средняя скорость автомобиля равна $\cfrac{2S}{\cfrac{S}{90}+\cfrac{S}{60}}=72$км/ч.\\
19. Пусть скорости покраски забора мальчиками равны $x,\ y$ и $z$ (забора в час). Тогда имеем систему уравнений $\begin{cases} x+y=\cfrac{1}{3},\\ y+z=\cfrac{1}{6},\\ x+z=\cfrac{1}{4}.\end{cases}$ откуда $x+y+z=\cfrac{\cfrac{1}{3}+\cfrac{1}{6}+\cfrac{1}{4}}{2}=\cfrac{3}{8}.$ Значит, втроём мальчики покрасят забор за $1:\cfrac{3}{8}=\cfrac{8}{3}$ч$=2$ч 20 мин.\\
20. Пусть скорости помывки окна девочками равны $x,\ y$ и $z$ (окна в минуту). Тогда имеем систему уравнений $\begin{cases} x+y=\cfrac{1}{20},\\ y+z=\cfrac{1}{15},\\ x+z=\cfrac{1}{12}.\end{cases}$ откуда $x+y+z=\cfrac{\cfrac{1}{20}+\cfrac{1}{15}+\cfrac{1}{12}}{2}=\cfrac{1}{10}.$ Значит, втроём девочки помоют окно за $1:\cfrac{1}{10}=10$мин.\\
21. Когда мальчик бежал вверх, он насчитал в $60:20=3$ раза больше ступенек, а значит потратил в 3 раза больше времени. При этом навстречу ему выползло в 3 раза больше ступенек, чем <<убежало>> при спуске. Если мальчик сбежит вниз 3 раза, то он насчитает 60 ступенек, а убегут от него столько же ступенек, сколько выпозло при подъёме. Поэтому $60+60=120$ --- это учетверённое количество ступенек эскалатора, то есть всего их $120:4=30.$\\
22. Когда мальчик бежал вверх, он насчитал в $70:30=\cfrac{7}{3}$ раза больше ступенек, а значит потратил в $\cfrac{7}{3}$ раза больше времени. При этом навстречу ему выползло в $\cfrac{7}{3}$ раза больше ступенек, чем <<убежало>> при спуске. Если мальчик сбежит вниз $\cfrac{7}{3}$ раза, то он насчитает 70 ступенек, а убегут от него столько же ступенек, сколько выпозло при подъёме. Поэтому $70+70=140$ --- это количество ступенек, умноженное на $\left(\cfrac{7}{3}+1\right)=\cfrac{10}{3},$ эскалатора, то есть всего их $140:\cfrac{10}{3}=42.$\\
23. Пусть это было число $\overline{abc}=100a+10b+c,$ тогда $100a+10b+c-a-b-c=189,\ 99a+9b=189,\ 11a+b=21.$ Если $a=1,$ то $b=10,$ что невозможно, так как это цифра. Если $a\geqslant2,$ то $11a\geqslant22>21.$ Значит, такого числа не существует и 189 получиться не могло.\\
24. Пусть это было число $\overline{abc}=100a+10b+c,$ тогда $100a+10b+c-a-b-c=180,\ 99a+9b=180,\ 11a+b=20.$ Это возможно, если $a=1,\ b=9,$ при этом цифра $c$ может быть какой угодно. Например, $190-(1+9+0)=180.$\\
25. $\cfrac{n^2+5n-8}{n+3}=\cfrac{n^2+3n+2n+6-14}{n+3}=n+2-\cfrac{14}{n+3}.$ Если это выражение является целым, то 14 делится на $n+3,$ то есть $n+3$ может быть равно 7 или 14, поэтому $n\in \{4;11\}.$\\
26. $\cfrac{-n^2+2n-31}{n+3}=\cfrac{-n^2-3n+5n+15-46}{n+3}=-n+5-\cfrac{46}{n+3}.$ Если это выражение является целым, то 46 делится на $n+3,$ то есть $n+3$ может быть равно 23 или 46, поэтому $n\in \{20;43\}.$\\
27. Пусть второй рабочий делает $x$ деталей в час, а первый --- $x+3.$ Тогда $\cfrac{180}{x}-\cfrac{180}{x+3}=3,$\\$ 180\left(\cfrac{1}{x}-\cfrac{1}{x+3}\right)=3,\
\cfrac{180\cdot3}{x^2+3x}=3,\ x^2+3x-180=0,\ x=12,\ x+1=13$ деталей в час.\\
28. Пусть второй рабочий делает $x$ деталей в час, а первый --- $x+1.$ Тогда $\cfrac{156}{x}-\cfrac{156}{x+1}=1,$\\$ 156\left(\cfrac{1}{x}-\cfrac{1}{x+1}\right)=1,\
\cfrac{156}{x^2+x}=1,\ x^2+x-156=0,\ x=12$ деталей в час.\\
29. В июле товар стоил $30000\cdot1,04\cdot0,96=29952$ рубля.\\
30. В августе товар стоил $20000\cdot1,06\cdot0,94=19928$ рублей.\\
31. Скорость плота равна скорости течения, то есть 50м/мин$=$3000м/ч$=$3км/ч. Тогда 9 км он пройдёт за $9:3=3$ часа. Лодка плыла на 30 минут меньше, то есть $\cfrac{5}{2}$ч. Пусть скорость лодки в неподвижной воде равна $x$км/ч, тогда $\cfrac{18}{x+3}+\cfrac{18}{x-3}=\cfrac{5}{2},\
18\left(\cfrac{1}{x+3}+\cfrac{1}{x-3}\right)=\cfrac{5}{2},\ \cfrac{18\cdot2x}{x^2-9}=\cfrac{5}{2},\ 72x=5x^2-45,\ 5x^2-72x-45=0,\ x=15$км/ч.\\
32. Скорость плота равна скорости течения, то есть 50м/мин$=$3000м/ч$=$3км/ч. Тогда 9 км он пройдёт за $7:3=\cfrac{7}{3}$ часа. Лодка плыла на 44 минуты меньше, то есть $\cfrac{7}{3}-\cfrac{11}{15}=\cfrac{8}{5}$ч. Пусть скорость лодки в неподвижной воде равна $x$км/ч, тогда $\cfrac{14}{x+3}+\cfrac{14}{x-3}=\cfrac{8}{5},\
14\left(\cfrac{1}{x+3}+\cfrac{1}{x-3}\right)=\cfrac{8}{5},\ \cfrac{14\cdot2x}{x^2-9}=\cfrac{8}{5},\ 140x=8x^2-72,\ 2x^2-35x-18=0,\ x=18$км/ч.\\
33. Пусть первого раствора взято $x$л, тогда второго было $100-x$л и $0,2x+0,7(100-x)=100\cdot0,5,\
0,2x+70-0,7x=50,\ 0,5x=20,\ x=40$л. Значит, первого раствора надо взять 40 литров, а второго --- $100-40=60$ литров.\\
34. Пусть добавили $x$кг чистого олова, тогда $15\cdot0,4=(15+x)\cdot0,3,\ 6=4,5+0,3x,\ x=5$кг.\\
35. Пусть в первом ящике $x$ синих шаров и $15-x$ красных, тогда во втором ящике $8-x$ синих шаров и $15-(8-x)=x+7$ красных. Тогда по условию задачи $\cfrac{15-x}{x}\cdot2=\cfrac{x+7}{8-x},\ (30-2x)(8-x)=x(x+7),\ 240-30x-16x+2x^2=x^2+7x,\ x^2-53x+240=0,\ x=5$ (второй корень $x=48>15).$ Значит, в первом ящике $15-5=10$ красных шаров, а во втором --- $5+7=12.$\\
36. Пусть в первом ящике $x$ жёлтых кубиков и $40-x$ зелёных, тогда во втором ящике $14-x$ жёлтых кубиков и $40-(14-x)=x+26$ зелёных. Тогда по условию задачи $\cfrac{40-x}{x}\cdot3=\cfrac{x+26}{14-x},\ (120-3x)(14-x)=x(x+26),\ 1680-120x-42x+3x^2=x^2+26x,\ x^2-94x+840=0,\ x=10$ (второй корень $x=84>40).$ Значит, в первом ящике $40-10=30$ зелёных кубиков, а во втором --- $10+26=36.$\\
37. Если товар стоил $x$ рублей, то по скидкам он в любом случае будет стоит $x\cdot0,5\cdot0,7=x\cdot0,7\cdot0,5=0,35x,$ то есть скидка составит $65\%.$\\
38. Если товар стоил $x$ рублей, то по скидкам он в любом случае будет стоит $x\cdot0,3\cdot0,8=x\cdot0,8\cdot0,3=0,24x,$ то есть скидка составит $76\%.$\\
39. Понятно, что $f(v)>f_0.$ Человек начнёт различать сигналы при $f(v)=593+7=600$гц и чем больше $v,$ тем больше $f(v)$ (знаменатель дроби уменьшается, значит вся дробь увеличивается). Значит, наименьшее возможное значение скорости $v$ --- это то, при котором $f(v)=\cfrac{f_0}{1-\frac{v}{c}}=600,\
\cfrac{593}{1-\cfrac{v}{300}}=600,\ 593=600-2v,\ v=\cfrac{7}{2}$м/с.\\
40. Понятно, что $f(v)>f_0.$ Человек начнёт различать сигналы при $f(v)=154+6=160$гц и чем больше $v,$ тем больше $f(v)$ (знаменатель дроби уменьшается, значит вся дробь увеличивается). Значит, наименьшее возможное значение скорости $v$ --- это то, при котором $f(v)=\cfrac{f_0}{1-\frac{v}{c}}=160,\
\cfrac{154}{1-\cfrac{v}{320}}=160,\ 154=160-\cfrac{v}{2},\ v=12$м/с.\\
41. Пусть скорость пассажирского поезда равна $x$км/ч, а скорость товарного --- $y$км/ч. Тогда расстояние от места встречи до Стамбула равно $3x,$ а до Вены --- $6y.$ При этом товарный поезд потратил на то, чтобы добраться до места встречи, на 3 часа меньше, то есть $\cfrac{3x}{y}+3=\cfrac{6y}{x}.$ Пусть $\cfrac{x}{y}=t,$ тогда $3t+3=\cfrac{6}{t},\ t^2+t=2,\ t^2+t-2=0,\ t=1.$ Значит, $x=y$ и оба поезда находились в пути $9x:x=9$ часов.\\
42. Пусть скорость Васи равна $x$м/мин, а скорость Пети --- $y$м/мин. Тогда расстояние от места встречи до пункта Б равно $11x,$ а до пункта А --- $9y.$ При этом Петя потратил на то, чтобы добраться до места встречи, на 2 минуты меньше, то есть $\cfrac{11x}{y}+2=\cfrac{9y}{x}.$ Пусть $\cfrac{x}{y}=t,$ тогда $11t+2=\cfrac{9}{t},\ 11t^2+2t=9,\ 11t^2+2t-9=0,\ t=\cfrac{9}{11}.$ Значит, $11x=9y$ и весь путь равен $11x+9y=18y$ или $11x+9y=22x,$ поэтому Вася находился в пути $22x:x=22$ минуты, а Петя $18y:y=18$ минут.\\
43. Пусть сторона листа равна $a,$ а сторона квадратика равна $x.$ Тогда периметр листа увеличился на $2x$ и $2x=0,05\cdot4a,\ x=0,1a.$ Площадь листа увеличилась на $x^2=0,01a,$ то есть на $1\%.$\\
44. Пусть сторона листа равна $a,$ а сторона квадратика равна $x.$ Тогда периметр листа увеличился на $2x$ и $2x=0,1\cdot4a,\ x=0,2a.$ Площадь листа уменьшилась на $x^2=0,04a,$ то есть на $4\%.$\\
45. Пусть объём бутыли равен $x$л, тогда в ней $0,12x$ соли. После первого долива в бутыли останется $0,12x-0,12$л соли и её концентрация будет равна $\cfrac{0,12x-0,12}{x}.$ После второго долива соли в бутыли будет $0,12x-0,12-\cfrac{0,12x-0,12}{x}=0,03x,$ значит $0,12x^2-0,12x-0,12x+0,12=0,03x^2,\ 0,09x^2-0,24x+0,12=0,\ 3x^2-8x+4=0,\ (x-2)(3x-2)=0,\ x=2$л (второй корень не подходит, так как $x>1$л).\\
46. Пусть объём фляги равен $x$л, тогда в ней $0,96x$ кислоты. После первого долива в фляге останется $0,96x-12\cdot0,96=0,96x-11,52$л кислоты и её концентрация будет равна $\cfrac{0,96x-11,52}{x}.$ После второго долива кислоты в фляге будет $0,96x-11,52-\cfrac{0,96x-11,52}{x}\cdot18=0,32x,$ значит $0,96x^2-11,52x-17,28x+207,36=0,32x^2,\ 0,64x^2-28,8x+207,36=0,\ x^2-45x+324=0,\ (x-9)(x-36)=0,\ x=36$л (второй корень не подходит, так как $x>18$л).\\
47. В баке получится $5+2=7$ чайных ложек удобрения на $3+10=13$ литров воды, значит на $65=13\cdot5$ литров понадобится $7\cdot5=35$ чайных ложек удобрения.\\
48. Через 3 года вклад станет равен $50000\cdot1,2^3=86400$ рублей.\\
49. Через 3 года вклад станет равен $70000\cdot1,1^3=93170$ рублей.\\
50. Пусть скорость автомобиля равна $x\text{ км/ч}.$ Первый велосипедист за час проедет $24\cdot1=24$ километра, а второй --- $18\cdot1=18$ километров. Тогда $\cfrac{24}{x-24}-\cfrac{18}{x-18}=\cfrac{1}{6}\Leftrightarrow\cfrac{24x-24\cdot18-18x+24\cdot18}{x^2-18x-24x+432}=\cfrac{1}{6}\Leftrightarrow 36x=x^2-42x+432\Leftrightarrow
x^2-78x+432=0\Leftrightarrow x=72\text{ км/ч}$ (второй корень равен $6\text{ км/ч},$ но с такой скоростью автомобиль никого бы не догнал).\\
51. Пусть катера равна $x\text{ км/ч}.$ Первая лодка за час проедет $12\cdot1=12$ километра, а вторая --- $9\cdot1=9$ километров. Тогда $\cfrac{12}{x-12}-\cfrac{9}{x-9}=\cfrac{1}{6}\Leftrightarrow\cfrac{12x-12\cdot9-9x+12\cdot9}{x^2-9x-12x+108}=\cfrac{1}{6}\Leftrightarrow 18x=x^2-21x+108\Leftrightarrow
x^2-39x+108=0\Leftrightarrow x=36\text{ км/ч}$ (второй корень равен $3\text{ км/ч},$ но с такой скоростью катер никого бы не догнал).\\
52. Пусть скорости работы 1-го, 2-го, 3-го и 4-го станка (в тиражах в час) равны $x,\ y,\ z$ и $t$ соответственно. Тогда имеем систему уравнений
$\begin{cases} x+y+t=\cfrac{1}{1\frac{48}{60}}=\cfrac{5}{9},\\ x+y+z=\cfrac{1}{2\frac{15}{60}}=\cfrac{4}{9},\\ z+t=\cfrac{1}{1,5}=\cfrac{2}{3}.\end{cases}$ Сложив все уравнения, получим соотношение $2x+2y+2z+2t=\cfrac{5}{9}+\cfrac{4}{9}+\cfrac{2}{3},\ 2(x+y+z+t)=\cfrac{5}{3},\ x+y+z+t=\cfrac{5}{6}.$ Значит, при совместной работе всех четырёх станков тираж будет напечатан за $\cfrac{1}{\frac{5}{6}}=\cfrac{6}{5}\text{ ч}=$ 1 ч 12 мин.\\
53. Пусть изначальная скорость автобуса была равна $x$км/ч, а весь путь --- $2S$км, тогда имеем систему уравнений
$\begin{cases}\cfrac{S-15}{x}+1,5+\cfrac{S+15}{x+20}=\cfrac{2S}{x},\\ \cfrac{2S}{x+9}=10.\end{cases}\Leftrightarrow
\begin{cases}\cfrac{5x+30}{x}+1,5+\cfrac{5x+60}{x+20}=\cfrac{10x+90}{x},\\ S=5x+45.\end{cases}\Leftrightarrow
\begin{cases}1,5+\cfrac{5x+60}{x+20}-\cfrac{5x+60}{x}=0,\\ S=5x+45.\end{cases}\Leftrightarrow
\begin{cases}1,5+(5x+60)\left(\cfrac{1}{x+20}-\cfrac{1}{x}\right)=0,\\ S=5x+45.\end{cases}\Leftrightarrow$\\$
\begin{cases}1,5-\cfrac{100x+1200}{x(x+20)}=0,\\ S=5x+45.\end{cases}\Leftrightarrow
\begin{cases}3x^2+60x-200x-2400=0,\\ S=5x+45.\end{cases}\Leftrightarrow
\begin{cases}3x^2-140x-2400=0,\\ S=5x+45.\end{cases}\Leftrightarrow$\\$
\begin{cases}(x-60)(3x+40)=0,\\ S=5x+45.\end{cases}\Leftrightarrow
\begin{cases}x=60,\\ S=345.\end{cases}$
Значит, изначальная скорость автобуса равна 60 км/ч, а расстояние между $A$ и $B$ равно $2\cdot345=690$км.\\
54. Пусть изначальная скорость автобуса была равна $x$км/ч, а весь путь --- $2S$км, тогда имеем систему уравнений
$\begin{cases}\cfrac{S+40}{x}+\cfrac{5}{6}+\cfrac{S-40}{x-5}=\cfrac{2S}{x}+1,\\ \cfrac{2S}{x-20}=8.\end{cases}\Leftrightarrow
\begin{cases}\cfrac{4x-40}{x}-\cfrac{1}{6}+\cfrac{4x-120}{x-5}=\cfrac{8x-160}{x},\\ S=4x-80.\end{cases}\Leftrightarrow
\begin{cases}\cfrac{4x-120}{x-5}-\cfrac{4x-120}{x}-\cfrac{1}{6}=0,\\ S=4x-80.\end{cases}\Leftrightarrow
\begin{cases}(4x-120)\left(\cfrac{1}{x-5}-\cfrac{1}{x}\right)-\cfrac{1}{6}=0,\\ S=4x-80.\end{cases}\Leftrightarrow$\\$
\begin{cases}\cfrac{20x-600}{x(x-5)}-\cfrac{1}{6}=0,\\ S=4x-80.\end{cases}\Leftrightarrow
\begin{cases}120x-3600-x^2+5x=0,\\ S=4x-80.\end{cases}\Leftrightarrow
\begin{cases}x^2-125x+3600=0,\\ S=4x-80.\end{cases}\Leftrightarrow$\\$
\begin{cases}(x-45)(x-80)=0,\\ S=4x-80.\end{cases}\Leftrightarrow
\left[\begin{array}{l}\begin{cases}x=45,\\ S=100.\end{cases}\\ \begin{cases}x=80,\\ S=240.\end{cases}\end{array}\right.$
Значит, изначальная скорость автобуса может быть равна 45 км/ч или 80 км/ч, а расстояние между $C$ и $D$ равно $2\cdot100=200$км или $2\cdot240=480$км соответственно.\\
55. Пусть в первом сосуде было $x$ мл раствора, а во втором --- $y,$ тогда имеем систему уравнений $\begin{cases}\cfrac{70}{x}-\cfrac{60}{y}=0,2,\\
x+y=600.\end{cases}\Leftrightarrow\begin{cases}\cfrac{70}{x}-\cfrac{60}{600-x}=0,2,\\
y=600-x.\end{cases}\Leftrightarrow\begin{cases}\cfrac{42000-70x-60x}{x(600-x)}=\cfrac{1}{5},\\
y=600-x.\end{cases}\Leftrightarrow$\\$\begin{cases}210000-650x=600x-x^2,\\
y=600-x.\end{cases}\Leftrightarrow\begin{cases}x^2-1250x+210000=0,\\
y=600-x.\end{cases}\Leftrightarrow\begin{cases}(x-200)(x-1050)=0,\\
y=600-x.\end{cases}\Leftrightarrow\begin{cases}x=200,\\
y=400.\end{cases}$ Значит, концентрация кислоты в первом растворе равна $70:200\cdot100\%=35\%,$ а во втором --- $60:400\cdot100\%=15\%.$\\
56. Пусть всего на предприятии было $x$ сотрудников, тогда сотрудников с высшим образованием было $0,8x.$ Следовательно, выполнено соотношение $0,85(x+30)=0,8x+30,\ 0,85x+25,5=0,8x+30,\ 0,05x=4,5,\ x=90.$ Таким образом, теперь на предприятии работает $90+30=120$ сотрудников.\\
57. Пусть всего в партии было $x$ изделий, тогда бракованных изделий было $0,09x.$ Следовательно, выполнено соотношение $0,025(x-40)=0,09x-40,\ 0,025x-1=0,09x-40,\ 0,065x=39,\ x=600.$ Таким образом, в партии осталось $600-40=560$ деталей.\\
58. Пусть поезда встретились в точке $C,$ а их скорости равны $x$ и $y$ соответственно, тогда $|AC|=8y,\ |CB|=50x.$ Так как поезда добрались до точки $C$ за одинаковое время, выполнено соотношение $\cfrac{8y}{x}=\cfrac{50x}{y},$ откуда $8y^2=50x^2,\ (2y)^2=(5x)^2,\ 2y=5x.$ Тогда $|AC|=8y=20x,$ и первый поезд доехал до места встречи за $20x:x=20$ часов.\\
59. Пусть велосипедисты встретились в точке $C,$ а их скорости равны $x$ и $y$ соответственно, тогда $|AC|=27y,\ |CB|=48x.$ Так как велосипедисты добрались до точки $C$ за одинаковое время, выполнено соотношение $\cfrac{27y}{x}=\cfrac{48x}{y},$ откуда $27y^2=48x^2,\ (3y)^2=(4x)^2,\ 3y=4x.$ Тогда $|AC|=27y=36x,$ и первый велосипедист доехал до места встречи за $36x:x=36$ минут.
\newpage
