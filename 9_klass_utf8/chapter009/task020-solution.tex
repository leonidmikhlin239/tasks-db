20. Пусть скорости помывки окна девочками равны $x,\ y$ и $z$ (окна в минуту). Тогда имеем систему уравнений $\begin{cases} x+y=\cfrac{1}{20},\\ y+z=\cfrac{1}{15},\\ x+z=\cfrac{1}{12}.\end{cases}$ откуда $x+y+z=\cfrac{\cfrac{1}{20}+\cfrac{1}{15}+\cfrac{1}{12}}{2}=\cfrac{1}{10}.$ Значит, втроём девочки помоют окно за $1:\cfrac{1}{10}=10$мин.\\
