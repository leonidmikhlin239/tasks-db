40. Перед отправкой тепловоз издал гудок с частотой $f_0=154$ Гц. Чуть позже издал гудок подъезжающий к платформе тепловоз. Из-за эффекта Допплера частота второго гудка $f$ больше первого: она зависит от скорости тепловоза по закону $f(v)=\cfrac{f_0}{1-\frac{v}{c}}$ (Гц), где $c$ --- скорость звука в м/с. Человек, стоящий на платформе, различает сигналы по тону, если они отличаются не менее, чем на 6 Гц. Определите, с какой минимальной скоростью приближался к платформе тепловоз, если человек смог различить сигналы, а $c=320$ м/с. Ответ выразите в м/с.\\
