10. Пусть весь путь равен $2S$км, а скорость первого автомобиля равна $x$км/ч, тогда имеем уравнение $\cfrac{2S}{x}=\cfrac{S}{x-16}+\cfrac{S}{96},\
192(x-16)=96x+x(x-16),\ x^2-112x+3072=0,\ x=48$ или $x=64.$ Так как $x>57,$ получаем, что скорость первого автомобиля была равна 64 км/ч.\\
