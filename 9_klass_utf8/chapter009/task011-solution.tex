11. Пусть скорость работы первого печника равна $x,$ а скорость работы второго --- $y$ (печей в час). Тогда имеем систему уравнений
$\begin{cases} x+y=\cfrac{1}{12},\\ 2x+3y=\cfrac{1}{5}.\end{cases}
\Leftrightarrow \begin{cases} y=\cfrac{1}{12}-x,\\ 2x+\cfrac{1}{4}-3x=\cfrac{1}{5}.\end{cases}
\Leftrightarrow \begin{cases} y=\cfrac{1}{30},\\ x=\cfrac{1}{20}.\end{cases}$
Значит, первый печник может сложить печь за $1:\cfrac{1}{20}=20$ часов.\\
