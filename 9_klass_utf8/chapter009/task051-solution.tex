51. Пусть катера равна $x\text{ км/ч}.$ Первая лодка за час проедет $12\cdot1=12$ километра, а вторая --- $9\cdot1=9$ километров. Тогда $\cfrac{12}{x-12}-\cfrac{9}{x-9}=\cfrac{1}{6}\Leftrightarrow\cfrac{12x-12\cdot9-9x+12\cdot9}{x^2-9x-12x+108}=\cfrac{1}{6}\Leftrightarrow 18x=x^2-21x+108\Leftrightarrow
x^2-39x+108=0\Leftrightarrow x=36\text{ км/ч}$ (второй корень равен $3\text{ км/ч},$ но с такой скоростью катер никого бы не догнал).\\
