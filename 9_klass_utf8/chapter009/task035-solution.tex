35. Пусть в первом ящике $x$ синих шаров и $15-x$ красных, тогда во втором ящике $8-x$ синих шаров и $15-(8-x)=x+7$ красных. Тогда по условию задачи $\cfrac{15-x}{x}\cdot2=\cfrac{x+7}{8-x},\ (30-2x)(8-x)=x(x+7),\ 240-30x-16x+2x^2=x^2+7x,\ x^2-53x+240=0,\ x=5$ (второй корень $x=48>15).$ Значит, в первом ящике $15-5=10$ красных шаров, а во втором --- $5+7=12.$\\
