36. Пусть в первом ящике $x$ жёлтых кубиков и $40-x$ зелёных, тогда во втором ящике $14-x$ жёлтых кубиков и $40-(14-x)=x+26$ зелёных. Тогда по условию задачи $\cfrac{40-x}{x}\cdot3=\cfrac{x+26}{14-x},\ (120-3x)(14-x)=x(x+26),\ 1680-120x-42x+3x^2=x^2+26x,\ x^2-94x+840=0,\ x=10$ (второй корень $x=84>40).$ Значит, в первом ящике $40-10=30$ зелёных кубиков, а во втором --- $10+26=36.$\\