54. Пусть изначальная скорость автобуса была равна $x$км/ч, а весь путь --- $2S$км, тогда имеем систему уравнений
$\begin{cases}\cfrac{S+40}{x}+\cfrac{5}{6}+\cfrac{S-40}{x-5}=\cfrac{2S}{x}+1,\\ \cfrac{2S}{x-20}=8.\end{cases}\Leftrightarrow
\begin{cases}\cfrac{4x-40}{x}-\cfrac{1}{6}+\cfrac{4x-120}{x-5}=\cfrac{8x-160}{x},\\ S=4x-80.\end{cases}\Leftrightarrow
\begin{cases}\cfrac{4x-120}{x-5}-\cfrac{4x-120}{x}-\cfrac{1}{6}=0,\\ S=4x-80.\end{cases}\Leftrightarrow
\begin{cases}(4x-120)\left(\cfrac{1}{x-5}-\cfrac{1}{x}\right)-\cfrac{1}{6}=0,\\ S=4x-80.\end{cases}\Leftrightarrow$\\$
\begin{cases}\cfrac{20x-600}{x(x-5)}-\cfrac{1}{6}=0,\\ S=4x-80.\end{cases}\Leftrightarrow
\begin{cases}120x-3600-x^2+5x=0,\\ S=4x-80.\end{cases}\Leftrightarrow
\begin{cases}x^2-125x+3600=0,\\ S=4x-80.\end{cases}\Leftrightarrow$\\$
\begin{cases}(x-45)(x-80)=0,\\ S=4x-80.\end{cases}\Leftrightarrow
\left[\begin{array}{l}\begin{cases}x=45,\\ S=100.\end{cases}\\ \begin{cases}x=80,\\ S=240.\end{cases}\end{array}\right.$
Значит, изначальная скорость автобуса может быть равна 45 км/ч или 80 км/ч, а расстояние между $C$ и $D$ равно $2\cdot100=200$км или $2\cdot240=480$км соответственно.\\
