\documentclass[12pt]{article}
%\usepackage{newlistok}
\usepackage[T1,T2A]{fontenc} %
%\usepackage[cp866]{inputenc} %
\usepackage[cp1251]{inputenc} %
\usepackage[russian]{babel} %
\usepackage{amsmath} %
\usepackage{amssymb}
\usepackage{pgfplots}
\graphicspath{{pictures/}}
\DeclareGraphicsExtensions{.pdf,.png,.jpg}
\pgfplotsset{compat=1.9}
\usepackage{tikz}
\textheight=257mm %
\textwidth =180mm
%\usepackage[active]{srcltx} %
\mathsurround=2pt %
%\textheight=185mm \textwidth =120mm %
\textheight=257mm %
\textwidth =190mm %
\hoffset=-18mm %
\voffset=-41mm
\author{Л.С.Михлин}
\title{Пособие для поступающих в 8---10 классы физико--математических школ}
\date{2020 --- ...}
\begin{document}
\maketitle
\newpage
\tableofcontents
\newpage
\section{Числовые выражения задачи}
$\begin{array}{ll}
1.\ 2\sqrt{7-4\sqrt{3}}+\sqrt{13-4\sqrt{3}},&
2.\ \sqrt{3-2\sqrt{2}}-\sqrt{2}+1,\\
3.\ 1998\cfrac{19}{6891}\cdot1997\cfrac{19}{6891}-1999\cfrac{19}{6891}\cdot1996\cfrac{19}{6891},&
4.\ (\sqrt[3]{49}+\sqrt[3]{7}+1)\cdot(\sqrt[3]{49}-1)\cdot(\sqrt[3]{49}-\sqrt[3]{7}+1),
\end{array}$\\
$\begin{array}{l}
5.\ 0,815\cdot\left(-\cfrac{2}{3}\right)-\cfrac{1}{6}\cdot(-4,385)+0,815\cdot\cfrac{1}{6}-(-4,385)\cdot\left(-\cfrac{2}{3}\right),\\
6.\ (-14,09)\cdot 2\cfrac{1}{6}-6,31\cdot\left(-1\cfrac{1}{2}\right)-2\cfrac{1}{6}\cdot6,31+\left(-1\cfrac{1}{2}\right)\cdot(-14,09),
\end{array}$\\
$\begin{array}{ll}
7.\ \cfrac{\sqrt{21+8\sqrt{5}}}{4+\sqrt{5}}\cdot\sqrt{9-4\sqrt{5}},&
8.\ \sqrt{19-6\sqrt{10}}\cdot\cfrac{3-\sqrt{7}}{\sqrt{16-6\sqrt{7}}},\\
9.\ \sqrt{6}+\sqrt{5}-\cfrac{1}{\sqrt{11-2\sqrt{30}}},&
10.\ \sqrt{7}-\sqrt{2}-\cfrac{5}{\sqrt{9+2\sqrt{14}}},\\
11.\ (-1,5)^{-3}-\left(\cfrac{2}{5}\right)^{-4}\cdot\left(\cfrac{2}{5}\right)^{3}-
\left(\left(\cfrac{4}{9}\right)^{0,5}\right)^0+16^{\frac{3}{4}}\cdot0,5,\\
12.\ \left(\cfrac{3}{5}\right)^{-3}\cdot\left(\cfrac{3}{5}\right)^{4}-
\left(\left(\cfrac{9}{25}\right)^{0}\right)^{0,5}-(-0,5)^{-3}-25^{1,5}\cdot0,2,\end{array}$ \\$\begin{array}{ll}
13.\ (-1)^{21}-81^\frac{3}{4}+\left(2^\frac{2}{3}\cdot2^\frac{1}{2}\right)^6-
16^{\frac{5}{4}}+\left(-\cfrac{1}{4}\right)^{-3},\\
14.\ (-1)^{18}+32^\frac{4}{5}+8\cdot27^\frac{1}{3}-\cfrac{1}{27}\cdot\left(3^\frac{1}{4}\cdot
3^\frac{1}{3}\right)^{12}-\left(-\cfrac{1}{6}\right)^{-3},\\
15.\ (\sqrt{2}-\sqrt{3})\sqrt{5+2\sqrt{6}},&
16.\ (\sqrt{5}-\sqrt{6})\sqrt{11+2\sqrt{30}},\end{array}$ \\$\begin{array}{ll}
17.\ \cfrac{\left(5\cfrac{4}{45}-4\cfrac{1}{6}\right):5\cfrac{8}{15}}{\left(4\cfrac{2}{3}+0,75\right)\cdot3\cfrac{9}{13}}
\cdot34\cfrac{2}{7}+\cfrac{0,3:0,01}{70}+\cfrac{2}{7},\\
18.\ \cfrac{\left(\cfrac{3}{5}+0,425-0,005\right):0,1}{30,5+\cfrac{1}{6}+3\cfrac{1}{3}}+
\cfrac{6\cfrac{3}{4}+5\cfrac{1}{2}}{26:3\cfrac{5}{7}}-0,05,\end{array}$ \\$\begin{array}{ll}
19.\ \cfrac{10}{\sqrt{5}-\sqrt{10}+\sqrt{20}+\sqrt{40}-\sqrt{80}},&
20.\ \cfrac{\sqrt{6}}{\sqrt{3}-\sqrt{6}-\sqrt{24}-\sqrt{48}+\sqrt{108}},\\
21.\ \cfrac{7,46^3+6,26^3}{13,72}-7,46\cdot 6,26,&
22.\ \cfrac{2,5^3-4,4^3}{1,9}+2,5^2+4,4^2,\\
23.\ \cfrac{3\cfrac{1}{3}\cdot1,9+19,5:4\cfrac{1}{2}}{\cfrac{62}{75}-0,16}:
\cfrac{3,5+4\cfrac{2}{3}+2\cfrac{2}{15}}{0,5\cdot\left(1\cfrac{1}{20}+4,1\right)},&
24.\ 2\sqrt{3}+0,25(\sqrt{21}-5)(\sqrt{7}+3\sqrt{3})+\cfrac{2\sqrt{7}-4}{1+\sqrt{7}},\\
25.\ \cfrac{1-\sqrt{10}}{\sqrt{2}+\sqrt{5}}+\cfrac{7}{2\sqrt{2}+1}-(11-5\sqrt{5})(2+\sqrt{5}),&
26.\ (36,5^2-27,5^2):\left(\cfrac{57^3+33^3}{90}-57\cdot 33 \right),\\
27.\ \left(\cfrac{97^3-53^3}{44}+97\cdot53\right):(152,5^2-27,5^2),&
28.\ \cfrac{\sqrt{7-4\sqrt{3}}}{\sqrt{2-\sqrt{3}}}\sqrt{2+\sqrt{3}},\\
29.\ \cfrac{\sqrt{7+4\sqrt{3}}}{\sqrt{2+\sqrt{3}}}\sqrt{2-\sqrt{3}},&
30.\ \left(\cfrac{1}{\sqrt{5}+\sqrt{6}}+\cfrac{1}{\sqrt{5}-2}\right):\left(1+\cfrac{\sqrt{6}}{2}\right),\\
31.\ \left(\cfrac{1}{\sqrt{10}+\sqrt{11}}+\cfrac{1}{\sqrt{10}-3}\right):\left(1+\cfrac{\sqrt{11}}{3}\right),&
32.\ \sqrt[6]{31+10\sqrt{6}}\cdot\sqrt[3]{5-\sqrt{6}},\\
33.\ \sqrt{175}-3\sqrt{3\cfrac{1}{9}}-6\sqrt{1,75},&
34.\ \left(\cfrac{\sqrt[3]{\sqrt{3}+\sqrt{6}}\sqrt[6]{9-6\sqrt{2}}-\sqrt[6]{18}}{\sqrt[6]{2}-1}\right)^3\\
35.\ \cfrac{1}{\sqrt[3]{25}+\sqrt[3]{5}+1},&
36.\ \cfrac{1}{\sqrt[3]{9}-2\sqrt[3]{3}+4},
\end{array}$\\
37. Сравните числа:\\
а) $\cfrac{1}{\sqrt{3}-\sqrt{2}}$ и $2\sqrt{2}.$\\
б) $\sqrt{10}-\sqrt{11}$ и $\sqrt{12}-\sqrt{13}.$\\
38. Сравните значения выражений $A=27894^2+1618^2$ и $B=27895^2+1617^2.$\\
39. Сравните значения выражений $A=191^6$ и $B=188\cdot189\cdot190\cdot192\cdot193\cdot194.$\\
40. $(2-\sqrt{5})\sqrt{9+4\sqrt{5}}.$ 41. $(\sqrt{7}-3)\sqrt{16+6\sqrt{7}}.$\\
42. $74,7\cdot\cfrac{2}{21}+(-105,3)\cdot2\cfrac{3}{7}-(-105,3)\cdot\cfrac{2}{21}-2\cfrac{3}{7}\cdot74,7.$\\
43. $6\cfrac{1}{10}\cdot2,391-0,109\cdot1\cfrac{5}{6}-1\cfrac{5}{6}\cdot2,391+0,109\cdot6\cfrac{1}{10}.$\\
44. Какое из чисел больше: $\sqrt{4+\sqrt{7}}-\sqrt{4-\sqrt{7}}$ или $\sqrt[3]{2}?$\\
$\begin{array}{ll}
45.\ \cfrac{2}{\sqrt{8-2\sqrt{15}}}-\cfrac{1}{\sqrt{22-4\sqrt{30}}}-\sqrt{5},&
46.\ \cfrac{10\sqrt{5}}{{\sqrt{38-12\sqrt{10}}+\sqrt{18}}},\\
47.\ \sqrt{\left(\cfrac{4^5}{16^{-2}\cdot2^{19}}\right)^{-1}}\cdot\sqrt{8}-(0,3)^{-1}+(2\sqrt{3})^{-2},&
48.\ \sqrt{\left(\cfrac{9^{12}}{3^{-5}\cdot27^{10}}\right)^{-1}}:\sqrt{27}+(\sqrt{6})^{-2}-(1,25)^{-1}.
\end{array}$\\
49. Какое из двух чисел больше, $\sqrt{3}+\sqrt{7}+\sqrt{21}$ или 9?\\
$\begin{array}{ll}
50.\ (\sqrt[3]{49}+\sqrt[3]{7}+1)(\sqrt[3]{49}-1)(\sqrt[3]{49}-\sqrt[3]{7}+1),&
51.\ (\sqrt[3]{36}+\sqrt[3]{6}+1)(\sqrt[3]{36}-1)(\sqrt[3]{36}-\sqrt[3]{6}+1),\\
52.\ \cfrac{\sqrt[4]{7\cdot\sqrt[3]{54}+15\cdot\sqrt[3]{128}}}{\sqrt[3]{4\cdot\sqrt[4]{32}}+\sqrt[3]{9\cdot\sqrt[4]{162}}},&
53.\ \cfrac{5\cdot\sqrt[3]{4\cdot\sqrt[3]{192}}+7\cdot\sqrt[3]{18\cdot\sqrt[3]{81}}}{\sqrt[3]{12\cdot
\sqrt[3]{24}+6\cdot\sqrt[3]{375}}}.
\end{array}$
\newpage
