\section{Уравнения решения}
1. $\cfrac{x+1}{2x-3}+\cfrac{x}{x+1}=\cfrac{x^2+6x-5}{2x^2-x-3}\Leftrightarrow \cfrac{x^2+2x+1+2x^2-3x}{(2x-3)(x+1)}=\cfrac{x^2+6x-5}{(2x-3)(x+1)}\Leftrightarrow$\\$
\begin{cases} 3x^2-x+1=x^2+6x-5,\\ x
eq\cfrac{3}{2},\ x
eq-1.\end{cases}\Leftrightarrow
\begin{cases} 2x^2-7x+6=0,\\ x
eq\cfrac{3}{2},\ x
eq-1.\end{cases}\Leftrightarrow
\begin{cases} (x-2)(2x-3)=0,\\ x
eq\cfrac{3}{2},\ x
eq-1.\end{cases}\Leftrightarrow x=2.$\\
2. $\cfrac{x+2}{2x-1}+\cfrac{x+1}{x+2}=\cfrac{x^2+8x+2}{2x^2+3x-2}\Leftrightarrow \cfrac{x^2+4x+4+2x^2+2x-x-1}{(2x-1)(x+2)}=\cfrac{x^2+8x+2}{(2x-1)(x+2)}\Leftrightarrow$\\$
\begin{cases} 3x^2+5x+3=x^2+8x+2,\\ x
eq\cfrac{1}{2},\ x
eq-2.\end{cases}\Leftrightarrow
\begin{cases} 2x^2-3x+1=0,\\ x
eq\cfrac{1}{2},\ x
eq-2.\end{cases}\Leftrightarrow
\begin{cases} (x-1)(2x-1)=0,\\ x
eq\cfrac{1}{2},\ x
eq-2.\end{cases}\Leftrightarrow x=1.$\\
3. $\sqrt{x-2}\cdot\sqrt{x-8}=4\Leftrightarrow \begin{cases} x^2-8x-2x+16=16,\\ x\geqslant8.\end{cases}
\Leftrightarrow \begin{cases} x(x-10)=0,\\ x\geqslant8.\end{cases}\Leftrightarrow x=10.$\\
4. $\sqrt{x-1}\cdot\sqrt{x-4}=2\Leftrightarrow \begin{cases} x^2-4x-x+4=4,\\ x\geqslant4.\end{cases}
\Leftrightarrow \begin{cases} x(x-5)=0,\\ x\geqslant4.\end{cases}\Leftrightarrow x=5.$\\
5. $||3+x|+2|=2x\Leftrightarrow|3+x|+2=2x\Leftrightarrow|3+x|=2x-2\Leftrightarrow\begin{cases}\left[\begin{array}{l} 3+x=2x-2,\\ 3+x=2-2x.\end{array}
ight.\\2x-2\geqslant0.\end{cases}\Leftrightarrow\begin{cases}\left[\begin{array}{l} x=5,\\ x=-\cfrac{1}{3}.\end{array}
ight.\\x\geqslant1.\end{cases}\Leftrightarrow x=5.$\\
6. $||2+x|+3|=3x\Leftrightarrow|2+x|+3=3x\Leftrightarrow|2+x|=3x-3\Leftrightarrow\begin{cases}\left[\begin{array}{l} 2+x=3x-3,\\ 2+x=3-3x.\end{array}
ight.\\3x-3\geqslant0.\end{cases}\Leftrightarrow\begin{cases}\left[\begin{array}{l} x=\cfrac{5}{2},\\ x=\cfrac{1}{4}.\end{array}
ight.\\x\geqslant1.\end{cases}\Leftrightarrow x=\cfrac{5}{2}.$\\
7. $\cfrac{z^2-z}{z^2-z+1}-\cfrac{z^2-z+2}{z^2-z-2}=1.$ Сделаем замену $z^2-z=t,$ тогда $\cfrac{t}{t+1}-\cfrac{t+2}{t-2}=1
\Leftrightarrow$\\$ \cfrac{t^2-2t-t^2-2t-t-2}{t^2-2t+t-2}=1\Leftrightarrow -5t-2=t^2-t-2\Leftrightarrow t=0.$ Тогда
$z^2-z=0,\ z(z-1)=0,\ z=0$ или $z=1.$\\
8. $\cfrac{x^2+2x+1}{x^2+2x+2}+\cfrac{x^2+2x+2}{x^2+2x+3}=\cfrac{7}{6}.$ Сделаем замену $x^2+2x+1=(x+1)^2=t\geqslant0,$ тогда $\cfrac{t}{t+1}+\cfrac{t+1}{t+2}=\cfrac{7}{6}\Leftrightarrow \cfrac{t^2+2t+t^2+2t+1}{t^2+2t+t+2}=\cfrac{7}{6}\Leftrightarrow
\cfrac{2t^2+4t+1}{t^2+3t+2}=\cfrac{7}{6}\Leftrightarrow 12t^2+24t+6=7t^2+21t+14\Leftrightarrow
5t^2+3t-8=0\Leftrightarrow t=1$ (так как второй корень $-\cfrac{8}{5}<0$). Тогда $(x+1)^2=1,\ x+1=\pm1,\ x=-2$ или $x=0.$\\
9. $\sqrt{x^2+8x-4}=\sqrt{4x-4}\Leftrightarrow\begin{cases} x^2+8x-4=4x-4,\\ 4x-4\geqslant0.\end{cases}
\Leftrightarrow\begin{cases} x(x+4)=0,\\ x\geqslant1.\end{cases}\Leftrightarrow x\in\{\varnothing\}.$\\
10. $\sqrt{x^2+9x-3}=\sqrt{3x-3}\Leftrightarrow\begin{cases} x^2+9x-3=3x-3,\\ 3x-3\geqslant0.\end{cases}
\Leftrightarrow\begin{cases} x(x+6)=0,\\ x\geqslant1.\end{cases}\Leftrightarrow x\in\{\varnothing\}.$\\
11. $|x-5|+|x-2|-|x+1|=-3\Leftrightarrow |x-5|+|x-2|=|x+1|-3\Leftrightarrow \left[\begin{array}{l}\begin{cases} 5-x-x+2=-x-1-3,\\ x\leqslant-1.\end{cases}\\
\begin{cases} 5-x-x+2=x+1-3,\\ -1<x\leqslant2.\end{cases}\\ \begin{cases} 5-x+x-2=x+1-3,\\ 2<x\leqslant5.\end{cases}\\
\begin{cases} x-5+x-2=x+1-3,\\ 5< x.\end{cases}\end{array}
ight.\Leftrightarrow \left[\begin{array}{l}\begin{cases} x=11,\\ x\leqslant-1.\end{cases}\\
\begin{cases} x=3,\\ -1<x\leqslant2.\end{cases}\\ \begin{cases} x=5,\\ 2<x\leqslant5.\end{cases}\\
\begin{cases} x=5,\\ 5< x.\end{cases}\end{array}
ight.\Leftrightarrow x=5.$\\
12. $|x-7|-|x|=2-|x+1|\Leftrightarrow|x-7|+|x+1|=|x|+2\Leftrightarrow \left[\begin{array}{l}\begin{cases} 7-x-x-1=-x+2,\\ x\leqslant-1.\end{cases}\\
\begin{cases} 7-x+x+1=-x+2,\\ -1<x\leqslant0.\end{cases}\\ \begin{cases} 7-x+x+1=x+2,\\ 0<x\leqslant7.\end{cases}\\
\begin{cases} x-7+x+1=x+2,\\ 7< x.\end{cases}\end{array}
ight.\Leftrightarrow \left[\begin{array}{l}\begin{cases} x=4,\\ x\leqslant-1.\end{cases}\\
\begin{cases} x=-6,\\ -1<x\leqslant0.\end{cases}\\ \begin{cases} x=6,\\ 0<x\leqslant7.\end{cases}\\
\begin{cases} x=8,\\ 7< x.\end{cases}\end{array}
ight.\Leftrightarrow x\in\{6; 8\}.$\\
13. $\cfrac{1}{x+2}-\cfrac{1}{x+4}=\cfrac{1}{x+3}-\cfrac{1}{x+1}\Leftrightarrow \cfrac{x+4-x-2}{x^2+4x+2x+8}=\cfrac{x+1-x-3}{x^2+x+3x+3}
\Leftrightarrow \cfrac{2}{x^2+6x+8}=\cfrac{-2}{x^2+4x+3}\Leftrightarrow \begin{cases} x^2+6x+8=-x^2-4x-3,\\ x
otin\{-4;-3;-2;-1\}.\end{cases}
\Leftrightarrow \begin{cases} 2x^2+10x+11=0,\\ x
otin\{-4;-3;-2;-1\}.\end{cases}\Leftrightarrow x\in \left\{\cfrac{-5-\sqrt{3}}{2}; \cfrac{-5+\sqrt{3}}{2}
ight\}.$\\
14. $\cfrac{3}{x-2}-\cfrac{4}{x-1}=\cfrac{1}{x-4}-\cfrac{2}{x-3}\Leftrightarrow \cfrac{3x-3-4x+8}{x^2-x-2x+2}=\cfrac{x-3-2x+8}{x^2-3x-4x+12}\Leftrightarrow
\cfrac{5-x}{x^2-3x+2}=\cfrac{5-x}{x^2-7x+12}$\\$\Leftrightarrow \left[\begin{array}{l} x=5,\\ \begin{cases} x^2-3x+2=x^2-7x+12,\\ x
otin\{1;2;3;4\}.\end{cases}\end{array}
ight.\Leftrightarrow x\in\left\{\cfrac{5}{2};5
ight\}.$\\
15. $(x+2)\sqrt{x^2-x-20}=6x+12\Leftrightarrow (x+2)\sqrt{x^2-x-20}=6(x+2).$ Если $x+2=0,$ то $x=-2,$ но тогда подкоренное выражение отрицательно, значит этот случай невозможен. Поэтому $\sqrt{x^2-x-20}=6,\ x^2-x-20=36,\ x^2-x-56=0,\ (x-8)(x+7)=0,\ x=-7$ или $x=8.$\\
16. $(x-1)\sqrt{x^2-x-6}=6x-6\Leftrightarrow(x-1)\sqrt{x^2-x-6}=6(x-1).$ Если $x-1=0,$ то $x=1,$ но тогда подкоренное выражение отрицательно, значит этот случай невозможен. Поэтому $\sqrt{x^2-x-6}=6,\ x^2-x-6=36,\ x^2-x-42=0,\ (x-7)(x+6)=0,\ x=-6$ или $x=7.$\\
17. $|7x-12|-|11-7x|=1\Leftrightarrow|7x-12|=|7x-11|+1\Leftrightarrow \left[\begin{array}{l}\begin{cases} 12-7x=11-7x+1,\\ x\leqslant\cfrac{11}{7}.\end{cases}\\
\begin{cases} 12-7x=7x-11+1,\\ \cfrac{11}{7}<x\leqslant\cfrac{12}{7}.\end{cases}\\ \begin{cases} 7x-12=7x-11+1,\\ \cfrac{12}{7}<x.\end{cases}\end{array}
ight.\Leftrightarrow \left[\begin{array}{l}\begin{cases} 0=0,\\ x\leqslant\cfrac{11}{7}.\end{cases}\\
\begin{cases} x=\cfrac{11}{7},\\ \cfrac{11}{7}<x\leqslant\cfrac{12}{7}.\end{cases}\\ \begin{cases} -12=-10,\\ \cfrac{12}{7}<x.\end{cases}\end{array}
ight.
\Leftrightarrow x\in \left(-\infty;\cfrac{11}{7}
ight].$\\
18. $|16-9x|-|9x-5|=11\Leftrightarrow|9x-16|=|9x-5|+11\Leftrightarrow \left[\begin{array}{l}\begin{cases} 16-9x=5-9x+11,\\ x\leqslant\cfrac{5}{9}.\end{cases}\\
\begin{cases} 16-9x=9x-5+11,\\ \cfrac{5}{9}<x\leqslant\cfrac{16}{9}.\end{cases}\\ \begin{cases} 9x-16=9x-5+11,\\ \cfrac{16}{9}<x.\end{cases}\end{array}
ight.\Leftrightarrow \left[\begin{array}{l}\begin{cases} 0=0,\\ x\leqslant\cfrac{5}{9}.\end{cases}\\
\begin{cases} x=\cfrac{5}{9},\\ \cfrac{5}{9}<x\leqslant\cfrac{16}{9}.\end{cases}\\ \begin{cases} -16=6,\\ \cfrac{16}{9}<x.\end{cases}\end{array}
ight.
\Leftrightarrow x\in \left(-\infty;\cfrac{5}{9}
ight].$\\
19. $min(2x^2-x-4; x^2+3x+1)=3x+12\Leftrightarrow\left[\begin{array}{l}\begin{cases}2x^2-x-4=3x+12,\\ 2x^2-x-4\leqslant x^2+3x+1.\end{cases}\\ \begin{cases}x^2+3x+1=3x+12,\\ x^2+3x+1\leqslant 2x^2-x-4.\end{cases}\end{array}
ight. \Leftrightarrow$\\$\left[\begin{array}{l}\begin{cases}2(x^2-2x-8)=0,\\ x^2-4x-5\leqslant 0.\end{cases}\\ \begin{cases} x^2=11,\\ x^2-4x-5\geqslant 0.\end{cases}\end{array}
ight.\Leftrightarrow\left[\begin{array}{l}\begin{cases}
\left[\begin{array}{l} x=4,\\ x=-2,\end{array}
ight.\\ x^2-4x-5\leqslant 0.\end{cases}\\ \begin{cases} x=\pm\sqrt{11},\\ x^2-4x-5\geqslant 0.\end{cases}\end{array}
ight.\Leftrightarrow x\in\{-\sqrt{11};4\}.$\\
20. $max(x^2+x-5;-2x^2+7x+4)=x-1\Leftrightarrow\left[\begin{array}{l}\begin{cases}x^2+x-5=x-1,\\ x^2+x-5\geqslant -2x^2+7x+4.\end{cases}\\ \begin{cases}-2x^2+7x+4=x-1,\\ -2x^2+7x+4\geqslant x^2+x-5.\end{cases}\end{array}
ight. \Leftrightarrow$\\$\left[\begin{array}{l}\begin{cases}x^2=4,\\ 3x^2-6x-9\geqslant 0.\end{cases}\\ \begin{cases} 2x^2-6x-5=0,\\ 3x^2-6x-9\leqslant0.\end{cases}\end{array}
ight.\Leftrightarrow\left[\begin{array}{l}\begin{cases}x=\pm2,\\ x^2-2x-3\geqslant 0.\end{cases}\\ \begin{cases} x=\cfrac{3\pm\sqrt{19}}{2},\\ x^2-2x-3\leqslant 0.\end{cases}\end{array}
ight.\Leftrightarrow x\in\left\{-2;\cfrac{3-\sqrt{19}}{2}
ight\}.$\\
21. $\cfrac{2}{x^2-x+1}=\cfrac{1}{x+1}+\cfrac{2x-1}{x^3+1}\Leftrightarrow \begin{cases} 2x+2=x^2-x+1+2x-1,\\ x
eq-1.\end{cases}
\Leftrightarrow \begin{cases} x^2-x-2=0,\\ x
eq-1.\end{cases}\Leftrightarrow$\\$\begin{cases} (x+1)(x-2)=0,\\ x
eq-1.\end{cases}\Leftrightarrow x=2.$\\
22. $\cfrac{2}{x+2}+\cfrac{1}{2}=-\cfrac{4}{x^2+2x}\Leftrightarrow\begin{cases} 4x+x^2+2x=-8,\\ x
eq0,\ x
eq-2.\end{cases}\Leftrightarrow
\begin{cases} x^2+6x+8=0,\\ x
eq0,\ x
eq-2.\end{cases}\Leftrightarrow\begin{cases} (x+2)(x+4)=0,\\ x
eq0,\ x
eq-2.\end{cases}$\\$\Leftrightarrow x=-4.$\\
23. $|2x^2-x+1|=x-1\Leftrightarrow \begin{cases} \left[\begin{array}{l}2x^2-x+1=x-1,\\ 2x^2-x+1=1-x.\end{array}
ight.\\ x-1\geqslant0.\end{cases}
\Leftrightarrow \begin{cases} \left[\begin{array}{l}2(x^2-x+1)=0,\\ x=0.\end{array}
ight.\\ x\geqslant1.\end{cases}\Leftrightarrow x\in\{\varnothing\}.$\\
24. $|2x^2-x+2|=x-2\Leftrightarrow \begin{cases} \left[\begin{array}{l}2x^2-x+2=x-2,\\ 2x^2-x+2=2-x.\end{array}
ight.\\ x-2\geqslant0.\end{cases}
\Leftrightarrow \begin{cases} \left[\begin{array}{l}2(x^2-x+2)=0,\\ x=0.\end{array}
ight.\\ x\geqslant2.\end{cases}\Leftrightarrow x\in\{\varnothing\}.$\\
25. $\sqrt{3x^2-6x+16}=2x-1\Leftrightarrow\begin{cases} 3x^2-6x+16=4x^2-4x+1,\\ 2x-1\geqslant0.\end{cases}
\Leftrightarrow\begin{cases} x^2+2x-15=0,\\ x\geqslant\cfrac{1}{2}.\end{cases}$\\$
\Leftrightarrow\begin{cases} (x+5)(x-3)=0,\\ x\geqslant\cfrac{1}{2}.\end{cases}
\Leftrightarrow x=3.$\\
26. $\sqrt{3x^2-11x+21}=2x-3\Leftrightarrow\begin{cases} 3x^2-11x+21=4x^2-12x+9,\\ 2x-3\geqslant0.\end{cases}
\Leftrightarrow\begin{cases} x^2-x-12=0,\\ x\geqslant\cfrac{3}{2}.\end{cases}$\\$
\Leftrightarrow\begin{cases} (x+3)(x-4)=0,\\ x\geqslant\cfrac{3}{2}.\end{cases}
\Leftrightarrow x=4.$\\
27. $\cfrac{2x+1}{1+x}=\cfrac{2}{x^2-1}\Leftrightarrow\cfrac{2x+1}{x+1}=\cfrac{2}{(x+1)(x-1)}\Leftrightarrow \begin{cases} 2x^2-2x+x-1=2,\\ x
eq-1\end{cases}
\Leftrightarrow \begin{cases} 2x^2-x-3=0,\\ x
eq-1\end{cases}$\\$\Leftrightarrow \begin{cases} (x+1)(2x-3)=0,\\ x
eq-1\end{cases}\Leftrightarrow x=\cfrac{3}{2}.$\\
28. $\cfrac{1-2x}{x-1}=-\cfrac{2}{x^2-1}\Leftrightarrow\cfrac{2x-1}{x-1}=\cfrac{2}{(x-1)(x+1)}\Leftrightarrow \begin{cases} 2x^2+2x-x-1=2,\\ x
eq1\end{cases}
\Leftrightarrow \begin{cases} 2x^2+x-3=0,\\ x
eq1\end{cases}$\\$\Leftrightarrow \begin{cases} (x-1)(2x+3)=0,\\ x
eq1\end{cases}\Leftrightarrow x=-\cfrac{3}{2}.$\\
29. $x^2+2x+\sqrt{x^2+2x+8}=12\Leftrightarrow\sqrt{x^2+2x+8}=12-(x^2+2x).$ Сделаем замену $t=x^2+2x,$ тогда $\sqrt{t+8}=12-t\Leftrightarrow
\begin{cases}t+8=t^2-24t+144,\\ 12-t\geqslant0.\end{cases}\Leftrightarrow
\begin{cases}t^2-25t+136=0,\\ t\leqslant12.\end{cases}\Leftrightarrow
\begin{cases}(t-8)(t-17)=0,\\ t\leqslant12.\end{cases}$\\$\Leftrightarrow t=8.$ Тогда $x^2+2x=8,\ x^2+2x-8=0,\ (x-2)(x+4)=0,\ x=-4$ или $x=2.$\\
30. $x^2-2x+\sqrt{x^2-2x+8}=12\Leftrightarrow\sqrt{x^2-2x+8}=12-(x^2-2x).$ Сделаем замену $t=x^2-2x,$ тогда $\sqrt{t+8}=12-t\Leftrightarrow
\begin{cases}t+8=t^2-24t+144,\\ 12-t\geqslant0.\end{cases}\Leftrightarrow
\begin{cases}t^2-25t+136=0,\\ t\leqslant12.\end{cases}\Leftrightarrow
\begin{cases}(t-8)(t-17)=0,\\ t\leqslant12.\end{cases}$\\$\Leftrightarrow t=8.$ Тогда $x^2-2x=8,\ x^2-2x-8=0,\ (x+2)(x-4)=0,\ x=-2$ или $x=4.$\\
31. $||x-2|-1|=x\Leftrightarrow\begin{cases}\left[\begin{array}{l}|x-2|-1=x,\\ |x-2|-1=-x.\end{array}
ight.\\ x\geqslant0.\end{cases}
\Leftrightarrow\begin{cases}\left[\begin{array}{l}|x-2|=x+1,\\ |x-2|=1-x.\end{array}
ight.\\ x\geqslant0.\end{cases}
\Leftrightarrow$\\$ \begin{cases}\left[\begin{array}{l}x-2=x+1,\\ x-2=-x-1,\\ \begin{cases}\left[\begin{array}{l}x-2=1-x,\\ x-2=x-1\end{array}
ight.\\ 1-x\geqslant0\end{cases}\end{array}
ight.\\ x\geqslant0.\end{cases}\Leftrightarrow x=\cfrac{1}{2}.$\\
32. $||x+2|-1|=-x\Leftrightarrow\begin{cases}\left[\begin{array}{l}|x+2|-1=-x,\\ |x+2|-1=x.\end{array}
ight.\\ -x\geqslant0.\end{cases}
\Leftrightarrow\begin{cases}\left[\begin{array}{l}|x+2|=1-x,\\ |x+2|=x+1.\end{array}
ight.\\ x\leqslant0.\end{cases}
\Leftrightarrow$\\$ \begin{cases}\left[\begin{array}{l}x+2=1-x,\\ x+2=x-1,\\ \begin{cases}\left[\begin{array}{l}x+2=x+1,\\ x+2=-x-1\end{array}
ight.\\ x+1\geqslant0\end{cases}\end{array}
ight.\\ x\leqslant0.\end{cases}\Leftrightarrow x=-\cfrac{1}{2}.$\\
33. $\begin{cases} x^2+3xy=1,\\ y-x=1.\end{cases}\Leftrightarrow\begin{cases} x^2+3x(x+1)=1,\\ y=x+1.\end{cases}
\Leftrightarrow\begin{cases} 4x^2+3x-1=0,\\ y=x+1.\end{cases}\Leftrightarrow\left[\begin{array}{l}\begin{cases} x=-1,\\ y=0.\end{cases}\\ \begin{cases} x=\cfrac{1}{4},\\ y=\cfrac{5}{4}.\end{cases}\end{array}
ight.$\\
34. $\begin{cases} x^2+3xy=1,\\ x-y=1.\end{cases}\Leftrightarrow\begin{cases} x^2+3x(x-1)=1,\\ y=x-1.\end{cases}
\Leftrightarrow\begin{cases} 4x^2-3x-1=0,\\ y=x-1.\end{cases}\Leftrightarrow\left[\begin{array}{l}\begin{cases} x=1,\\ y=0.\end{cases}\\ \begin{cases} x=-\cfrac{1}{4},\\ y=-\cfrac{5}{4}.\end{cases}\end{array}
ight.$\\
35. $\cfrac{44}{4-x^2}+\cfrac{2x+7}{x-2}=\cfrac{3-x}{x+2}\Leftrightarrow \begin{cases}44-(2x+7)(x+2)=(3-x)(2-x),\\ x
eq\pm2.\end{cases}
\Leftrightarrow$\\$ \begin{cases}44-2x^2-4x-7x-14=6-3x-2x+x^2,\\ x
eq\pm2.\end{cases}
\Leftrightarrow \begin{cases}3x^2+6x-24=0,\\ x
eq\pm2.\end{cases}
\Leftrightarrow \begin{cases}3(x-2)(x+4)=0,\\ x
eq\pm2.\end{cases}$\\$\Leftrightarrow x=-4.$\\
36. $\cfrac{x}{x+1}-\cfrac{9x+13}{x^2-2x-3}=\cfrac{5}{3-x}\Leftrightarrow \begin{cases} x^2-3x-9x-13=-5x-5,\\ x
eq-1,\ x
eq3.\end{cases}
\Leftrightarrow \begin{cases} x^2-7x-8=0,\\ x
eq-1,\ x
eq3.\end{cases}
\Leftrightarrow$\\$\begin{cases} (x+1)(x-8)=0,\\ x
eq-1,\ x
eq3.\end{cases}\Leftrightarrow x=8.$\\
37. $(x+1)\sqrt{1+4x-x^2}=x^2-1\Leftrightarrow (x+1)\sqrt{1+4x-x^2}=(x+1)(x-1).$ Если $x=-1,$ то подкоренное выражение отрицательно, значит этот корень не подходит и $\sqrt{1+4x-x^2}=x-1\Leftrightarrow\begin{cases}1+4x-x^2=x^2-2x+1,\\ x-1\geqslant0.\end{cases}
\Leftrightarrow\begin{cases}2x(x-3)=0,\\ x\geqslant1.\end{cases}\Leftrightarrow
x=3.$\\
38. $(x^2-8x)\sqrt{7-x}=x(x^2-9x+8)\Leftrightarrow x(x-8)\sqrt{7-x}=x(x-8)(x-1).$ При $x=0$ подкоренное выражение положительно, а при $x=8$ --- отрицательно, поэтому корнем является только $x=0.$ В другом случае $\sqrt{7-x}=x-1\Leftrightarrow \begin{cases} 7-x=x^2-2x+1,\\ x-1\geqslant0.\end{cases}
\Leftrightarrow \begin{cases} x^2-x-6=0,\\ x\geqslant1.\end{cases}
\Leftrightarrow \begin{cases} (x-3)(x+2)=0,\\ x\geqslant1.\end{cases}\Leftrightarrow x=3.$

Таким образом, $x\in\{0; 3\}.$\\
39. $\cfrac{1}{x^2+2x+4}+\cfrac{1}{x-2}=\cfrac{x^2-2x+4}{x^3-8}\Leftrightarrow \begin{cases}x-2+x^2+2x+4=x^2-2x+4,\\ x
eq2.\end{cases}
\Leftrightarrow \begin{cases}5x=2,\\ x
eq2.\end{cases}\Leftrightarrow$\\$ x=\cfrac{2}{5}.$\\
40. $\cfrac{x^2+3x+9}{x^3+27}-\cfrac{1}{x+3}=\cfrac{2}{x^2-3x+9}\Leftrightarrow \begin{cases}x^2+3x+9-x^2+3x-9=2x+6,\\ x
eq-3.\end{cases}
\Leftrightarrow\begin{cases} 4x=6,\\ x
eq-3.\end{cases}\Leftrightarrow x=\cfrac{3}{2}.$\\
41. $(x^2+3x+1)(x^2+3x+3)+1=0.$ Сделаем замену $t=x^2+3x+1,$ тогда $t(t+2)+1=0,\ t^2+2t+1=0,\ (t+1)^2=0,\ t=-1.$ Поэтому $x^2+3x+1=-1,\
x^2+3x+2=0,\ (x+1)(x+2)=0,\ x=-2$ или $x=-1.$\\
42. $(x^2-5x+2)(x^2-5x-1)=28.$ Сделаем замену $t=x^2-5x-1,$ тогда $t(t+3)=28,\ t^2+3t-28=0,\ (t+7)(t-4)=0, t=-7$ или $t=4.$ Поэтому $x^2-5x-1=-7,\ x^2-5x+6=0,\
(x-3)(x-2)=0,\ x=2$ или $x=3;$ или $x^2-5x-1=4,\ x^2-5x-5=0,\ x=\cfrac{5\pm3\sqrt{5}}{2}.$\\
43. $x-3+4\sqrt{x-3}=12.$ Сделаем замену $t=\sqrt{x-3}\geqslant0,$ тогда $t^2+4t=12,\ t^2+4t-12=0,\ (t+6)(t-2)=0,\ t=2.$ Поэтому $\sqrt{x-3}=2,\ x-3=4,\ x=7.$\\
44. $x+2-13\sqrt{x+2}=-42.$ Сделаем замену $t=\sqrt{x+2}\geqslant0,$ тогда $t^2-13t=-42,\ t^2-13t+42=0,\ (t-6)(t-7)=0,\ t=6$ или $t=7.$ Поэтому $\sqrt{x+2}=6,\ x+2=36,\ x=34$ или $\sqrt{x+2}=7,\ x+2=49,\ x=47.$\\
45. $\cfrac{11x-x^2-8}{2x^2-x-3}=\cfrac{x+1}{2x-3}+\cfrac{x}{x+1}\Leftrightarrow \begin{cases} 11x-x^2-8=x^2+2x+1+2x^2-3x,\\ x
eq\cfrac{3}{2},\ x
eq-1.\end{cases}$\\$
\Leftrightarrow \begin{cases} (2x-3)^2=0,\\ x
eq\cfrac{3}{2},\ x
eq-1.\end{cases}\Leftrightarrow x\in\{\varnothing\}.$\\
46. $\cfrac{9x-x^2+2}{2x^2+3x-2}=\cfrac{x+2}{2x-1}+\cfrac{x+1}{x+2}\Leftrightarrow\begin{cases} 9x-x^2+2=x^2+4x+4+2x^2+2x-x-1,\\ x
eq\cfrac{1}{2},\ x
eq-2.\end{cases}$\\$\Leftrightarrow\begin{cases} (2x-1)^2=0,\\ x
eq\cfrac{1}{2},\ x
eq-2.\end{cases}\Leftrightarrow x\in\{\varnothing\}.$\\
47. $(x-1)^2+3|x-1|-4=0.$ Сделаем замену $t=|x-1|\geqslant0,$ тогда $t^2+3t-4=0,\ (t+4)(t-1)=0,\ t=1.$ Поэтому $|x-1|=1,\ x-1=\pm1,\ x=0$ или $x=2.$\\
48. $(x-2)^2+|x-2|-6=0.$ Сделаем замену $t=|x-2|\geqslant0,$ тогда $t^2+t-6=0,\ (t+3)(t-2)=0,\ t=2.$ Поэтому $|x-2|=2,\ x-2=\pm2,\ x=0$ или $x=4.$\\
49. $\cfrac{x}{\sqrt{1-x^2}}-2\cfrac{\sqrt{1-x^2}}{x}=-1.$ Сделаем замену $t=\cfrac{x}{\sqrt{1-x^2}},$ тогда $t-\cfrac{2}{t}=-1,\ t^2-2=-t,\ t^2+t-2=0,\ (t+2)(t-1)=0,\ t=-2$ или $t=1.$ В первом случае $\cfrac{x}{\sqrt{1-x^2}}=-2,\ \sqrt{1-x^2}=-\cfrac{x}{2},$\\$ \begin{cases} 1-x^2=\cfrac{x^2}{4},\\ -\cfrac{x}{2}\geqslant 0.\end{cases}\Leftrightarrow\begin{cases} x^2=\cfrac{4}{5},\\ x\leqslant 0.\end{cases}\Leftrightarrow x=-\cfrac{2\sqrt{5}}{5}.$ Во втором случае $\cfrac{x}{\sqrt{1-x^2}}=1,\ \sqrt{1-x^2}=x,$\\$\begin{cases} 1-x^2=x^2,\\ x\geqslant 0.\end{cases}\Leftrightarrow
\begin{cases} x^2=\cfrac{1}{2},\\ x\geqslant 0.\end{cases}\Leftrightarrow x=\cfrac{\sqrt{2}}{2}.$\\
50. $\cfrac{x-1}{\sqrt{x}}-2\cfrac{\sqrt{x}}{x-1}=1.$ Сделаем замену $t=\cfrac{x-1}{\sqrt{x}},$ тогда $t-\cfrac{2}{t}=1,\ t^2-2=t,\ t^2-t-2=0,\ (t-2)(t+1)=0,\ t=-1$ или $t=2.$ В первом случае $\cfrac{x-1}{\sqrt{x}}=-1,\ \sqrt{x}=1-x,$\\$ \begin{cases} x=1-2x+x^2,\\ 1-x\geqslant0.\end{cases}\Leftrightarrow
\begin{cases} x=1-2x+x^2,\\ 1-x\geqslant0.\end{cases}\Leftrightarrow\begin{cases} x^2-3x+1=0,\\ x\leqslant1.\end{cases}\Leftrightarrow x=\cfrac{3-\sqrt{5}}{2}.$
Во втором случае $\cfrac{x-1}{\sqrt{x}}=2,\ \sqrt{x}=\cfrac{x-1}{2},\ \begin{cases} x=\cfrac{x^2-2x+1}{4},\\ \cfrac{x-1}{2}\geqslant0.\end{cases}\Leftrightarrow
\begin{cases} x^2-6x+1=0,\\ x\geqslant1.\end{cases}\Leftrightarrow x=3+2\sqrt{2}.$\\
51. $\sqrt{7-x}=x-1\Leftrightarrow \begin{cases} 7-x=x^2-2x+1,\\ x-1\geqslant0.\end{cases}\Leftrightarrow \begin{cases} x^2-x-6=0,\\ x\geqslant1.\end{cases}
\Leftrightarrow \begin{cases} (x-3)(x+2)=0,\\ x\geqslant1.\end{cases}\Leftrightarrow$\\$x=3.$\\
52. $x-\sqrt{x+1}=5\Leftrightarrow \sqrt{x+1}=x-5\Leftrightarrow \begin{cases} x+1=x^2-10x+25,\\ x-5\geqslant0.\end{cases}
\Leftrightarrow \begin{cases} x^2-11x+24=0,\\ x\geqslant5.\end{cases}\Leftrightarrow \begin{cases} (x-8)(x-3)=0,\\ x\geqslant5.\end{cases}
\Leftrightarrow x=8.$\\
53. $|x+3|-|x-5|+|2x-5|=6\Leftrightarrow |x+3|+|2x-5|=|x-5|+6\Leftrightarrow \left[\begin{array}{l}\begin{cases} -x-3-2x+5=-x+5+6,\\ x\leqslant-3.\end{cases}\\
\begin{cases} x+3-2x+5=-x+5+6,\\ -3<x\leqslant\cfrac{5}{2}.\end{cases}\\ \begin{cases} x+3+2x-5=-x+5+6,\\ \cfrac{5}{2}<x\leqslant5.\end{cases}\\
\begin{cases} x+3+2x-5=x-5+6,\\ 5< x.\end{cases}\end{array}
ight.\Leftrightarrow \left[\begin{array}{l}\begin{cases} x=-\cfrac{9}{2},\\ x\leqslant-3.\end{cases}\\
\begin{cases} 8=11,\\ -3<x\leqslant\cfrac{5}{2}.\end{cases}\\ \begin{cases} x=\cfrac{13}{4},\\ \cfrac{5}{2}<x\leqslant5.\end{cases}\\
\begin{cases} x=\cfrac{3}{2},\\ 5< x.\end{cases}\end{array}
ight.\Leftrightarrow x\in\left\{-\cfrac{9}{2};\cfrac{13}{4}
ight\}.$\\
54. $|x-2|-|x+4|+|2x-3|=1\Leftrightarrow|x-2|+|2x-3|=|x+4|+1\Leftrightarrow \left[\begin{array}{l}\begin{cases} -x+2-2x+3=-x-4+1,\\ x\leqslant-4.\end{cases}\\
\begin{cases} -x+2-2x+3=x+4+1,\\ -4<x\leqslant\cfrac{3}{2}.\end{cases}\\ \begin{cases} -x+2+2x-3=x=4+1,\\ \cfrac{3}{2}<x\leqslant2.\end{cases}\\
\begin{cases} x-2+2x-3=x+4+1,\\ 2<x.\end{cases}\end{array}
ight.\Leftrightarrow \left[\begin{array}{l}\begin{cases} x=4,\\ x\leqslant-4.\end{cases}\\
\begin{cases} x=0,\\ -4<x\leqslant\cfrac{3}{2}.\end{cases}\\ \begin{cases} -1=5,\\ \cfrac{3}{2}<x\leqslant2.\end{cases}\\
\begin{cases} x=5,\\ 2< x.\end{cases}\end{array}
ight.\Leftrightarrow x\in\{0; 5\}.$\\
55. $\cfrac{6}{(x+1)(x+2)}+\cfrac{8}{(x-1)(x+4)}=1\Leftrightarrow \cfrac{6}{x^2+3x+2}+\cfrac{8}{x^2+3x-4}=1.$ Сделаем замену $t=x^2+3x-4,$ тогда $\cfrac{6}{t+6}+\cfrac{8}{t}=1,\ 6t+8t+48=t^2+6t,\ t^2-8t-48=0,\ (t+4)(t-12)=0,\ t=-4$ или $t=12.$ В первом случае $x^2+3x-4=-4,\ x(x+3)=0,\ x=-3$ или $x=0.$ Во втором случае $x^2+3x-4=12,\ x^2+3x-16=0,\ x=\cfrac{-3\pm\sqrt{73}}{2}.$\\
56. $\cfrac{16}{(x+6)(x-1)}-\cfrac{20}{(x+2)(x+3)}=1\Leftrightarrow\cfrac{16}{x^2+5x-6}-\cfrac{20}{x^2+5x+6}=1.$ Сделаем замену $t=x^2+5x-6,$ тогда $\cfrac{16}{t}-\cfrac{20}{t+12}=1,\ 16t+192-20t=t^2+12t,\ t^2+16t-192=0,\ (t-8)(t+24)=0,\ t=-24$ или $t=8.$ В первом случае $x^2+5x-6=-24,\ x^2+5x+18=0,\ x\in\{\varnothing\}.$ Во втором случае $x^2+5x-6=8,\ x^2+5x-14=0,\ (x+7)(x-2)=0,\ x=-7$ или $x=2.$\\
57. $\cfrac{x-4}{\sqrt{x}}-\cfrac{\sqrt{x}}{x-4}=\cfrac{8}{3}.$ Сделаем замену $t=\cfrac{x-4}{\sqrt{x}},$ тогда $t-\cfrac{1}{t}=\cfrac{8}{3},\ t^2-1=\cfrac{8}{3}t,\
 t^2-\cfrac{8}{3}t-1=0,\ t=-\cfrac{1}{3}$ или $t=3.$ В первом случае $\cfrac{x-4}{\sqrt{x}}=-\cfrac{1}{3},\ \sqrt{x}=12-3x,\ \begin{cases} x=144-72x+9x^2,\\ 12-3x\geqslant0.\end{cases}\Leftrightarrow\begin{cases} 9x^2-73x+144=0,\\ x\leqslant4.\end{cases}\Leftrightarrow x=\cfrac{73-\sqrt{145}}{18}.$ Во втором случае $
\cfrac{x-4}{\sqrt{x}}=3,\ \sqrt{x}=\cfrac{x-4}{3},$\\$ \begin{cases} x=\cfrac{x^2-8x+16}{9},\\ \cfrac{x-4}{3}\geqslant0.\end{cases}\Leftrightarrow
\begin{cases} x^2-17x+16=0,\\ x\geqslant4.\end{cases}\Leftrightarrow\begin{cases} (x-16)(x-1)=0,\\ x\geqslant4.\end{cases}\Leftrightarrow x=16.$\\
58. $\sqrt{12-\cfrac{2}{x}-\cfrac{4x+1}{x+4}}=\cfrac{2}{x}+\cfrac{4x+1}{4+x}.$ Сделаем замену $t=\cfrac{2}{x}+\cfrac{4x+1}{4+x},$ тогда $\sqrt{12-t}=t\Leftrightarrow$\\$ \begin{cases} 12-t=t^2,\\ t\geqslant0.\end{cases}\Leftrightarrow \begin{cases} t^2+t-12=0,\\ t\geqslant0.\end{cases}
\Leftrightarrow \begin{cases} (t+4)(t-3)=0,\\ t\geqslant0.\end{cases}\Leftrightarrow t=3.$ Поэтому $\cfrac{2}{x}+\cfrac{4x+1}{4+x}=3,$\\$
8+2x+4x^2+x=12x+3x^2,\ x^2-9x+8=0,\ (x-8)(x-1)=0,\ x=1$ или $x=8.$\\
59. $2x^2+3x-17=2(2-\sqrt{7})^2+3(2-\sqrt{7})-17.$ Пусть $2-\sqrt{7}=a,$ тогда $2x^2+3x-17=2a^2+3a-17,\ 2(x-a)(x+a)+3(x-a)=0,\ (x-a)(2x+2a+3)=0,\ x=a=2-\sqrt{7}$ или $x=-\cfrac{3}{2}-a=-\cfrac{3}{2}-2+\sqrt{7}=\sqrt{7}-\cfrac{7}{2}.$\\
60. $\sqrt{2x-1}=x-2\Leftrightarrow\begin{cases} 2x-1=x^2-4x+4,\\ x-2\geqslant0.\end{cases}\Leftrightarrow\begin{cases} x^2-6x+5=0,\\ x\geqslant2.\end{cases}
\Leftrightarrow\begin{cases} (x-5)(x-1)=0,\\ x\geqslant2.\end{cases}\Leftrightarrow x=5.$\\
61. $4x^2-9x-11=4(\sqrt{2}+\sqrt{3})^2-9(\sqrt{2}+\sqrt{3})-11.$ Пусть $\sqrt{2}+\sqrt{3}=a,$ тогда $4x^2-9x-11=4a^2-9a-11,\ 4(x-a)(x+a)-9(x-a)=0,\
(x-a)(4x+4a-9)=0,\ x=a=\sqrt{2}+\sqrt{3}$ или $x=\cfrac{9}{4}-a=\cfrac{9}{4}-\sqrt{2}-\sqrt{3}.$\\
62. $\sqrt{1+4x}=x+1\Leftrightarrow\begin{cases}1+4x=x^2+2x+1,\\ x+1\geqslant0.\end{cases}
\Leftrightarrow\begin{cases}x(x-2),\\ x\geqslant-1.\end{cases}\Leftrightarrow x\in\{0; 2\}.$\\
63. $\begin{cases}x^2-y=\cfrac{3}{4},\\ y^2+x=0,75.\end{cases}$ Так как $\cfrac{3}{4}=0,75,$ имеем $x^2-y=y^2+x,\ (x-y)(x+y)-(x+y)=0,\ (x+y)(x-y-1)=0,\ y=-x$ или $y=x-1.$ В первом случае $x^2+x=\cfrac{3}{4},\ x^2+x-\cfrac{3}{4}=0,\ x=\cfrac{1}{2}$ и $y=-\cfrac{1}{2}$ или $x=-\cfrac{3}{2}$ и $y=\cfrac{3}{2}.$ Во втором случае $x^2-x+1=\cfrac{3}{4},\ x^2-x+\cfrac{1}{4}=0,\ \left(x-\cfrac{1}{2}
ight)^2=0,\ x=\cfrac{1}{2}$ и $y=\cfrac{1}{2}-1=-\cfrac{1}{2}.$\\
64. $\begin{cases} \cfrac{6}{x+y}+\cfrac{5}{x-y}=7,\\ \cfrac{3}{x+y}-\cfrac{2}{x-y}=-1.\end{cases}$ Пусть $\cfrac{1}{x+y}=a,\ \cfrac{1}{x-y}=b,$ тогда
$\begin{cases} 6a+5b=7,\\ 3a-2b=-1.\end{cases}\Leftrightarrow$\\$ \begin{cases} 12a+10b=14,\\ 15a-10b=-5.\end{cases}\Leftrightarrow
\begin{cases} 27a=9,\\ 15a-10b=-5.\end{cases}\Leftrightarrow \begin{cases} a=\cfrac{1}{3},\\ b=1.\end{cases}$ Значит,
$\begin{cases} x+y=3,\\ x-y=1.\end{cases}\Leftrightarrow\begin{cases} 2x=4,\\ x-y=1.\end{cases}\Leftrightarrow\begin{cases} x=2,\\ y=1.\end{cases}$\\
65. $\left(\cfrac{x^2-3x+2}{x}
ight)^2-x=\cfrac{2-x}{x}\Leftrightarrow\left(x+\cfrac{2}{x}-3
ight)^2=x+\cfrac{2}{x}-1.$ Сделаем замену $t=x+\cfrac{2}{x},$ тогда
$(t-3)^2=t-1,\ t^2-6t+9=t-1,\ t^2-7t+10=0,\ (t-5)(t-2)=0,\ t=2$ или $t=5.$ В первом случае $x+\cfrac{2}{x}=2,\ x^2+2=2x,\ x^2-2x+2=0,\ x\in\{\varnothing\}.$ Во втором случае $x+\cfrac{2}{x}=5,\ x^2+2=5x,\ x^2-5x+2=0,\ x=\cfrac{5\pm\sqrt{17}}{2}.$\\
66. $\left(\cfrac{x^2-4x+7}{x}
ight)^2-x=\cfrac{7+2x}{x}\Leftrightarrow\left(x+\cfrac{7}{x}-4
ight)^2=x+\cfrac{7}{x}+2.$ Сделаем замену $t=x+\cfrac{7}{x},$ тогда
$(t-4)^2=t+2,\ t^2-8t+16=t+2,\ t^2-9t+14=0,\ (t-2)(t-7)=0,\ t=2$ или $t=7.$ В первом случае $x+\cfrac{7}{x}=2,\ x^2+7=2x,\ x^2-2x+7=0,\ x\in\{\varnothing\}.$ Во втором случае $x+\cfrac{7}{x}=7,\ x^2+7=7x,\ x^2-7x+7=0,\ x=\cfrac{7\pm\sqrt{21}}{2}.$\\
67. $\sqrt{x-1}=x-3\Leftrightarrow\begin{cases}x-1=x^2-6x+9,\\x-3\geqslant0.\end{cases}\Leftrightarrow\begin{cases}x^2-7x+10=0,\\x\geqslant3.\end{cases}
\Leftrightarrow\begin{cases}(x-5)(x-2)=0,\\x\geqslant3.\end{cases}\Leftrightarrow x=5.$\\
68. $\sqrt{x+2}=x\Leftrightarrow\begin{cases}x+2=x^2,\\x\geqslant0.\end{cases}\Leftrightarrow\begin{cases}x^2-x-2=0,\\x\geqslant0.\end{cases}
\Leftrightarrow\begin{cases}(x-2)(x+1)=0,\\x\geqslant0.\end{cases}\Leftrightarrow x=2.$\\
69. $|3-x|+|2x-5|=6\Leftrightarrow |x-3|+|2x-5|=6\Leftrightarrow \left[\begin{array}{l}\begin{cases} -x+3-2x+5=6,\\ x\leqslant\cfrac{5}{2}.\end{cases}\\
\begin{cases} 3-x+2x-5=6,\\ \cfrac{5}{2}<x\leqslant3.\end{cases}\\ \begin{cases} x-3+2x-5=6,\\ 3<x.\end{cases}\end{array}
ight.
\Leftrightarrow \left[\begin{array}{l}\begin{cases} x=\cfrac{2}{3},\\ x\leqslant\cfrac{5}{2}.\end{cases}\\
\begin{cases} x=8,\\ \cfrac{5}{2}<x\leqslant3.\end{cases}\\ \begin{cases} x=\cfrac{14}{3},\\ 3<x.\end{cases}\end{array}
ight.
\Leftrightarrow x\in\left\{\cfrac{2}{3};\cfrac{14}{3}
ight\}.$\\
70. $|2-x|+|2x-3|=1\Leftrightarrow |x-2|+|2x-3|=1\Leftrightarrow \left[\begin{array}{l}\begin{cases} -x+2-2x+3=1,\\ x\leqslant\cfrac{3}{2}.\end{cases}\\
\begin{cases} 2-x+2x-3=1,\\ \cfrac{3}{2}<x\leqslant2.\end{cases}\\ \begin{cases} x-2+2x-3=1,\\ 2<x.\end{cases}\end{array}
ight.
\Leftrightarrow \left[\begin{array}{l}\begin{cases} x=\cfrac{4}{3},\\ x\leqslant\cfrac{3}{2}.\end{cases}\\
\begin{cases} x=2,\\ \cfrac{3}{2}<x\leqslant2.\end{cases}\\ \begin{cases} x=2,\\ 2<x.\end{cases}\end{array}
ight.
\Leftrightarrow x\in\left\{\cfrac{4}{3};2
ight\}.$\\
71. $\cfrac{2x}{x-1}+\cfrac{3}{x-2}=6\Leftrightarrow \begin{cases}2x^2-4x+3x-3=6x^2-12x-6x+12,\\ x
eq1,\ x
eq2.\end{cases}
\Leftrightarrow \begin{cases}4x^2-17x+15=0,\\ x
eq1,\ x
eq2.\end{cases}
\Leftrightarrow \begin{cases}(4x-5)(x-3),\\ x
eq1,\ x
eq2.\end{cases}\Leftrightarrow x\in\left\{\cfrac{5}{4};3
ight\}.$\\
72. $\cfrac{4}{x+1}+\cfrac{3x}{x-2}=-1\Leftrightarrow\begin{cases}4x-8+3x^2+3x=-x^2+2x-x+2,\\ x
eq-1,\ x
eq2.\end{cases}
\Leftrightarrow\begin{cases}4x^2+6x-10=0,\\ x
eq-1,\ x
eq2.\end{cases}
\Leftrightarrow\begin{cases}2(2x+5)(x-1)=0,\\ x
eq-1,\ x
eq2.\end{cases}\Leftrightarrow x\in\left\{-\cfrac{5}{2};1
ight\}.$\\
73. $\sqrt{2x+5}-\sqrt{2x}=1\Big|\cdot(\sqrt{2x+5}+\sqrt{2x})\Leftrightarrow 2x+5-2x=\sqrt{2x+5}+\sqrt{2x},\ \sqrt{2x+5}+\sqrt{2x}=5.$ Тогда имеем место система уравнений $\begin{cases}\sqrt{2x+5}-\sqrt{2x}=1,\\ \sqrt{2x+5}+\sqrt{2x}=5.\end{cases}\Rightarrow 2\sqrt{2x+5}=6,\ 2x+5=9,\ x=2.$\\
74. $2\sqrt{x-3}+\sqrt{x+1}=2.$ Чтобы выражения под корнями были неотрицательны, необходимо выполнение условия $x\geqslant3.$ При $x=3$ левая часть уравнения равна 2, а при увеличении значения $x$ она увеличивается, значит $x=3$ --- единственный корень.\\
75. $\cfrac{x^2+4}{x}+\cfrac{x}{x^2+3x+4}+\cfrac{11}{2}=0\Leftrightarrow x+\cfrac{4}{x}+\cfrac{1}{x+\cfrac{4}{x}+3}+\cfrac{11}{2}=0.$ Сделаем замену $t=x+\cfrac{4}{x},$ тогда $t+\cfrac{1}{t+3}+\cfrac{11}{2}=0,\ t^2+3t+1+\cfrac{11}{2}t+\cfrac{33}{2}=0,\ t^2+\cfrac{17}{2}t+\cfrac{35}{2}=0,\
2t^2+17t+35=0,\ (2t+7)(t+5)=0,\ t=-5$ или $t=-\cfrac{7}{2}.$ В первом случае $x+\cfrac{4}{x}=-5,\ x^2+4=-5x,\ x^2+5x+4=0,\ (x+4)(x+1)=0,\ x=-4$ или $x=-1.$ Во втором случае $x+\cfrac{4}{x}=-\cfrac{7}{2},\ x^2+4=-\cfrac{7}{2}x,\ x^2+\cfrac{7}{2}x+4=0,\ x\in\{\varnothing\}.$\\
76. $\cfrac{4(x^2+1)}{x^2-10x+1}-\cfrac{5x}{x^2+1}+\cfrac{7}{2}=0\Leftrightarrow\cfrac{4}{1-10\cfrac{x}{x^2+1}}-5\cfrac{x}{x^2+1}+\cfrac{7}{2}=0.$ Сделаем замену $t=\cfrac{x}{x^2+1},$ тогда $\cfrac{4}{1-10t}-5t+\cfrac{7}{2}=0\Leftrightarrow 4-5t+50t^2+\cfrac{7}{2}-35t=0,\
50t^2-40t+\cfrac{15}{2}=0,\ 20t^2-16t+3=0,\ (2t-1)(10t-3)=0,\ t=\cfrac{1}{2}$ или $t=\cfrac{3}{10}.$ В первом случае $\cfrac{x}{x^2+1}=\cfrac{1}{2},\
2x=x^2+1,\ x^2-2x+1=0,\ (x-1)^2=0,\ x=1.$ Во втором случае $\cfrac{x}{x^2+1}=\cfrac{3}{10},\ 10x=3x^2+3,\ 3x^2-10x+3=0,\ (x-3)(3x-1)=0,\ x=\cfrac{1}{3}$ или $x=3.$\\
77. $(x+1)(x^2+5x+6)=x+2\Leftrightarrow(x+1)(x+2)(x+3)=x+2\Leftrightarrow(x+2)(x^2+3x+x+3-1)=0\Leftrightarrow(x+2)(x^2+4x+2)=0\Leftrightarrow
x\in\{-2-\sqrt{2}; -2; \sqrt{2}-2\}.$\\
78. $(x+3)(x^2+3x+2)=x+1\Leftrightarrow(x+3)(x+2)(x+1)=x+1\Leftrightarrow (x+1)(x^2+2x+3x+6-1)=0\Leftrightarrow(x+1)(x^2+5x+5)=0\Leftrightarrow
x\in\left\{\cfrac{-5-\sqrt{5}}{2}; -1; \cfrac{-5+\sqrt{5}}{2}
ight\}.$\\
79. $(x^2-x-6)\sqrt{3x-2}=0\Leftrightarrow (x-3)(x+2)\sqrt{3x-2}=0\Leftrightarrow x\in\left\{\cfrac{2}{3};3
ight\},$ так как один из множителей должен быть равен нулю, но подкоренное выражение при этом должно быть неотрицательно.\\
80. $(x^2-x-2)\sqrt{2x+1}=0\Leftrightarrow (x-2)(x+1)\sqrt{2x+1}=0\Leftrightarrow x\in\left\{-\cfrac{1}{2};2
ight\},$ так как один из множителей должен быть равен нулю, но подкоренное выражение при этом должно быть неотрицательно.\\
81. $\sqrt{12-x}=-x\Leftrightarrow \begin{cases} 12-x=x^2,\\ -x\geqslant0.\end{cases}\Leftrightarrow \begin{cases} x^2+x-12=0,\\ x\leqslant0.\end{cases}
\Leftrightarrow \begin{cases} (x+4)(x-3)=0,\\ x\leqslant0.\end{cases}\Leftrightarrow x=-4.$\\
82. $\sqrt{x+2}=-x\Leftrightarrow \begin{cases} x+2=x^2,\\ -x\geqslant0.\end{cases}\Leftrightarrow \begin{cases} x^2-x-2=0,\\ x\leqslant0.\end{cases}
\Leftrightarrow \begin{cases} (x+1)(x-2)=0,\\ x\leqslant0.\end{cases}\Leftrightarrow x=-1.$\\
83. $(3x-2)(x-1)=4(x-1)^2\Leftrightarrow (x-1)(3x-2-4x+4)=0\Leftrightarrow (x-1)(2-x)=0\Leftrightarrow x=1$ или $x=2.$\\
84. $(3x+2)(x+1)=2(x+1)^2\Leftrightarrow(x+1)(3x+2-2x-2)=0\Leftrightarrow (x+1)x=0\Leftrightarrow x=-1$ или $x=0.$\\
85. $\cfrac{5}{x^2+2x+4}=\cfrac{1}{x-2}-\cfrac{4x+4}{x^3-8}\Leftrightarrow \begin{cases} 5x-10=x^2+2x+4-4x-4,\\ x
eq2.\end{cases}
\Leftrightarrow \begin{cases} x^2-7x+10=0,\\ x
eq2.\end{cases}\Leftrightarrow \begin{cases} (x-2)(x-5)=0,\\ x
eq2.\end{cases}\Leftrightarrow
x=5.$\\
86. $\cfrac{4}{x^2+3x+9}=\cfrac{1}{x-3}-\cfrac{6x+9}{x^3-27}\Leftrightarrow \begin{cases} 4x-12=x^2+3x+9-6x-9,\\ x
eq3.\end{cases}
\Leftrightarrow \begin{cases} x^2-7x+12=0,\\ x
eq3.\end{cases}\Leftrightarrow \begin{cases} (x-3)(x-4)=0,\\ x
eq3.\end{cases}\Leftrightarrow
x=4.$\\
87. $\cfrac{2x-7}{x^2-9x+14}-\cfrac{1}{x^2-3x+2}=\cfrac{1}{x-1}\Leftrightarrow
\cfrac{2x-7}{(x-2)(x-7)}-\cfrac{1}{(x-2)(x-1)}=\cfrac{1}{x-1}\Leftrightarrow$\\$
\begin{cases} 2x^2-7x-2x+7-x+7=x^2-7x-2x=14,\\
x
otin\{1;2;7\}.\end{cases}\Leftrightarrow
\begin{cases} x^2-x=0,\\
x
otin\{1;2;7\}.\end{cases}\Leftrightarrow
\begin{cases} x(x-1)=0,\\
x
otin\{1;2;7\}.\end{cases}\Leftrightarrow x=0.$\\
88. $\cfrac{2x+7}{x^2+5x-6}+\cfrac{3}{x^2+9x+18}=\cfrac{1}{x+3}\Leftrightarrow
\cfrac{2x+7}{(x+6)(x-1)}+\cfrac{3}{(x+6)(x+3)}=\cfrac{1}{x+3}\Leftrightarrow$\\$
\begin{cases} 2x^2+7x+6x+21+3x-3=x^2-x+6x-6,\\
x
otin\{-6;-3;1\}.\end{cases}\Leftrightarrow
\begin{cases} x^2+11x+24=0,\\
x
otin\{-6;-3;1\}.\end{cases}\Leftrightarrow
\begin{cases} (x+8)(x+3),\\
x
otin\{-6;-3;1\}.\end{cases}$\\$\Leftrightarrow x=-8.$\\
89. $\cfrac{1}{|x^2-5x+6|}=\cfrac{|x-1,5|}{x^2-5x+6}.$ У левой дроби положительны и числитель, и знаменатель, а у правой точно положителен числитель, а значит должен быть положителен и знаменатель. Но тогда $|x^2-5x+6|=x^2-5x+6>0$ и $1=|x-1,5|,\ x-1,5=\pm1,\ x=0,5$ или $x=2,5.$ Проверим найденные корни: $0,5^2-5\cdot0,5+6>0,$ а $2,5^2-5\cdot2,5+6<0,$ поэтому единственным корнем уравнения является $x=0,5.$\\
90. $\cfrac{|2x-1|}{8-x-x^2}-\cfrac{4}{|x^2+x-8|}=0\Leftrightarrow\cfrac{|2x-1|}{8-x-x^2}=\cfrac{4}{|x^2+x-8|}.$ У правой дроби положительны и числитель, и знаменатель, а у левой точно положителен числитель, а значит должен быть положителен и знаменатель. Но тогда $|x^2+x-8|=8-x-x^2>0$ и $|2x-1|=4,\ 2x-1=\pm4,\ x=-\cfrac{3}{2}$ или $x=\cfrac{5}{2}.$ Проверим найденные корни: $8+\cfrac{3}{2}-\cfrac{9}{4}>0,$ а $8-\cfrac{5}{2}-\cfrac{25}{4}<0,$ поэтому единственным корнем уравнения является $x=-\cfrac{3}{2}.$\\
91. $\sqrt{5x+1}=x-1\Leftrightarrow\begin{cases}5x+1=x^2-2x+1,\\ x-1\geqslant0.\end{cases}
\Leftrightarrow\begin{cases}x(x-7),\\ x\geqslant1.\end{cases}\Leftrightarrow x=7.$\\
92. $|3x+2|=|7x-4|\Leftrightarrow \left[\begin{array}{l} 3x+2=7x-4,\\ 3x+2=-7x+4.\end{array}
ight.
\Leftrightarrow \left[\begin{array}{l} x=\cfrac{3}{2},\\ x=\cfrac{1}{5}.\end{array}
ight.$\\
93. $|4x-7|=|2x+3|\Leftrightarrow \left[\begin{array}{l} 4x-7=2x+3,\\ 4x-7=-2x-3.\end{array}
ight.
\Leftrightarrow \left[\begin{array}{l} x=5,\\ x=\cfrac{2}{3}.\end{array}
ight.$\\
94. $x^2-x=14-\cfrac{24}{x^2-x}.$ Сделаем замену $t=x^2-x,$ тогда $t=14-\cfrac{24}{t},\ t^2=14t-24=0,\ t^2-14t+24=0,\ (t-2)(t-12)=0,\ t=2$ или $t=12.$ В первом случае $x^2-x=2,\ x^2-x-2=0,\ (x-2)(x+1)=0,\ x=-1$ или $x=2.$ Во втором случае $x^2-x=12,\ x^2-x-12=0,\ (x-4)(x+3)=0,\ x=-3$ или $x=4.$ Меньшим корнем является $x=-3.$\\
95. $x^2-5x=30-\cfrac{144}{x^2-5x}.$ Сделаем замену $t=x^2-5x,$ тогда $t=30-\cfrac{144}{t},\ t^2=30t-144,\ t^2-30t+144=0,\ (t-6)(t-24)=0,\ t=6$ или $t=24.$ В первом случае $x^2-5x=6,\ x^2-5x-6=0,\ (x-6)(x+1)=0,\ x=-1$ или $x=6.$ Во втором случае $x^2-5x=24,\ x^2-5x-24=0,\ (x-8)(x+3)=0,\ x=-3$ или $x=8.$ Меньшим корнем является $x=-3.$\\
96. $\cfrac{4}{|x^2+10x|}=\cfrac{1}{25}+\cfrac{2}{5x}\Leftrightarrow\cfrac{4}{|x(x+10)|}=\cfrac{x+10}{25x}\Leftrightarrow
\left[\begin{array}{l}\begin{cases} \cfrac{4}{x(x+10)}=\cfrac{x+10}{25x},\\ x(x+10)>0.\end{cases}\\
\begin{cases} \cfrac{-4}{x(x+10)}=\cfrac{x+10}{25x},\\ x(x+10)<0.\end{cases}\end{array}
ight.\Leftrightarrow$\\$
\left[\begin{array}{l}\begin{cases} 100=(x+10)^2,\\ x(x+10)>0.\end{cases}\\
\begin{cases} -100=(x+10)^2,\\ x(x+10)<0.\end{cases}\end{array}
ight.\Leftrightarrow x=-20.$\\
97. $\cfrac{25}{|x^2+5x|}=1+\cfrac{5}{x}\Leftrightarrow\cfrac{25}{|x^2+5x|}=\cfrac{x+5}{x}\Leftrightarrow
\left[\begin{array}{l}\begin{cases} \cfrac{25}{x(x+5)}=\cfrac{x+5}{x},\\ x(x+5)>0.\end{cases}\\
\begin{cases} \cfrac{-25}{x(x+5)}=\cfrac{x+5}{x},\\ x(x+5)<0.\end{cases}\end{array}
ight.\Leftrightarrow$\\$
\left[\begin{array}{l}\begin{cases} 25=(x+5)^2,\\ x(x+5)>0.\end{cases}\\
\begin{cases} -25=(x+5)^2,\\ x(x+5)<0.\end{cases}\end{array}
ight.\Leftrightarrow x=-10.$\\
98. $\cfrac{1}{\cfrac{1}{\cfrac{1}{\sqrt{5}-2k}+4}+4}+4=\sqrt{5}+2\Leftrightarrow
\cfrac{1}{\cfrac{1}{\sqrt{5}-2k}+4}+4=\cfrac{1}{\sqrt{5}+2-4}=\cfrac{1}{\sqrt{5}-2}=\cfrac{\sqrt{5}+2}{5-4}=\sqrt{5}+2\Leftrightarrow$\\$
\cfrac{1}{\sqrt{5}-2k}+4=\cfrac{1}{\sqrt{5}+2-4}=\cfrac{1}{\sqrt{5}-2}=\cfrac{\sqrt{5}+2}{5-4}=\sqrt{5}+2\Leftrightarrow
\sqrt{5}-2k=\cfrac{1}{\sqrt{5}+2-4}=\cfrac{1}{\sqrt{5}-2}=\cfrac{\sqrt{5}+2}{5-4}=\sqrt{5}+2\Leftrightarrow k=-1.$\\
99. $4-\cfrac{1}{4-\cfrac{1}{4-\cfrac{1}{2k-\sqrt{3}}}}=2-\sqrt{3}\Leftrightarrow
4-\cfrac{1}{4-\cfrac{1}{2k-\sqrt{3}}}=\cfrac{1}{4-2+\sqrt{3}}=\cfrac{1}{2+\sqrt{3}}=\cfrac{2-\sqrt{3}}{4-3}=2-\sqrt{3}\Leftrightarrow$\\$
4-\cfrac{1}{2k-\sqrt{3}}=\cfrac{1}{4-2+\sqrt{3}}=\cfrac{1}{2+\sqrt{3}}=\cfrac{2-\sqrt{3}}{4-3}=2-\sqrt{3}\Leftrightarrow
2k-\sqrt{3}=\cfrac{1}{4-2+\sqrt{3}}=\cfrac{1}{2+\sqrt{3}}=\cfrac{2-\sqrt{3}}{4-3}=2-\sqrt{3}\Leftrightarrow k=1.$\\
100. $\left(\cfrac{x-1}{x+1}
ight)^2+\left(\cfrac{x}{x+3}
ight)^2=\cfrac{2(x^2-x)}{x^2+4x+3}.$ Пусть $a=\cfrac{x-1}{x+1},\ b=\cfrac{x}{x+3},$ тогда $a^2+b^2=2ab,\ a^2-2ab+b^2=0,\ (a-b)^2=0,\ a=b.$ Поэтому $\cfrac{x-1}{x+1}=\cfrac{x}{x+3},\ x^2+3x-x-3=x^2+x,\ x=3.$\\
101. $\left(\cfrac{x+1}{x-1}
ight)^2+\left(\cfrac{x}{x+5}
ight)^2=\cfrac{2x^2+2x}{x^2+4x-5}.$  Пусть $a=\cfrac{x+1}{x-1},\ b=\cfrac{x}{x+5},$ тогда $a^2+b^2=2ab,\ a^2-2ab+b^2=0,\ (a-b)^2=0,\ a=b.$ Поэтому $\cfrac{x+1}{x-1}=\cfrac{x}{x+5},\ x^2+5x+x+5=x^2-x,\ x=-\cfrac{5}{7}.$\\
102. $\sqrt{x+1}+3\sqrt{3x+7}=17-x.$ Левая часть уравнения представляет из себя возрастающую на ОДЗ функцию, а правая --- убывающую. Значит, у уравнения может быть только один корень, подбором найдём $x=3.$\\
103. $x^2+2x+2|x+1|=7\Leftrightarrow (x+1)^2+2|x+1|-8=0.$ Сделаем замену $t=|x+1|\geqslant0,$ тогда $t^2+2t-8=0,\ (t+4)(t-2)=0,\ t=2.$ Поэтому $|x+1|=2,\ x+1=\pm2,\ x=-3$ или $x=1.$\\
104. $(345x^2+137x-208)\sqrt{3x-2}=0.$ Подбором найдём корень квадратного уравнения $x=-1,$ тогда по теореме Виета второй корень равен $\cfrac{208}{345}.$ При этом должно выполняться неравенство $3x-2\geqslant0,\ x\geqslant\cfrac{2}{3}.$ Но $-1<\cfrac{2}{3}$ и $\cfrac{208}{345}<\cfrac{2}{3},$ так как $624<670,$ поэтому единственным корнем уравнения является $x=\cfrac{2}{3}.$\\
105. $\cfrac{8x-4x^2}{1-x^2}=\cfrac{x^3-4x}{x+1}\Leftrightarrow\cfrac{4x(x-2)}{(x-1)(x+1)}=\cfrac{x(x-2)(x+2)}{x+1}\Leftrightarrow \begin{cases}
\left[\begin{array}{l} x=2,\\ x=0,\\ 4=(x+2)(x-1).\end{array}
ight.\\ x
eq-1.\end{cases}\Leftrightarrow$\\$ \begin{cases}
\left[\begin{array}{l} x=2,\\ x=0,\\ 4=x^2-x+2x-2.\end{array}
ight.\\ x
eq-1.\end{cases}\Leftrightarrow \begin{cases}
\left[\begin{array}{l} x=2,\\ x=0,\\ (x+3)(x-2)=0.\end{array}
ight.\\ x
eq-1.\end{cases}\Leftrightarrow x\in\left\{-3; 0; 2
ight\}.$\\
106. $(x-4)\sqrt{x^2-x-6}=6x-24\Leftrightarrow (x-4)\sqrt{x^2-x-6}=6(x-4).$ При $x=4$ подкоренное выражение положительно, значит он подходит. В другом случае $\sqrt{x^2-x-6}=6,\ x^2-x-6=36,\ x^2-x-42=0,\ (x-7)(x+6)=0,\ x=-6$ или $x=7.$\\
107. $(x+6)\sqrt{x^2-x-20}=6x+36\Leftrightarrow (x+6)\sqrt{x^2-x-20}=6(x+6).$ При $x=-6$ подкоренное выражение положительно, значит он подходит. В другом случае
$\sqrt{x^2-x-20}=6,\ x^2-x-20=36,\ x^2-x-56,\ (x-8)(x+7)=0,\ x=-7$ или $x=8.$\\
108. $(x^2-3x)^2-2(x^2-3x)=8.$ Сделаем замену $t=x^2-3x,$ тогда $t^2-2t=8,\ t^2-2t-8=0,\ (t-4)(t+2)=0,\ t=-2$ или $t=4.$ В первом случае $x^2-3x=-2,\ x^2-3x+2=0,\
(x-2)(x-1)=0,\ x=1$ или $x=2.$ Во втором случае  $x^2-3x=4,\ x^2-3x-4=0,\ (x-4)(x+1)=0,\ x=-1$ или $x=4.$\\
109. $(x^2+4x)^2-2(x^2+4x)=15.$ Сделаем замену $t=x^2+4x,$ тогда $t^2-2t=15,\ t^2-2t-15=0,\ (t-5)(t+3)=0,\ t=-3$ или $t=5.$ В первом случае $x^2+4x=-3,\ x^2+4x+3=0,\ (x+3)(x+1)=0,\ x=-3$ или $x=-1.$ Во втором случае $x^2+4x=5,\ x^2+4x-5=0,\ (x+5)(x-1)=0,\ x=-5$ или $x=1.$\\
110. $4x^2-9x-11=4(\sqrt{5}+\sqrt{3})^2-9(\sqrt{5}+\sqrt{3})-11.$ Пусть $a=\sqrt{5}+\sqrt{3},$ тогда $4x^2-9x-11=4a^2-9a-11,\ 4(x-a)(x+a)-9(x-a)=0,\ (x-a)(4x+4a-9)=0,\ x=a=\sqrt{5}+\sqrt{3}$ или $x=\cfrac{9}{4}-a=\cfrac{9}{4}-\sqrt{5}-\sqrt{3}.$\\
111. $2x^2+3x-17=2(2-\sqrt{5})^2+3(2-\sqrt{5})-17.$ Пусть $a=2-\sqrt{5},$ тогда $2x^2+3x-17=2a^2+3a-17,\ 2(x-a)(x+a)+3(x-a)=0,\ (x-a)(2x+2a+3)=0,\ x=a=2-\sqrt{5}$ или $x=-a-\cfrac{3}{2}=-2+\sqrt{5}-\cfrac{3}{2}=\sqrt{5}-\cfrac{7}{2}.$\\
112. $\sqrt{2x^2+x+6}+\sqrt{2x^2+x-9}=5\Big|\cdot(\sqrt{2x^2+x+6}-\sqrt{2x^2+x-9})\Leftrightarrow 2x^2+x+6-2x^2-x+9=5(\sqrt{2x^2+x+6}-\sqrt{2x^2+x-9})
\Leftrightarrow \sqrt{2x^2+x+6}-\sqrt{2x^2+x-9}=3.$ Получаем систему уравнений $\begin{cases} \sqrt{2x^2+x+6}+\sqrt{2x^2+x-9}=5,\\
\sqrt{2x^2+x+6}-\sqrt{2x^2+x-9}=3.\end{cases}\Rightarrow2\sqrt{2x^2+x+6}=8,\ 2x^2+x+6=16,\ 2x^2+x-10=0,\ (x-2)(2x+5)=0,\ x=2$ или $x=-\cfrac{5}{2}.$\\
113. $\sqrt{x^2+3x+6}-\sqrt{x^2+3x-1}=1\Big|\cdot(\sqrt{x^2+3x+6}+\sqrt{x^2+3x-1})\Leftrightarrow x^2+3x+6-x^2-3x+1=\sqrt{x^2+3x+6}+\sqrt{x^2+3x-1}
\Leftrightarrow\sqrt{x^2+3x+6}+\sqrt{x^2+3x-1}=7.$ Получаем систему уравнений $\begin{cases}\sqrt{x^2+3x+6}-\sqrt{x^2+3x-1}=1,\\
\sqrt{x^2+3x+6}+\sqrt{x^2+3x-1}=7.\end{cases}\Rightarrow2\sqrt{x^2+3x+6}=8,\ x^2+3x+6=16,\ x^2+3x-10=0,\ (x+5)(x-2)=0,\ x=-5$ или $x=2.$\\
114. $2\cdot\left(\cfrac{x-1}{x+2}
ight)^2-\left(\cfrac{x+1}{x-2}
ight)^2=\cfrac{x^2-1}{x^2-4}.$ Сделаем замену $a=\cfrac{x-1}{x+2},\ b=\cfrac{x+1}{x-2}.$ Тогда $2a^2-b^2=ab,\ 2a^2-b^2-ab=0,\ (a-b)(a+b)+a(a-b)=0,\ (a-b)(2a+b)=0, a=b$ или $b=-2a.$ В первом случае $\cfrac{x-1}{x+2}=\cfrac{x+1}{x-2},\ x^2-x-2x+2=x^2+x+2x+2,\ -6x=0,\ x=0.$ Во втором случае $\cfrac{x+1}{x-2}=\cfrac{2-2x}{x+2},\ x^2+x+2x+2=2x-2x^2-4+4x,\ 3x^2-3x+6=0,\ x^2-x+2=0, D<0.$ Значит, единственное решение $x=0.$\\
115. $x^3-6x^2+12x-1=0\Leftrightarrow x^3-6x^2+12x-8+7=0 \Leftrightarrow (x-2)^3=-7 \Leftrightarrow x=2-\sqrt[3]{7}.$\\
116. $\sqrt{7-x}=x-2\Leftrightarrow\begin{cases} 7-x=x^2-4x+4,\\ x-2\geqslant0.\end{cases}\Leftrightarrow\begin{cases} x^2-3x-3=0,\\ x\geqslant2.\end{cases}\Leftrightarrow\begin{cases} x=\cfrac{3\pm\sqrt{21}}{2},\\ x\geqslant2.\end{cases}\Rightarrow x=\cfrac{3+\sqrt{21}}{2}.$\\
117. $\sqrt{x+4}=x+1\Leftrightarrow\begin{cases} x+4=x^2+2x+1,\\ x+1\geqslant0.\end{cases}\Leftrightarrow\begin{cases} x^2+x-3=0,\\ x\geqslant-1.\end{cases}\Leftrightarrow\begin{cases} x=\cfrac{-1\pm\sqrt{13}}{2},\\ x\geqslant-1.\end{cases}\Rightarrow x=\cfrac{-1+\sqrt{13}}{2}.$\\
118. $|x+3|-|5-x|+|2x-5|=6\Leftrightarrow |x+3|+|2x-5|=|x-5|+6\Leftrightarrow$\\$ \left[\begin{array}{l}\begin{cases} -x-3-2x+5=-x+5+6,\\ x\leqslant-3.\end{cases}\\
\begin{cases} x+3-2x+5=-x+5+6,\\ -3<x\leqslant\cfrac{5}{2}.\end{cases}\\ \begin{cases} x+3+2x-5=-x+5+6,\\ \cfrac{5}{2}<x\leqslant5.\end{cases}\\
\begin{cases} x+3+2x-5=x-5+6,\\ 5< x.\end{cases}\end{array}
ight.\Leftrightarrow \left[\begin{array}{l}\begin{cases} -2x=9,\\ x\leqslant-3.\end{cases}\\
\begin{cases} 8=11,\\ -3<x\leqslant\cfrac{5}{2}.\end{cases}\\ \begin{cases} 4x=13,\\ \cfrac{5}{2}<x\leqslant5.\end{cases}\\
\begin{cases} 2x=3,\\ 5< x.\end{cases}\end{array}
ight.\Rightarrow x\in\left\{-\cfrac{9}{2};\cfrac{13}{4}
ight\}.$\\
119. $|2x-3|+|2-x|-|x+4|=1\Leftrightarrow |2x-3|+|x-2|=|x+4|+1\Leftrightarrow$\\$ \left[\begin{array}{l}\begin{cases} -2x+3-x+2=-x-4+1,\\ x\leqslant-4.\end{cases}\\
\begin{cases} -2x+3-x+2=x+4+1,\\ -4<x\leqslant\cfrac{3}{2}.\end{cases}\\ \begin{cases} 2x-3-x+2=x+4+1,\\ \cfrac{3}{2}<x\leqslant2.\end{cases}\\
\begin{cases} 2x-3+x-2=x+4+1,\\ 2< x.\end{cases}\end{array}
ight.\Leftrightarrow \left[\begin{array}{l}\begin{cases} -2x=-8,\\ x\leqslant-4.\end{cases}\\
\begin{cases} -3x=0,\\ -4<x\leqslant\cfrac{3}{2}.\end{cases}\\ \begin{cases} -1=5,\\ \cfrac{3}{2}<x\leqslant2.\end{cases}\\
\begin{cases} 2x=10,\\ 2< x.\end{cases}\end{array}
ight.
x\in\left\{0;5
ight\}.$\\
120. $\sqrt{5-x}=x-2\Leftrightarrow\begin{cases} 5-x=x^2-4x+4,\\ x-2\geqslant0.\end{cases}\Leftrightarrow\begin{cases} x^2-3x-1=0,\\ x\geqslant2.\end{cases}\Leftrightarrow\begin{cases} x=\cfrac{3\pm\sqrt{13}}{2},\\ x\geqslant2.\end{cases}\Rightarrow x=\cfrac{3+\sqrt{13}}{2}.$\\
121. $(x^2+2x)^2-(x+1)^2=55,\ (x^2+2x)^2-(x^2+2x+1)=55.$ Сделаем замену $x^2+2x=t,$ тогда $t^2-t-1=55,\ t^2-t-56=0,\ (t-8)(t+7)=0,\ t=-7$ или $t=8.$ В первом случае
$x^2+2x=-7,\ x^2+2x+7=0,\ D<0.$ Во втором случае $x^2+2x=8,\ x^2+2x-8=0,\ (x+4)(x-2)=0,\ x=-4$ или $x=2.$\\
122. $|x^2+15x-64|=-15x\Leftrightarrow \begin{cases}\left[\begin{array}{l} x^2+15x-64=-15x,\\ x^2+15x-64=15x.\end{array}
ight.\\ -15x\geqslant0.\end{cases}
\Leftrightarrow \begin{cases}\left[\begin{array}{l} x^2+30x-64=0,\\ x^2-64=0.\end{array}
ight.\\ x\leqslant0.\end{cases}
\Leftrightarrow$\\$ \begin{cases}\left[\begin{array}{l} (x+32)(x-2)=0,\\ (x+8)(x-8)=0.\end{array}
ight.\\ x\leqslant0.\end{cases}\Rightarrow
x\in\{-32;-8\}.$\\
123. $(\sqrt{x^2+1}-1)^2=3\sqrt{x^2+1}+7.$ Сделаем замену $t=\sqrt{x^2+1}\geqslant0,$ тогда
$(t-1)^2=3t+7,\ t^2-2t+1=3t+7,\ t^2-5t-6=0,\ (t-6)(t+1)=0,\ t=6.$ Значит, $\sqrt{x^2+1}=6,\ x^2+1=36,\ x^2=35,\ x=\pm\sqrt{35}.$\\
124. $(x^2+x-2)\sqrt{x+1}=(x-1)(x^2+3x+2)\Leftrightarrow(x-1)(x+2)\sqrt{x+1}=(x-1)(x+2)(x+1)\Leftrightarrow
(x-1)(x+2)(\sqrt{x+1}-x-1)=0\Leftrightarrow
\begin{cases}\left[\begin{array}{l} x=1,\\ x=-2,\\ \sqrt{x+1}=x+1.\end{array}
ight.\\ x\geqslant-1.\end{cases}\Leftrightarrow
\begin{cases}\left[\begin{array}{l} x=1,\\ x+1=x^2+2x+1.\end{array}
ight.\\ x\geqslant-1.\end{cases}\Leftrightarrow
\begin{cases}\left[\begin{array}{l} x=1,\\ x(x+1)=0.\end{array}
ight.\\ x\geqslant-1.\end{cases}\Leftrightarrow x\in\{-1;0;1\}$\\
125. $(x^2-3x-4)\sqrt{x-2}=(x+1)(x^2-6x+8)\Leftrightarrow(x-4)(x+1)\sqrt{x-2}=(x+1)(x-2)(x-4)\Leftrightarrow
(x-4)(x+1)(\sqrt{x-2}-x+2)=0\Leftrightarrow
\begin{cases}\left[\begin{array}{l} x=4,\\ x=-1,\\ \sqrt{x-2}=x-2.\end{array}
ight.\\ x\geqslant2.\end{cases}\Leftrightarrow
\begin{cases}\left[\begin{array}{l} x=4,\\ x-2=x^2-4x+4.\end{array}
ight.\\ x\geqslant2.\end{cases}\Leftrightarrow
\begin{cases}\left[\begin{array}{l} x=4,\\ (x-2)(x-3)=0.\end{array}
ight.\\ x\geqslant2.\end{cases}\Leftrightarrow x\in\{2;3;4\}$\\
126. $\left(x+\cfrac{1}{x}
ight)^2+3x=4-\cfrac{3}{x}\Leftrightarrow \left(x+\cfrac{1}{x}
ight)^2+3\left(x+\cfrac{1}{x}
ight)-4=0\Leftrightarrow
\begin{cases} t=x+\cfrac{1}{x},\\ t^2+3t-4=0.\end{cases}\Leftrightarrow$\\$
\begin{cases} t=x+\cfrac{1}{x},\\ (t+4)(t-1)=0.\end{cases}\Rightarrow \left[\begin{array}{l} x+\cfrac{1}{x}=-4,\\ x+\cfrac{1}{x}=1.\end{array}
ight.
\Rightarrow \left[\begin{array}{l} x^2+4x+1=0,\\ x^2-x+1=0.\end{array}
ight.\Rightarrow x\in\{-2\pm\sqrt{3}\}.$\\
127. $\left(x+\cfrac{1}{x}
ight)^2-2x=3+\cfrac{2}{x}\Leftrightarrow \left(x+\cfrac{1}{x}
ight)^2-2\left(x+\cfrac{1}{x}
ight)-3=0\Leftrightarrow
\begin{cases} t=x+\cfrac{1}{x},\\ t^2-2t-3=0.\end{cases}\Leftrightarrow$\\$
\begin{cases} t=x+\cfrac{1}{x},\\ (t-3)(t+1)=0.\end{cases}\Rightarrow \left[\begin{array}{l} x+\cfrac{1}{x}=-1,\\ x+\cfrac{1}{x}=3.\end{array}
ight.
\Rightarrow \left[\begin{array}{l} x^2+x+1=0,\\ x^2-3x+1=0.\end{array}
ight.\Rightarrow x\in\left\{\cfrac{3\pm\sqrt{5}}{2}
ight\}.$\\
128. $(x^2-2x)^2-3x^2+6x=4\Leftrightarrow (x^2-2x)^2-3(x^2-2x)-4=0.$ Пусть $t=x^2-2x,$ тогда $t^2-3t-4=0,\ (t-4)(t+1)=0,\ t=-1$ или $t=4.$ В первом случае $x^2-2x=-1,\ x^2-2x+1=0,\ (x-1)^2=0,\ x=1.$ Во втором случае $x^2-2x=4,\ x^2-2x-4=0,\ x=1\pm\sqrt{5}.$\\
129. $x^2+x^{10}=2x^{12}\Leftrightarrow x^2(1+x^8)=2x^{12}\Leftrightarrow \left[\begin{array}{l} x=0,\\ 1+x^8=2x^{10} \end{array}
ight.$ Если $|x|>1,$ то $2x^{10}=x^{10}+x^{10}>1+x^8.$ Если $|x|<1,$ то $2x^{10}=x^{10}+x^{10}<1+x^8.$ Значит, $|x|$ может быть равен только 1, значения $x=\pm1$ подходят. Таким образом, ответом является множество $\{-1;0;1\}.$\\
130. $(2x^2-x-4)^2+16x^2-8x=17\Leftrightarrow(2x^2-x-4)^2+8(2x^2-x-4)=-15\Leftrightarrow\begin{cases}2x^2-x-4=t,\\ t^2+8t+15=0.\end{cases}\Leftrightarrow
\begin{cases}2x^2-x-4=t,\\ (t+5)(t+3)=0.\end{cases}\Leftrightarrow\left[\begin{array}{l}2x^2-x-4=-3,\\2x^2-x-4=-5.\end{array}
ight.
\Leftrightarrow\left[\begin{array}{l}2x^2-x-1=0,\\2x^2-x+1=0.\end{array}
ight.\Leftrightarrow (x-1)(2x+1)=0 \Leftrightarrow
x\in\left\{-\cfrac{1}{2};1
ight\}.$\\
131. $2\cdot(x^2+2x-5)^2+3x^2+6x=69\Leftrightarrow2\cdot(x^2+2x-5)^2+3\cdot(x^2+2x-5)=54\Leftrightarrow\begin{cases}x^2+2x-5=t,\\ 2t^2+3t-54=0.\end{cases}\Leftrightarrow
\begin{cases}x^2+2x-5=t,\\ (2t-9)(t+6)=0.\end{cases}\Leftrightarrow\left[\begin{array}{l}x^2+2x-5=\cfrac{9}{2},\\x^2+2x-5=-6.\end{array}
ight.
\Leftrightarrow\left[\begin{array}{l}2x^2+4x-19=0,\\x^2+2x+1=0.\end{array}
ight.\Leftrightarrow \left[\begin{array}{l}2x^2+4x-19=0,\\(x+1)^2=0.\end{array}
ight.\Leftrightarrow
x\in\left\{\cfrac{-2\pm\sqrt{42}}{2};-1
ight\}.$\\
132. $(x^2-9)\cdot\sqrt{10-3x-x^2}=x^3-9x\Leftrightarrow (x-3)(x+3)\cdot\sqrt{10-3x-x^2}=x(x-3)(x+3).$ Из возможных корней $x=-3$ и $x=3$ подходит только $x=-3,$ так как при $x=3$ под корнем оказывается отрицательное выражение. В другом случае $\sqrt{10-3x-x^2}=x \Leftrightarrow\begin{cases}10-3x-x^2=x^2,\\ x\geqslant0.\end{cases}\Leftrightarrow\begin{cases}2x^2+3x-10=0,\\ x\geqslant0.\end{cases}\Leftrightarrow\begin{cases}x=\cfrac{-3\pm\sqrt{89}}{4},\\ x\geqslant0.\end{cases}\Leftrightarrow x=\cfrac{-3+\sqrt{89}}{4}.$ Таким образом,\\ $x\in\left\{-3; \cfrac{-3+\sqrt{89}}{4}
ight\}.$\\
133. $(3x^2-12)\cdot\sqrt{x^2-4x-5}=8x-2x^3\Leftrightarrow 3(x-2)(x+2)\cdot\sqrt{x^2-4x-5}=2x(2-x)(2+x).$ Из возможных корней $x=-2$ и $x=2$ подходит только $x=-2,$ так как при $x=2$ под корнем оказывается отрицательное выражение. В другом случае $3\sqrt{x^2-4x-5}=-2x \Leftrightarrow\begin{cases}9x^2-36x-45=4x^2,\\ x\leqslant0.\end{cases}\Leftrightarrow\begin{cases}5x^2-36x-45=0,\\ x\leqslant0.\end{cases}\Leftrightarrow\begin{cases}x=\cfrac{18\pm3\sqrt{61}}{5},\\ x\leqslant0.\end{cases}\Leftrightarrow x=\cfrac{18-3\sqrt{61}}{5}.$ Таким образом,\\ $x\in\left\{-2; \cfrac{18-3\sqrt{61}}{5}
ight\}.$\\
134. $\sqrt{3x-2}=2-x\Leftrightarrow\begin{cases} 3x-2=4-4x+x^2,\\2-x\geqslant0.\end{cases}\Leftrightarrow
\begin{cases} x^2-7x+6=0,\\x\leqslant2.\end{cases}\Leftrightarrow\begin{cases} (x-6)(x-1)=0,\\x\leqslant2.\end{cases}
\Leftrightarrow x=1.$\\
135. $\sqrt{5x+2x^2}-x=2\Leftrightarrow\sqrt{5x+2x^2}=x+2\Leftrightarrow\begin{cases}5x+2x^2=x^2+4x+4,\\ x+2\geqslant0.\end{cases}
\Leftrightarrow\begin{cases}x^2+x-4,\\ x\geqslant-2.\end{cases}
\Leftrightarrow\begin{cases}x=\cfrac{-1\pm\sqrt{17}}{2},\\ x\geqslant-2.\end{cases}
\Leftrightarrow x=\cfrac{-1+\sqrt{17}}{2}.$\\
136. $\cfrac{2|x|-6}{|x|+3}=\cfrac{x^2-9}{|x|+7}\Leftrightarrow \begin{cases} t=|x|,\\ \cfrac{2(t-3)}{t+3}=\cfrac{(t-3)(t+3)}{t+7}. \end{cases}\Leftrightarrow
\begin{cases} t=|x|,\\ \left[\begin{array}{l} t=3,\\ t^2+4t-5=0.\end{array}
ight. \end{cases}\Leftrightarrow$\\$
\begin{cases} t=|x|,\\ \left[\begin{array}{l} t=3,\\ (t+5)(t-1)=0.\end{array}
ight. \end{cases}\Leftrightarrow
\left[\begin{array}{l} |x|=3,\\ |x|=1,\\ |x|=-5.\end{array}
ight.\Leftrightarrow x\in\{\pm 3; \pm1\}.$\\
137. $\cfrac{4|x|-16}{|x|+4}=\cfrac{x^2-16}{|x|+7}\Leftrightarrow \begin{cases} t=|x|,\\ \cfrac{4(t-4)}{t+4}=\cfrac{(t-4)(t+4)}{t+7}. \end{cases}\Leftrightarrow
\begin{cases} t=|x|,\\ \left[\begin{array}{l} t=4,\\ t^2+4t-12=0.\end{array}
ight. \end{cases}\Leftrightarrow$\\$
\begin{cases} t=|x|,\\ \left[\begin{array}{l} t=4,\\ (t-2)(t+6)=0.\end{array}
ight. \end{cases}\Leftrightarrow
\left[\begin{array}{l} |x|=4,\\ |x|=-6,\\ |x|=2.\end{array}
ight.\Leftrightarrow x\in\{\pm 2; \pm4\}.$\\
138. $(x^2-1)^2+x^2=2x^3-2x\Leftrightarrow(x^2-1)^2-2x(x^2-1)+x^2=0\Leftrightarrow (x^2-1-x)^2=0\Leftrightarrow x^2-x-1=0\Leftrightarrow x=\cfrac{1\pm\sqrt{5}}{2}.$\\
139. $(x^2-3)^2+x^2=2x^3-6x\Leftrightarrow(x^2-3)^2-2x(x^2-3)+x^2=0\Leftrightarrow (x^2-3-x)^2=0\Leftrightarrow x^2-x-3=0\Leftrightarrow x=\cfrac{1\pm\sqrt{13}}{2}.$\\
140. $\cfrac{x+1}{x-1}=\cfrac{x^2-4x+3}{x-3}\Leftrightarrow
\cfrac{x+1}{x-1}=\cfrac{(x-3)(x-1)}{x-3}\Leftrightarrow
\begin{cases} \cfrac{x+1}{x-1}=x-1,\\ x
eq3.\end{cases}\Leftrightarrow
\begin{cases} x+1=x^2-2x+1,\\ x
otin\{1;3\}.\end{cases}$\\$\Leftrightarrow
\begin{cases} x(x-3)=0,\\ x
otin\{1;3\}.\end{cases}\Leftrightarrow x=0.$\\
141. $\cfrac{x+4}{x-2}=\cfrac{x^2-7x+10}{x-5}\Leftrightarrow
\cfrac{x+4}{x-2}=\cfrac{(x-5)(x-2)}{x-5}\Leftrightarrow
\begin{cases} \cfrac{x+4}{x-2}=x-2,\\ x
eq5.\end{cases}\Leftrightarrow
\begin{cases} x+4=x^2-4x+4,\\ x
otin\{2;5\}.\end{cases}$\\$\Leftrightarrow
\begin{cases} x(x-5)=0,\\ x
otin\{2;5\}.\end{cases}\Leftrightarrow x=0.$\\
142. $\cfrac{\sqrt{x}+1}{2\sqrt{x}-1}=\cfrac{x+3\sqrt{x}+2}{\sqrt{x}}\Leftrightarrow
\begin{cases}\cfrac{t+1}{2t-1}=\cfrac{t^2+3t+2}{t},\\ t=\sqrt{x}\geqslant0.\end{cases}\Leftrightarrow
\begin{cases}\cfrac{t+1}{2t-1}=\cfrac{(t+1)(t+2)}{t},\\ t=\sqrt{x}\geqslant0.\end{cases}\Leftrightarrow$\\$
\begin{cases}\cfrac{1}{2t-1}=\cfrac{t+2}{t},\\ t=\sqrt{x}\geqslant0.\end{cases}\Leftrightarrow
\begin{cases}t=2t^2+4t-t-2,\\ t=\sqrt{x}>0,\ t
eq\cfrac{1}{2}.\end{cases}\Leftrightarrow
\begin{cases}t^2+t-1=0,\\ t=\sqrt{x}>0,\ t
eq\cfrac{1}{2}.\end{cases}\Leftrightarrow
\begin{cases}t=\cfrac{-1\pm\sqrt{5}}{2},\\ t=\sqrt{x}>0,\ t
eq\cfrac{1}{2}.\end{cases}
\Leftrightarrow x=\left(\cfrac{-1+\sqrt{5}}{2}
ight)^2=
\cfrac{1-2\sqrt{5}+5}{4}=\cfrac{3-\sqrt{5}}{2}.$\\
143. $\cfrac{2\sqrt{x}+1}{\sqrt{x}-1}=\cfrac{2x+3\sqrt{x}+1}{\sqrt{x}}\Leftrightarrow
\begin{cases}\cfrac{2t+1}{t-1}=\cfrac{2t^2+3t+1}{t},\\ t=\sqrt{x}\geqslant0.\end{cases}\Leftrightarrow
\begin{cases}\cfrac{2t+1}{t-1}=\cfrac{(t+1)(2t+1)}{t},\\ t=\sqrt{x}\geqslant0.\end{cases}\Leftrightarrow$\\$
\begin{cases}\cfrac{1}{t-1}=\cfrac{t+1}{t},\\ t=\sqrt{x}\geqslant0.\end{cases}\Leftrightarrow
\begin{cases}t=t^2-1,\\ t=\sqrt{x}>0,\ t
eq1.\end{cases}\Leftrightarrow
\begin{cases}t^2-t-1=0,\\ t=\sqrt{x}>0,\ t
eq1.\end{cases}\Leftrightarrow
\begin{cases}t=\cfrac{1\pm\sqrt{5}}{2},\\ t=\sqrt{x}>0,\ t
eq1.\end{cases}
\Leftrightarrow x=\left(\cfrac{1+\sqrt{5}}{2}
ight)^2=
\cfrac{1+2\sqrt{5}+5}{4}=\cfrac{3+\sqrt{5}}{2}.$\\
144. $(x^2+x-2)\left(\cfrac{x^3-8}{x-2}-2(\sqrt{x^2+2x-3})^2
+x-4
ight)=0.$ Если первый множитель равен нулю, то $x^2+x-2=0,\ (x+2)(x-1)=0,\ x=-2$ или $x=1.$ При $x=-2$ подкоренное выражение во второй скобке отрицательно, значит этот корень не подходит, при $x=1$ это выражение равно нулю, поэтому корень $x=1$ подходит. Теперь приравняем к нулю вторую скобку: $\cfrac{x^3-8}{x-2}-2(\sqrt{x^2+2x-3})^2
+x-4=0\Leftrightarrow \begin{cases} x^2+2x+4-2(x^2+2x-3)+x-4=0,\\
x
eq2,\\ x^2+2x-3\geqslant 0.\end{cases}\Leftrightarrow \begin{cases} -x^2-x+6=0,\\
x
eq2,\\ (x+3)(x-1)\geqslant 0.\end{cases}\Leftrightarrow$\\$ \begin{cases} -(x+3)(x-2)=0,\\
x
eq2,\\ (x+3)(x-1)\geqslant 0.\end{cases}\Leftrightarrow x=-3.$ Таким образом, окончательный ответ $x\in\{-3;1\}.$\\
145. $\begin{cases} x^2+3y+z=-8,\\ x+y^2+5z=-12,\\ x+y+z^2=6.\end{cases}\Rightarrow x^2+3y+z+x+y^2+5z+x+y+z^2=-8-12+6 \Leftrightarrow x^2+2x+y^2+4y+z^2+6z+14=0\Leftrightarrow (x+1)^2+(y+2)^2+(z+3)^2=0\Leftrightarrow \begin{cases} x=-1,\\ y=-2,\\ z=-3.\end{cases}$

ewpage
