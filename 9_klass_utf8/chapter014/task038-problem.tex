38. $(x^2-8x)\sqrt{7-x}=x(x^2-9x+8)\Leftrightarrow x(x-8)\sqrt{7-x}=x(x-8)(x-1).$ При $x=0$ подкоренное выражение положительно, а при $x=8$ --- отрицательно, поэтому корнем является только $x=0.$ В другом случае $\sqrt{7-x}=x-1\Leftrightarrow \begin{cases} 7-x=x^2-2x+1,\\ x-1\geqslant0.\end{cases}
\Leftrightarrow \begin{cases} x^2-x-6=0,\\ x\geqslant1.\end{cases}
\Leftrightarrow \begin{cases} (x-3)(x+2)=0,\\ x\geqslant1.\end{cases}\Leftrightarrow x=3.$

Таким образом, $x\in\{0; 3\}.$\\
