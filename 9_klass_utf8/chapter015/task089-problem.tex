89.  $\cfrac{2|x|}{x+1}>-x.$ Если $x>0,$ то левая часть неравенства положительна, а правая отрицательна, поэтому все $x>0$ подходят и надо разобрать только случай $x<0$ (тогда $|x|=-x$). Значение $x=0$ не подходит, так как в этом случае левая часть равна правой.

$\cfrac{-2x}{x+1}>-x\Leftrightarrow \cfrac{-2x+x^2+x}{x+1}>0\Leftrightarrow \cfrac{x(x-1)}{x+1}>0.$ Так как $x<0,$ то и $x-1<0,$ а значит необходимо, чтобы выполнялось неравенство $x+1>0,$ то есть $x>-1.$ Таким образом, окончательным ответом является $x\in(-1;0)\cup (0;+\infty).$\\
