20. Дробь несократима, если НОД её числителя и знаменателя равен единице. При этом\\ НОД$(a,b)=$НОД$(a-b,b),$ поэтому у всех дробей можно вычесть из знаменателя числитель и проверять на несократимость полученные дроби: $\cfrac{6}{n+2},\ \cfrac{7}{n+2},\ \ldots, \cfrac{29}{n+2}.$ Они все несократимы, если число $n+2$ не имеет общих делителей с числами от 6 до 29. Для этого число $n+2$ должно быть наименьшим простым числом, превосходящим 29, тогда $n+2=31,\ n=29.$\\
