30. $(x^2+1)^4+(x-2)^2=1.$ Заметим, что $(x^2+1)^4\geqslant1,$ а $(x-2)^2\geqslant0.$ Значит, в уравнении равенство может выполняться только если оба неравенства обращаются в равенства. Но первое неравенство становится равенством при $x=0,$ а второе --- при $x=2,$ поэтому уравнение решений не имеет.\\