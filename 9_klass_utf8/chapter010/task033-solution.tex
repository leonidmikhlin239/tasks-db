33. $3x^2+5xy-2y^2=x(3x-y)+2y(3x-y)=(x+2y)(3x-y).$ Если это число является простым, то $3x-y=1,\ y=3x-1$ и простым должно быть число $x+2(3x-1)=7x-2.$ Так как $y=3x-1\leqslant30,$ получим неравенство $3x\leqslant31,\ x\leqslant 10$ (так как $x$ является натуральным числом). При чётных $x$ число $7x-2$ является чётным и не может быть простым, значит достаточно перебрать нечётные числа от 1 до 9. При $x=1,\ y=3\cdot1-1=2$ получим $7\cdot1-2=5$ простое число, значит пара $(1;2)$ подходит. При $x=3,\ y=3\cdot3-1=8$ получим $7\cdot3-2=19,$ значит пара $(3;8)$ подходит. При $x=5,\ y=3\cdot5-1=14$ получим $7\cdot5-2=33$ составное число, значит пара $(5;14)$ не подходит. При $x=7,\ y=3\cdot7-1=20$ получим $7\cdot7-2=47$ простое число, значит пара $(7;20)$ подходит. При $x=9,\ y=3\cdot9-1=26$ получим $7\cdot9-2=61$ простое число, значит пара $(9;26)$ подходит. Итого, ответом является множество $\{(1;2); (3;8); (7;20); (9;26)\}.$\\
