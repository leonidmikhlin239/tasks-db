12. Нули получаются от умножения 2 и 5. Двоек в множителях чисел от 1 до 101 достаточно много, количество нулей равно количеству пятёрок. Среди чисел от 1 до 101 содержится 20 чисел, делящихся на 5, и 4 числа, делящихся на 25, а значит $20+4=24$ пятёрки. Таким образом, $101!$ будет кончаться на 24 нуля.\\