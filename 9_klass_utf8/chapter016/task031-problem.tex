31. Корни уравнения лежат по разные стороны от 2 тогда и только тогда, когда выполняется неравенство $(x_1-2)(x_2-2)<0,\ x_1x_2-2(x_1+x_2)+4<0.$ По теореме Виета имеем равенства $x_1+x_2=5-a,\ x_1x_2=a^2-a.$ Поэтому $a^2-a+2a-10+4<0,\
a^2+a-6<0,\ (a+3)(a-2)<0,\ a\in(-3;2).$ Кроме того, необходимо проверить, что при этих значениях $a$ уравнение имеет два корня. Для это рассмотрим неравенство $D>0,$ то есть $(a-5)^2-4(a^2-a)>0,\ a^2-10a+25-4a^2+4a>0,\ 3a^2+6a-25<0,\ 3(a+1)^2<28.$ При полученных ранее значениях $a$ это неравенство выполняется, так как максимальное значение левой части равно $3\cdot9=27<28.$\\
