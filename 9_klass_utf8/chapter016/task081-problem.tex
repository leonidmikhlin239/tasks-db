81. $(\sqrt{x}-2)(ax+2)(3x-2a)=0\Leftrightarrow\begin{cases} \left[\begin{array}{l}x=-\cfrac{2}{a},\\ x=\cfrac{2a}{3},\\ x=4.\end{array}
ight.\\ x\geqslant0.\end{cases}$ Один корень $x=4$ у этого уравнения есть всегда, значит из двух оставшихся должен подходить ровно один, или они должны совпадать друг с другом или с 4. Знаки этих корней всегда разные, значит подходить оба они никогда не могут. Если $-\cfrac{2}{a}=4,$ то $a=-\cfrac{1}{2}$ и у уравнения только один корень. Если $\cfrac{2a}{3}=4,$ то $a=6$ и у уравнения также только один корень. Совпадать друг с другом корни не могут, так как имеют разный знак, если $a=0,$ то один корень не существует, но другой подходит. Значит, это уравнение имеет два корня при $a
otin\left\{-\cfrac{1}{2};6
ight\}.$\\
