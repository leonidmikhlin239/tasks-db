23. Корни уравнения лежат по разные стороны от 1 тогда и только тогда, когда выполняется неравенство $(x_1-1)(x_2-1)<0,\ x_1x_2-(x_1+x_2)+1<0.$ По теореме Виета имеем равенства $x_1+x_2=\cfrac{2a+1}{1-a^2},\ x_1x_2=\cfrac{3}{1-a^2}.$ Поэтому $\cfrac{3}{1-a^2}-\cfrac{2a+1}{1-a^2}+1<0,\
\cfrac{3-2a-a^2}{1-a^2}<0,\ \cfrac{(a+3)(a-1)}{(a-1)(a+1)}<0,\ a\in(-3;-1).$ Кроме того, необходимо проверить, что при этих значениях $a$ уравнение имеет два корня. Старший коэффициент обнуляется при $a=\pm1,$ которые в полученные интервал не входят. Остаётся проверить неравенство $D>0,$ то есть $(2a+1)^2+12(a^2-1)>0,$ что верно, так как квадрат всегда неотрицателен, а $a^2$ при полученных значениях $a$ больше 1.\\
