24. Корни уравнения лежат по разные стороны от $(-1)$ тогда и только тогда, когда выполняется неравенство $(x_1+1)(x_2+1)<0,\ x_1x_2+(x_1+x_2)+1<0.$ По теореме Виета имеем равенства $x_1+x_2=\cfrac{3b-1}{4-b^2},\ x_1x_2=\cfrac{7}{4-b^2}.$ Поэтому $\cfrac{7}{4-b^2}+\cfrac{3b-1}{4-b^2}+1<0,\
\cfrac{10+3b-b^2}{4-b^2}<0,\ \cfrac{(b+2)(b-5)}{(b-2)(b+2)}<0,\ b\in(2;5).$ Кроме того, необходимо проверить, что при этих значениях $b$ уравнение имеет два корня. Старший коэффициент обнуляется при $b=\pm2,$ которые в полученные интервал не входят. Остаётся проверить неравенство $D>0,$ то есть $(3b-1)^2+28(b^2-4)>0,$ что верно, так как квадрат всегда неотрицателен, а $b^2$ при полученных значениях $b$ больше 4.\\
