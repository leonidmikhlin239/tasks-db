72. а) Сначала проверим значение $a,$ при котором это уравнение линейно: при $a=0$ получим $3x+1=0,\ x=-\cfrac{1}{3},$ значит оно подходит. Для квадратного уравнения проверим наличие корней: $D\geqslant0,\ (a-3)^2-4a=a^2-6a+9-4a=a^2-10a+9=(a-1)(a-9)\geqslant0,\ a\in(-\infty;1]\cup[9;+\infty).$ То, что все корни отрицательны, равносильно тому, что отрицательна их сумма и положительно произведение. Тогда по теореме Виета имеем неравенства $\begin{cases} \cfrac{1}{a}>0,\\ \cfrac{a-3}{a}<0.\end{cases} \Leftrightarrow a\in (0;3).$ Таким образом, итоговым ответом будет $a\in[0;1].$\\
б) Квадратное уравнение имеет два различных корня, если $a
eq0$ и $D>0,$ то есть \\$\begin{cases}a
eq0,\\ (a-1)(a-9)>0.\end{cases}\Leftrightarrow a\in(-\infty;0)\cup(0;1)\cup(9;+\infty).$\\
в) Это уравнение имеет один корень, если у числителя один корень (не равный 2) или если один из двух корней числителя равен 2 (и отбрасывается по ОДЗ). Из предыдущих пунктов у числителя один корень при $a\in\{0; 1; 9\}.$ Подставим $x=2:\ 4a-2(a-3)+1=0,\ 4a-2a+6+1=0,\ a=-\cfrac{7}{2}.$ Таким образом, итоговый ответ $a\in \left\{-\cfrac{7}{2};0; 1; 9
ight\}.$\\
