89. Пусть уравнение этой прямой имеет вид $y=kx+b,$ тогда $10k+b=0,\ kx_1+b=x_1^2,\ kx_2+b=x_2^2,$ откуда $b=-10k$ и $x_1^2-x_2^2=k(x_1-x_2),\
(x_1-x_2)(x_1+x_2)=k(x_1-x_2),\ x_1+x_2=k,$ так как точки различны.
 Сложив второе и третье равенство, получим соотношение $k(x_1+x_2)+2b=x_1^2+x_2^2,\ k^2+2b=(x_1+x_2)^2-2x_1x_2,\ k^2+2b=k^2-2x_1x_2,\ x_1x_2=-b.$ Таким образом, $\cfrac{1}{x_1}+\cfrac{1}{x_2}=\cfrac{x_1+x_2}{x_1x_2}=\cfrac{k}{-b}=\cfrac{k}{10k}=\cfrac{1}{10}.$\\
