\section{Нестандартные задачи решения}
1. а) $\log_28=3$ так как $2^3=8.$\\
б) $2^{\log_23}=3$ по определению логарифма (мы возводим 2 в ту степень, в которую её надо возводить для получения 3).\\
в) $4^{\log_25}=(2^2)^{\log_25}=(2^{\log_25})^2=5^2=25.$\\
г) $2^{\log_5 3\cdot\log_25}=(2^{\log_25})^{\log_5 3}=5^{\log_5 3}=3.$\\
2. а) $\log_3 27=3$ так как $3^3=27.$\\
б) $3^{\log_32}=2$ по определению логарифма (мы возводим 3 в ту степень, в которую её надо возводить для получения 2).\\
в) $9^{\log_37}=(3^2)^{\log_37}=(3^{\log_37})^2=7^2=49.$\\
г) $3^{\log_5 3\cdot\log_3 5}=(3^{\log_35})^{\log_5 3}=5^{\log_5 3}=3.$\\
3. Сумма первых 13 натуральных чисел $1+2+\ldots+13$ равна $\cfrac{13\cdot14}{2}=91.$ Значит, необходимо её увеличить только на 1, это можно сделать единственным способом: взять вместо числа 13 число 14. Таким образом, ответом является сумма $1+2+\ldots+12+14=92.$\\
4. Сумма первых 11 натуральных чисел $1+2+\ldots+11$ равна $\cfrac{11\cdot12}{2}=66.$ Значит, необходимо её увеличить только на 1, это можно сделать единственным способом: взять вместо числа 11 число 12. Таким образом, ответом является сумма $1+2+\ldots+10+12=67.$\\
5. Если инициатором был Саша, то ложны оба его утверждения и по одному утверждению остальных ребят, то есть 4 из 6. Если инициатором был Дима, то верны все утверждения Саши и Лёши, то есть 4 из 6. Если инициатором был Лёша, то верны оба утверждения Димы и одно Саши, то ест 3 из 6, что соответствует условию. Значит, инициатором был Лёша.\\
6. Если инициатором бы Дима, то Саша говорит только правду, что невозможно. Если инициатором был Лёша, то  Дима говорит только правду, что невозможно. Если инициатором был Саша, то он лжёт, а двое других говорят правду только наполовину, что соответствует условию задачи. Значит, инициатором был Саша.\\
7. $6*14=(-8+14)*14=-8+14*14=-8.$\\
8. $8*12=(-4+12)*12=-4+12*12=-4.$\\
9. Число тем меньше, чем меньше его первая цифра, значит остальные цифры должны быть как можно больше. Таким образом, искомое число равно 499.\\
10. Число тем меньше, чем меньше его первая цифра, значит остальные цифры должны быть как можно больше. Таким образом, искомое число равно 599.\\
11. Нули получаются от умножения 2 и 5. Двоек в множителях чисел от 1 до 100 достаточно много, количество нулей равно количеству пятёрок. Среди чисел от 1 до 100 содержится 20 чисел, делящихся на 5, и 4 числа, делящихся на 25, а значит $20+4=24$ пятёрки. Таким образом, $100!$ будет кончаться на 24 нуля.\\
12. Нули получаются от умножения 2 и 5. Двоек в множителях чисел от 1 до 101 достаточно много, количество нулей равно количеству пятёрок. Среди чисел от 1 до 101 содержится 20 чисел, делящихся на 5, и 4 числа, делящихся на 25, а значит $20+4=24$ пятёрки. Таким образом, $101!$ будет кончаться на 24 нуля.\\
13. $0,123^3+0,124^3+0,125^3>0,1^3+0,1^3+0,1^3=0,003>0,002856.$\\
14. $0,131^3+0,132^3+0,133^3>0,1^3+0,1^3+0,1^3=0,003>0,002976.$\\
15. Если мы выбираем одну точку на первой прямой, а две на второй, то количество треугольников равно $6\cdot\cfrac{3\cdot2}{2}=18.$ Если мы выбираем одну точку на второй прямой и две на первой, то количество треугольников равно $3\cdot\cfrac{6\cdot5}{2}=45.$ Значит, всего треугольников $18+45=63.$\\
16. Если мы выбираем одну точку на первой прямой, а две на второй, то количество треугольников равно $5\cdot\cfrac{4\cdot3}{2}=30.$ Если мы выбираем одну точку на второй прямой и две на первой, то количество треугольников равно $4\cdot\cfrac{5\cdot4}{2}=40.$ Значит, всего треугольников $30+40=70.$\\
17. $(a+1)(b+1)=2ab,\ ab+a+b+1=2ab,\ ab-a-b=1.$ Тогда $\cfrac{(a^2-1)(b^2-1)}{ab}=$\\$=\cfrac{(a-1)(a+1)(b-1)(b+1)}{ab}=\cfrac{(a+1)(b+1)}{ab}\cdot(ab-a-b+1)=2\cdot2=4.$\\
18. $(a-1)(b-1)=2ab,\ ab-a-b+1=2ab,\ ab+a+b=1.$ Тогда $\cfrac{(a^2-1)(b^2-1)}{ab}=$\\$=\cfrac{(a-1)(a+1)(b-1)(b+1)}{ab}=\cfrac{(a-1)(b-1)}{ab}\cdot(ab+a+b+1)=2\cdot2=4.$\\
19. Дробь несократима, если НОД её числителя и знаменателя равен единице. При этом\\ НОД$(a,b)=$НОД$(a-b,b),$ поэтому у всех дробей можно вычесть из знаменателя числитель и проверять на несократимость полученные дроби: $\cfrac{7}{n+2},\ \cfrac{8}{n+2},\ \ldots, \cfrac{31}{n+2}.$ Они все несократимы, если число $n+2$ не имеет общих делителей с числами от 7 до 31. Для этого число $n+2$ должно быть наименьшим простым числом, превосходящим 31, тогда $n+2=37,\ n=35.$\\
20. Дробь несократима, если НОД её числителя и знаменателя равен единице. При этом\\ НОД$(a,b)=$НОД$(a-b,b),$ поэтому у всех дробей можно вычесть из знаменателя числитель и проверять на несократимость полученные дроби: $\cfrac{6}{n+2},\ \cfrac{7}{n+2},\ \ldots, \cfrac{29}{n+2}.$ Они все несократимы, если число $n+2$ не имеет общих делителей с числами от 6 до 29. Для этого число $n+2$ должно быть наименьшим простым числом, превосходящим 29, тогда $n+2=31,\ n=29.$\\
21. а) $f(1;3;5)+f(1;3;5)=f(2;3;5)+2\cdot3\cdot5=f(2;3;5)+30.$ Тогда $f(2;3;5)=46+46-30=62=f(3;2;5).$\\
б) $f(3;2;5)+f(3;2;5)=f(6;2;5)+2\cdot2\cdot5=f(6;2;5)+20,$ значит $f(6;2;5)=62+62-20=104=f(5;2;6).$ Также
$f(5;2;6)+f(5;2;6)=f(10;2;6)+2\cdot2\cdot6=f(10;2;6)+24,$ откуда $f(10;2;6)=104+104-24=184=f(2;6;10).$\\
22. а) $f(2;3;5)+f(2;3;5)=f(4;3;5)+2\cdot3\cdot5=f(4;3;5)+30.$ Тогда $f(4;3;5)=62+62-30=94=f(3;4;5).$\\
б) $f(3;4;5)+f(3;4;5)=f(6;4;5)+2\cdot4\cdot5=f(6;4;5)+40,$ значит $f(6;4;5)=94+94-40=148=f(5;4;6).$ Также
$f(5;4;6)+f(5;4;6)=f(10;4;6)+2\cdot4\cdot6=f(10;4;6)+48,$ откуда $f(10;4;6)=148+148-48=248=f(4;6;10).$\\
23. $2^{10}+5^{12}=(2^5)^2+2\cdot2^5\cdot5^6+(5^6)^2-2\cdot2^5\cdot5^6=(2^5+5^6)^2-(2^3\cdot5^3)^2=
(2^5+5^6-2^3\cdot5^3)(2^5+5^6+2^3\cdot5^3).$ Значит, это число является составным.\\
24. Сумма первых 15 натуральных чисел $1+2+\ldots+15$ равна $\cfrac{15\cdot16}{2}=120.$ Значит, необходимо её увеличить только на 1, это можно сделать единственным способом: взять вместо числа 15 число 16. Таким образом, ответом является сумма $1+2+\ldots+14+16=121.$\\
25. Если $\sqrt{50-n^2}$ является целым, то $50-n^2$ должно быть некоторым полным квадратом, не превосходящим 50. Так как $8^2>50,$ это может быть только число $1,\ 4,\ 9,\ 16,\ 25,\ 36$ или 49. Перебором найдём, что подходят только числа 1, 25 и 49, значит $n\in\{1;5;7\}.$\\
26. Если $\sqrt{52-n^2}$ является целым, то $52-n^2$ должно быть некоторым полным квадратом, не превосходящим 52. Так как $8^2>52,$ это может быть только число $1,\ 4,\ 9,\ 16,\ 25,\ 36$ или 49. Перебором найдём, что подходят только числа 16 и 36, значит $n\in\{4;6\}.$\\
27. а) Если на последнее место поставить 0, то на первое место 5 вариантов, на второе 4, на третье 3, на четвёртое 2, то есть всего $5\cdot4\cdot3\cdot2=120$ чисел. Если на последнее место поставить 2 или 4 (2 варианта), то на первое место 4 варианта (без 0), на второе опять 4 (одна цифра использована, но можно поставить 0), на третье 3, на четвёртое 2, то есть всего $2\cdot4\cdot4\cdot3\cdot2=192$ варианта. Значит, всего можно составить $120+192=312$ чисел.\\
б) На первое место можно поставить 5 цифр (без 0), на второе-четвёртое по 6, на пятое 3 (только 0, 2 или 4), значит всего можно составить $5\cdot6\cdot6\cdot6\cdot3=3240$ чисел.\\
28. а) Если на последнее место поставить 0, то на первое место 5 вариантов, на второе 4, на третье 3, на четвёртое 2, то есть всего $5\cdot4\cdot3\cdot2=120$ чисел. Если на последнее место поставить 5, то на первое место 4 варианта (без 0), на второе опять 4 (одна цифра использована, но можно поставить 0), на третье 3, на четвёртое 2, то есть всего $4\cdot4\cdot3\cdot2=96$ вариантов. Значит, всего можно составить $120+96=216$ чисел.\\
б) На первое место можно поставить 5 цифр (без 0), на второе-четвёртое по 6, на пятое 3 (только 0 или 5), значит всего можно составить $5\cdot6\cdot6\cdot6\cdot2=2160$ чисел.\\
29. а) Например, $111=(1+1+1)\cdot3.$\\
б) Пусть трёхзначное число равно $\overline{abc},$ тогда $37(a+b+c)=100a+10b+c,\ 27b+36c=63a,\ 3b+4c=7a, 3(b-a)=4(a-c).$ Значит, $b-a$ кратно 4. Если $b-a=0,$ то $a-c=0,$ все числа вида $\overline{aaa}$ подходят. Если $b-a=4,$ то $a-c=3,$ то есть $a=c+3$ и $b=a+4=c+7,$ это возможно только при $c\in\{0;1;2\},$ подходят числа 370, 481, 592. Если $b-a=8,$ то $a-c=6,$ то есть $a=c+6$ и $b=a+8=c+14,$ что невозможно. Таким образом, подходят только числа 111, 222, 333, 444, 555, 666, 777, 888, 999, 370, 481 и 592.\\
30. $(x^2+1)^4+(x-2)^2=1.$ Заметим, что $(x^2+1)^4\geqslant1,$ а $(x-2)^2\geqslant0.$ Значит, в уравнении равенство может выполняться только если оба неравенства обращаются в равенства. Но первое неравенство становится равенством при $x=0,$ а второе --- при $x=2,$ поэтому уравнение решений не имеет.\\
31. Пусть исходное число равно $\overline{ab}=10a+b.$ Тогда имеем систему уравнений \\$\begin{cases} 10a+b=7\cdot(a+b)+3,\\ 10a+b=1\cdot(10b+a)+36.\end{cases}\Leftrightarrow \begin{cases} 3a-6b=3,\\ 9a-9b=36.\end{cases}\Leftrightarrow \begin{cases} a-2b=1,\\ a-b=4.\end{cases}
\Leftrightarrow \begin{cases} a=7,\\ b=3.\end{cases}$ Таким образом, искомое число равно 73.\\
32. Пусть исходное число равно $\overline{ab}=10a+b.$ Тогда имеем систему уравнений \\$\begin{cases} 10a+b=6\cdot(a+b)+11,\\ 10b+a=4\cdot(a+b)+3.\end{cases}\Leftrightarrow \begin{cases} 4a-5b=11,\\ -a+2b=1.\end{cases}\Leftrightarrow \begin{cases} 4(2b-1)-5b=11,\\ a=2b-1.\end{cases}
\Leftrightarrow \begin{cases} a=9,\\ b=5.\end{cases}$ Таким образом, искомое число равно 95.\\
33. $3x^2+5xy-2y^2=x(3x-y)+2y(3x-y)=(x+2y)(3x-y).$ Если это число является простым, то $3x-y=1,\ y=3x-1$ и простым должно быть число $x+2(3x-1)=7x-2.$ Так как $y=3x-1\leqslant30,$ получим неравенство $3x\leqslant31,\ x\leqslant 10$ (так как $x$ является натуральным числом). При чётных $x$ число $7x-2$ является чётным и не может быть простым, значит достаточно перебрать нечётные числа от 1 до 9. При $x=1,\ y=3\cdot1-1=2$ получим $7\cdot1-2=5$ простое число, значит пара $(1;2)$ подходит. При $x=3,\ y=3\cdot3-1=8$ получим $7\cdot3-2=19,$ значит пара $(3;8)$ подходит. При $x=5,\ y=3\cdot5-1=14$ получим $7\cdot5-2=33$ составное число, значит пара $(5;14)$ не подходит. При $x=7,\ y=3\cdot7-1=20$ получим $7\cdot7-2=47$ простое число, значит пара $(7;20)$ подходит. При $x=9,\ y=3\cdot9-1=26$ получим $7\cdot9-2=61$ простое число, значит пара $(9;26)$ подходит. Итого, ответом является множество $\{(1;2); (3;8); (7;20); (9;26)\}.$\\
34. Всего двузначных чисел с нечётной цифрой десятков $5\cdot10=50$ (5 возможных вариантов для первой цифры и 10 вариантов для второй). Из них сумму цифр 11 имеют
4 числа: 38, 56, 74 и 92. Значит, искомая вероятность равна $4:50=0,08.$\\
35. Всего двузначных чисел с чётной цифрой десятков $4\cdot10=40$ (4 возможных варианта для первой цифры и 10 вариантов для второй). Из них сумму цифр 10 имеют
4 числа: 28, 46, 64 и 82. Значит, искомая вероятность равна $4:40=0,1.$\\
36. Нет, неверно. Пусть концентрация вещества в первой цистерне равна $20\%,$ во второй $90\%,$ в третьей $10\%,$ а в четвёртой --- $80\%.$ Пусть после использования в первой и четвёртой цистерне осталось по 5 т раствора, а во второй и третьей --- по 1 т. Тогда и в первом, и во втором сосуде будет по $1+5=6$т
раствора, при этом в первом сосуде будет $5\cdot0,2+1\cdot0,9=1,9$т вещества, а во втором --- $1\cdot0,1+5\cdot0,8=4,1$т вещества, поэтому концентрация в нём будет больше.\\
37. Пусть исходное число равно $\overline{ab},$ тогда имеем систему уравнений $\begin{cases}10a+b=7a+7b+3,\\ 10a+b=10b+a+36.\end{cases}\Leftrightarrow
\begin{cases}a-2b=1,\\ a-b=4.\end{cases}\Leftrightarrow
\begin{cases}a=7,\\ b=3.\end{cases}$ Значит, исходное число равно 73.\\
38. Рассмотрим самого тяжёлого гнома. Его первое утверждение верно, но он должен лгать, поэтому неверно его второе утверждение, то есть нет гнома ниже него, значит он самый высокий: обязательно верно утверждение В.\\
39. Если $n=p_1^{a_1}p_2^{\alpha_2}\ldots p_k^{\alpha_k},$ то количество его натуральных делителей равно $(\alpha_1+1)(\alpha_2+1)\ldots (\alpha_k+1).$ Так как число 5 является простым, это произведение может равняться пяти, только если $n=p_1^4.$ Наибольшая четвёртая степень, являющаяся трёхзначным числом, это $5^4=625.$\\
40. а) На первое место можно поставить 5 цифр, как и на второе, значит их $5\cdot5=25.$\\
б) На первое место можно поставить 4 цифры (без 0), на второе тоже 4 (нельзя повторить первую, но можно поставить 0), а на третье --- 3 (нельзя повторить первую и вторую), значит всего таких чисел $4\cdot4\cdot3=48.$\\
41. а) На первое место можно поставить 5 цифр, а на второе 4 (нельзя повторить первую цифру), значит их $5\cdot4=20.$\\
б) На первое место можно поставить 4 цифры (без 0), на второе и третье --- по 5, значит всего таких чисел $4\cdot5\cdot5=100.$\\
42. Всего пар результатов $6\cdot6=36,$ а подходят нам только 5 вариантов $1+5=2+4=3+3=4+2=5+1=6,$ поэтому вероятность равна $\cfrac{5}{36}.$\\
43. Всего пар результатов $6\cdot6=36,$ а подходят нам только 4 варианта $1+4=2+3=3+2=4+1=5,$ поэтому вероятность равна $\cfrac{4}{36}=\cfrac{1}{9}.$\\
44. Чёрных шаров не может быть больше 11 (иначе среди этих хотя бы 12 шаров не найдётся ни одного белого), а белых шаров не может быть больше 19 (иначе среди этих хотя бы 20 шаров не найдётся ни одного чёрного). Так как всего шаров $30=11+19,$ чёрных шаров должно быть ровно 11, а белых --- ровно 19.\\
45. Запишем свойство: $x*(3*3)=(x*3)\cdot3,$ а значит $x*3=\cfrac{x*1}{3}.$ Запишем свойства ещё раз: $x*1=x*(x*x)=(x*x)\cdot x=1\cdot x=x,$ а значит
$\cfrac{x}{3}=2024,\ x=6072.$

ewpage
