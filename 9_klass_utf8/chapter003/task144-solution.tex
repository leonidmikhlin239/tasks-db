144. $(x^2+x-2)\left(\cfrac{x^3-8}{x-2}-2(\sqrt{x^2+2x-3})^2
+x-4\right)=0.$ Если первый множитель равен нулю, то $x^2+x-2=0,\ (x+2)(x-1)=0,\ x=-2$ или $x=1.$ При $x=-2$ подкоренное выражение во второй скобке отрицательно, значит этот корень не подходит, при $x=1$ это выражение равно нулю, поэтому корень $x=1$ подходит. Теперь приравняем к нулю вторую скобку: $\cfrac{x^3-8}{x-2}-2(\sqrt{x^2+2x-3})^2
+x-4=0\Leftrightarrow \begin{cases} x^2+2x+4-2(x^2+2x-3)+x-4=0,\\
x\neq2,\\ x^2+2x-3\geqslant 0.\end{cases}\Leftrightarrow \begin{cases} -x^2-x+6=0,\\
x\neq2,\\ (x+3)(x-1)\geqslant 0.\end{cases}\Leftrightarrow$\\$ \begin{cases} -(x+3)(x-2)=0,\\
x\neq2,\\ (x+3)(x-1)\geqslant 0.\end{cases}\Leftrightarrow x=-3.$ Таким образом, окончательный ответ $x\in\{-3;1\}.$\\
