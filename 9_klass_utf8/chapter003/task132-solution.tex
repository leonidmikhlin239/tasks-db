132. $(x^2-9)\cdot\sqrt{10-3x-x^2}=x^3-9x\Leftrightarrow (x-3)(x+3)\cdot\sqrt{10-3x-x^2}=x(x-3)(x+3).$ Из возможных корней $x=-3$ и $x=3$ подходит только $x=-3,$ так как при $x=3$ под корнем оказывается отрицательное выражение. В другом случае $\sqrt{10-3x-x^2}=x \Leftrightarrow\begin{cases}10-3x-x^2=x^2,\\ x\geqslant0.\end{cases}\Leftrightarrow\begin{cases}2x^2+3x-10=0,\\ x\geqslant0.\end{cases}\Leftrightarrow\begin{cases}x=\cfrac{-3\pm\sqrt{89}}{4},\\ x\geqslant0.\end{cases}\Leftrightarrow x=\cfrac{-3+\sqrt{89}}{4}.$ Таким образом,\\ $x\in\left\{-3; \cfrac{-3+\sqrt{89}}{4}\right\}.$\\