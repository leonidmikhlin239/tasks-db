35. Пусть искомая функция имеет вид $f(x)=ax^2+bx+c.$ Подставим точку $A:\ 0+0+c=-2,\ c=-2.$ Единственное значение парабола принимает только в вершине, исходя из этого составим систему уравнений:
$\begin{cases} 4a-2b-2=4,\\ a\cdot\cfrac{b^2}{4a^2}-b\cdot\cfrac{b}{2a}-2=-4.\end{cases}\Leftrightarrow
\begin{cases} b=2a-3,\\ \cfrac{-b^2}{4a}=-2.\end{cases}\Leftrightarrow
\begin{cases} b=2a-3,\\ (2a-3)^2=8a.\end{cases}\Leftrightarrow
\begin{cases} b=2a-3,\\ 4a^2-20a+9=0.\end{cases}\Leftrightarrow
\left[\begin{array}{l}\begin{cases} a=\cfrac{1}{2},\\ b=-2.\end{cases}\\\begin{cases} a=\cfrac{9}{2},\\ b=6.\end{cases}\end{array}
ight.$\\
Таким образом, искомая функция равна $\frac{1}{2}x^2-2x-2$ или $\frac{9}{2}x^2+6x-2.$\\
