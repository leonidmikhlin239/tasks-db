36. Пусть искомая функция имеет вид $f(x)=ax^2+bx+c.$ У квадратичной функции совпадают значения, симметричные относительно вершины, значит $-\cfrac{b}{2a}=\cfrac{-1+2}{2}=\cfrac{1}{2},\ b=-a.$ Наибольшее значение квадратичной функции достигается в вершине (при это должно выполняться неравенство $a<0).$ Исходя из этого, составим систему уравнений (второе уравнение получим из того, что парабола проходит через точку $(1;1))$: $\begin{cases}
a\cdot\cfrac{1}{4}-a\cdot\cfrac{1}{2}+c=3,\\ a-a+c=1.\end{cases}\Leftrightarrow
\begin{cases}
a=-8,\\ c=1.\end{cases}$\\
Таким образом, искомая функция равна $-8x^2+8x+1.$\\
