3. Если $\alpha-\cfrac{\pi}{4}$ находится в $III$ четверти, то $\alpha+\cfrac{\pi}{4}$ находится в $IV$ четверти, его косинус положителен и равен $\cos\left(\alpha+\cfrac{\pi}{4}\right)=\sqrt{1-\cfrac{144}{169}}=\cfrac{5}{13}.$ Тогда $\sin\left(2\alpha+\cfrac{\pi}{2}\right)=2\left(-\cfrac{12}{13}\right)\cdot
\cfrac{5}{13}=-\cfrac{120}{169}=\cos(2\alpha).$ Так как $\alpha+\cfrac{\pi}{4}$ находится в $IV$ четверти, его синус равен $-\cfrac{12}{13},$ это угол близкий
к $\cfrac{3\pi}{2},$ значит $\alpha$ находится в $III$ четверти, поэтому его косинус отрицателен и равен $\cos(\alpha)=-\sqrt{\cfrac{-\cfrac{120}{169}+1}{2}}=-\cfrac{7\sqrt{2}}{26}.$\\