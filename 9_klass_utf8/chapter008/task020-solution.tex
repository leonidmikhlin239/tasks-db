20. Если $\alpha-\cfrac{\pi}{4}$ находится в I четверти, то $\alpha-\cfrac{\pi}{4}-\cfrac{\pi}{2}=\alpha-\cfrac{3\pi}{4}$ находится в IV четверти и $\sin\left(\alpha-\cfrac{3\pi}{4}\right)=-\sqrt{1-\cfrac{25}{169}}=-\cfrac{12}{13}.$ Тогда $\sin(\alpha)=\sin\left(\alpha-\cfrac{3\pi}{4}+\cfrac{3\pi}{4}\right)=
-\cfrac{12}{13}\cdot\left(-\cfrac{\sqrt{2}}{2}\right)-\cfrac{5}{13}\cdot\cfrac{\sqrt{2}}{2}=\cfrac{7\sqrt{2}}{26}.$\\
