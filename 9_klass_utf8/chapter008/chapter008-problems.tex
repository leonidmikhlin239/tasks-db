\section{Тригонометрия задачи}
1. Упростить $\cfrac{\cos(3\alpha)+\cos(4\alpha)+\cos(5\alpha)}{\sin(3\alpha)+\sin(4\alpha)+\sin(5\alpha)}.$\\
2. Упростить $\cfrac{\cos(6\alpha)+\cos(8\alpha)+\cos(10\alpha)}{\sin(6\alpha)+\sin(8\alpha)+\sin(10\alpha)}.$\\
3. Найти $\cos(\alpha),$ если $\sin\left(\alpha+\cfrac{\pi}{4}\right)=-\cfrac{12}{13},$ а угол $\alpha-\cfrac{\pi}{4}$ находится в $III$ четверти.\\
4. Найти $\sin(\alpha),$ если $\cos\left(\alpha-\cfrac{3\pi}{4}\right)=\cfrac{5}{13},$ а угол $\alpha-\cfrac{\pi}{4}$ находится в $II$ четверти.\\
5. Пусть $tg(2\alpha)=\cfrac{1}{\sqrt{3}}.$ Найти все значения, которые может принимать $\sin(\alpha)+\cos(\alpha).$\\
6. Пусть $ctg(2\alpha)=\sqrt{3}.$ Найти все значения, которые может принимать $\sin(\alpha)-\cos(\alpha).$\\
7. Построить график функции $f(x)=-\sqrt{\sin^2\left(\cfrac{x}{2}\right)}+\cfrac{\cos\left(\cfrac{x}{2}\right)-\cos\left(\cfrac{3x}{2}\right)}
{\sqrt{2-2\cos(2x)}}.$\\
8.\ Построить график функции $f(x)=\cfrac{\cos\left(\cfrac{3x}{2}\right)-\cos\left(\cfrac{x}{2}\right)}
{\sqrt{\cos^2(x)+\sin^2(x)-2\cos(2x)+1}}+\left|\sin\left(\cfrac{x}{2}\right)\right|.$\\
9. Упростить: $\cfrac{1-\cos(2\alpha)+\sin(2\alpha)}{1+\cos(2\alpha)+\sin(2\alpha)}.$\\
10. Упростить: $\cfrac{1+\cos(\alpha)+\cos(2\alpha)+\cos(3\alpha)}{\cos(\alpha)+\cos(2\alpha)}.$\\
11. Вычислить: $tg(41^\circ)\cdot tg(42^\circ)\cdot \ldots \cdot tg(48^\circ)\cdot tg(49^\circ).$\\
12. Вычислить: $ctg(41^\circ)\cdot ctg(42^\circ)\cdot \ldots \cdot ctg(48^\circ)\cdot ctg(49^\circ).$\\
13. Упростить: $\cfrac{\sin(\alpha)}{1+\cos(\alpha)}+ctg(\alpha).$\\
14. Упростить: $\cfrac{\cos(\alpha)}{1+\sin(\alpha)}+tg(\alpha).$\\
15. Вычислите: $ctg(160^\circ)\cdot tg(20^\circ)\cdot ctg(135^\circ).$\\
16. Вычислите: $ctg(140^\circ)\cdot tg(40^\circ)\cdot tg(135^\circ).$\\
17. Найдите $tg(\alpha),$ если $\cos(\alpha)=-\cfrac{5}{13},\ 90^\circ<\alpha<180^\circ.$\\
18. Найдите $ctg(\alpha),$ если $\sin(\alpha)=\cfrac{5}{13},\ 90^\circ<\alpha<180^\circ.$\\
19. Вычислите $\cos(\alpha),$ если $\sin\left(\alpha+\cfrac{\pi}{6}\right)=-\cfrac{13}{14},$ а $\left(\alpha-\cfrac{\pi}{3}\right)$ --- угол $III$ четверти.\\
20. Вычислите $\sin(\alpha),$ если $\cos\left(\alpha-\cfrac{3\pi}{4}\right)=\cfrac{5}{13},$ а $\left(\alpha-\cfrac{\pi}{4}\right)$ --- угол $I$ четверти.\\
21. Дано: $\sin(\alpha)=0,28,\ \cfrac{\pi}{2}<\alpha<\pi.$ Найдите $\sin(2\alpha).$\\
22. Дано: $\cos(\alpha)=\cfrac{5}{13},\ -\cfrac{\pi}{2}<\alpha<0.$ Найдите $\sin(2\alpha).$\\
23. Дано: $\sin(\alpha)=-0,28,\ \pi<\alpha<\cfrac{3\pi}{2}.$ Найдите $\sin(2\alpha).$\\
24. Дано: $\cos(\alpha)=-\cfrac{5}{13},\ \pi<\alpha<\cfrac{3\pi}{2}.$ Найдите $\sin(2\alpha).$\\
25. Дано: $\sin(\alpha)=\cfrac{12}{13},\ \cfrac{\pi}{2}<\alpha<\pi.$ Найдите $\cos(2\alpha).$\\
26. Дано: $\cos(\alpha)=-\cfrac{5}{13},\ \cfrac{\pi}{2}<\alpha<\pi.$ Найдите $\sin(2\alpha).$\\
27. В треугольнике $ABC$ угол $C$ равен $90^\circ,\ \sin(A)=\cfrac{7}{25}.$ Найдите $\cos(A).$\\
28. В треугольнике $ABC$ угол $C$ равен $90^\circ,\ \sin(A)=\cfrac{24}{25}.$ Найдите $\sin(B).$\\
29. Вычислите $\sin(2\alpha),$ если $tg(\alpha)=3.$\\
30. Вычислите $\cos(2\alpha),$ если $ctg(\alpha)=0,5.$\\
31. Найти $\cfrac{\sin(\alpha)+2\cos(\alpha)}{\sin(\alpha)-\cos(\alpha)},$ если $tg(\alpha)=\cfrac{5}{4}.$
\newpage
