5. $tg(2\alpha)=\cfrac{2tg(\alpha)}{1-tg^2(\alpha)}=\cfrac{1}{\sqrt{3}},\ 1-tg^2(\alpha)=2\sqrt{3}tg(\alpha),\ tg^2(\alpha)+2\sqrt{3}tg(\alpha)-1=0,\
tg(\alpha)=2-\sqrt{3}$ или $tg(\alpha)=-2-\sqrt{3}.$ В первом случае $\sin(\alpha)=(2-\sqrt{3})\cos(\alpha)$ и $(4-4\sqrt{3}+3)\cos^2(\alpha)+\cos^2(\alpha)=1,\
\cos^2(\alpha)=\cfrac{1}{4(2-\sqrt{3})}=\cfrac{4+2\sqrt{3}}{8},\ \cos(\alpha)=\pm\cfrac{\sqrt{3}+1}{2\sqrt{2}}.$ Тогда $\sin(\alpha)=\cfrac{2\sqrt{3}+2-3-\sqrt{3}}{2\sqrt{2}}=\cfrac{\sqrt{3}-1}{2\sqrt{2}}$ или $\sin(\alpha)=\cfrac{1-\sqrt{3}}{2\sqrt{2}}.$ Поэтому $\sin(\alpha)+\cos(\alpha)=\cfrac{\sqrt{3}-1}{2\sqrt{2}}+\cfrac{\sqrt{3}+1}{2\sqrt{2}}=\cfrac{\sqrt{6}}{2}$ или $\sin(\alpha)+\cos(\alpha)=\cfrac{1-\sqrt{3}}{2\sqrt{2}}+\cfrac{-1-\sqrt{3}}{2\sqrt{2}}=-\cfrac{\sqrt{6}}{2}.$ Во втором случае $\sin(\alpha)=(-2-\sqrt{3})\cos(\alpha)$ и $(4+4\sqrt{3}+3)\cos^2(\alpha)+\cos^2(\alpha)=1,\
\cos^2(\alpha)=\cfrac{1}{4(2+\sqrt{3})}=\cfrac{4-2\sqrt{3}}{8},\ \cos(\alpha)=\pm\cfrac{\sqrt{3}-1}{2\sqrt{2}}.$ Тогда $\sin(\alpha)=\cfrac{-2\sqrt{3}+2-3+\sqrt{3}}{2\sqrt{2}}=\cfrac{-\sqrt{3}-1}{2\sqrt{2}}$ или $\sin(\alpha)=\cfrac{\sqrt{3}+1}{2\sqrt{2}}.$ Поэтому $\sin(\alpha)+\cos(\alpha)=\cfrac{-\sqrt{3}-1}{2\sqrt{2}}+\cfrac{\sqrt{3}-1}{2\sqrt{2}}=-\cfrac{\sqrt{2}}{2}$ или $\sin(\alpha)+\cos(\alpha)=\cfrac{\sqrt{3}+1}{2\sqrt{2}}+\cfrac{1-\sqrt{3}}{2\sqrt{2}}=\cfrac{\sqrt{2}}{2}.$ Таким образом, $\sin(\alpha)+\cos(\alpha)\in\left\{\pm\cfrac{\sqrt{2}}{2};\pm\cfrac{\sqrt{6}}{2}\right\}.$\\
