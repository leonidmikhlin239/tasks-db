19. Если $\alpha-\cfrac{\pi}{3}$ находится в III четверти, то $\alpha-\cfrac{\pi}{3}+\cfrac{\pi}{2}=\alpha+\cfrac{\pi}{6}$ находится в IV четверти и $\cos\left(\alpha+\cfrac{\pi}{6}\right)=\sqrt{1-\cfrac{169}{196}}=\cfrac{3\sqrt{3}}{14}.$ Тогда $\cos(\alpha)=\cos\left(\alpha+\cfrac{\pi}{6}-\cfrac{\pi}{6}\right)=
\cfrac{3\sqrt{3}}{14}\cdot\cfrac{\sqrt{3}}{2}-\cfrac{13}{14}\cdot\cfrac{1}{2}=-\cfrac{1}{7}.$\\
