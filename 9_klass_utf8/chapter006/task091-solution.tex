91. Так как у параболы $f$ ветви направлены вверх, а у $g$ --- вниз, имеем соотношения $a_1>0>a_2.$ Так как значения $f$ и $g$ при $x=0$ совпадают, имеем равенство $c_1=c_2.$ Так как параболы касаются, уравнение $f(x)=g(x)$ имеет ровно одно решение $x=0.$ Но если $a_1x^2+b_1x+c_1=a_2x^2+b_2x+c_2,$ то с учётом равенства $c_1=c_2$ и того, что $a_1-a_2\neq0,$ получим второй корень $x=\cfrac{b_2-b_1}{a_1-a_2}.$ Так как второго корня быть не может, он совпадает с первым $x=0,$ а значит $b_1=b_2.$\\
