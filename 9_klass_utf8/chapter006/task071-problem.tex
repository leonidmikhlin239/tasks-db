72. Может ли парабола, приведённая на рисунке (абсцисса её вершины равна 3), быть графиком функции $y=a(x-2)(x-5),$ где $a\neq0?$
$$\begin{tikzpicture}[scale=1]
\begin{axis}[
    axis lines = middle,
    legend pos={south west},
    %xlabel = {$x$},
    %xlabel style={below right},
    %ylabel = {$y$},
    %title={$\text{Рис. 2}$},
    ymin=-3,
    ymax=5.5,
    xmin=-1,
    xmax=6,
    %xtick=\empty,
	%ytick=\empty,
    ]
	\addplot[domain=-1:6, samples=100, color=black] {(0.2*(3*x^2-18*x)+4)};
	%\addplot[domain=-5:1, samples=100, color=blue] {0.6*(x^2+4*x+2)};
%\addlegendentry{$\text{Рис. 1}$};
\end{axis}
\draw (1.2,5.7) node {\scriptsize $y$};
\draw (7,2.2) node {\scriptsize $x$};
\end{tikzpicture}$$
