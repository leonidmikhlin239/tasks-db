1. Переходы туристов представляют из себя арифметическую прогрессию с $a_1=1800,\ d=-200,$\\$S_n=4800.$ Тогда $4800=1800n+\cfrac{n(n-1)}{2}\cdot(-200),\
n^2-19n+48=0,\ n=3$ (второй корень $n=16$ говорит о том, что туристы, продолжив подниматься с той же закономерностью, начнут двигаться в обратном направлении и окажутся на перевале второй раз). Значит, туристы достигнут перевала за три дня.\\
