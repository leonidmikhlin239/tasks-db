13. Трёхзначные числа, заканчивающиеся на 3, представляют из себя арифметическую прогрессию с $a_1=103,\ a_n=993,\ d=10,\ n=\cfrac{993-103}{10}+1=90.$ Их сумма равна $S_{90}=\cfrac{103+993}{2}\cdot90=49320.$ Трёхзначные числа, заканчивающиеся на 3 и делящиеся на 7, представляют из себя арифметическую прогрессию с $a_1=133,\ a_n=973,\ d=70,\ n=\cfrac{973-133}{70}+1=13.$ Их сумма равна $\cfrac{133+973}{2}\cdot13=7189.$ Таким образом, сумма всех трёхзначных чисел, не делящихся на 7 и имеющих последней цифру 3, равна $49320-7189=42131.$\\
