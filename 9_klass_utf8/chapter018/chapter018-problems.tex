\section{Прогрессии решения}
1. Переходы туристов представляют из себя арифметическую прогрессию с $a_1=1800,\ d=-200,$\\$S_n=4800.$ Тогда $4800=1800n+\cfrac{n(n-1)}{2}\cdot(-200),\
n^2-19n+48=0,\ n=3$ (второй корень $n=16$ говорит о том, что туристы, продолжив подниматься с той же закономерностью, начнут двигаться в обратном направлении и окажутся на перевале второй раз). Значит, туристы достигнут перевала за три дня.\\
2. Сплавы геологов представляют из себя арифметическую прогрессию с $a_1=40,\ d=-5,\ S_n=130.$ Тогда $130=40n+\cfrac{n(n-1)}{2}\cdot(-5),\
5n^2-85n+260=0,\ n=4$ (второй корень $n=13$ говорит о том, что геологи, продолжив сплавляться с той же закономерностью, начнут двигаться в обратном направлении и окажутся в конце реки второй раз). Значит, геологи сплавятся за четыре дня.\\
3. Натуральные числа, дающие при делении на 7 остаток 5 и не превосходящие 165, представляют из себя арифметическую прогрессию с $a_1=5,\ a_n=159,\ d=7.$ Найдём $n:\ n=\cfrac{159-5}{7}+1=23.$ Тогда $S_{23}=\cfrac{5+159}{2}\cdot23=1886.$\\
4. Натуральные числа, дающие при делении на 11 остаток 3 и не превосходящие 300, представляют из себя арифметическую прогрессию с $a_1=3,\ a_n=300,\ d=11.$ Найдём $n:\ n=\cfrac{300-3}{11}+1=28.$ Тогда $S_{28}=\cfrac{3+300}{2}\cdot28=4242.$\\
5. Все трёхзначные числа представляют из себя арифметическую прогрессию с $a_1=100,\ a_n=999,\ d=1,\ n=\cfrac{999-100}{1}+1=900.$ Тогда $S_{900}=\cfrac{999+100}{2}\cdot900=494550.$ Трёхзначные числа, кратные 5, представляют из себя арифметическую прогрессию с $a_1=100,\ a_n=995,\ d=5,\ n=\cfrac{995-100}{5}+1=180.$ Тогда $S_{180}=\cfrac{995+100}{2}\cdot180=98550.$ Таким образом, сумма всех трёхзначных чисел, не кратных 5, равна $494550-98550=396000.$\\
6. Все двузначные числа представляют из себя арифметическую прогрессию с $a_1=10,\ a_n=99,\ d=1,\ n=\cfrac{99-10}{1}+1=90.$ Тогда $S_{90}=\cfrac{99+10}{2}\cdot90=4905.$ Двузначные числа, кратные 3, представляют из себя арифметическую прогрессию с $a_1=12,\ a_n=99,\ d=3,\ n=\cfrac{99-12}{3}+1=30.$ Тогда $S_{30}=\cfrac{99+12}{2}\cdot30=1665.$ Таким образом, сумма всех двузначных чисел, не кратных 3, равна $4905-1665=3240.$\\
7. $b_5^2=b_3b_7=2\cdot32=64\Rightarrow b_5=8.$\\
8. $b_6^2=b_2b_{10}=16\cdot4=64\Rightarrow b_6=8.$\\
9. Если они являются последовательными членами арифметической прогрессии в порядке\\ $a^2;\ 4a;\ 2a+5,$ то $2\cdot4a=a^2+2a+5,\ a^2-6a+5=0,\ a=5$ или $a=1.$ Если средним членом является $a^2,$ то $2a^2=4a+2a+5,\ 2a^2-6a-5=0,\ a=\cfrac{3\pm\sqrt{19}}{2}.$ Если же средним членом является $2a+5,$ то $2(2a+5)=a^2+4a,\ a^2=10,\ a=\pm\sqrt{10}.$\\
10. Если они являются последовательными членами арифметической прогрессии в порядке\\ $a^2;\ 3a;\ a+4,$ то $2\cdot3a=a^2+a+4,\ a^2-5a+4=0,\ a=4$ или $a=1.$ Если средним членом является $a^2,$ то $2a^2=3a+a+4,\ 2a^2-4a-4=0,\ a=1\pm\sqrt{3}.$ Если же средним членом является $a+4,$ то $2(a+4)=a^2+3a,\ a^2+a-8=0,\ a=\cfrac{-1\pm\sqrt{33}}{2}.$\\
11. Составим систему уравнений: $\begin{cases}a_1+2d=10,\\a_1+7d=30.\end{cases}\Leftrightarrow\begin{cases}a_1+2d=10,\\5d=20.\end{cases}
\Leftrightarrow\begin{cases}a_1=2,\\d=4.\end{cases}$ Тогда $S_n=2n+\cfrac{n(n-1)}{2}\cdot4=242,\ n^2-n+n=121,\ n^2=121,\ n=11.$\\
12. Составим систему уравнений: $\begin{cases}a_1+2d=21,\\a_1+8d=51.\end{cases}\Leftrightarrow\begin{cases}a_1+2d=21,\\6d=30.\end{cases}
\Leftrightarrow\begin{cases}a_1=11,\\d=5.\end{cases}$ Тогда $S_n=11n+\cfrac{n(n-1)}{2}\cdot5=396,\ 5n^2-5n+22n=792,\ 5n^2+17n-792=0,\ n=11.$\\
13. Трёхзначные числа, заканчивающиеся на 3, представляют из себя арифметическую прогрессию с $a_1=103,\ a_n=993,\ d=10,\ n=\cfrac{993-103}{10}+1=90.$ Их сумма равна $S_{90}=\cfrac{103+993}{2}\cdot90=49320.$ Трёхзначные числа, заканчивающиеся на 3 и делящиеся на 7, представляют из себя арифметическую прогрессию с $a_1=133,\ a_n=973,\ d=70,\ n=\cfrac{973-133}{70}+1=13.$ Их сумма равна $\cfrac{133+973}{2}\cdot13=7189.$ Таким образом, сумма всех трёхзначных чисел, не делящихся на 7 и имеющих последней цифру 3, равна $49320-7189=42131.$\\
14. Трёхзначные числа, заканчивающиеся на 5, представляют из себя арифметическую прогрессию с $a_1=105,\ a_n=995,\ d=10,\ n=\cfrac{995-105}{10}+1=90.$ Их сумма равна $S_{90}=\cfrac{105+995}{2}\cdot90=49500.$ Трёхзначные числа, заканчивающиеся на 5 и делящиеся на 11, представляют из себя арифметическую прогрессию с $a_1=165,\ a_n=935,\ d=110,\ n=\cfrac{935-165}{110}+1=8.$ Их сумма равна $\cfrac{165+935}{2}\cdot8=4400.$ Таким образом, сумма всех трёхзначных чисел, не делящихся на 11 и имеющих последней цифру 5, равна $49500-4400=45100.$\\
15. Составим систему уравнений: $\begin{cases}b_1q=6,\\b_1q^3=24.\end{cases}\Leftrightarrow\begin{cases}b_1q=6,\\q^2=4.\end{cases}
\Leftrightarrow\left[\begin{array}{l}\begin{cases}b_1=3,\\q=2.\end{cases}\\ \begin{cases}b_1=-3,\\q=-2.\end{cases}\end{array}
ight.$ В первом случае
$S_8=3\cdot\cfrac{2^8-1}{2-1}=765,$ а во втором $S_8=(-3)\cdot\cfrac{(-2)^8-1}{-2-1}=255.$\\
16. Составим систему уравнений: $\begin{cases}b_1q^2=54,\\b_1q^4=6.\end{cases}\Leftrightarrow\begin{cases}b_1q^2=54,\\q^2=\cfrac{1}{9}.\end{cases}
\Leftrightarrow\left[\begin{array}{l}\begin{cases}b_1=486,\\q=\cfrac{1}{3}.\end{cases}\\ \begin{cases}b_1=486,\\q=-\cfrac{1}{3}.\end{cases}\end{array}
ight.$ В первом случае
$S_6=486\cdot\cfrac{\cfrac{1}{3^6}-1}{\cfrac{1}{3}-1}=728,$ а во втором $S_6=486\cdot\cfrac{\cfrac{1}{3^6}-1}{-\cfrac{1}{3}-1}=364.$\\
17. $a_7+a_{13}=2a_{10}=16,\ a_{10}=8.$\\
18. $a_5+a_{17}=2a_{11}=20,\ a_{11}=10.$\\
19. Составим систему уравнений: $\begin{cases}a_1+5d=\cfrac{3}{4},\\a_1+9d=\cfrac{7}{4}.\end{cases}\Leftrightarrow\begin{cases}a_1+5d=\cfrac{3}{4},\\4d=1.\end{cases}
\Leftrightarrow\begin{cases}a_1=-\cfrac{1}{2},\\d=\cfrac{1}{4}.\end{cases}$ Тогда $S_6=6\cdot\left(-\cfrac{1}{2}
ight)+\cfrac{6\cdot5}{2}\cdot\cfrac{1}{4}=\cfrac{3}{4}.$\\
20. Составим систему уравнений: $\begin{cases}a_1+2d=-1,\\a_1+4d=3.\end{cases}\Leftrightarrow\begin{cases}a_1+2d=-1,\\2d=4.\end{cases}
\Leftrightarrow\begin{cases}a_1=-5,\\d=2.\end{cases}$ Тогда $S_{10}=10\cdot\left(-5
ight)+\cfrac{10\cdot9}{2}\cdot2=40.$\\
21. Найдём $d=\cfrac{1,8-1,2}{3}=0,2.$ Тогда $S_6=6\cdot1,2+\cfrac{6\cdot5}{2}\cdot0,2=10,2.$\\
22. $a_6-a_4=2d=6,\ d=3.$ Тогда $a_1+3=-5,\ a_1=-8.$ Таким образом, $S_{10}=10\cdot(-8)+\cfrac{10\cdot9}{2}\cdot3=55.$\\
23. Нормы покраски представляют из себя арифметическую прогрессию, в которой $S_n=240,\ a_1+a_n=60.$ Тогда $\cfrac{60}{2}\cdot n=240,\ n=8$ дней.\\
24. Расстояния, проползаемые улиткой, представляют из себя арифметическую прогрессию, в которой $S_n=150,\ a_1+a_n=10.$ Тогда $\cfrac{10}{2}\cdot n=150,\ n=30$ дней.\\
25. Пусть первое число равно $x,$ тогда знаменатель геометрической прогрессии равен $\cfrac{6}{x}$ и третье число равно $\cfrac{36}{x}.$ Значит, арифметическую прогрессию составляют числа $x,\ 6,\ \cfrac{36}{x}-x.$ Если средний член равен 6, то $2\cdot6=x+\cfrac{36}{x}-x,\ x=3$ и разность арифметической прогрессии равна $6-3=3.$ Если средний член равен $x,$ то $2x=6+\cfrac{36}{x}-x,\ 3x^2-6x-36=0,\ x^2-2x-12=0,\ x=1\pm\sqrt{13}.$ Тогда разность арифметической прогрессии равна либо $6-(1+\sqrt{13})=5-\sqrt{13},$ либо $6-(1-\sqrt{13})=5+\sqrt{13}.$ Если средний член равен $\cfrac{36}{x}-x,$ то $2\cdot\left(\cfrac{36}{x}-x
ight)=x+6,\ 3x^2+6x-72=0,\ x^2+2x-24=0,\ x=4$ или $x=-6.$ Тогда разность арифметической прогрессии равна либо $5-4=1,$ либо $0-(-6)=6.$\\
26. Пусть первое число равно $x,$ а разность арифметической прогрессии равна $d,$ тогда изначально числа были равны $x,\ x+d,\ x+2d.$ Полученные числа равны $x,\ x+d,\ 3x+3d=3(x+d).$ Если числа $x+d$ и $3(x+d)$ являются последовательными членами этой геометрической прогрессии, то её знаменатель может быть равен 3 или $\cfrac{1}{3}$ в зависимости от того, в каком порядке они расположены. Если же они не последовательные члены, то между ними расположен $x$ и $x^2=(x+d)\cdot3(x+d),\
x^2=3x^2+6xd+3d^2,\ 2x^2+6xd+3d^2=0,\ 2\left(\cfrac{x}{d}
ight)^2+6\left(\cfrac{x}{d}
ight)+3=0,\ \cfrac{x}{d}=\cfrac{-3\pm\sqrt{3}}{2}.$ В первом случае знаменатель геометрической прогрессии может быть равен $\cfrac{x+d}{x}=1+\cfrac{d}{x}=1+\cfrac{2}{-3+\sqrt{3}}=\cfrac{\sqrt{3}-1}{\sqrt{3}(1-\sqrt{3})}=-\cfrac{1}{\sqrt{3}}$ или $-\sqrt{3},$ если числа расположены в обратном порядке. Во втором случае $\cfrac{x+d}{x}=1+\cfrac{d}{x}=1+\cfrac{2}{-3-\sqrt{3}}=\cfrac{-\sqrt{3}-1}{-3-\sqrt{3}}=\cfrac{1}{\sqrt{3}}$ или
$\sqrt{3},$ если числа расположены в обратном порядке.
Таким образом, знаменатель геометрической прогрессии может быть равен $3,\ \cfrac{1}{3},\ -\sqrt{3},\ -\cfrac{1}{\sqrt{3}},\ \sqrt{3},\ \cfrac{1}{\sqrt{3}}.$\\
27. Составим систему уравнений: $\begin{cases}S_{10}=30,\\ a_7^2=a_4a_5.\end{cases}\Leftrightarrow
\begin{cases}\cfrac{a_1+a_{10}}{2}\cdot10=30,\\ (a_1+6d)^2=(a_1+3d)(a_1+4d).\end{cases}\Leftrightarrow$\\$
\begin{cases}2a_1+9d=6,\\ a_1^2+12a_1d+36d^2=a_1^2+7a_1d+12d^2.\end{cases}\Leftrightarrow
\begin{cases}2a_1+9d=6,\\ d(5a_1+24d)=0.\end{cases}\Leftrightarrow
\begin{cases}10a_1+45d=30,\\ 10a_1+48d=0.\end{cases}\Rightarrow
3d=-30\Rightarrow d=-10.$\\
28. Составим систему уравнений: $\begin{cases}S_{13}=130,\\ a_{10}^2=a_4a_7.\end{cases}\Leftrightarrow
\begin{cases}\cfrac{a_1+a_{13}}{2}\cdot13=130,\\ (a_1+9d)^2=(a_1+3d)(a_1+6d).\end{cases}\Leftrightarrow$\\$
\begin{cases}2a_1+12d=20,\\ a_1^2+18a_1d+81d^2=a_1^2+9a_1d+18d^2.\end{cases}\Leftrightarrow
\begin{cases}a_1+6d=10,\\ 9d(a_1+7d)=0.\end{cases}\Leftrightarrow
\begin{cases}a_1+6d=10,\\ a_1+7d=0.\end{cases}\Rightarrow
d=-10.$\\
29. Если $a_5+a_9=40,$ то $2a_1+12d=40,\ a_1+6d=20.$ Тогда $a_3+a_7+a_{11}=3a_1+18d=3(a_1+6d)=3\cdot20=60.$\\
30. Если $a_4+a_6=38,$ то $2a_1+8d=38,\ a_1+4d=19.$ Тогда $a_2+a_5+a_{8}=3a_1+12d=3(a_1+4d)=3\cdot19=57.$\\
31. $a_1+a_3+\ldots+a_{21}=a_2+a_4+\ldots+ a_{20}+15,\ a_1+(a_3-a_2)+(a_5-a_4)+\ldots+(a_{21}-a_{20})=15,\ a_1+10d=15.$ Составим систему уравнений:
$\begin{cases} a_1+10d=15,\\ a_{20}=3a_9.\end{cases}\Leftrightarrow
\begin{cases} a_1+10d=15,\\ a_1+19d=3a_1+24d.\end{cases}\Leftrightarrow
\begin{cases} a_1+10d=15,\\ -2a_1=5d.\end{cases}\Leftrightarrow
\begin{cases} a_1-4a_1=15,\\ -2a_1=5d.\end{cases}\Leftrightarrow
\begin{cases} a_1=-5,\\ d=2.\end{cases}$
Тогда $a_{12}=a_1+11d=-5+22=17.$\\
32. $2(a_2+a_4+\ldots+a_{18})=a_1+a_2+\ldots+ a_{19}+10,\ (a_2-a_1)+(a_4-a_3)+(a_6-a_5)+\ldots+(a_{18}-a_{17})-a_{19}=10,\ -a_1-9d=10.$ Составим систему уравнений:
$\begin{cases} -a_1-9d=10,\\ a_{3}=2a_4.\end{cases}\Leftrightarrow
\begin{cases} -a_1-9d=10,\\ a_1+2d=2a_1+6d.\end{cases}\Leftrightarrow
\begin{cases} -a_1-9d=10,\\ -a_1=4d.\end{cases}\Leftrightarrow
\begin{cases} 4d-9d=10,\\ -a_1=4d.\end{cases}\Leftrightarrow
\begin{cases} d=-2,\\ a_1=8.\end{cases}$
Тогда $a_{13}=a_1+12d=8-24=-16.$\\
33. $a_2=0,88a_1,\ a_1+d=0,88a_1.\ d=-0,12a_1.$ Тогда $a_5=a_1+4d=a_1-0,48a_1=0,52a_1.$ Тогда первый член составляет от пятого $\cfrac{a_1}{a_5}\cdot100\%=
\cfrac{a_1}{0,52a_1}\cdot100\%=\cfrac{2500}{13}\%.$\\
34. Составим систему уравнений: $\begin{cases} b_1\cdot\cfrac{q^3-1}{q-1}=21,\\ b_1^2\cdot\cfrac{q^6-1}{q^2-1}=189.\end{cases}\Leftrightarrow
\begin{cases} b_1(q^2+q+1)=21,\\ b_1^2(q^4+q^2+1)=189.\end{cases}\Leftrightarrow$\\$
\begin{cases} b_1^2(q^4+2q^3+3q^2+2q+1)=441,\\ b_1^2(q^4+q^2+1)=189.\end{cases}\Rightarrow
\cfrac{q^4+2q^3+3q^2+2q+1}{q^4+q^2+1}=\cfrac{441}{189}=\cfrac{7}{3}\Leftrightarrow
3q^4+6q^3+9q^2+6q+3=7q^4+7q^2+7\Leftrightarrow
4q^4-6q^3-2q^2-6q+4=0\Leftrightarrow 2q^4-3q^3-q^2-3q+2=0\Big|:q^2 \Leftrightarrow
2q^2-3q-1-\cfrac{3}{q}+\cfrac{2}{q^2}=0 \Leftrightarrow 2\left(q^2+\cfrac{1}{q^2}
ight)-3\left(q+\cfrac{1}{q}
ight)-1=0.$
Сделаем замену $q+\cfrac{1}{q}=t,$ тогда $t^2=q^2+2+\cfrac{1}{q^2}$ и $2(t^2-2)-3t-1=0,\ 2t^2-3t-5=0,\ t=\cfrac{5}{2}$ или $t=-1.$ В первом случае $q+\cfrac{1}{q}=
\cfrac{5}{2},\ 2q^2-5q+2=0,\ q=\cfrac{1}{2}$ или $q=2.$ Так как прогрессия убывающая, подходит только $q=\cfrac{1}{2},$ тогда $b_1=\cfrac{21}{\cfrac{1}{4}+\cfrac{1}{2}+1}=12.$  Во втором случае $q+\cfrac{1}{q}=-1,\ q^2+q+1=0,\ q\in\varnothing.$\\
35. Пусть стороны треугольника равны $x,\ qx$ и $q^2x.$ По обратной теореме Пифагора для того, чтобы он был прямоугольным, достаточно выполнения равенства
$x^2+q^2x^2=q^4x^2,\ q^4-q^2-1=0.$ У этого биквадратного уравнения есть положительный корень $q^2=\cfrac{1+\sqrt{5}}{2},$ а значит существует подходящее $q=\sqrt{\cfrac{1+\sqrt{5}}{2}},$ поэтому такой прямоугольный треугольник существует.\\
36. Все трёхзначные числа представляют из себя арифметическую прогрессию с $a_1=100,\ a_n=999,\ d=1,\ n=\cfrac{999-100}{1}+1=900.$ Тогда $S_{900}=\cfrac{999+100}{2}\cdot900=494550.$ Трёхзначные числа, кратные 11, представляют из себя арифметическую прогрессию с $a_1=110,\ a_n=990,\ d=11,\ n=\cfrac{990-110}{11}+1=81.$ Тогда $S_{81}=\cfrac{990+110}{2}\cdot81=44550.$ Таким образом, сумма всех трёхзначных чисел, не кратных 11, равна $494550-44550=450000.$\\
37. Если число кратно 5 и кратно 7, оно кратно 35. Трёхзначные числа, кратные 7, представляют из себя арифметическую прогрессию с $a_1=105,\ a_n=994,\ d=7,\ n=\cfrac{994-105}{7}+1=128.$ Их сумма равна $S_{128}=\cfrac{105+994}{2}\cdot128=70336.$ Трёхзначные числа, кратные 35, представляют из себя арифметическую прогрессию с $a_1=105,\ a_n=980,\ d=35,\ n=\cfrac{980-105}{35}+1=26.$ Их сумма равна $S_{26}=\cfrac{105+980}{2}\cdot26=14105.$ Таким образом, сумма всех трёхзначных чисел, кратных 7, но не кратных 5, равна $70336-14105=56231.$\\
38. Если число кратно 5 и кратно 7, оно кратно 35. Трёхзначные числа, кратные 5, представляют из себя арифметическую прогрессию с $a_1=100,\ a_n=995,\ d=5,\ n=\cfrac{995-100}{5}+1=180.$ Их сумма равна $S_{180}=\cfrac{100+995}{2}\cdot180=98550.$ Трёхзначные числа, кратные 35, представляют из себя арифметическую прогрессию с $a_1=105,\ a_n=980,\ d=35,\ n=\cfrac{980-105}{35}+1=26.$ Их сумма равна $S_{26}=\cfrac{105+980}{2}\cdot26=14105.$ Таким образом, сумма всех трёхзначных чисел, кратных 5, но не кратных 7, равна $98550-14105=84445.$\\
39. Если число чётно и оно делится на 3, то оно делится на 6. Такие числа представляют из себя арифметическую прогрессию, в которой $d=6,\ a_1=12=6\cdot2,\ a_n=96=6\cdot16,\ n=16-2+1=15.$ Тогда $S_{15}=\cfrac{12+96}{2}\cdot15=810.$\\
40. Двузначные числа, делящиеся на 7, представляют из себя арифметическую прогрессию, в которой $d=7,\ a_1=14=7\cdot2,\ a_n=98=7\cdot14,\ n=14-2+1=13.$ Тогда
$S_{13}=\cfrac{14+98}{2}\cdot13=728.$\\
41. $a_5a_7=(a_1+4d)(a_1+6d)=(4+4d)(4+6d)=16+24d+16d+24d^2=24d^2+40d+16.$ Наименьшее значение парабола принимает в своей вершине $d=-\cfrac{b}{2a}=-\cfrac{40}{48}=-\cfrac{5}{6}.$\\
42. Трёхзначные числа, дающие при делении на 11 остаток 5, представляют из себя арифметическую прогрессию с $a_1=104,\ a_n=995,\ d=11.$ Найдём $n:\ n=\cfrac{995-104}{11}+1=82.$ Тогда $S_{82}=\cfrac{104+995}{2}\cdot82=45059.$\\
43. Трёхзначные числа, дающие при делении на 13 остаток 7, представляют из себя арифметическую прогрессию с $a_1=111,\ a_n=995,\ d=13.$ Найдём $n:\ n=\cfrac{995-111}{13}+1=69.$ Тогда $S_{69}=\cfrac{111+995}{2}\cdot69=38157.$\\
44. Пусть разность арифметической прогрессии равна $d,$ тогда $b=a+d,\ c=a+3d,$ и имеет место система уравнений
$\begin{cases} a(a+3d)=(a+d)^2,\\ a+a+d+a+3d=35.\end{cases}\Leftrightarrow\begin{cases} a^2+3ad=a^2+2ad+d^2,\\ 3a+4d=35.\end{cases}\Leftrightarrow
\begin{cases} ad=d^2,\\ 3a+4d=35.\end{cases}$\\$\Leftrightarrow\begin{cases} a=d,\\ 7a=35.\end{cases}
\Leftrightarrow\begin{cases} a=5,\\ d=5.\end{cases}$ Таким образом, $a=5,\ b=5+5=10,\ c=5+15=20.$\\
45. Пусть разность арифметической прогрессии равна $d,$ тогда $b=a+d,\ c=a+5d,$ и имеет место система уравнений
$\begin{cases} a(a+5d)=(a+d)^2,\\ a+a+5d-a-d=13.\end{cases}\Leftrightarrow\begin{cases} a^2+5ad=a^2+2ad+d^2,\\ a+4d=13.\end{cases}\Leftrightarrow
\begin{cases} 3ad=d^2,\\ a+4d=13.\end{cases}$\\$\Leftrightarrow\begin{cases} d=3a,\\ 13a=13.\end{cases}
\Leftrightarrow\begin{cases} a=1,\\ d=3.\end{cases}$ Таким образом, $a=1,\ b=1+3=4,\ c=1+15=16.$\\
46. Пусть разность прогрессии равна $d,$ тогда имеем систему уравнений\\ $\begin{cases} a_1+16d+a_1+22d=400,\\a_1+19d+a_1+107d=224.\end{cases}\Leftrightarrow
\begin{cases} 2a_1+38d=400,\\2a_1+126d=224.\end{cases}\Leftrightarrow
\begin{cases} a_1=200-19d,\\88d=-176.\end{cases}\Leftrightarrow
\begin{cases} a_1=238,\\ d=-2.\end{cases}.$ Тогда $S_{239}=239\cdot238+\cfrac{239\cdot238}{2}\cdot(-2)=0.$\\
47. Пусть первый член прогрессии равен $a_1,$ а её разность равна $d,$ тогда имеет место соотношение:
$(a_1+6d)(a_1+7d)=(a_1+4d)(a_1+8d),\ a_1^2+7a_1d+6a_1d+42d^2=a_1^2+8a_1d+4a_1d+32d^2,\ 10d^2=-a_1d,\ a_1=-10d.$ Тогда $a_{11}=a_1+10d=-10d+10d=0.$\\
48. Пусть исходное число имеет вид $\overline{abc},$ тогда
$\overline{abc}-792=\overline{cba},$ то есть
$100a+10b+c-792=100c+10b+a,\ 99a-99c=792,\ a-c=8,\ a=c+8.$ Тогда $b=c+4$ и числа $c,\ c+2$ и $c+8$ образуют геометрическую прогрессию, откуда $(c+2)^2=c(c+8),\ c^2+4c+4=c^2+8,\ c=1.$ Тогда $b=1+4=5,\ a=1+8=9$ и исходное число равно 951.\\
49. Пусть исходное число имеет вид $\overline{abc},$ тогда
$\overline{abc}-792=\overline{cba},$ то есть
$100a+10b+c-792=100c+10b+a,\ 99a-99c=792,\ a-c=8,\ a=c+8.$ Так как числа $c+4,\ b,\ c$ образуют арифметическую прогрессию, $b=c+2.$ Тогда $(c+2)^2=c(c+8),\ c^2+4c+4=c^2+8c,\ 4c=4,\ c=1,\ a=1+8=9,\ b=1+2=3.$ Значит, исходное число равно 931.

ewpage
