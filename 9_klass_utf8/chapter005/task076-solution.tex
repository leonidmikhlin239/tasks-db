76. Выразим $x$ через $y:\ y^2x-y^2+4xy+6x-2y=3,\ x(y^2+4y+6)=y^2+2y+3,\ x=\cfrac{y^2+2y+3}{y^2+4y+6}.$ Числитель и знаменатель получившейся дроби всегда положительны, поэтому найти её минимальное значение --- это то же, что найти максимальное значение обратной дроби $\cfrac{y^2+4y+6}{y^2+2y+3}=1+\cfrac{2y+3}{y^2+2y+3}.$ Второе слагаемое не превосходит 1, так как $2y+3\leqslant y^2+2y+3,$ причём равенство достигается при $y^2=0,$ то есть $y=0.$ Значит, наибольшее значение обратной дроби равно 2, а наименьшее значение $x$ равно $\cfrac{1}{2}$ и достигается оно в паре $\left(\cfrac{1}{2};0\right).$\\
