67. $\sqrt{2+x-x^2}(x^2-(a+1)x+a)=0,\ \sqrt{(1+x)(2-x)}(x-1)(x-a)=0.$ У этого уравнения точно есть корни $-1,\ 1$ и 2 при любом значении $a.$ Также у него
может появиться четвёртый корень $x=a,$ если $a\notin\{-1;1;2\}$ и $a\in[-1;2].$ Таким образом, при $a\in(-1;1)\cup(1;2): 4$ корня, а при всех остальных --- 3.\\
