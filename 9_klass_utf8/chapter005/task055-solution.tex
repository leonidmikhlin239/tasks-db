55. Корни уравнения лежат по разные стороны от 1 тогда и только тогда, когда выполняется неравенство $(x_1-1)(x_2-1)<0,\ x_1x_2-(x_1+x_2)+1<0.$ По теореме Виета имеем равенства $x_1+x_2=2b+3,\ x_1x_2=-b-6.$ Поэтому $-b-6-2b-3+1<0,\
b>-\cfrac{8}{3}.$ Кроме того, необходимо проверить, что при этих значениях $b$ уравнение имеет два корня. Решим неравенство $D>0,$ то есть $(2b+3)^2+4(b+6)>0,\
4b^2+12b+9+4b+24>0,\ 4b^2+16b+33=(2b+4)^2+17>0.$ Оно выполняется при любых значениях $b,$ значит $b\in\left(-\cfrac{8}{3};+\infty\right).$\\