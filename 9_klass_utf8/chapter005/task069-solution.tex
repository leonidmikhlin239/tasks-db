69. Так как $D=4+16|a^2-5|>0$ при любых значениях $a,$ два корня у этого уравнения есть всегда. По теореме Виета имеем равенство $x_1x_2=\cfrac{-|a^2-5|}{4}.$ Тогда $\cfrac{-|a^2-5|}{4}<a\Leftrightarrow |a^2-5|>-4a \Leftrightarrow \left[\begin{array}{l} a^2-5>-4a,\\ a^2-5<4a.\end{array}\right.\Leftrightarrow \left[\begin{array}{l} (a-1)(a+5)>0,\\ (a+1)(a-5)<0.\end{array}\right.\Leftrightarrow a\in(-\infty;-5)\cup(-1;+\infty).$\\