32. Корни уравнения лежат по разные стороны от $(-1)$ тогда и только тогда, когда выполняется неравенство $(x_1+1)(x_2+1)<0,\ x_1x_2+(x_1+x_2)+1<0.$ По теореме Виета имеем равенства $x_1+x_2=a-7,\ x_1x_2=a^2-6a.$ Поэтому $a^2-6a+a-7+1<0,\ a^2-5a-6<0,\ (a-6)(a+1)<0,\ a\in(-1;6).$ Кроме того, необходимо проверить, что при этих значениях $a$ уравнение имеет два корня. Для это рассмотрим неравенство $D>0,$ то есть $(a-7)^2-4(a^2-6a)>0,\ a^2-14a+49-4a^2+24a>0,\ 3a^2-10a-49<0,\ \left(a-\cfrac{5}{3}\right)^2<\cfrac{172}{9}.$ При полученных ранее значениях $a$ это неравенство выполняется, так как максимальное значение левой части равно $\left(6-\cfrac{5}{3}\right)^2=\cfrac{169}{9}<\cfrac{172}{9}.$\\
