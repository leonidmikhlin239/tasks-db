83.  Раз $|y|=x(2-x),$ должно выполняться неравенство $x(2-x)\geqslant0,$ поэтому $x\in[0;2].$ При этом $y=2x-x^2$ или $y=x^2-2x,$ а значит $x+y=3x-x^2$ или $x+y=x^2-x.$ У первой параболы вершина в точке $x=\cfrac{3}{2},$ значение в ней равно $\cfrac{9}{4}.$ Её наименьшее значение достигается на левом конце отрезка при $x=0$ и оно также равно 0. У второй параболы вершина в точке $x=\cfrac{1}{2},$ значение в ней равно $-\cfrac{1}{4}.$ Её наибольшее значение достигается на правом конце отрезка при $x=2$ и оно равно 2. Таким образом, выражение $x+y$ может принимать значения от $-\cfrac{1}{4}$ до $\cfrac{9}{4}.$\\
