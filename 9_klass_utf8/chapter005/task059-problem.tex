59. Дано уравнение $(a-1)x^2+4(a+1)x+a-4=0.$\\
а) При каких значениях $a$ уравнение имеет единственное решение?\\
б) При $a=2$ найдите $x_1^3+x_2^3,$ где $x_1,\ x_2$ --- корни данного уравнения.\\
в) При $a=-2$ найдите все значения параметра $b,$ для которых решение неравенства\\ $(a-1)x^2+4(a+1)x+a-4\geqslant b$ --- отрезок.\\
