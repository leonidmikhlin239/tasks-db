59. а) Уравнение имеет единственное решение при $a-1=0,\ a=1$ или при $\cfrac{D}{4}=4(a+1)^2-(a-1)(a-4)=4a^2+8a+4-a^2+a+4a-4=3a^2+13a=a(3a+13)=0,\ a\in\left\{-\cfrac{13}{3};0\right\}.$\\
б) При $a=2$ уравнение имеет вид $x^2+12x-2=0.$ По теореме Виета $x_1+x_2=-12,\ x_1x_2=-2,$ тогда $x_1^3+x_2^3=(x_1+x_2)(x_1^2-x_1x_2+x_2^2)=(x_1+x_2)((x_1+x_2)^2-3x_1x_2)=(-12)\cdot(144+6)=-1800.$\\
в) При $a=-2$ неравенство примет вид $-3x^2-4x-6\geqslant b.$ Так как ветви этой параболы направлены вниз, решением будет отрезок в том случае, если значение $b$ меньше значения в вершине. Вершина находится в точке $x_{\text{верш}}=-\cfrac{-4}{-6}=-\cfrac{2}{3},\ y_{\text{верш}}=-3\cdot\cfrac{4}{9}+4\cdot\cfrac{2}{3}-6=-\cfrac{14}{3}.$ Значит, ответом является множество $\left(-\infty;-\cfrac{14}{3}\right).$\\
