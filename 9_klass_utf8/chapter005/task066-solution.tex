66. Корни уравнения лежат по разные стороны от $(-1)$ тогда и только тогда, когда выполняется неравенство $(x_1+1)(x_2+1)<0,\ x_1x_2+(x_1+x_2)+1<0.$ По теореме Виета имеем равенства $x_1+x_2=a-7,\ x_1x_2=a^2-6a+4.$ Поэтому $a^2-6a+4+a-7+1<0,\ a^2-5a-2<0,\ a\in\left(\cfrac{5-\sqrt{33}}{2};\cfrac{5+\sqrt{33}}{2}\right).$ Кроме того, необходимо проверить, что при этих значениях $a$ уравнение имеет два корня. Для это рассмотрим неравенство $D>0,$ то есть $(a-7)^2-4(a^2-6a+4)>0,\ a^2-14a+49-4a^2+24a-16>0,\ 3a^2-10a-33<0,\ a\in\left(\cfrac{5-2\sqrt{31}}{3};\cfrac{5+2\sqrt{31}}{3}\right).$ При полученных ранее значениях $a$ это неравенство выполняется, так как $\cfrac{5-2\sqrt{31}}{3}<\cfrac{5-\sqrt{33}}{2}<\cfrac{5+\sqrt{33}}{2}<\cfrac{5+2\sqrt{31}}{3}.$\\
