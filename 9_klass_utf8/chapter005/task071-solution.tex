71. Пусть $f(x)=x^2+4x+2,\ g(x)=-\sqrt{x+2}.$ График функции $f(x)$ представляет из себя параболу с вершиной в точке $x_{\text{верш}}=-\cfrac{4}{2}=-2,$ значит на луче $[-2;+\infty)$ функция $f(x)$ возрастает. Функция $g(x)$ определена на $[-2;+\infty)$ и на всём этом множестве убывает. При этом $f(-2)=-2,\ g(-2)=0$ и $f(-1)=g(-1)=-1.$ Значит, $m(x)=\begin{cases} f(x),\ x\in[-2;-1],\\ g(x),\ x\in(-1;+\infty).\end{cases}$ Таким образом, наибольшее значение функция $m(x)$ принимает при $x=-1$ и оно равно $-1.$\\
