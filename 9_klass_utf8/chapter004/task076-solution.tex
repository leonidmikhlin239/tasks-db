76. Если знаменатель правой части отрицателен $(x<2),$ то левая часть больше при любом $x$ из области определения: $x^2-5x+6=(x-2)(x-3)\neq0,\ x\neq2$ и $x\neq3.$ Значит, подходит интервал $(-\infty;2).$ Если знаменатель правой дроби положителен $(x>2),$ то необходимо выполнение неравенства $|x^2-5x+6|\leqslant x-2\Leftrightarrow \begin{cases} x^2-5x+6\leqslant x-2,\\ x^2-5x+6\geqslant 2-x.\end{cases}\Leftrightarrow \begin{cases} x^2-6x+8\leqslant0,\\ x^2-4x+4\geqslant0.\end{cases}
\Leftrightarrow \begin{cases} (x-2)(x-4)\leqslant0,\\ (x-2)^2\geqslant0.\end{cases}\Rightarrow x\in(2;3)\cup(3;4].$ Таким образом, итоговый ответ $x\in(-\infty;2)\cup (2;3)\cup(3;4].$\newpage\noindent
