78. $(x-1)x(x+1)(x+2)\leqslant8\Leftrightarrow (x^2+x-2)(x^2+x)\leqslant8.$ Сделаем замену $t=x^2+x,$ тогда $(t-2)t\leqslant8\Leftrightarrow
t^2-2t-8\leqslant0\Leftrightarrow(t-4)(t+2)\leqslant0\Leftrightarrow t\in[-2;4]\Leftrightarrow \begin{cases} x^2+x\geqslant-2,\\ x^2+x\leqslant 4.\end{cases}
\Leftrightarrow \begin{cases} x^2+x+2\geqslant0,\\ x^2+x-4\leqslant 0.\end{cases}\Leftrightarrow \left(x-\cfrac{-1-\sqrt{17}}{2}\right)\left(x-\cfrac{-1+\sqrt{17}}{2}\right)\leqslant0\Leftrightarrow x\in \left[\cfrac{-1-\sqrt{17}}{2};\cfrac{-1+\sqrt{17}}{2}\right].$ Целыми решениями этого неравенства являются числа от $-2$ до 1, их сумма равна $-2.$\\