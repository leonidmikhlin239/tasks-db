53. $\cfrac{|x-3|(4x^2+7x-2)}{10+3x-x^2}\leqslant0\Leftrightarrow\cfrac{|x-3|(4x-1)(x+2)}{-(x+2)(x-5)}\leqslant0.$ Применив метод интервалов, найдём ответ: $x\in(-\infty;-2)\cup\left(-2;\cfrac{1}{4}\right]\cup\{3\}\cup(5;+\infty).$
\begin{figure}[ht!]
\center{\includegraphics[scale=0.35]{ner9-53.png}}
\end{figure}\\
