4. Если $\alpha-\cfrac{\pi}{4}$ находится во $II$ четверти, то $\alpha-\cfrac{3\pi}{4}$ находится в $I$ четверти, его синус положителен и равен $\sin\left(\alpha-\cfrac{3\pi}{4}
ight)=\sqrt{1-\cfrac{25}{169}}=\cfrac{12}{13}.$ Тогда $\sin\left(2\alpha-\cfrac{3\pi}{2}
ight)=2\cdot\cfrac{12}{13}\cdot
\cfrac{5}{13}=\cfrac{120}{169}=\cos(2\alpha).$ Так как $\alpha-\cfrac{3\pi}{4}$ находится в $I$ четверти, его синус равен $\cfrac{12}{13},$ это угол близкий
к $\cfrac{\pi}{2},$ значит $\alpha$ находится в $III$ четверти, поэтому его синус отрицателен и равен $\sin(\alpha)=-\sqrt{\cfrac{1-\cfrac{120}{169}}{2}}=-\cfrac{7\sqrt{2}}{26}.$\\
