6. $ctg(2\alpha)=\cfrac{ctg^2(\alpha)-1}{2ctg(\alpha)}=\sqrt{3},\ ctg^2(\alpha)-1=2\sqrt{3}ctg(\alpha),\ ctg^2(\alpha)-2\sqrt{3}ctg(\alpha)-1=0,\
ctg(\alpha)=2+\sqrt{3}$ или $ctg(\alpha)=-2+\sqrt{3}.$ В первом случае $\cos(\alpha)=(2+\sqrt{3})\sin(\alpha)$ и $(4+4\sqrt{3}+3)\sin^2(\alpha)+\sin^2(\alpha)=1,\
\sin^2(\alpha)=\cfrac{1}{4(2+\sqrt{3})}=\cfrac{4-2\sqrt{3}}{8},\ \sin(\alpha)=\pm\cfrac{\sqrt{3}-1}{2\sqrt{2}}.$ Тогда $\cos(\alpha)=\cfrac{2\sqrt{3}-2+3-\sqrt{3}}{2\sqrt{2}}=\cfrac{\sqrt{3}+1}{2\sqrt{2}}$ или $\cos(\alpha)=\cfrac{-1-\sqrt{3}}{2\sqrt{2}}.$ Поэтому $\sin(\alpha)-\cos(\alpha)=\cfrac{\sqrt{3}-1}{2\sqrt{2}}-\cfrac{\sqrt{3}+1}{2\sqrt{2}}=-\cfrac{\sqrt{2}}{2}$ или $\sin(\alpha)-\cos(\alpha)=\cfrac{1-\sqrt{3}}{2\sqrt{2}}-\cfrac{-1-\sqrt{3}}{2\sqrt{2}}=\cfrac{\sqrt{2}}{2}.$ Во втором случае $\cos(\alpha)=(-2+\sqrt{3})\sin(\alpha)$ и $(4-4\sqrt{3}+3)\sin^2(\alpha)+\sin^2(\alpha)=1,\
\sin^2(\alpha)=\cfrac{1}{4(2-\sqrt{3})}=\cfrac{4+2\sqrt{3}}{8},\ \sin(\alpha)=\pm\cfrac{\sqrt{3}+1}{2\sqrt{2}}.$ Тогда $\cos(\alpha)=\cfrac{-2\sqrt{3}-2+3+\sqrt{3}}{2\sqrt{2}}=\cfrac{1-\sqrt{3}}{2\sqrt{2}}$ или $\cos(\alpha)=\cfrac{\sqrt{3}-1}{2\sqrt{2}}.$ Поэтому $\sin(\alpha)-\cos(\alpha)=\cfrac{\sqrt{3}+1}{2\sqrt{2}}-\cfrac{1-\sqrt{3}}{2\sqrt{2}}=\cfrac{\sqrt{6}}{2}$ или $\sin(\alpha)-\cos(\alpha)=\cfrac{-\sqrt{3}-1}{2\sqrt{2}}-\cfrac{\sqrt{3}-1}{2\sqrt{2}}=-\cfrac{\sqrt{6}}{2}.$ Таким образом, $\sin(\alpha)-\cos(\alpha)\in\left\{\pm\cfrac{\sqrt{2}}{2};\pm\cfrac{\sqrt{6}}{2}
ight\}.$\\
